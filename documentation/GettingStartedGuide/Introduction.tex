\section{Introduction}\label{sec:introduction}

This document is a help guide for \CNAME\, age-structured population modelling software package. This short document is aimed at users who are new to \CNAME. As the names suggests \CNAME\ is a generalised tool for carrying out age-structured population dynamics models, including fisheries assessments and other population dynamics problems. 

\CNAME is very generalised, highly flexible, and therefore can be a bit daunting at first sight. It has a large number of run modes, settings, and user defined population dynamics choices that can turned on and off, depending on the circumstances and the user requirements. While there is no requirement for a user to see or understand the underlying code base, it has been written so that is is well tested and great effort has been put into developing a code base that can be easily interpreted and understood by even novice programmers

\CNAME is open source, and is covered under the GNU GPL 2.0 licence. See the terms and conditions in the user manual \citep{CASAL2}, or type \texttt{casal2 -v} into the command prompt. There is also supplementary information that may be useful to have access while reading this. This includes a more comprehensive and detailed user manual which goes into \CNAME\ in more depth. 

Also coming soon will be a document on how \CNAME\ has inbuilt testing of code, and another validation document comparing \CNAME\ predecessor with \CNAME\. These can all be found in \CNAME\ file that is installed with the installer. By this stage we are assuming you have installed \CNAME\ using the \texttt{casal2\_setup.exe}. This will install \CNAME\ into your path and create a folder in a location of your choice containing, source code, examples, a manual, an R library, this document, an executable and auxiliary libraries.
