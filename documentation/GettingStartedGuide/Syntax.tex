\section{Syntax of a CASAL2 file}\label{Sec:stru}

A general structure of \CNAME\ files are that they are split into blocks of subcommands. A block always starts with \command{} symbol. Blocks describe different aspects of the model, fundamental blocks to have in the model are \command{model}, \command{initialisation\_phase}, \command{categories}, \command{time\_step}, and \command{process}. Within each block there will be subcommands some will be optional and important subcommands will be mandatory. An example of subcommand is shown for the \command{model} block,

{\small{\begin{verbatim}
		@model
		type age		## is the model age or length based?
		min_age 1		## minimum age in model
		max_age 17		## maximum age in model
		age_plus true	## is the last age group a plus group?
		start_year 1972	## the first year of the model
		final_year 2013 	## the first year of the model
		initialisation_phases phase1	
		## The label for the block @intialisation_phase
		time_steps step1 step2
		## Labels for the block @time_step
		\end{verbatim}}}
The subcommands are all the options that follow \command{model}, then there is a space which is where the value for the subcommand goes, i.e. \texttt{min\_age} specifies the minimum age in the model and we have set that equal to one but could be any integer. This brings up a useful concept to understand. Different subcommands can take different types of parameters, they can be of type int, double, string and vector. For information about which parameter type a subcommand takes, you should read the syntax section of the manual, there is a field labelled type. If you use the wrong type for a subcommand, for example \texttt{min\_age 1.5}, you will get an error. A line beginning with \texttt{\#} is a comment and that line is ignored by \CNAME. To comment out multi-lines the user can use the curly braces \{\}, everything between these braces will be ignored by \CNAME. It is a useful tool for annotating models. 

