\section{\I{The estimation section}\label{sec:estimation-section}}

\subsection{Role of the estimation section\label{sec:role-of-the-estimation-section}}
\CH
\KL This text from SPM, a rewrite of CASAL. OK to use? \KLend

The role of the estimation section is to define the tasks carried out by \CAS: 

\begin{enumerate}
  \item Define the objective function (see Section \ref{sec:objective-function})
  \item Define the parameters to be estimated (see Section \ref{sec:estimate-free-parameters})
  \item Calculate a point estimate, i.e., the maximum posterior density estimate (MPD) (see Section \ref{sec:estimate-MPD})\index{Maximum posterior density estimate (MPD)}\index{MPD (Maximum posterior density estimate)}.
  \item Calculate a posterior profile selected parameters, i.e., find, for each of a series of values of a parameter, allowing the other estimated parameters to vary, the minimum value of the objective function (see Section \ref{sec:estimate-profiles}\index{Profiles}).
  \item Generate an MCMC\index{Monte Carlo Markov Chain (MCMC)} sample from the posterior distribution (see Section \ref{sec:estimate-MCMC}\index{MCMC}).
  \item Calculate the approximate covariance matrix of the parameters as the inverse of the minimizer\textquoteright{}s approximation to the \I{Hessian}, and the corresponding correlation matrix (see Section \ref{sec:estimate-MPD}\index{Covariance matrix}).
\end{enumerate}

The estimation section defines the objective function, parameters of the model, and the method of estimation (point estimates, Bayesian posteriors, profiles, etc.). The objective function is based on a goodness-of-fit measure of the model to observations, priors and penalties. See the observation section for a description of the observations, likelihoods, priors and penalties. 

\subsection{\I{The objective function}\label{sec:objective-function}}
\CH
\KL This text from SPM. Does not cover all text in S6.7 of CASAL, e.g. max likelihood. OK to use? Additions required? \KLend

In Bayesian estimation, the objective function is a negative log-posterior,

\begin{equation}
Objective(p)= - \sum\limits_i {\log \left[ {L\left( {{\bf{p}}|O_i } \right)} \right]}  - \log \left[ {\pi \left( {\bf{p}} \right)} \right]
\end{equation}

where $\pi$ is the joint prior density of the parameters $p$.

The contribution to the objective function from the likelihoods are defined in Section \ref{sec:likelihoods}. In addition to likelihoods, priors (see Section \ref{sec:priors}) and penalties (see Section \ref{sec:penalties}) are components of the objective function. 

Penalties can be used to ensure that the exploitation rate constraints on mortality events (i.e., fisheries) are not breached (otherwise there is nothing to prevent the model from having abundances so low that the recorded mortalities could not have been taken), penalties on category transitions (to ensure there are enough individuals to move), and possibly penalties to encourage estimated values to be similar or smooth, etc.

\subsection{\I{Specifying the parameters to be estimated}\label{sec:estimate-free-parameters}}
\CH
\KL This text from SPM. Does not cover all the text in CASAL, e.g. catchabilities.  OK to use? Additions required? \KLend

The estimable parameters that will be estimated are defined using \command{estimate} commands (see Section \ref{sec:estimation-syntax}). An \command{estimate} command-block looks like\index{Estimating parameters},

{\small{\begin{verbatim}
  @estimate process[MyRecruitment].r0
  lower_bound 1000
  upper_bound 100000
  prior uniform
\end{verbatim}}}

See Section \ref{sec:parameter-names} for instructions on how to generate the parameter name. At least one parameter to be estimated if doing an estimation, profile, or MCMC run. Initial values for the parameters to be estimated will still need to be provided, and these are used as the starting values for the minimiser. However, these may be overwritten if you provide a set of alternative starting values (i.e., using  \texttt{\cas\ -i}, see Section \ref{sec:command-line-arguments}).

All parameters are estimated within bounds\index{Bounds}. For each parameter to be estimated, you need to specify the bounds and the prior\index{Priors} (Section \ref{sec:priors}). Note that the bounds and prior for each parameter refer to the values of the parameters, not the actual values resulting from the application of the parameter to an equation. \textbf{Bounds should be carefully chosen as they effect the space in which the minimisers search over. Minimisers convert lower and upper bound to -1,1 space. Needs to be referenced}. If estimating only some elements of a vector, either define the elements of the vector to be estimated (see \ref{sec:parameter-names}) or fix the others by setting the bounds equal.

The estimation of parameters can be phased. Here, some of the estimated parameters are initially held fixed, and a minimisation is carried out. Next, some or all of the remaining parameters that were initially held fixed are freed, and another minimisation is carried out. This process continues until all phases have been carried out.

\subsection{\I{Point estimation}\label{sec:estimate-MPD}}
\CH
\KL This text from SPM. A rewrite of what's in CASAL and set out differntly. OK to use? Additions required? \KLend

Point estimation is invoked with \texttt{\cas\ -e}. Mathematically, it is an attempt to find a minimum of the objective function\index{Objective function}. SPM has two algorithms for solving (minimising) the optimisation problem. The first uses a quasi-Newton minimiser built which is a slightly modified implementation of the main algorithm of Dennis Jr. \& Schnabel \citep{779}\index{Finite differences minimiser}, while the second uses a genetic algorithm developed by Storn \& Price \citep{1442}, the differential evolution minimiser \index{Differential evolution minimiser}. However, the second minimiser does not produce an estimate of the covariance matrix, and hence cannot be used to start an MCMC.

\subsubsection{\I{The numerical differences minimiser}}
\CH
The minimiser has three kinds of (non-error) exit status: 

\begin{enumerate}
\item Successful convergence (suggests you have found a local minimum, at least).
\item Convergence failure (you have not reached a local minimum, though you may deem yourself to be `close enough' at your own risk).
\item Convergence unclear (the minimiser halted but was unable to determine if convergence occurred. You may be at a local minimum, although you should check by restarting the minimiser at the final values of the estimated parameters).
\end{enumerate}

You can choose the maximum number of quasi-Newton iterations\index{Quasi-Newton iterations} and objective function evaluations\index{Objective function evaluations} allotted to the minimiser. If it exceeds either limit, it exits with a convergence failure\index{Convergence failure}. We recommend large numbers of evaluations and iterations (at least the defaults of 300 and 1000) unless you successfully reach convergence with less. You can also specify an alternative starting point of the minimiser using \texttt{\cas\ -i}.

We want to stress that this is a local optimisation algorithm trying to solve a global optimisation problem. What this means is that, even if you get a 'successful convergence'\index{Successful convergence} message, your solution may be only a local minimum\index{Local minimums}, not a global one. To diagnose this problem, try doing multiple runs from different starting points and comparing the results, or doing profiles of one or more key parameters and seeing if any of the profiled estimates finds a better optimum than than the original point estimate.

The approximate covariance matrix\index{Covariance matrix} of the estimated parameters can be calculated as the inverse of the minimiser's approximation to the \I{Hessian}, and the corresponding correlation matrix\index{Correlation matrix} is also calculated. Be aware that

\begin{itemize}
\item the Hessian approximation develops over many minimiser steps, so if the minimiser has only run for a small number of iterations the covariance matrix can be a very poor approximation
\item the inverse Hessian is not a good approximation to the covariance matrix of the estimated parameters, and may not be useful to construct, for example, confidence intervals. 
\end{itemize}

Also note that if an estimated parameter has equal lower and upper bounds, it will have entries of `0' in the covariance matrix and \texttt{NaN} or \texttt{-1.\#IND} (depending on the operating system) in the correlation matrix. 

\subsubsection{\I{The differential evolution minimiser}}
\CH
The differential evolution minimiser is a simple population based, stochastic function minimizer, but is claimed to be quite powerful in solving minimisation problems. It is a method of mathematical optimization of multidimensional functions and belongs to the class of evolution strategy optimizers. Initially, the procedure randomly generates and evaluates a number of solution vectors (the population size), each with $p$ parameters. Then, for each generation (iteration), the algorithm creates a candidate solution for each existing solution by random mutation and uniform crossover. The random mutation generates a new solution by multiplying the difference between two randomly selected solution vectors by some scale factor, then adding the result to a third vector. Then an element-wise crossover takes place with probability $P_{cr}$, to generate a potential candidate solution. If this is better than the initial solution vector, it replaces it, otherwise the original solution is retained. The algorithm is terminated after either a predefined number of generations (\argument{max\_generations}) or when the maximum difference between the scaled individual parameters from the candidate solutions from all populations is less than some predefined amount \argument{tolerance}.

The differential evolution minimiser can be good at finding global minimums in surfaces that may have local minima. However, the speed of the minimiser, and the ability to find a good minima depend on the number of initial `populations'. Some authors recommend that the number of populations be set at about $10*p$, where $p$ is the number of free parameters. However, depending on your problem, you may find that you may need more, or that less will suffice.

We note that there is no proof of convergence for the differential evolution solver, but several papers have found it to be an efficient method of solving multidimensional problems. Our (limited) experience suggests that it can often find a better minima and may be faster or longer (depending on the actual model specification) at finding a solution when compared with the numerical differences minimiser. Comparisons with auto-differentiation minimisers or other more sophisticated algorithms have not been made. 

\subsubsection{\I{The Betadiff minimiser}}
An auto-differentiable minimiser for non-linear models.


\subsection{\I{Posterior profiles}\label{sec:estimate-profiles}}

\CH
\KL This text from SPM. A rewrite of what's in CASAL, except for par. 5 and 7. OK to use? Additions required? \KLend


If profiles are requested \texttt{\cas\ -p}, \CAS\ will first calculate a point estimate. For each scalar parameter or, in the case of vectors or selectivities, the element of the parameter to be profiled, \CAS\ will fix its value at a sequence of $n$ evenly spaced numbers ($step$) between a specified lower and upper bounds $l$ and $u$, and calculate a point estimate at each value. 

By default $step=10$, and $(l, u)=($lower bound on parameter plus $(range/(2n))$, upper bound on parameter less $(range/(2n))$. Each minimisation starts at the final parameter values from the previous resulting value of the parameter being profiled. \CAS\ will report the objective function for each parameter value. Note that an initial point estimate should be compared with the profile, not least to check that none of the other points along the profile have a better objective function value than the initial `minimum'.

You specify which parameters are to be profiled, and optionally the number of steps, lower bound, and upper bound for each. In the case of vector parameters, you will also need to specify the element of the vector being profiled. 

You can also supply the initial starting point for the estimation using \texttt{\cas\ -i \emph{file}} --- this may improve the minimiser performance for the profiles.

If you get an implausible profile, it may be a result of not using enough iterations in the minimiser or a poor choice of minimiser control variables (e.g., the minimiser tolerance). It also may be useful to try both if the minimisers in \CAS\ and compare the results.

\subsection{\I{Bayesian estimation}\label{sec:estimate-MCMC}}
\CH
\KL This text is from SPM and is nearly verbatim S6.5 in CASASL, but two large sections exluded: ...request covariate matrix change adaptively... and from ...multivariate t dist... onwards. OK to use/ Additions required? \KLend

\CAS\ can use a \I{Monte Carlo Markov Chain (MCMC)}\index{MCMC} to generate a sample from the posterior distribution of the estimated parameters \texttt{\cas\ -m} and output the sampled values to a file (optionally keeping only every $n$th set of values).

As \CAS\ has no post-processing capabilities. \CAS\ cannot produce MCMC convergence diagnostics (use a package such as \href{http://www.public-health.uiowa.edu/boa}{BOA}) or plot/summarize the posterior distributions of the output quantities (for example, using a general-purpose statistical or spreadsheet package such as \href{http://www.insightful.com}{S-Plus}, \href{http://www.r-project.org}{\R}, or \href{http://www.microsoft.com}{Microsoft Excel}).

Bayesian methodology\index{Bayesian estimation} and MCMC are both large and complex topics, and we do not describe either properly here. See Gelman et al. \citeyearpar{823} and Gilks et al. \citeyearpar{143} for details of both Bayesian analysis and MCMC methods. In addition, see Punt \& Hilborn \citeyearpar{828} for an introduction to quantitative fish stock assessment using Bayesian methods. 

This section only briefly describes the MCMC algorithms used in \CAS. See Section \ref{sec:estimation-syntax-MCMC} for a better description of the sequence of \CAS\ commands used in a full Bayesian analysis.

\CAS\ uses a straightforward implementation of the Metropolis-Hastings algorithm \citep{823,143}. The Metropolis-Hastings algorithm attempts to draw a sample from a Bayesian posterior distribution, and calculates the posterior density $\pi$, scaled by an unknown constant. The algorithm generates a `chain' or sequence of values. Typically the beginning of the chain is discarded and every $N$th element of the remainder is taken as the posterior sample. The chain is produced by taking an initial point $x_0$ and repeatedly applying the following rule, where $x_i$ is the current point: 

\begin{itemize}
\item Draw a candidate step s from a proposal distribution J, which should be symmetric i.e., $J(-s)=J(s)$.
\item Calculate $r=min(\pi(x_i+s)/\pi(x_i),1)$. 
\item Let $x_i+1=x_i+s$ with probability $r$, or $x_i$ with probability $1-r$.
\end{itemize}

An initial point estimate is produced before the chain starts, which is done so as to calculate the approximate covariance matrix of the estimated parameters (as the inverse Hessian), and may also be used as the starting point of the chain. 

The user can specify the starting point of the point estimate minimiser using \texttt{\cas\ -i}. Don't start it too close to the actual estimate (either by using \texttt{\cas\ -i}, or by changing the initial parameter values in \config) as it takes a few iterations to form a reasonable approximation to the Hessian. 

There are two options for the starting point of the Markov Chain: 

\begin{itemize}
\item Start from the point estimate.
\item Start from a random point near the point estimate (the point is generated from a multivariate normal distribution, centred on the point estimate, with covariance equal to the inverse Hessian times a user-specified constant). This may be useful if the chain gets `stuck' at the point estimate, or if you wish to generate multiple chains from  for later MCMC diagnostic tests.
\item Start from a point specified by the user with \texttt{\cas\ -i} \COO
\end{itemize}

The chain moves in natural space, i.e., no transformations are applied to the estimated parameters. The default proposal distribution is a multivariate t centred on the current point, with covariance matrix equal to a matrix based on the approximate covariance produced by the minimiser, times some stepsize factor. The following steps define the initial covariance matrix of the proposal distribution: 

\begin{itemize}
\item The covariance matrix is taken as the inverse of the approximate Hessian from the quasi-Newton minimiser.
\item The covariance matrix is modified so as to decrease all correlations greater than \commandsub{mcmc}{max\_correlation} down to \commandsub{mcmc}{max\_correlation}, and similarly to increase all correlations less than  -\commandsub{mcmc}{max\_correlation} up to -\commandsub{mcmc}{max\_correlation} (the \commandsub{mcmc}{max\_correlation} parameter defaults to 0.8). This should help to avoid getting `stuck' in a lower-dimensional subspace.

\item The covariance matrix is then modified either by,

\begin{itemize}
\item if \commandsubarg{mcmc}{adjustment\_method}{covariance}: that if the variance of the $i$th parameter is non-zero and less than \commandsub{mcmc}{min\_difference} times the difference between the parameters' lower and upper bound, then the variance is changed, without changing the associated correlations, to $k=$min\_diff$(upper\_bound_i-lower\_bound_i)$. This is done by setting \[
{\mathop{\rm Cov}\nolimits} \left( {i,j} \right)^\prime   = {{{\mathop{\rm sqrt}\nolimits} \left( k \right){\mathop{\rm Cov}\nolimits} \left( {i,j} \right)} \mathord{\left/
{\vphantom {{{\mathop{\rm sqrt}\nolimits} \left( k \right){\mathop{\rm Cov}\nolimits} \left( {i,j} \right)} {{\mathop{\rm sd}\nolimits} \left( i \right)}}} \right.
\kern-\nulldelimiterspace} {{\mathop{\rm sd}\nolimits} \left( i \right)}}
\]
for $i \ne j$, and ${\mathop{\rm var}} \left( i \right)^\prime   = k$

\item if \commandsubarg{mcmc}{adjustment\_method}{correlation}: that if the variance of the $i$th parameter is non-zero and less than \commandsub{mcmc}{min\_difference} times the difference between the parameters' lower and upper bound, then its variance is changed to $k=min\_diff(upper\_bound_i-lower\_bound_i)$. This differs from (i) above in that the effect of this option is that it also modifies the resulting correlations between the $i$th parameter and all other parameters.
\end{itemize}

This allows each estimated parameter to move in the MCMC even if its variance is very small according to the inverse Hessian. In both cases, the \commandsub{mcmc}{min\_difference} parameter defaults to $0.0001$.

\item The \commandsub{mcmc}{stepsize} (a scalar factor applied to the covariance matrix to improve the acceptance probability) is chosen by the user. The default is $2.4d^{-0.5}$ where $d$ is the number of estimated parameters, as recommended by Gelman et al. \citep{823}. However, you may find that a smaller value may often be better. 
\end{itemize}

The proposal distribution can also change adaptively during the chain, using two different mechanisms. Both are offered as means of improving the convergence properties of the chain. It is important to note that any adaptive behaviour must finish before the end of the burn-in period, i.e., the proposal distribution must be finalised before the kept portion of the chain starts. The adaptive mechanisms are as follows: 

\begin{enumerate}
\item You can request that the stepsize change adaptively at one or more sample numbers. At each adaptation, the stepsize is doubled if the acceptance rate since the last adaptation is more than $0.5$, or halved if the acceptance rate is less than $0.2$. (See Gelman et al. \citep{823} for justification.)

\item You can request that the entire covariance matrix change adaptively at one or more sample numbers. At each adaptation, it is replaced with a matrix based on the sample covariance of an earlier section of the chain. The theory here is that the covariance of a portion of chain could potentially be a better estimate of the covariance of the posterior distribution than the inverse Hessian. \COO
\end{enumerate}

The procedure used to choose the sample of points is as follows. First, all points on the chain so far are taken. All points in an initial user-specified period are discarded. The assumption is that the chain will have started moving during this period - if this is incorrect and the chain has still not moved by the end of this period, it is a fatal error and \CAS\ stops. The remaining set of points must contain at least some user-specified number of transitions - if this is incorrect and the chain has not moved this often, it is again a fatal error. If this test is passed, the set of points is systematically sub-sampled down to 1000 points (it must be at least this long to start with). \COO

The variance-covariance matrix of this sub-sample of chain is calculated. As above, correlations greater than \commandsub{mcmc}{max\_correlation} are reduced to \commandsub{mcmc}{max\_correlation}, correlations less than \commandsub{mcmc}{max\_correlation} are increased to  \commandsub{mcmc}{max\_correlation}, and very small non-zero variances are increased (\commandsub{mcmc}{covariance\_adjustment} and \commandsub{mcmc}{min\_difference}. The result is the new variance-covariance matrix of the proposal distribution. \COO

The stepsize parameter is now on a completely different scale, and must be reset. It is set to a user-specified value (which may or may not be the same as the initial stepsize). We recommend that some of the stepsize adaptations are set to occur after this, so that the stepsize can be readjusted to an appropriate value which gives good acceptance probabilities with the new matrix. \COO

All modified versions of the covariance matrix are printed to the standard output, but only the initial covariance matrix (inverse Hessian) is saved to the objectives file. The number of covariance modifications by each iteration is recorded as a column on the objectives file. \COO 

The probability of acceptance for each jump is $0$ if it would move out of the bounds, or $1$ if it improves the posterior, or (new posterior/old posterior) otherwise. You can specify how often the position of the chain is recorded using the keep parameter. For example, with keep $10$, only every $10$th sample is recorded. 

You have the option to specify that some of the estimated parameters are fixed during the MCMC. If the chain starts at the point estimate or at a random location, these fixed parameters are set to their values at the point estimate.

If you specify the start of the chain using \texttt{\cas\ -i}, these fixed parameters are set to the values in the file. \COO

The posterior sample can be used for (projections (Section \ref{sec:projections})\COO) or 
simulations (Section \ref{sec:simulation-observations}) with the values supplied using \texttt{\cas\ -i \emph{file}}.

\CA
\begin{lstlisting}
A multivariate t distribution is available as an alternative to the multivariate normal proposal distribution. If you request multivariate t proposals, you may want to change the degrees of freedom from the default of 4. As the degrees of freedom decrease, the t distribution becomes more heavy tailed. This may lead to better convergence properties.

Having produced one or more Markov chains and looked at the diagnostics, reload all the chain output files into CASAL and use them to generate a single posterior sample (using -C). At this stage, the first burn_in iterations for each chain are discarded (so, with keep 10, burn_in 1000, the first 1000 recorded samples are discarded for each chain). Unless a very large value of keep was originally chosen, it will be necessary to further reduce the size of the posterior sample (possibly down to several hundred) such that it can be analysed in a reasonable amount of time. This is done by sub-sampling. You specify the size of the sub-sample to be produced (or else no sub-sampling is done). You have the option to generate a systematic sub-sample (i.e., every nth point is kept) or a random sub-sample (the former is recommended except with prior re-weighting, when the latter must be used).

Given a posterior (sub)sample, CASAL can calculate a list of output quantities for each sample point (see Section 7.2). These quantities can be dumped into a file (using casal -v) and read into an external software package where the posterior distributions can be plotted and/or summarised. 

The posterior sample can also be used for projections (Section 7.3) and stochastic yield calculations (Section 7.5). The advantage of this is that the parameter uncertainty, as expressed in your posterior distribution, can be included into the risk and yield estimates.

It is possible to investigate the results that would have been obtained if a different prior had been specified This is called prior re-weighting and is done by calculating the ratio of the new prior to the original prior for each point in the posterior sample, then using these ratios as probability weights when generating a random (not systematic) sub-sample with casal -C. Prior re-weighting is applicable only if the new prior is zero in every part of the parameter space for which the original prior was zero. Also, it is likely to be numerically unstable unless the new prior is very small in every part of the parameter space for which the original prior was very small.
\end{lstlisting}



\subsection{Priors\label{sec:priors}}

In a Bayesian analysis, you need to give a prior for every parameter that is being estimated. There are no default priors.  

Note that when some of these priors are parameterised in terms of mean, c.v., and standard deviation, these refer to the parameters of the distribution before bounds are applied. The moments of the prior after the bounds are applied may differ.

\CAS\ has the following priors (expressed in terms of their contribution to the objective function): 

\begin{enumerate}
\item{Uniform\index{Uniform prior}\index{Priors ! Uniform}}

\begin{equation}
 - \log \left(\pi \left(p \right) \right) = 0
\end{equation}

\item{Uniform-log\index{Uniform-log prior}\index{Priors ! Uniform-log}} (i.e., $\log(p) \sim \text{uniform}$)

\begin{equation}
 - \log \left(\pi \left(p \right) \right) = \log \left( p \right)
\end{equation}

\item{Normal\index{Normal prior}\index{Priors ! Normal} with mean $\mu$ and c.v. $c$}

\begin{equation}
 - \log \left(\pi \left(p \right) \right) = 0.5\left(\frac{p - \mu}{c\mu} \right)^2 
\end{equation}

\item{Normal with mean $\mu$ and standard deviation $\sigma$}

\begin{equation}
 - \log \left(\pi \left(p \right) \right) = 0.5\left(\frac{p - \mu}
{\sigma }\right)^2
\end{equation}

\item{Lognormal\index{Lognormal prior}\index{Priors ! Lognormal} with mean $\mu$ and c.v. $c$} 

\begin{equation}
 - \log \left(\pi \left(p \right) \right) = \log \left( p \right) + 0.5 \left( \frac{\log \left( p / \mu \right)}{s} + \frac{s}{2} \right)^2
\end{equation}

where $s$ is the standard deviation of $\log(p)$ and $s= \sqrt{\log \left(1+c^2 \right)}$.

\CA
\begin{lstlisting}
6.	Normal-log with log(p) having mean m and standard deviation s, 
\end{lstlisting}

\item{Beta\index{Beta prior}\index{Priors ! Beta} with mean $\mu$ and standard deviation $\sigma$, and range parameters $A$ and $B$}

\begin{equation}
 - \log \left(\pi \left( p \right) \right) = \left( 1 - m \right) \log \left( p - A \right) + \left( 1 - n \right)\log \left( B - p \right)
\end{equation}

where $\nu  = \frac{\mu  - A}{B - A}$, and $\tau = \frac{\left(\mu -A \right)\left(B - \mu \right)}{\sigma ^2} - 1$ and then $\mu=\tau \nu$ and $n=\tau(1-\nu)$. Note that the beta prior is undefined when $\tau \leq 0$.

\end{enumerate}

\CA
\begin{lstlisting}
Vectors of parameters can be independently (but not necessarily identically) distributed according to any of the above forms, in which case the joint negative-log-prior for the vector is the sum of the negative-log-priors of the components. Values of each parameter need to be specified for each element of the vector. 

In addition, for a vector p of n identically distributed parameters (for example, YCS) the following priors are allowed: 

1.	Multivariate normal from a stationary AR(1) process with parameters 
.
.
.

2.	Multivariate normal-log, where log(p) forms a stationary AR(1) process as per 1. above, with parameters
.
.
.
3.	Multivariate normal-log with mean 1, where E(pi)=1 and log(p) forms a stationary AR(1) process as for the multivariate normal above, with parameters 
.
.
.
\end{lstlisting}





\subsection{Penalties\label{sec:penalties}}
\CH
\KL This text from SPM, currently iSAM coded as described here.  CASAL documentation (S6.7.6) includes 11 penalties. Will this be included? \KLend

Penalties can be used to encourage or discourage parameter values or model outputs that are unlikely to be sensible, by adding a penalty to the objective function. For example, parameter estimates that do not allow a known mortality event to remove enough individuals from the population can be discouraged with an event mortality penalty. \CAS\ requires penalty functions for processes that move or shift a \emph{number} of individuals between categories or from the partition.

For most penalties, you need to specify a multiplier, and the objective function is increased by this multiplier times the penalty value as described below. In some cases you will need to make the multiplier quite large to prohibit some model behaviour. 

Currently, the penalties for the processes \commandlabsubarg{process}{type}{event\_mortality} and \commandlabsubarg{process}{type}{category\_transition} are the only penalties implemented. 

For both of these processes, two types of penalty can be defined, natural scale (the default) and log scale. Both of these types add a penalty value of the squared difference between the observed value (i.e., the actual number of individuals to be removed in an event mortality process or the actual number of individuals to shift in a category transition process), and the number that were moved (if less than or equal), times the penalty multiplier.

The natural scale penalty just uses at the squared difference on a natural scale, while the log scale penalty uses the squared difference of the logged values. 
