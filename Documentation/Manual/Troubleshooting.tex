\section{Troubleshooting\label{sec:trouble-shooting}}

\subsection{Introduction}

\subsection{Reporting errors\label{sec:reporting-errors}}

When reporting a bug or problem to the \CNAME\ development team at \email, please address the following points.

\subsection{Guidelines for reporting a problem with \CNAME\label{sec:error-guidelines}}

\begin{enumerate}
\item Detail the version of \CNAME\ are you using? e.g., \CNAME\ \VER Microsoft Windows executable''

\item What operating system or environment are you using? e.g., ``IBM-PC Intel CPU running Microsoft Windows 10 Enterprise, Service Pack 1''.

\item Give a brief one-line description of the problem, e.g., ``a segmentation fault was reported''.

\item If the problem is reproducible, please list the exact steps required to cause it, remembering to include the relevant \CNAME\ configuration file, other input files, and any out generated. Specify the \emph{exact} command line arguments that were used, e.g., ``Using the command \texttt{***.-* -*} reports a segmentation fault. The \config s are attached.''

\item If the problem is not reproducible (only happened once, or occasionally for no apparent reason), please describe the circumstances in which it occurred and the symptoms observed (but note it is much harder to reproduce and hence fix non-reproducible bugs, but if several reports are made over time that relate to the same thing, then this may help to track down the problem), e.g., ``\CNAME\ crashed, but I cannot reproduce how I did it. It seemed to be related to a local network crash but I cannot be sure.''

\item If the problem causes any error messages to appear, please give the \emph{exact} text displayed, e.g., \texttt{segmentation fault (core dumped)}.

\item Remember to attach all relevant input and output files so that the problem can be reproduced (it can helpful to compress these into a single file). Without these, it is usually not possible to determine the cause of the problem, and we are unlikely to provide any assistance. Note that it is helpful to be as specific as possible when describing the problem.

\end{enumerate}
