\section{Population command and subcommand syntax\label{sec:population-syntax}}

\subsection{\I{Model structure}}
\defComLab{model}{Define an object of type \emph{Model}}.
\defRef{sec:Model}
\label{syntax:Model}

\defSub{type}{Type of model (only type=age is currently implemented)}
\defType{String}
\defDefault{age}

\defSub{base\_weight\_units}{Define the units for the base weight measurement unit (grams, kilograms (kgs), or tonnes). This will be the default unit of any weight input values}
\defType{String}
\defDefault{tonnes}

\defSub{threads}{The number of threads to use for this model}
\defType{Non-negative integer}
\defDefault{1}
\defLowerBound{1 (inclusive)}

\subsubsection{Model of type Age}
\commandlabsubarg{model}{type}{Age}.
\label{syntax:Model-Age}

\defSub{start\_year}{Define the first year of the model, immediately following initialisation}
\defType{Non-negative integer}
\defDefault{No default}
\defValue{Defines the first year of the model, must be $\ge 1000$, e.g. 1990}

\defSub{final\_year}{Define the final year of the model, excluding years in the projection period}
\defType{Non-negative integer}
\defDefault{No default}
\defValue{Defines the last year of the model, i.e., the model is run from start\_year to final\_year}

\defSub{min\_age}{Minimum age of individuals in the population}
\defType{Non-negative integer}
\defDefault{0}
\defValue{$0 \le$ age\textlow{min} $\le$ age\textlow{max}}

\defSub{max\_age}{Maximum age of individuals in the population}
\defType{Non-negative integer}
\defDefault{0}
\defValue{$0 \le$ age\textlow{min} $\le$ age\textlow{max}}

\defSub{age\_plus}{Define the oldest age or extra length midpoint (plus group size) as a plus group}
\defType{Boolean}
\defDefault{true}
\defValue{true, false}

\defSub{initialisation\_phases}{Define the labels of the phases of the initialisation}
\defType{Vector of strings}
\defDefault{true}
\defValue{A list of valid labels defined by \texttt{@initialisation\_phase}}

\defSub{time\_steps}{Define the labels of the time steps, in the order that they are applied, to form the annual cycle}
\defType{Vector of strings}
\defDefault{No default}
\defValue{A list of valid labels defined by \texttt{@time\_step}}

\defSub{projection\_final\_year}{Define the final year of the model when running projections}
\defType{Non-negative integer}
\defDefault{0}
\defValue{A value greater than final\_year}

\defSub{length\_bins}{The minimum length in each length bin}
\defType{Vector of real numbers (estimable)}
\defDefault{true}
\defValue{$0 \le$ length\textlow{min} $\le$ length\textlow{max}}

\defSub{length\_plus}{Specify whether there is a length plus group or not}
\defType{Boolean}
\defDefault{true}
\defValue{true, false}

\defSub{length\_plus\_group}{Mean length of length plus group}
\defType{Real number (estimable)}
\defDefault{0}
\defValue{length\textlow{max} $<$ length\_plus\_group}

 

\subsection{\I{Initialisation}}
\defComLab{initialisation\_phase}{Define an object of type \emph{Initialisation\_Phase}}.
\defRef{sec:Initialisation}
\label{syntax:InitialisationPhase}

\defSub{label}{The label of the initialisation phase}
\defType{String}
\defDefault{No default}

\defSub{type}{The type of initialisation}
\defType{String}
\defDefault{iterative}

\subsubsection{Initialisation Phase of type Cinitial}
\commandlabsubarg{initialisation\_phase}{type}{Cinitial}.
\defRef{sec:InitialisationPhase-Cinitial}
\label{syntax:InitialisationPhase-Cinitial}

\defSub{categories}{The list of categories for the Cinitial initialisation}
\defType{Vector of strings}
\defDefault{No default}

\defSub{table}{The table of data specifying the initial values by age}
\defType{Data table with label = n}
\defDefault{No default}
\defValue{A $n x m$ matrix, where $n=$ the categories and $m=$ the number of ages defined in the model. Rows of values specify the initial value of the partition. The table ends with `end\_table'}
\defNote{See \ref{sec:DataTable} for more details on specifying data tables}

\subsubsection{Initialisation Phase of type Derived}
\commandlabsubarg{initialisation\_phase}{type}{Derived}.
\defRef{sec:InitialisationPhase-Derived}
\label{syntax:InitialisationPhase-Derived}

\defSub{insert\_processes}{Specifies the additional processes that are not in the annual cycle, but should be be inserted into this initialisation phase}
\defType{Vector of strings}
\defDefault{true}

\defSub{exclude\_processes}{Specifies the processes in the annual cycle that should be excluded from this initialisation phase}
\defType{Vector of strings}
\defDefault{true}

\defSub{casal\_initialisation\_switch}{Run an extra annual cycle to evaluate equilibrium SSBs. Warning: if true, this may not correctly evaluate the equilibrium state. Set to true if replicating a CASAL model}
\defType{Boolean}
\defDefault{false}

\subsubsection{Initialisation Phase of type Iterative}
\commandlabsubarg{initialisation\_phase}{type}{Iterative}.
\defRef{sec:InitialisationPhase-Iterative}
\label{syntax:InitialisationPhase-Iterative}

\defSub{years}{The number of iterations (years) over which to execute this initialisation phase}
\defType{Non-negative integer}
\defDefault{No default}

\defSub{insert\_processes}{The processes in the annual cycle to be include in this initialisation phase}
\defType{Vector of strings}
\defDefault{true}

\defSub{exclude\_processes}{The processes in the annual cycle to be excluded from this initialisation phase}
\defType{Vector of strings}
\defDefault{true}

\defSub{convergence\_years}{The iteration (year) when the test for convergence ($\lambda$) is evaluated}
\defType{Vector of non-negative integers}
\defDefault{true}

\defSub{lambda}{The maximum value of the absolute sum of differences (lambda) between the partition at year-1 and year that indicates successful convergence}
\defType{Real number}
\defDefault{0.0}

\subsubsection{Initialisation Phase of type State\_Category\_By\_Age}
\commandlabsubarg{initialisation\_phase}{type}{State\_Category\_By\_Age}.
\defRef{sec:InitialisationPhase-StateCategoryByAge}
\label{syntax:InitialisationPhase-StateCategoryByAge}

\defSub{categories}{The list of categories for the category state initialisation}
\defType{Vector of strings}
\defDefault{No default}

\defSub{min\_age}{The minimum age of values supplied in the definition of the category state}
\defType{Non-negative integer}
\defDefault{No default}

\defSub{max\_age}{The maximum age of values supplied in the definition of the category state}
\defType{Non-negative integer}
\defDefault{No default}

\defSub{table}{The table of data specifying the initial values by age}
\defType{Data table with label = n}
\defDefault{No default}
\defValue{A $n x m$ matrix, where $n=$ the categories and $m=$ the number of ages defined in the model. Rows of values specify the initial value of the partition. The table ends with `end\_table'}
\defNote{See \ref{sec:DataTable} for more details on specifying data tables}

 

\subsection{\I{Categories}}
\defComLab{categories}{Define an object of type \emph{Categories}}.
\defRef{sec:Categories}
\label{syntax:Categories}

\defSub{format}{The format that the category names use}
\defType{String}
\defDefault{No default}

\defSub{names}{The names of the categories}
\defType{Vector of strings}
\defDefault{No default}
\defNote{Category names must start with a letter or underscore}

\defSub{age\_lengths}{The age-length relationship labels for each category}
\defType{Vector of strings}
\defDefault{true}
\defValue{Valid labels of age-weight relationships}

\defSub{age\_weight}{The age-weight relationships labels for each category}
\defType{Vector of strings}
\defDefault{true}
\defValue{Valid labels of age-weight relationships}

 

\subsection{\I{Time-steps}}
\defComLab{Time\_Step}{Define an object of type \emph{Time\_Step}}.
\defRef{sec:TimeStep}
\label{syntax:TimeStep}

\defSub{label}{The label of the time step}
\defType{String}
\defDefault{No default}

\defSub{processes}{The labels of the processes that occur in this time step, in the order that they occur}
\defType{Vector of strings}
\defDefault{No default}

 

\subsection{\I{Processes}}
\defComLab{Process}{Define an object of type \emph{Process}}.
\defRef{sec:Process}
\label{syntax:Process}

\defSub{label}{The label of the process}
\defType{String}
\defDefault{No default}

\defSub{type}{The type of process}
\defType{String}
\defDefault{No default}

\subsubsection{Process of type Ageing}
\commandlabsubarg{Process}{type}{Ageing}.
\defRef{sec:Process-Ageing}
\label{syntax:Process-Ageing}

\defSub{categories}{The labels of the categories to age}
\defType{Vector of strings}
\defDefault{No default}

\subsubsection{Process of type Load\_Partition}
\commandlabsubarg{Process}{type}{Load\_Partition}.
\defRef{sec:Process-LoadPartition}
\label{syntax:Process-LoadPartition}

The Load\_Partition type has no additional subcommands.

\subsubsection{Process of type Maturation}
\commandlabsubarg{Process}{type}{Maturation}.
\defRef{sec:Process-Maturation}
\label{syntax:Process-Maturation}

\defSub{from}{The list of categories to mature from}
\defType{Vector of strings}
\defDefault{No default}

\defSub{to}{The list of categories to mature to}
\defType{Vector of strings}
\defDefault{No default}

\defSub{selectivities}{The list of selectivities to use for maturation}
\defType{Vector of strings}
\defDefault{No default}

\defSub{years}{The years to be associated with the maturity rates}
\defType{Vector of non-negative integers}
\defDefault{No default}

\defSub{rates}{The rates to mature for each year}
\defType{Vector of real numbers (estimable)}
\defDefault{No default}

\subsubsection{Process of type Mortality\_Constant\_Rate}
\commandlabsubarg{Process}{type}{Mortality\_Constant\_Rate}.
\defRef{sec:Process-MortalityConstantRate}
\label{syntax:Process-MortalityConstantRate}

\defSub{categories}{The list of category labels}
\defType{Vector of strings}
\defDefault{No default}

\defSub{m}{The mortality rates}
\defType{Real number (estimable)}
\defDefault{No default}
\defLowerBound{0.0 (inclusive)}

\defSub{time\_step\_proportions}{The time step proportions for the mortality rates}
\defType{Vector of real numbers}
\defDefault{false}
\defLowerBound{0.0 (inclusive)}
\defUpperBound{1.0 (inclusive)}
\defValue{The time step proportions must sum to one. If only one value is supplied, then the each time step is allocated an equal proportion. Otherwise the number of values must equal the number of time steps}

\defSub{relative\_m\_by\_age}{The list of mortality by age ogive labels for the categories}
\defType{Vector of strings}
\defDefault{No default}

\subsubsection{Process of type Mortality\_Event}
\commandlabsubarg{Process}{type}{Mortality\_Event}.
\defRef{sec:Process-MortalityEvent}
\label{syntax:Process-MortalityEvent}

\defSub{categories}{The categories}
\defType{Vector of strings}
\defDefault{No default}

\defSub{years}{The years in which to apply the mortality process}
\defType{Vector of non-negative integers}
\defDefault{No default}

\defSub{catches}{The number of removals (catches) to apply for each year}
\defType{Vector of real numbers (estimable)}
\defDefault{No default}

\defSub{u\_max}{The maximum exploitation rate ($U\_{max}$)}
\defType{Real number}
\defDefault{0.99}
\defLowerBound{0.0 (inclusive)}
\defUpperBound{1.0 (inclusive)}

\defSub{selectivities}{The list of selectivities}
\defType{Vector of strings}
\defDefault{No default}

\defSub{penalty}{The label of the penalty to apply if the total number of removals cannot be taken}
\defType{String}
\defDefault{No default}

\subsubsection{Process of type Mortality\_Event\_Biomass}
\commandlabsubarg{Process}{type}{Mortality\_Event\_Biomass}.
\defRef{sec:Process-MortalityEventBiomass}
\label{syntax:Process-MortalityEventBiomass}

\defSub{categories}{The category labels}
\defType{Vector of strings}
\defDefault{No default}

\defSub{selectivities}{The labels of the selectivities for each of the categories}
\defType{Vector of strings}
\defDefault{No default}

\defSub{years}{The years in which to apply the mortality process}
\defType{Vector of non-negative integers}
\defDefault{No default}

\defSub{catches}{The biomass of removals (catches) to apply for each year}
\defType{Vector of real numbers (estimable)}
\defDefault{No default}

\defSub{u\_max}{The maximum exploitation rate ($U\_{max}$)}
\defType{Real number (estimable)}
\defDefault{0.99}
\defLowerBound{0.0 (inclusive)}
\defUpperBound{1.0 (inclusive)}

\defSub{penalty}{The label of the penalty to apply if the total biomass of removals cannot be taken}
\defType{String}
\defDefault{No default}

\subsubsection{Process of type Mortality\_Holling\_Rate}
\commandlabsubarg{Process}{type}{Mortality\_Holling\_Rate}.
\defRef{sec:Process-MortalityHollingRate}
\label{syntax:Process-MortalityHollingRate}

\defSub{prey\_categories}{The prey categories labels}
\defType{Vector of strings}
\defDefault{No default}

\defSub{predator\_categories}{The predator categories labels}
\defType{Vector of strings}
\defDefault{No default}

\defSub{is\_abundance}{Is vulnerable amount of prey and predator an abundance [true] or biomass [false]}
\defType{Boolean}
\defDefault{true}

\defSub{a}{Parameter a}
\defType{Real number (estimable)}
\defDefault{No default}
\defLowerBound{0.0 (inclusive)}

\defSub{b}{Parameter b}
\defType{Real number (estimable)}
\defDefault{No default}
\defLowerBound{0.0 (inclusive)}

\defSub{x}{This parameter controls the functional form: Holling function type 2 (x=2) or 3 (x=3), or generalised (Michaelis Menten, x>=1)}
\defType{Real number (estimable)}
\defDefault{No default}
\defLowerBound{1.0 (inclusive)}

\defSub{u\_max}{The maximum exploitation rate ($U\_{max}$)}
\defType{Real number}
\defDefault{0.99}
\defLowerBound{0.0 (inclusive)}
\defUpperBound{1.0 (exclusive)}

\defSub{prey\_selectivities}{The selectivities for prey categories}
\defType{Vector of strings}
\defDefault{true}

\defSub{predator\_selectivities}{The selectivities for predator categories}
\defType{Vector of strings}
\defDefault{true}

\defSub{penalty}{The label of penalty}
\defType{String}
\defDefault{No default}

\defSub{years}{The years in which to apply the mortality process}
\defType{Vector of non-negative integers}
\defDefault{No default}

\subsubsection{Process of type Mortality\_Initialisation\_Event}
\commandlabsubarg{Process}{type}{Mortality\_Initialisation\_Event}.
\defRef{sec:Process-MortalityInitialisationEvent}
\label{syntax:Process-MortalityInitialisationEvent}

\defSub{categories}{The categories}
\defType{Vector of strings}
\defDefault{No default}

\defSub{catch}{The number of removals (catches) to apply for each year}
\defType{Real number (estimable)}
\defDefault{No default}

\defSub{u\_max}{The maximum exploitation rate ($U\_{max}$)}
\defType{Real number}
\defDefault{0.99}
\defLowerBound{0.0 (inclusive)}
\defUpperBound{1.0 (exclusive)}

\defSub{selectivities}{The list of selectivities}
\defType{Vector of strings}
\defDefault{No default}

\defSub{penalty}{The label of the penalty to apply if the total number of removals cannot be taken}
\defType{String}
\defDefault{No default}

\subsubsection{Process of type Mortality\_Initialisation\_Event\_Biomass}
\commandlabsubarg{Process}{type}{Mortality\_Initialisation\_Event\_Biomass}.
\defRef{sec:Process-MortalityInitialisationEventBiomass}
\label{syntax:Process-MortalityInitialisationEventBiomass}

\defSub{categories}{The categories}
\defType{Vector of strings}
\defDefault{No default}

\defSub{catch}{The number of removals (catches) to apply for each year}
\defType{Real number (estimable)}
\defDefault{No default}

\defSub{u\_max}{The maximum exploitation rate ($U\_{max}$)}
\defType{Real number}
\defDefault{0.99}
\defLowerBound{0.0 (inclusive)}
\defUpperBound{1.0 (inclusive)}

\defSub{selectivities}{The list of selectivities}
\defType{Vector of strings}
\defDefault{No default}

\defSub{penalty}{The label of the penalty to apply if the total number of removals cannot be taken}
\defType{String}
\defDefault{No default}

\subsubsection{Process of type Mortality\_Instantaneous}
\commandlabsubarg{Process}{type}{Mortality\_Instantaneous}.
\defRef{sec:Process-MortalityInstantaneous}
\label{syntax:Process-MortalityInstantaneous}

\defSub{categories}{The categories for instantaneous mortality}
\defType{Vector of strings}
\defDefault{No default}

\defSub{m}{The natural mortality rates for each category}
\defType{Real number (estimable)}
\defDefault{No default}
\defLowerBound{0.0 (inclusive)}

\defSub{time\_step\_proportions}{The time step proportions for natural mortality}
\defType{Vector of real numbers}
\defDefault{true}
\defLowerBound{0.0 (inclusive)}
\defUpperBound{1.0 (inclusive)}
\defValue{Proportions must sum to one}

\defSub{relative\_m\_by\_age}{The M-by-age selectivities to apply to each of the categories for natural mortality}
\defType{Vector of strings}
\defDefault{No default}

\subsubsection{Process of type Mortality\_Instantaneous\_Retained}
\commandlabsubarg{Process}{type}{Mortality\_Instantaneous\_Retained}.
\defRef{sec:Process-MortalityInstantaneousRetained}
\label{syntax:Process-MortalityInstantaneousRetained}

\defSub{categories}{The categories for instantaneous mortality}
\defType{Vector of strings}
\defDefault{No default}

\defSub{m}{The natural mortality rates for each category}
\defType{Real number (estimable)}
\defDefault{No default}
\defLowerBound{0.0 (inclusive)}

\defSub{time\_step\_proportions}{The time step proportions for natural mortality}
\defType{Vector of real numbers}
\defDefault{No default}
\defLowerBound{0.0 (inclusive)}
\defUpperBound{1.0 (inclusive)}
\defValue{Proportions must sum to one}

\defSub{relative\_m\_by\_age}{The M-by-age selectivities to apply on the categories for natural mortality}
\defType{Vector of strings}
\defDefault{No default}

\subsubsection{Process of type Mortality\_Prey\_Suitability}
\commandlabsubarg{Process}{type}{Mortality\_Prey\_Suitability}.
\defRef{sec:Process-MortalityPreySuitability}
\label{syntax:Process-MortalityPreySuitability}

\defSub{prey\_categories}{The prey categories labels}
\defType{Vector of strings}
\defDefault{No default}

\defSub{predator\_categories}{The predator categories labels}
\defType{Vector of strings}
\defDefault{No default}

\defSub{consumption\_rate}{The predator consumption rate}
\defType{Real number (estimable)}
\defDefault{No default}
\defLowerBound{0.0 (inclusive)}
\defUpperBound{1.0 (inclusive)}

\defSub{electivities}{The prey electivities}
\defType{Vector of real numbers (estimable)}
\defDefault{No default}
\defLowerBound{0.0 (inclusive)}
\defUpperBound{1.0 (inclusive)}

\defSub{u\_max}{The maximum exploitation rate ($U\_{max}$)}
\defType{Real number}
\defDefault{0.99}
\defLowerBound{0.0 (inclusive)}
\defUpperBound{1.0 (exclusive)}

\defSub{prey\_selectivities}{The selectivities for prey categories}
\defType{Vector of strings}
\defDefault{No default}

\defSub{predator\_selectivities}{The selectivities for predator categories}
\defType{Vector of strings}
\defDefault{No default}

\defSub{penalty}{The label of the penalty}
\defType{String}
\defDefault{No default}

\defSub{years}{The year that process occurs}
\defType{Vector of non-negative integers}
\defDefault{No default}

\subsubsection{Process of type null process}
\commandlabsubarg{Process}{type}{null\_preocess}.
\defRef{sec:Process-Null}
\label{syntax:Process-Null}

The null\_process  type has no additional subcommands. Note that this process does nothing. It is included primarily as a means of replacing other processes with "no action" to allow for testing of alternative scenarios]

\subsubsection{Process of type Recruitment\_Beverton\_Holt}
\commandlabsubarg{Process}{type}{Recruitment\_Beverton\_Holt}.
\defRef{sec:Process-RecruitmentBevertonHolt}
\label{syntax:Process-RecruitmentBevertonHolt}

\defSub{categories}{The category labels}
\defType{Vector of strings}
\defDefault{No default}

\defSub{r0}{R0, the mean recruitment used to scale annual recruits or initialise the model}
\defType{Real number (estimable)}
\defDefault{No default}
\defLowerBound{0.0 (inclusive)}
\defValue{Use either R0 or B0, but not both}

\defSub{b0}{B0, the SSB corresponding to R0, and used to scale annual recruits or initialise the model}
\defType{Real number (estimable)}
\defDefault{No default}
\defLowerBound{0.0 (inclusive)}
\defValue{Use either R0 or B0, but not both}

\defSub{proportions}{The proportion for each category}
\defType{Real number (estimable)}
\defDefault{No default}

\defSub{age}{The age at recruitment}
\defType{Non-negative integer}
\defDefault{No default}

\defSub{ssb\_offset}{The spawning biomass year offset}
\defType{Non-negative integer}
\defDefault{No default}

\defSub{steepness}{Steepness (h)}
\defType{Real number (estimable)}
\defDefault{1.0}
\defLowerBound{0.2 (inclusive)}
\defUpperBound{1.0 (inclusive)}

\defSub{ssb}{The SSB label (i.e., the derived quantity label)}
\defType{String}
\defDefault{No default}

\defSub{b0\_initialisation\_phase}{The initialisation phase label that B0 is from}
\defType{String}
\defDefault{No default}

\defSub{ycs\_values}{The YCS values}
\defType{Vector of real numbers (estimable)}
\defDefault{No default}

\defSub{ycs\_years}{The recruitment years. A vector of years that relates to the year of the spawning event that created this cohort}
\defType{Vector of non-negative integers}
\defDefault{false}

\defSub{standardise\_ycs\_years}{The years that are included for year class standardisation}
\defType{Vector of non-negative integers}
\defDefault{true}

\subsubsection{Process of type Recruitment\_Beverton\_Holt\_With\_Deviations}
\commandlabsubarg{Process}{type}{Recruitment\_Beverton\_Holt\_With\_Deviations}.
\defRef{sec:Process-RecruitmentBevertonHoltWithDeviations}
\label{syntax:Process-RecruitmentBevertonHoltWithDeviations}

\defSub{categories}{The category labels}
\defType{Vector of strings}
\defDefault{No default}

\defSub{r0}{R0, the mean recruitment used to scale annual recruits or initialise the model}
\defType{Real number (estimable)}
\defDefault{No default}
\defLowerBound{0.0 (inclusive)}
\defValue{Use either R0 or B0, but not both}

\defSub{b0}{B0, the SSB corresponding to R0, and used to scale annual recruits or initialise the model}
\defType{Real number (estimable)}
\defDefault{No default}
\defLowerBound{0.0 (inclusive)}
\defValue{Use either R0 or B0, but not both}

\defSub{proportions}{The proportion for each category}
\defType{Real number (estimable)}
\defDefault{No default}

\defSub{age}{The age at recruitment}
\defType{Non-negative integer}
\defDefault{true}

\defSub{ssb\_offset}{The spawning biomass year offset}
\defType{Non-negative integer}
\defDefault{No default}

\defSub{steepness}{Steepness (h)}
\defType{Real number (estimable)}
\defDefault{1.0}
\defLowerBound{0.2 (inclusive)}
\defUpperBound{1.0 (inclusive)}

\defSub{ssb}{The SSB label (i.e., the derived quantity label)}
\defType{String}
\defDefault{No default}
\defValue(A valid derived quantity)

\defSub{sigma\_r}{The standard deviation of recruitment, $\sigma_R$}
\defType{Real number (estimable)}
\defDefault{No default}
\defLowerBound{0.0 (inclusive)}

\defSub{b\_max}{The maximum bias adjustment}
\defType{Real number (estimable)}
\defDefault{0.85}
\defLowerBound{0.0 (inclusive)}
\defUpperBound{1.0 (inclusive)}

\defSub{last\_year\_with\_no\_bias}{The last year with no bias adjustment}
\defType{Non-negative integer}
\defDefault{false}

\defSub{first\_year\_with\_bias}{The first year with full bias adjustment}
\defType{Non-negative integer}
\defDefault{false}

\defSub{last\_year\_with\_bias}{The last year with full bias adjustment}
\defType{Non-negative integer}
\defDefault{false}

\defSub{first\_recent\_year\_with\_no\_bias}{The first recent year with no bias adjustment}
\defType{Non-negative integer}
\defDefault{false}

\defSub{b0\_initialisation\_phase}{The initialisation phase label that B0 is from}
\defType{String}
\defDefault{No default}

\defSub{deviation\_values}{The recruitment deviation values}
\defType{Vector of real numbers (estimable)}
\defDefault{No default}

\defSub{deviation\_years}{The recruitment years. A vector of years that relates to the year of the spawning event that created this cohort}
\defType{Vector of non-negative integers}
\defDefault{false}

\subsubsection{Process of type Recruitment\_Constant}
\commandlabsubarg{Process}{type}{Recruitment\_Constant}.
\defRef{sec:Process-RecruitmentConstant}
\label{syntax:Process-RecruitmentConstant}

\defSub{categories}{The categories}
\defType{Vector of strings}
\defDefault{No default}

\defSub{proportions}{The proportion for each category}
\defType{Real number (estimable)}
\defDefault{true}

\defSub{age}{The age at recruitment}
\defType{Non-negative integer}
\defDefault{No default}

\defSub{r0}{R0, the recruitment used for annual recruits and initialise the model}
\defType{Real number (estimable)}
\defDefault{No default}
\defLowerBound{0.0 (inclusive)}

\subsubsection{Process of type Survival\_Constant\_Rate}
\commandlabsubarg{Process}{type}{Survival\_Constant\_Rate}.
\defRef{sec:Process-SurvivalConstantRate}
\label{syntax:Process-SurvivalConstantRate}

\defSub{categories}{The list of categories}
\defType{Vector of strings}
\defDefault{No default}

\defSub{s}{The survival rates}
\defType{Real number (estimable)}
\defDefault{No default}
\defLowerBound{0.0 (inclusive)}
\defUpperBound{1.0 (inclusive)}

\defSub{time\_step\_proportions}{The time step proportions for the survival rate $S$}
\defType{Vector of real numbers}
\defDefault{true}
\defLowerBound{0.0 (exclusive)}
\defUpperBound{1.0 (inclusive)}
\defValue{The proportions must sum to one}

\defSub{selectivities}{The selectivity labels for each category}
\defType{Vector of strings}
\defDefault{No default}

\subsubsection{Process of type Tag\_By\_Age}
\commandlabsubarg{Process}{type}{Tag\_By\_Age}.
\defRef{sec:Process-TagByAge}
\label{syntax:Process-TagByAge}

\defSub{from}{The categories that are selected for tagging (i.e, transition from)}
\defType{Vector of strings}
\defDefault{No default}

\defSub{to}{The categories that have tags (i.e., transition to)}
\defType{Vector of strings}
\defDefault{No default}

\defSub{min\_age}{The minimum age tagged}
\defType{Non-negative integer}
\defDefault{No default}

\defSub{max\_age}{The maximum age tagged}
\defType{Non-negative integer}
\defDefault{No default}

\defSub{penalty}{The penalty label}
\defType{String}
\defDefault{No default}

\defSub{u\_max}{The maximum exploitation rate ($U\_{max}$)}
\defType{Real number}
\defDefault{0.99}
\defLowerBound{0.0 (inclusive)}
\defUpperBound{1.0 (exclusive)}

\defSub{years}{The years to execute the tagging in}
\defType{Vector of non-negative integers}
\defDefault{No default}

\defSub{initial\_mortality}{The initial mortality value}
\defType{Real number (estimable)}
\defDefault{0.0}
\defLowerBound{0.0 (inclusive)}

\defSub{initial\_mortality\_selectivity}{The initial mortality selectivity label}
\defType{String}
\defDefault{No default}

\defSub{loss\_rate}{The loss rate}
\defType{Vector of real numbers (estimable)}
\defDefault{No default}

\defSub{loss\_rate\_selectivities}{The loss rate selectivity label}
\defType{Vector of strings}
\defDefault{true}

\defSub{selectivities}{The selectivity labels}
\defType{Vector of strings}
\defDefault{No default}

\defSub{n}{N}
\defType{Vector of real numbers (estimable)}
\defDefault{true}

\subsubsection{Process of type Tag\_By\_Length}
\commandlabsubarg{Process}{type}{Tag\_By\_Length}.
\defRef{sec:Process-TagByLength}
\label{syntax:Process-TagByLength}

\defSub{from}{The categories to transition from}
\defType{Vector of strings}
\defDefault{No default}

\defSub{to}{The categories to transition to}
\defType{Vector of strings}
\defDefault{No default}

\defSub{penalty}{The penalty label}
\defType{String}
\defDefault{No default}

\defSub{u\_max}{The maximum exploitation rate ($U\_{max}$)}
\defType{Real number (estimable)}
\defDefault{0.99}
\defLowerBound{0.0 (inclusive)}
\defUpperBound{1.0 (exclusive)}

\defSub{years}{The years to execute the transition in}
\defType{Vector of non-negative integers}
\defDefault{No default}

\defSub{initial\_mortality}{}
\defType{Real number (estimable)}
\defDefault{0.0}
\defLowerBound{0.0 (inclusive)}
\defUpperBound{1.0 (inclusive)}

\defSub{initial\_mortality\_selectivity}{}
\defType{String}
\defDefault{No default}

\defSub{selectivities}{}
\defType{Vector of strings}
\defDefault{No default}

\defSub{n}{}
\defType{Vector of real numbers (estimable)}
\defDefault{true}

\defSub{tolerance}{Tolerance for checking the proportions to move in each year}
\defType{Real number (estimable)}
\defDefault{1e-5}
\defLowerBound{0 (inclusive)}
\defUpperBound{1.0 (inclusive)}

\subsubsection{Process of type Tag\_Loss}
\commandlabsubarg{Process}{type}{Tag\_Loss}.
\defRef{sec:Process-TagLoss}
\label{syntax:Process-TagLoss}

\defSub{categories}{The list of categories}
\defType{Vector of strings}
\defDefault{No default}

\defSub{tag\_loss\_rate}{The tag loss rates}
\defType{Vector of real numbers (estimable)}
\defDefault{No default}
\defLowerBound{0.0 (inclusive)}

\defSub{time\_step\_ratio}{The time step ratios for tag loss}
\defType{Vector of real numbers (estimable)}
\defDefault{true}
\defLowerBound{0.0 (inclusive)}
\defUpperBound{1.0 (inclusive)}

\defSub{tag\_loss\_type}{The type of tag loss}
\defType{String}
\defDefault{No default}

\defSub{selectivities}{The selectivities}
\defType{Vector of strings}
\defDefault{No default}

\defSub{year}{The year the first tagging release process was executed}
\defType{Non-negative integer}
\defDefault{No default}

\subsubsection{Process of type Transition\_Category}
\commandlabsubarg{Process}{type}{Transition\_Category}.
\defRef{sec:Process-TransitionCategory}
\label{syntax:Process-TransitionCategory}

\defSub{from}{The from category}
\defType{Vector of strings}
\defDefault{No default}

\defSub{to}{The to category}
\defType{Vector of strings}
\defDefault{No default}

\defSub{proportions}{The proportions}
\defType{Real number (estimable)}
\defDefault{No default}
\defLowerBound{0.0 (inclusive)}
\defUpperBound{1.0 (inclusive)}

\defSub{selectivities}{The selectivity names}
\defType{Vector of strings}
\defDefault{No default}

\subsubsection{Process of type Transition\_Category\_By\_Age}
\commandlabsubarg{Process}{type}{Transition\_Category\_By\_Age}.
\defRef{sec:Process-TransitionCategoryByAge}
\label{syntax:Process-TransitionCategoryByAge}

\defSub{from}{The categories to transition from}
\defType{Vector of strings}
\defDefault{No default}

\defSub{to}{The categories to transition to}
\defType{Vector of strings}
\defDefault{No default}

\defSub{min\_age}{The minimum age to transition}
\defType{Non-negative integer}
\defDefault{No default}

\defSub{max\_age}{The maximum age to transition}
\defType{Non-negative integer}
\defDefault{No default}

\defSub{penalty}{The penalty label}
\defType{String}
\defDefault{No default}

\defSub{u\_max}{The maximum exploitation rate ($U\_{max}$)}
\defType{Real number (estimable)}
\defDefault{0.99}
\defLowerBound{0.0 (inclusive)}
\defUpperBound{1.0 (inclusive)}

\defSub{years}{The years to execute the transition in}
\defType{Vector of non-negative integers}
\defDefault{No default}

\subsubsection{Process of type Unit\_Tester}
\commandlabsubarg{Process}{type}{Unit\_Tester}.
\defRef{sec:Process-UnitTester}
\label{syntax:Process-UnitTester}

The Unit\_Tester type has no additional subcommands.
 

\subsection{\I{Time varying parameters}}
\defComLab{Time\_Varying}{Define an object of type \emph{Time\_Varying}}.
\defRef{sec:TimeVarying}
\label{syntax:TimeVarying}

\defSub{label}{The label of the time-varying object}
\defType{String}
\defDefault{No default}

\defSub{type}{The type of the time-varying object}
\defType{String}
\defDefault{No default}

\defSub{parameter}{The name of the parameter to vary in each year}
\defType{String}
\defDefault{No default}

\defSub{years}{The years in which to vary the parameter}
\defType{Vector of non-negative integers}
\defDefault{No default}

\subsubsection{Time Varying of type Annual Shift}
\commandlabsubarg{Time\_Varying}{type}{Annual\_Shift}.
\defRef{sec:TimeVarying-AnnualShift}
\label{syntax:TimeVarying-AnnualShift}

\defSub{a}{Parameter A}
\defType{Real number (estimable)}
\defDefault{No default}

\defSub{b}{Parameter B}
\defType{Real number (estimable)}
\defDefault{No default}

\defSub{c}{Parameter C}
\defType{Real number (estimable)}
\defDefault{No default}

\defSub{scaling\_years}{The scaling years}
\defType{Vector of non-negative integers}
\defDefault{The years in which to vary the parameter}

\defSub{values}{The values}
\defType{Vector of real numbers (estimable)}
\defDefault{No default}

\subsubsection{Time Varying of type Constant}
\commandlabsubarg{Time\_Varying}{type}{Constant}.
\defRef{sec:TimeVarying-Constant}
\label{syntax:TimeVarying-Constant}

\defSub{values}{The value to assign to addressable}
\defType{Vector of real numbers (estimable)}
\defDefault{No default}

\subsubsection{Time Varying of type Exogenous}
\commandlabsubarg{Time\_Varying}{type}{Exogenous}.
\defRef{sec:TimeVarying-Exogenous}
\label{syntax:TimeVarying-Exogenous}

\defSub{a}{The shift parameter}
\defType{Real number (estimable)}
\defDefault{No default}

\defSub{exogenous\_variable}{The values of exogenous variable for each year}
\defType{Vector of real numbers (estimable)}
\defDefault{No default}

\subsubsection{Time Varying of type Linear}
\commandlabsubarg{Time\_Varying}{type}{Linear}.
\defRef{sec:TimeVarying-Linear}
\label{syntax:TimeVarying-Linear}

\defSub{slope}{The slope of the linear trend (i.e., the additive amount per year)}
\defType{Real number (estimable)}
\defDefault{No default}

\defSub{intercept}{The intercept of the linear trend (, i.e. the value in the first year)}
\defType{Real number (estimable)}
\defDefault{No default}

\subsubsection{Time Varying of type Random Draw}
\commandlabsubarg{Time\_Varying}{type}{Random\_Draw}.
\defRef{sec:TimeVarying-RandomDraw}
\label{syntax:TimeVarying-RandomDraw}

\defSub{mean}{The mean ($\mu$) of the random draw distribution}
\defType{Real number (estimable)}
\defDefault{0}

\defSub{sigma}{The standard deviation ($\sigma$)  of the random draw distribution}
\defType{Real number (estimable)}
\defDefault{1.0}
\defValue{A positive real number}

\defSub{lower\_bound}{The lower bound for the random draw}
\defType{Real number (estimable)}
\defDefault{No default}

\defSub{upper\_bound}{The upper bound for the random draw}
\defType{Real number (estimable)}
\defDefault{No default}

\defSub{distribution}{The distribution type}
\defType{String}
\defDefault{normal}
\defValue{Only the normal and lognormal are implemented}

\subsubsection{Time Varying of type Random Walk}
\commandlabsubarg{Time\_Varying}{type}{Random\_Walk}.
\defRef{sec:TimeVarying-RandomWalk}
\label{syntax:TimeVarying-RandomWalk}

\defSub{mean}{The mean ($\mu$) of the random walk distribution}
\defType{Real number (estimable)}
\defDefault{0.0}

\defSub{sigma}{The standard deviation ($\sigma$)  of the random walk distribution}
\defType{Real number (estimable)}
\defDefault{1.0}
\defValue{A positive real number}

\defSub{lower\_bound}{The lower bound for the random walk}
\defType{Real number (estimable)}
\defDefault{No default}

\defSub{upper\_bound}{The upper bound for the random walk}
\defType{Real number (estimable)}
\defDefault{No default}

\defSub{rho}{The autocorrelation parameter ($\rho$)  of the random walk distribution}
\defType{Real number (estimable)}
\defDefault{1}

\defSub{distribution}{The distribution type}
\defType{String}
\defDefault{normal}
\defValue{Only the normal distribution is implemented}



\subsection{\I{Derived quantities}}
\defComLab{derived\_quantity}{Define an object of type \emph{derived\_quantity}}

\defSub{label} {Label of the derived quantity}
\defType{string}
\defDefault{No Default}

\defSub{type} {Type of derived quantity}
\defType{string}
\defDefault{No Default}

\defSub{time\_step} {The time step in which to calculate the derived quantity after}
\defType{string}
\defDefault{No Default}

\defSub{categories} {The list of categories to use when calculating the derived quantity}
\defType{string vector}
\defDefault{No Default}

\defSub{selectivities} {A list of one selectivity}
\defType{string vector}
\defDefault{No Default}

\defSub{time\_step\_proportion} {Proportion through the mortality block of the time step when calculated}
\defType{constant}
\defDefault{0.5}
\defLowerBound{0.0 (inclusive)}
\defUpperBound{1.0 (inclusive)}

\defSub{time\_step\_proportion\_method} {Method for interpolating for the proportion through the mortality block}
\defType{string}
\defDefault{weighted\_sum}
\defAllowedValues{weighted\_sum, weighted\_product}

\subsubsection[Abundance]{\commandlabsubarg{derived\_\_quantity}{type}{abundance}}

\subsubsection[Biomass]{\commandlabsubarg{derived\_\_quantity}{type}{biomass}}

 

\subsection{\I{Age-size relationship}}
\defComLab{age\_size}{Define an object type Age\_Size}

\defSub{label} {Label}
\defType{string}
\defDefault{No default}

\defSub{type} {Type}
\defType{string}
\defDefault{No default}

\subsubsection[None]{\commandlabsubarg{age\_size}{type}{none}}

\defSub{size\_weight} {Not Implemented}
\defType{string}
\defDefault{No default}

\subsubsection[Schnute]{\commandlabsubarg{age\_size}{type}{schnute}}

\defSub{tau2} {TBA}
\defType{double}
\defDefault{No default}

\defSub{tau1} {TBA}
\defType{double}
\defDefault{No default}

\defSub{y1} {TBA}
\defType{double}
\defDefault{No default}

\defSub{distribution} {TBA}
\defType{string}
\defDefault{normal}

\defSub{y2} {TBA}
\defType{double}
\defDefault{No default}

\defSub{cv} {TBA}
\defType{double}
\defDefault{0.0}

\defSub{by\_length} {TBA}
\defType{bool}
\defDefault{true}

\defSub{a} {TBA}
\defType{double}
\defDefault{No default}

\defSub{size\_weight} {TBA}
\defType{string}
\defDefault{No default}

\defSub{b} {TBA}
\defType{double}
\defDefault{No default}

\subsubsection[Von Bertalanffy]{\commandlabsubarg{age\_size}{type}{von\_bertalanffy}}

\defSub{linf} {TBA}
\defType{double}
\defDefault{No default}

\defSub{distribution} {TBA}
\defType{string}
\defDefault{normal}

\defSub{t0} {TBA}
\defType{double}
\defDefault{No default}

\defSub{cv} {TBA}
\defType{double}
\defDefault{0.0}

\defSub{k} {TBA}
\defType{double}
\defDefault{No default}

\defSub{by\_length} {TBA}
\defType{bool}
\defDefault{true}

\defSub{size\_weight} {TBA}
\defType{string}
\defDefault{No default}

 

\subsection{\I{Size-weight}}
\defComLab{size\_weight}{Define an object type Size\_Weight}

\defSub{label} {Label}
\defType{string}
\defDefault{No default}

\defSub{type} {Type}
\defType{string}
\defDefault{No default}

\subsubsection[Basic]{\commandlabsubarg{size\_weight}{type}{basic}}

\defSub{a} {A}
\defType{double}
\defDefault{No default}

\defSub{b} {B}
\defType{double}
\defDefault{No default}

\subsubsection[None]{\commandlabsubarg{size\_weight}{type}{none}}



\subsection{\I{Selectivities}}
\defComLab{Selectivity}{Define an object of type \emph{Selectivity}}.
\defRef{sec:Selectivity}
\label{syntax:Selectivity}

\defSub{label}{The label for the selectivity}
\defType{String}
\defDefault{No default}

\defSub{type}{The type of selectivity}
\defType{String}
\defDefault{No default}

\defSub{length\_based}{Is the selectivity length based?}
\defType{Boolean}
\defDefault{false}

\defSub{intervals}{The number of quantiles to evaluate a length-based selectivity over the age-length distribution}
\defType{Non-negative integer}
\defDefault{5}

\defSub{values}{}
\defType{Vector of addressables}
\defDefault{No default}

\defSub{length\_values}{}
\defType{Vector of addressables}
\defDefault{No default}

\subsubsection{Selectivity of type All\_Values}
\commandlabsubarg{Selectivity}{type}{All\_Values}.
\defRef{sec:Selectivity-AllValues}
\label{syntax:Selectivity-AllValues}

\defSub{v}{The v parameter}
\defType{Vector of real numbers (estimable)}
\defDefault{No default}

\subsubsection{Selectivity of type All\_Values\_Bounded}
\commandlabsubarg{Selectivity}{type}{All\_Values\_Bounded}.
\defRef{sec:Selectivity-AllValuesBounded}
\label{syntax:Selectivity-AllValuesBounded}

\defSub{l}{The low value (L)}
\defType{Non-negative integer}
\defDefault{No default}

\defSub{h}{The high value (H)}
\defType{Non-negative integer}
\defDefault{No default}

\defSub{v}{The v parameter}
\defType{Vector of real numbers (estimable)}
\defDefault{No default}

\subsubsection{Selectivity of type Compound\_All}
\commandlabsubarg{Selectivity}{type}{Compound\_All}.
\defRef{sec:Selectivity-CompoundAll}
\label{syntax:Selectivity-CompoundAll}

\defSub{a50}{The mean (mu)}
\defType{Real number (estimable)}
\defDefault{No default}

\defSub{ato95}{The sigma L parameter}
\defType{Real number (estimable)}
\defDefault{No default}
\defLowerBound{0.0 (exclusive)}

\defSub{a\_min}{The sigma R parameter}
\defType{Real number (estimable)}
\defDefault{No default}
\defLowerBound{0.0 (exclusive)}

\defSub{alpha}{The maximum value of the selectivity}
\defType{Real number (estimable)}
\defDefault{1.0}
\defLowerBound{0.0 (exclusive)}

\subsubsection{Selectivity of type Compound\_Left}
\commandlabsubarg{Selectivity}{type}{Compound\_Left}.
\defRef{sec:Selectivity-CompoundLeft}
\label{syntax:Selectivity-CompoundLeft}

\defSub{a50}{The mean (mu)}
\defType{Real number (estimable)}
\defDefault{No default}

\defSub{ato95}{The sigma L parameter}
\defType{Real number (estimable)}
\defDefault{No default}
\defLowerBound{0.0 (exclusive)}

\defSub{a\_min}{The sigma R parameter}
\defType{Real number (estimable)}
\defDefault{No default}
\defLowerBound{0.0 (exclusive)}

\defSub{left\_mean}{The maximum value of the selectivity}
\defType{Real number (estimable)}
\defDefault{1.0}
\defLowerBound{0.0 (exclusive)}

\defSub{sigma}{The maximum value of the selectivity}
\defType{Real number (estimable)}
\defDefault{1.0}
\defLowerBound{0.0 (exclusive)}

\defSub{alpha}{The maximum value of the selectivity}
\defType{Real number (estimable)}
\defDefault{1.0}
\defLowerBound{0.0 (exclusive)}

\subsubsection{Selectivity of type Compound\_Middle}
\commandlabsubarg{Selectivity}{type}{Compound\_Middle}.
\defRef{sec:Selectivity-CompoundMiddle}
\label{syntax:Selectivity-CompoundMiddle}

\defSub{a50}{The mean (mu)}
\defType{Real number (estimable)}
\defDefault{No default}

\defSub{ato95}{The sigma L parameter}
\defType{Real number (estimable)}
\defDefault{No default}
\defLowerBound{0.0 (exclusive)}

\defSub{a\_min}{The sigma R parameter}
\defType{Real number (estimable)}
\defDefault{No default}
\defLowerBound{0.0 (exclusive)}

\defSub{left\_mean}{The maximum value of the selectivity}
\defType{Real number (estimable)}
\defDefault{1.0}
\defLowerBound{0.0 (exclusive)}

\defSub{to\_right\_mean}{The maximum value of the selectivity}
\defType{Real number (estimable)}
\defDefault{1.0}
\defLowerBound{0.0 (exclusive)}

\defSub{sigma}{The maximum value of the selectivity}
\defType{Real number (estimable)}
\defDefault{1.0}
\defLowerBound{0.0 (exclusive)}

\defSub{alpha}{The maximum value of the selectivity}
\defType{Real number (estimable)}
\defDefault{1.0}
\defLowerBound{0.0 (exclusive)}

\subsubsection{Selectivity of type Compound\_Right}
\commandlabsubarg{Selectivity}{type}{Compound\_Right}.
\defRef{sec:Selectivity-CompoundRight}
\label{syntax:Selectivity-CompoundRight}

\defSub{a50}{The mean (mu)}
\defType{Real number (estimable)}
\defDefault{No default}

\defSub{ato95}{The sigma L parameter}
\defType{Real number (estimable)}
\defDefault{No default}
\defLowerBound{0.0 (exclusive)}

\defSub{a\_min}{The sigma R parameter}
\defType{Real number (estimable)}
\defDefault{No default}
\defLowerBound{0.0 (exclusive)}

\defSub{left\_mean}{The maximum value of the selectivity}
\defType{Real number (estimable)}
\defDefault{1.0}
\defLowerBound{0.0 (exclusive)}

\defSub{to\_right\_mean}{The maximum value of the selectivity}
\defType{Real number (estimable)}
\defDefault{1.0}
\defLowerBound{0.0 (exclusive)}

\defSub{sigma}{The maximum value of the selectivity}
\defType{Real number (estimable)}
\defDefault{1.0}
\defLowerBound{0.0 (exclusive)}

\defSub{alpha}{The maximum value of the selectivity}
\defType{Real number (estimable)}
\defDefault{1.0}
\defLowerBound{0.0 (exclusive)}

\subsubsection{Selectivity of type Constant}
\commandlabsubarg{Selectivity}{type}{Constant}.
\defRef{sec:Selectivity-Constant}
\label{syntax:Selectivity-Constant}

\defSub{c}{The constant value}
\defType{Real number (estimable)}
\defDefault{No default}

\subsubsection{Selectivity of type Constant.Mock}
\commandlabsubarg{Selectivity}{type}{Constant.Mock}.
\defRef{sec:Selectivity-Constant.Mock}
\label{syntax:Selectivity-Constant.Mock}

The Constant.Mock type has no additional subcommands.
\subsubsection{Selectivity of type Double\_Exponential}
\commandlabsubarg{Selectivity}{type}{Double\_Exponential}.
\defRef{sec:Selectivity-DoubleExponential}
\label{syntax:Selectivity-DoubleExponential}

\defSub{x0}{The X0 parameter}
\defType{Real number (estimable)}
\defDefault{No default}

\defSub{x1}{The X1 parameter}
\defType{Real number (estimable)}
\defDefault{No default}

\defSub{x2}{The X2 parameter}
\defType{Real number (estimable)}
\defDefault{No default}

\defSub{y0}{The Y0 parameter}
\defType{Real number (estimable)}
\defDefault{No default}
\defLowerBound{0.0 (exclusive)}

\defSub{y1}{The Y1 parameter}
\defType{Real number (estimable)}
\defDefault{No default}
\defLowerBound{0.0 (exclusive)}

\defSub{y2}{The Y2 parameter}
\defType{Real number (estimable)}
\defDefault{No default}
\defLowerBound{0.0 (exclusive)}

\defSub{alpha}{The maximum value of the selectivity}
\defType{Real number (estimable)}
\defDefault{1.0}
\defLowerBound{0.0 (exclusive)}

\subsubsection{Selectivity of type Double\_Normal}
\commandlabsubarg{Selectivity}{type}{Double\_Normal}.
\defRef{sec:Selectivity-DoubleNormal}
\label{syntax:Selectivity-DoubleNormal}

\defSub{mu}{The mean (mu)}
\defType{Real number (estimable)}
\defDefault{No default}

\defSub{sigma\_l}{The sigma L parameter}
\defType{Real number (estimable)}
\defDefault{No default}
\defLowerBound{0.0 (exclusive)}

\defSub{sigma\_r}{The sigma R parameter}
\defType{Real number (estimable)}
\defDefault{No default}
\defLowerBound{0.0 (exclusive)}

\defSub{alpha}{The maximum value of the selectivity}
\defType{Real number (estimable)}
\defDefault{1.0}
\defLowerBound{0.0 (exclusive)}

\subsubsection{Selectivity of type Double\_Normal\_Plateau}
\commandlabsubarg{Selectivity}{type}{Double\_Normal\_Plateau}.
\defRef{sec:Selectivity-DoubleNormalPlateau}
\label{syntax:Selectivity-DoubleNormalPlateau}

\defSub{sigma\_l}{The sigma L parameter}
\defType{Real number (estimable)}
\defDefault{No default}
\defLowerBound{0.0 (exclusive)}

\defSub{sigma\_r}{The sigma R parameter}
\defType{Real number (estimable)}
\defDefault{No default}
\defLowerBound{0.0 (exclusive)}

\defSub{a1}{The a1 parameter}
\defType{Real number (estimable)}
\defDefault{No default}
\defLowerBound{0.0 (exclusive)}

\defSub{a2}{The a2 parameter}
\defType{Real number (estimable)}
\defDefault{No default}
\defLowerBound{0.0 (exclusive)}

\defSub{alpha}{The maximum value of the selectivity}
\defType{Real number (estimable)}
\defDefault{1.0}
\defLowerBound{0.0 (inclusive)}

\subsubsection{Selectivity of type Double\_Normal\_SS3}
\commandlabsubarg{Selectivity}{type}{Double\_Normal\_SS3}.
\defRef{sec:Selectivity-DoubleNormalSS3}
\label{syntax:Selectivity-DoubleNormalSS3}

\defSub{peak}{beginning size (or age) for the plateau (in cm or year).}
\defType{Real number (estimable)}
\defDefault{No default}
\defLowerBound{0.0 (exclusive)}

\defSub{y0}{Selectivity at first bin, as logistic between 0 and 1}
\defType{Real number (estimable)}
\defDefault{No default}
\defLowerBound{-20 (inclusive)}
\defUpperBound{0.0 (inclusive)}

\defSub{y1}{Selectivity at last bin, as logistic between 0 and 1}
\defType{Real number (estimable)}
\defDefault{No default}
\defLowerBound{-20 (inclusive)}
\defUpperBound{10.0 (inclusive)}

\defSub{ascending}{Ascending width: parameter value is ln(width).}
\defType{Real number (estimable)}
\defDefault{No default}

\defSub{descending}{Descending width: parameter value is ln(width).}
\defType{Real number (estimable)}
\defDefault{No default}

\defSub{width}{width of plateau, as logistic between peak and maximum length (or age)}
\defType{Real number (estimable)}
\defDefault{No default}

\defSub{l}{The low value (L)}
\defType{Real number (estimable)}
\defDefault{No default}
\defLowerBound{0.0 (exclusive)}

\defSub{h}{The high value (H)}
\defType{Real number (estimable)}
\defDefault{No default}
\defLowerBound{1.0 (exclusive)}

\defSub{alpha}{The maximum value of the selectivity}
\defType{Real number (estimable)}
\defDefault{1.0}
\defLowerBound{0.0 (inclusive)}

\subsubsection{Selectivity of type Increasing}
\commandlabsubarg{Selectivity}{type}{Increasing}.
\defRef{sec:Selectivity-Increasing}
\label{syntax:Selectivity-Increasing}

\defSub{l}{The low value (L)}
\defType{Non-negative integer}
\defDefault{No default}

\defSub{h}{The high value (H)}
\defType{Non-negative integer}
\defDefault{No default}

\defSub{v}{The v parameter}
\defType{Vector of real numbers (estimable)}
\defDefault{No default}

\defSub{alpha}{The maximum value of the selectivity}
\defType{Real number (estimable)}
\defDefault{1.0}
\defLowerBound{0.0 (exclusive)}

\subsubsection{Selectivity of type Inverse\_Logistic}
\commandlabsubarg{Selectivity}{type}{Inverse\_Logistic}.
\defRef{sec:Selectivity-InverseLogistic}
\label{syntax:Selectivity-InverseLogistic}

\defSub{a50}{a50}
\defType{Real number (estimable)}
\defDefault{No default}

\defSub{ato95}{ato95}
\defType{Real number (estimable)}
\defDefault{No default}
\defLowerBound{0.0 (exclusive)}

\defSub{alpha}{The maximum value of the selectivity}
\defType{Real number (estimable)}
\defDefault{1.0}
\defLowerBound{0.0 (exclusive)}

\subsubsection{Selectivity of type Knife\_Edge}
\commandlabsubarg{Selectivity}{type}{Knife\_Edge}.
\defRef{sec:Selectivity-KnifeEdge}
\label{syntax:Selectivity-KnifeEdge}

\defSub{e}{The edge value}
\defType{Real number (estimable)}
\defDefault{No default}

\defSub{alpha}{The maximum value of the selectivity}
\defType{Real number (estimable)}
\defDefault{1.0}

\subsubsection{Selectivity of type Logistic}
\commandlabsubarg{Selectivity}{type}{Logistic}.
\defRef{sec:Selectivity-Logistic}
\label{syntax:Selectivity-Logistic}

\defSub{a50}{The a50 parameter}
\defType{Real number (estimable)}
\defDefault{No default}

\defSub{ato95}{The ato95 parameter}
\defType{Real number (estimable)}
\defDefault{No default}
\defLowerBound{0.0 (exclusive)}

\defSub{alpha}{The maximum value of the selectivity}
\defType{Real number (estimable)}
\defDefault{1.0}
\defLowerBound{0.0 (exclusive)}

\subsubsection{Selectivity of type Logistic\_Producing}
\commandlabsubarg{Selectivity}{type}{Logistic\_Producing}.
\defRef{sec:Selectivity-LogisticProducing}
\label{syntax:Selectivity-LogisticProducing}

\defSub{l}{The low value (L)}
\defType{Non-negative integer}
\defDefault{No default}

\defSub{h}{The high value (H)}
\defType{Non-negative integer}
\defDefault{No default}

\defSub{a50}{the a50 parameter}
\defType{Real number (estimable)}
\defDefault{No default}

\defSub{ato95}{The ato95 parameter}
\defType{Real number (estimable)}
\defDefault{No default}
\defLowerBound{0.0 (exclusive)}

\defSub{alpha}{The maximum value of the selectivity}
\defType{Real number (estimable)}
\defDefault{1.0}
\defLowerBound{0.0 (exclusive)}

 

\section{Estimation command and subcommand syntax\label{sec:estimation-syntax}}

\subsection{\I{Estimation methods}}
\defComLab{estimate}{Define an object of type \emph{estimate}}

\defSub{label} {The label of the estimate}
\defType{string}
\defDefault{""}

\defSub{type} {The prior type for the estimate}
\defType{string}
\defDefault{No Default}

\defSub{parameter} {The name of the parameter to estimate in the model}
\defType{string}
\defDefault{No Default}

\defSub{lower\_bound} {The lower bound for the parameter}
\defType{constant}
\defDefault{No Default}

\defSub{upper\_bound} {The upper bound for the parameter}
\defType{constant}
\defDefault{No Default}

\defSub{same} {List of parameters that are constrained to have the same value as this parameter}
\defType{string vector}
\defDefault{""}

\defSub{mcmc} {Indicates if this parameter is fixed at the point estimate during an MCMC run}
\defType{boolean}
\defDefault{false}

\defSub{transformation} {Type of simple transformation to apply to estimable}
\defType{string}
\defDefault{""}

\defSub{transform\_with\_jacobian} {Apply jacobian during transformation}
\defType{boolean}
\defDefault{false}

\subsubsection[Beta]{\commandlabsubarg{estimate}{type}{beta}}

\defSub{mu} {Beta prior  mean (mu) parameter}
\defType{estimable}
\defDefault{No Default}

\defSub{sigma} {Beta prior variance (sigma) parameter}
\defType{estimable}
\defDefault{No Default}
\defLowerBound{0.0 (exclusive)}

\defSub{a} {Beta prior lower bound of the range (A) parameter}
\defType{constant}
\defDefault{No Default}

\defSub{b} {Beta prior upper bound of the range (B) parameter}
\defType{constant}
\defDefault{No Default}

\subsubsection[Lognormal]{\commandlabsubarg{estimate}{type}{lognormal}}

\defSub{mu} {The lognormal prior mean (mu) parameter}
\defType{estimable}
\defDefault{No Default}
\defLowerBound{0.0 (exclusive)}

\defSub{cv} {The Lognormal variance (CV) parameter}
\defType{estimable}
\defDefault{No Default}
\defLowerBound{0.0 (exclusive)}

\subsubsection[Normal]{\commandlabsubarg{estimate}{type}{normal}}

\defSub{mu} {The normal prior mean (mu) parameter}
\defType{estimable}
\defDefault{No Default}

\defSub{cv} {The normal variance (standard devation) parameter}
\defType{estimable}
\defDefault{No Default}
\defLowerBound{0.0 (exclusive)}

\subsubsection[Normal By Stdev]{\commandlabsubarg{estimate}{type}{normal\_by\_stdev}}

\defSub{mu} {The normal prior mean (mu) parameter}
\defType{estimable}
\defDefault{No Default}

\defSub{sigma} {The normal variance (standard devation) parameter}
\defType{estimable}
\defDefault{No Default}
\defLowerBound{0.0 (exclusive)}

\subsubsection[Normal Log]{\commandlabsubarg{estimate}{type}{normal\_log}}

\defSub{mu} {The normal-log prior mean (mu) parameter}
\defType{estimable}
\defDefault{No Default}

\defSub{sigma} {The normal-log prior variance (standard deviation) parameter}
\defType{estimable}
\defDefault{No Default}
\defLowerBound{0.0 (exclusive)}

\subsubsection[Uniform]{\commandlabsubarg{estimate}{type}{uniform}}

\subsubsection[Uniform Log]{\commandlabsubarg{estimate}{type}{uniform\_log}}



\subsection{\I{Point estimation}}
\defComLab{minimiser}{Define an object of type \emph{minimiser}}

\defSub{label} {The minimiser label}
\defType{string}
\defDefault{No Default}

\defSub{type} {The type of minimiser to use}
\defType{string}
\defDefault{No Default}

\defSub{active} {Indicates if this minimiser is active}
\defType{boolean}
\defDefault{false}

\defSub{covariance} {Indicates if a covariance matrix should be generated}
\defType{boolean}
\defDefault{true}

\subsubsection[A D O L C]{\commandlabsubarg{minimiser}{type}{adolc}}

\defSub{iterations} {Maximum number of iterations}
\defType{integer}
\defDefault{1000}

\defSub{evaluations} {Maximum number of evaluations}
\defType{integer}
\defDefault{4000}

\defSub{tolerance} {Tolerance of the gradient for convergence}
\defType{constant}
\defDefault{0.02}

\defSub{step\_size} {Minimum Step-size before minimisation fails}
\defType{constant}
\defDefault{1e-7}

\subsubsection[Beta Diff]{\commandlabsubarg{minimiser}{type}{betadiff}}

\defSub{iterations} {Maximum number of iterations}
\defType{integer}
\defDefault{1000}

\defSub{evaluations} {Maximum number of evaluations}
\defType{integer}
\defDefault{4000}

\defSub{tolerance} {Tolerance of the gradient for convergence}
\defType{constant}
\defDefault{2e-3}

\subsubsection[C P P A D]{\commandlabsubarg{minimiser}{type}{cppad}}

\subsubsection[D E Solver]{\commandlabsubarg{minimiser}{type}{de\_solver}}

\defSub{population\_size} {The number of candidate solutions to have in the population}
\defType{non-negative integer}
\defDefault{No Default}

\defSub{crossover\_probability} {Define the minimisers crossover probability}
\defType{constant}
\defDefault{0.9}

\defSub{difference\_scale} {The scale to apply to new solutions when comparing candidates}
\defType{constant}
\defDefault{0.02}

\defSub{max\_generations} {The maximum number of iterations to run}
\defType{non-negative integer}
\defDefault{No Default}

\defSub{tolerance} {The total variance between the population and best candidate before acceptance}
\defType{constant}
\defDefault{0.01}

\defSub{method} {The type of candidate generation method to use}
\defType{string}
\defDefault{""}
\defValue{not\_yet\_implemented}

\subsubsection[D Lib]{\commandlabsubarg{minimiser}{type}{d\_lib}}

\subsubsection[Gamma Diff]{\commandlabsubarg{minimiser}{type}{numerical\_differences}}

\defSub{iterations} {Maximum number of iterations}
\defType{integer}
\defDefault{1000}

\defSub{evaluations} {Maximum number of evaluations}
\defType{integer}
\defDefault{4000}

\defSub{tolerance} {Tolerance of the gradient for convergence}
\defType{constant}
\defDefault{0.02}

\defSub{step\_size} {Minimum Step-size before minimisation fails}
\defType{constant}
\defDefault{1e-7}



\subsection{\I{Monte Carlo Markov Chain (MCMC)}\label{sec:estimation-syntax-MCMC}}
\defComLab{MCMC}{Define an object of type \emph{MCMC}}.
\defRef{sec:MCMC}
\label{syntax:MCMC}

\defSub{label}{The label of the MCMC}
\defType{String}
\defDefault{No default}

\defSub{type}{The MCMC method}
\defType{String}
\defDefault{No default}

\defSub{length}{The number of iterations for the MCMC (including the burn in period)}
\defType{Non-negative integer}
\defDefault{No default}
\defLowerBound{1 (inclusive)}

\defSub{burn\_in}{The number of iterations for the burn\_in period of the MCMC}
\defType{Non-negative integer}
\defDefault{0}
\defLowerBound{0 (inclusive)}

\defSub{active}{Indicates if this is the active MCMC algorithm}
\defType{Boolean}
\defDefault{true}

\defSub{step\_size}{Initial step-size (as a multiplier of the approximate covariance matrix)}
\defType{Real number}
\defDefault{The default is $2.4d^{-0.5}$}
\defLowerBound{0 (inclusive)}

\defSub{start}{The covariance multiplier for the starting point of the MCMC}
\defType{Real number}
\defDefault{0.0}
\defLowerBound{0.0 (inclusive)}
\defValue{If zero, then the MCMC starts at the point estimate (i.e., the MPD). Otherwise a random (multivariate normal) jump from the point estimate with \subcommand{start} used as the standard deviation multiplier}

\defSub{keep}{The spacing between recorded values in the MCMC}
\defType{Non-negative integer}
\defDefault{1}
\defLowerBound{1 (inclusive)}

\defSub{max\_correlation}{The maximum absolute correlation in the covariance matrix of the proposal distribution}
\defType{Real number}
\defDefault{0.8}
\defLowerBound{0.0 (exclusive)}
\defUpperBound{1.0 (inclusive)}

\defSub{covariance\_adjustment\_method}{The method for adjusting small variances in the covariance proposal matrix}
\defType{String}
\defDefault{correlation}
\defValue{Either covariance, correlation, or none}

\defSub{correlation\_adjustment\_diff}{The minimum non-zero variance times the range of the bounds in the covariance matrix of the proposal distribution}
\defType{Real number}
\defDefault{0.0001}
\defLowerBound{0.0 (exclusive)}

\defSub{proposal\_distribution}{The shape of the proposal distribution (either the t or the normal distribution)}
\defType{String}
\defDefault{t}
\defValue{Either t or normal}

\defSub{df}{The degrees of freedom of the multivariate t proposal distribution}
\defType{Non-negative integer}
\defDefault{4}
\defLowerBound{1}

\defSub{adapt\_stepsize\_at}{The iteration numbers in which to check and resize the MCMC stepsize}
\defType{Vector of non-negative integers}
\defDefault{true}
\defLowerBound{0 (inclusive)}
\defValue{If zero, then no step size adaption is applied. Otherwise must be a positive integer less than less than the burn\_in}

\defSub{adapt\_stepsize\_method}{The method to use to adapt the step size}
\defType{String}
\defDefault{ratio}

\defSub{adapt\_covariance\_matrix\_at}{The iteration number in which to adapt the covariance matrix}
\defType{Non-negative integer}
\defDefault{0}
\defLowerBound{0 (inclusive)}
\defValue{If zero, then no covariance matrix adaption is applied. Otherwise must be a positive integer that is less than the burn\_in}

\subsubsection{MCMC of type Hamiltonian\_Monte\_Carlo}
\commandlabsubarg{MCMC}{type}{Hamiltonian\_Monte\_Carlo}.
\label{syntax:MCMC-HamiltonianMonteCarlo}

\defSub{leapfrog\_steps}{Number of leapfrog steps}
\defType{Non-negative integer}
\defDefault{1}
\defLowerBound{0 (exclusive)}

\defSub{leapfrog\_delta}{Amount to leapfrog per step}
\defType{Real number}
\defDefault{1e-7}
\defLowerBound{0 (exclusive)}

\defSub{gradient\_step\_size}{Step size to use when calculating gradient}
\defType{Real number}
\defDefault{1e-7}
\defLowerBound{1e-13 (inclusive)}

\subsubsection{MCMC of type Random\_Walk\_Metropolis\_Hastings}
\commandlabsubarg{MCMC}{type}{Random\_Walk}.
\defRef{sec:MCMC-RandomWalkMetropolisHastings}
\label{syntax:MCMC-RandomWalkMetropolisHastings}

The Random\_Walk type has no additional subcommands.


\subsection{\I{Profiles}}
\defComLab{profile}{Define an object of type \emph{profile}}

\defSub{label} {Label}
\defType{string}
\defDefault{""}

\defSub{steps} {The number of steps to take between the lower and upper bound}
\defType{non-negative integer}
\defDefault{No Default}

\defSub{lower\_bound} {The lower bounds}
\defType{constant}
\defDefault{No Default}

\defSub{upper\_bound} {The upper bounds}
\defType{constant}
\defDefault{No Default}

\defSub{parameter} {The system parameter to profile}
\defType{string}
\defDefault{No Default}



\subsection{\I{Defining catchability constants}}
\defComLab{catchability}{Define an object type Catchability}

\defSub{label} {Label}
\defType{string}
\defDefault{No default}

\defSub{q} {The catchability amount}
\defType{double}
\defDefault{No default}



\subsection{\I{Defining penalties}}
\defComLab{penalty}{Define an object of type \emph{penalty}}

\defSub{label} {The label of the penalty}
\defType{string}
\defDefault{No Default}

\defSub{type} {The type of penalty}
\defType{string}
\defDefault{No Default}

\subsubsection[Process]{\commandlabsubarg{penalty}{type}{process}}

\defSub{multiplier} {The penalty multiplier}
\defType{constant}
\defDefault{1.0}

\defSub{log\_scale} {Indicates if the sums of squares is calculated on the log scale}
\defType{boolean}
\defDefault{false}



\subsection{\I{Defining priors on parameter ratios, differences and means}}
\defComLab{Additional\_Prior}{Define an object of type \emph{Additional\_Prior}}.
\defRef{sec:AdditionalPrior}
\label{syntax:AdditionalPrior}

\defSub{label}{The label for the additional prior}
\defType{String}
\defDefault{No default}

\defSub{parameter}{The name of the parameter for the additional prior}
\defType{String}
\defDefault{No default}

\defSub{type}{The additional prior type}
\defType{String}
\defDefault{No default}

\subsubsection{Additional\_Prior of type Beta}
\commandlabsubarg{Additional\_Prior}{type}{Beta}.
\defRef{sec:AdditionalPrior-Beta}
\label{syntax:AdditionalPrior-Beta}

\defSub{mu}{Beta distribution mean $\mu$ parameter}
\defType{Real number (estimable)}
\defDefault{No default}

\defSub{sigma}{Beta distribution variance $\sigma$ parameter}
\defType{Real number (estimable)}
\defDefault{No default}
\defLowerBound{0.0 (exclusive)}

\defSub{a}{Beta distribution lower bound, of the range $A$ parameter}
\defType{Real number (estimable)}
\defDefault{No default}

\defSub{b}{Beta distribution upper bound of the range $B$ parameter}
\defType{Real number (estimable)}
\defDefault{No default}

\subsubsection{Additional\_Prior of type Element\_Difference}
\commandlabsubarg{Additional\_Prior}{type}{Element\_Difference}.
\defRef{sec:AdditionalPrior-ElementDifference}
\label{syntax:AdditionalPrior-ElementDifference}

\defSub{second\_parameter}{The name of the second parameter for comparing}
\defType{String}
\defDefault{No default}

\defSub{multiplier}{Multiply the penalty by this factor}
\defType{Real number (estimable)}
\defDefault{1}

\subsubsection{Additional\_Prior of type Log\_Normal}
\commandlabsubarg{Additional\_Prior}{type}{Log\_Normal}.
\defRef{sec:AdditionalPrior-LogNormal}
\label{syntax:AdditionalPrior-LogNormal}

\defSub{mu}{The lognormal prior mean (mu) parameter}
\defType{Real number (estimable)}
\defDefault{No default}
\defLowerBound{0.0 (exclusive)}

\defSub{cv}{The lognormal CV parameter}
\defType{Real number (estimable)}
\defDefault{No default}
\defLowerBound{0.0 (exclusive)}

\subsubsection{Additional\_Prior of type Uniform\_Log}
\commandlabsubarg{Additional\_Prior}{type}{Uniform\_Log}.
\defRef{sec:AdditionalPrior-UniformLog}
\label{syntax:AdditionalPrior-UniformLog}

The Uniform\_Log type has no additional subcommands.
\subsubsection{Additional\_Prior of type Vector\_Average}
\commandlabsubarg{Additional\_Prior}{type}{Vector\_Average}.
\defRef{sec:AdditionalPrior-VectorAverage}
\label{syntax:AdditionalPrior-VectorAverage}

\defSub{method}{Which calculation method to use: k, l, or m}
\defType{String}
\defDefault{k}

\defSub{k}{The k value to use in the calculation}
\defType{Real number (estimable)}
\defDefault{No default}

\defSub{multiplier}{Multiplier for the penalty amount}
\defType{Real number (estimable)}
\defDefault{1}

\subsubsection{Additional\_Prior of type Vector\_Smoothing}
\commandlabsubarg{Additional\_Prior}{type}{Vector\_Smoothing}.
\defRef{sec:AdditionalPrior-VectorSmoothing}
\label{syntax:AdditionalPrior-VectorSmoothing}

\defSub{log\_scale}{Should the sums of squares be calculated on the log scale?}
\defType{Boolean}
\defDefault{false}

\defSub{multiplier}{Multiply the penalty by this factor}
\defType{Real number (estimable)}
\defDefault{1}

\defSub{lower\_bound}{The first element to apply the penalty to in the vector}
\defType{Non-negative integer}
\defDefault{0}

\defSub{upper\_bound}{The last element to apply the penalty to in the vector}
\defType{Non-negative integer}
\defDefault{0}

\defSub{r}{Penalty applied to rth differences}
\defType{Non-negative integer}
\defDefault{2}



\section{Observation command and subcommand syntax\label{sec:observation-syntax}}

%Note: this code auto generated by the iSAM build system. 

\subsection{\I{Observation types}}

The observation types available are,

\begin{description}
  \item Observations of proportions of individuals by age class
  \item Observations of proportions of individuals between categories within each age class
  \item Relative and absolute abundance observations
  \item Relative and absolute biomass observations
\end{description}

Each type of observation requires a set of subcommands and arguments specific to that process.

\defComLab{Observation}{Define an object of type \emph{Observation}}.
\defRef{sec:Observation}
\label{syntax:Observation}

\defSub{label}{The label of the observation}
\defType{String}
\defDefault{No default}

\defSub{type}{The type of observation}
\defType{String}
\defDefault{No default}

\defSub{likelihood}{The type of likelihood to use}
\defType{String}
\defDefault{No default}

\defSub{categories}{The category labels to use}
\defType{Vector of strings}
\defDefault{true}

\defSub{delta}{The robustification value (delta) for the likelihood}
\defType{Real number (estimable)}
\defDefault{utilities::math::DELTA}
\defLowerBound{0.0 (inclusive)}

\defSub{simulation\_likelihood}{The simulation likelihood to use}
\defType{String}
\defDefault{No default}

\defSub{likelihood\_multiplier}{The likelihood score multiplier}
\defType{Real number (estimable)}
\defDefault{1.0}
\defLowerBound{0.0 (inclusive)}

\defSub{error\_value\_multiplier}{The error value multiplier for likelihood}
\defType{Real number (estimable)}
\defDefault{1.0}
\defLowerBound{0.0 (inclusive)}

\subsubsection{Observation of type Abundance}
\commandlabsubarg{Observation}{type}{Abundance}.
\defRef{sec:Observation-Abundance}
\label{syntax:Observation-Abundance}

\defSub{time\_step}{The label of the time step that the observation occurs in}
\defType{String}
\defDefault{No default}

\defSub{catchability}{The label of the catchability coefficient (q)}
\defType{String}
\defDefault{No default}

\defSub{selectivities}{The labels of the selectivities}
\defType{Vector of strings}
\defDefault{true}

\defSub{process\_error}{The process error}
\defType{Real number (estimable)}
\defDefault{0.0}
\defLowerBound{0.0 (inclusive)}

\defSub{years}{The years for which there are observations}
\defType{Vector of non-negative integers}
\defDefault{No default}

\subsubsection{Observation of type Age\_Length}
\commandlabsubarg{Observation}{type}{Age\_Length}.
\defRef{sec:Observation-AgeLength}
\label{syntax:Observation-AgeLength}

\defSub{year}{The year this observation occurs in.}
\defType{Non-negative integer}
\defDefault{No default}

\defSub{time\_step}{The label of the time step that the observation occurs in}
\defType{String}
\defDefault{No default}

\defSub{selectivities}{The labels of the selectivities}
\defType{Vector of strings}
\defDefault{true}

\defSub{process\_errors}{The process error}
\defType{Vector of real numbers (estimable)}
\defDefault{true}

\defSub{ageing\_error}{The label of ageing error to use}
\defType{String}
\defDefault{No default}

\defSub{numerator\_categories}{The categories sampled (used in the numerator for the observation)}
\defType{Vector of strings}
\defDefault{true}

\defSub{sample\_type}{The sampling type used to collect this data}
\defType{String}
\defDefault{length}

\defSub{time\_step\_proportion}{The proportion through the mortality block of the time step when the observation is evaluated}
\defType{Real number (estimable)}
\defDefault{0.5}
\defLowerBound{0.0 (inclusive)}
\defUpperBound{1.0 (inclusive)}

\defSub{ages}{Vector of individual ages}
\defType{Vector of non-negative integers}
\defDefault{false}

\defSub{lengths}{Vector of individual lengths. Has a one to one relationship with ages}
\defType{Vector of real numbers}
\defDefault{false}

\subsubsection{Observation of type Biomass}
\commandlabsubarg{Observation}{type}{Biomass}.
\defRef{sec:Observation-Biomass}
\label{syntax:Observation-Biomass}

\defSub{time\_step}{The label of the time step that the observation occurs in}
\defType{String}
\defDefault{No default}

\defSub{catchability}{The label of the catchability coefficient (q)}
\defType{String}
\defDefault{No default}

\defSub{selectivities}{The labels of the selectivities}
\defType{Vector of strings}
\defDefault{true}

\defSub{process\_error}{The process error}
\defType{Real number (estimable)}
\defDefault{0.0}
\defLowerBound{0.0 (inclusive)}

\defSub{age\_weight\_labels}{The labels for the \command{$age\_weight$} block which corresponds to each category, to use the weight calculation method for biomass calculations)}
\defType{Vector of strings}
\defDefault{No default}

\defSub{years}{The years of the observed values}
\defType{Vector of non-negative integers}
\defDefault{No default}

\subsubsection{Observation of type Process\_Removals\_By\_Age}
\commandlabsubarg{Observation}{type}{Process\_Removals\_By\_Age}.
\defRef{sec:Observation-ProcessRemovalsByAge}
\label{syntax:Observation-ProcessRemovalsByAge}

\defSub{min\_age}{The minimum age}
\defType{Non-negative integer}
\defDefault{No default}

\defSub{max\_age}{The maximum age}
\defType{Non-negative integer}
\defDefault{No default}

\defSub{plus\_group}{Is the maximum age the age plus group}
\defType{Boolean}
\defDefault{true}

\defSub{time\_step}{The label of time-step that the observation occurs in}
\defType{Vector of strings}
\defDefault{No default}

\defSub{years}{The years for which there are observations}
\defType{Vector of non-negative integers}
\defDefault{No default}

\defSub{process\_errors}{The label of process error to use}
\defType{Vector of real numbers (estimable)}
\defDefault{true}

\defSub{ageing\_error}{The label of the ageing error to use}
\defType{String}
\defDefault{No default}

\defSub{method\_of\_removal}{The label of the observed method of removals}
\defType{Vector of strings}
\defDefault{No default}

\defSub{mortality\_process}{The label of the mortality process for the observation}
\defType{String}
\defDefault{No default}

\defSub{simulated\_data\_sum\_to\_one}{Whether simulated data is discrete or scaled by totals to be proportions for each year}
\defType{Boolean}
\defDefault{true}

\defSub{sum\_to\_one}{Scale year (row) observed values by the total, so they sum = 1}
\defType{Boolean}
\defDefault{false}

\subsubsection{Observation of type Process\_Removals\_By\_Age\_Retained}
\commandlabsubarg{Observation}{type}{Process\_Removals\_By\_Age\_Retained}.
\defRef{sec:Observation-ProcessRemovalsByAgeRetained}
\label{syntax:Observation-ProcessRemovalsByAgeRetained}

\defSub{min\_age}{The minimum age}
\defType{Non-negative integer}
\defDefault{No default}

\defSub{max\_age}{The maximum age}
\defType{Non-negative integer}
\defDefault{No default}

\defSub{plus\_group}{Is the maximum age the age plus group?}
\defType{Boolean}
\defDefault{true}

\defSub{time\_step}{The label of the time step that the observation occurs in}
\defType{Vector of strings}
\defDefault{No default}

\defSub{years}{The years for which there are observations}
\defType{Vector of non-negative integers}
\defDefault{No default}

\defSub{process\_errors}{The label of the process error to use}
\defType{Vector of real numbers (estimable)}
\defDefault{true}

\defSub{ageing\_error}{The label of the ageing error to use}
\defType{String}
\defDefault{No default}

\defSub{method\_of\_removal}{The label of observed method of removals}
\defType{Vector of strings}
\defDefault{No default}

\defSub{mortality\_process}{The label of the mortality instantaneous process for the observation}
\defType{String}
\defDefault{No default}

\defSub{simulated\_data\_sum\_to\_one}{Whether simulated data is discrete or scaled by totals to be proportions for each year}
\defType{Boolean}
\defDefault{true}

\defSub{sum\_to\_one}{Scale year (row) observed values by the total, so they sum = 1}
\defType{Boolean}
\defDefault{false}

\subsubsection{Observation of type Process\_Removals\_By\_Age\_Retained\_Total}
\commandlabsubarg{Observation}{type}{Process\_Removals\_By\_Age\_Retained\_Total}.
\defRef{sec:Observation-ProcessRemovalsByAgeRetainedTotal}
\label{syntax:Observation-ProcessRemovalsByAgeRetainedTotal}

\defSub{min\_age}{The minimum age}
\defType{Non-negative integer}
\defDefault{No default}

\defSub{max\_age}{The maximum age}
\defType{Non-negative integer}
\defDefault{No default}

\defSub{plus\_group}{Is the maximum age the age plus group?}
\defType{Boolean}
\defDefault{true}

\defSub{time\_step}{The label of the time step that the observation occurs in}
\defType{Vector of strings}
\defDefault{No default}

\defSub{years}{The years for which there are observations}
\defType{Vector of non-negative integers}
\defDefault{No default}

\defSub{process\_errors}{The label of the process error to use}
\defType{Vector of real numbers (estimable)}
\defDefault{true}

\defSub{ageing\_error}{The label of the ageing error to use}
\defType{String}
\defDefault{No default}

\defSub{method\_of\_removal}{The label of observed method of removals}
\defType{Vector of strings}
\defDefault{No default}

\defSub{mortality\_process}{The label of the mortality instantaneous process for the observation}
\defType{String}
\defDefault{No default}

\defSub{simulated\_data\_sum\_to\_one}{Whether simulated data is discrete or scaled by totals to be proportions for each year}
\defType{Boolean}
\defDefault{true}

\defSub{sum\_to\_one}{Scale year (row) observed values by the total, so they sum = 1}
\defType{Boolean}
\defDefault{false}

\subsubsection{Observation of type Process\_Removals\_By\_Length}
\commandlabsubarg{Observation}{type}{Process\_Removals\_By\_Length}.
\defRef{sec:Observation-ProcessRemovalsByLength}
\label{syntax:Observation-ProcessRemovalsByLength}

\defSub{time\_step}{The time step to execute in}
\defType{Vector of strings}
\defDefault{No default}

\defSub{years}{The years for which there are observations}
\defType{Vector of non-negative integers}
\defDefault{No default}

\defSub{process\_errors}{The process error}
\defType{Vector of real numbers (estimable)}
\defDefault{true}

\defSub{method\_of\_removal}{The label of observed method of removals}
\defType{String}
\defDefault{No default}

\defSub{length\_bins}{The length bins}
\defType{Vector of real numbers}
\defDefault{No default}

\defSub{plus\_group}{Is the last length bin a plus group? (defaults to @model value)}
\defType{Boolean}
\defDefault{model}

\defSub{mortality\_process}{The label of the mortality instantaneous process for the observation}
\defType{String}
\defDefault{No default}

\defSub{simulated\_data\_sum\_to\_one}{Whether simulated data is discrete or scaled by totals to be proportions for each year}
\defType{Boolean}
\defDefault{true}

\defSub{sum\_to\_one}{Scale year (row) observed values by the total, so they sum = 1}
\defType{Boolean}
\defDefault{false}

\subsubsection{Observation of type Process\_Removals\_By\_Length\_Retained}
\commandlabsubarg{Observation}{type}{Process\_Removals\_By\_Length\_Retained}.
\defRef{sec:Observation-ProcessRemovalsByLengthRetained}
\label{syntax:Observation-ProcessRemovalsByLengthRetained}

\defSub{time\_step}{The time step to execute in}
\defType{String}
\defDefault{No default}

\defSub{years}{The years for which there are observations}
\defType{Vector of non-negative integers}
\defDefault{No default}

\defSub{process\_errors}{The process error}
\defType{Vector of real numbers (estimable)}
\defDefault{true}

\defSub{method\_of\_removal}{The label of observed method of removals}
\defType{String}
\defDefault{No default}

\defSub{length\_bins}{The length bins}
\defType{Vector of real numbers}
\defDefault{No default}

\defSub{plus\_group}{Is the last length bin a plus group? (defaults to @model value)}
\defType{Boolean}
\defDefault{model}

\defSub{mortality\_process}{The label of the mortality instantaneous process for the observation}
\defType{String}
\defDefault{No default}

\defSub{simulated\_data\_sum\_to\_one}{Whether simulated data is discrete or scaled by totals to be proportions for each year}
\defType{Boolean}
\defDefault{true}

\defSub{sum\_to\_one}{Scale year (row) observed values by the total, so they sum = 1}
\defType{Boolean}
\defDefault{false}

\subsubsection{Observation of type Process\_Removals\_By\_Length\_Retained\_Total}
\commandlabsubarg{Observation}{type}{Process\_Removals\_By\_Length\_Retained\_Total}.
\defRef{sec:Observation-ProcessRemovalsByLengthRetainedTotal}
\label{syntax:Observation-ProcessRemovalsByLengthRetainedTotal}

\defSub{time\_step}{The time step to execute in}
\defType{String}
\defDefault{No default}

\defSub{years}{The years for which there are observations}
\defType{Vector of non-negative integers}
\defDefault{No default}

\defSub{process\_errors}{The process error}
\defType{Vector of real numbers (estimable)}
\defDefault{true}

\defSub{method\_of\_removal}{The label of observed method of removals}
\defType{String}
\defDefault{No default}

\defSub{length\_bins}{The length bins}
\defType{Vector of real numbers}
\defDefault{No default}

\defSub{plus\_group}{Is the last length bin a plus group? (defaults to @model value)}
\defType{Boolean}
\defDefault{model}

\defSub{mortality\_process}{The label of the mortality instantaneous process for the observation}
\defType{String}
\defDefault{No default}

\defSub{simulated\_data\_sum\_to\_one}{Whether simulated data is discrete or scaled by totals to be proportions for each year}
\defType{Boolean}
\defDefault{true}

\defSub{sum\_to\_one}{Scale year (row) observed values by the total, so they sum = 1}
\defType{Boolean}
\defDefault{false}

\subsubsection{Observation of type Process\_Removals\_By\_Weight}
\commandlabsubarg{Observation}{type}{Process\_Removals\_By\_Weight}.
\defRef{sec:Observation-ProcessRemovalsByWeight}
\label{syntax:Observation-ProcessRemovalsByWeight}

\defSub{mortality\_process}{The label of the mortality instantaneous process for the observation}
\defType{String}
\defDefault{No default}

\defSub{method\_of\_removal}{The label of observed method of removals}
\defType{String}
\defDefault{No default}

\defSub{time\_step}{The time step to execute in}
\defType{String}
\defDefault{No default}

\defSub{years}{The years for which there are observations}
\defType{Vector of non-negative integers}
\defDefault{No default}

\defSub{process\_errors}{The process error}
\defType{Vector of real numbers (estimable)}
\defDefault{true}

\defSub{length\_weight\_cv}{The CV for the length-weight relationship}
\defType{Real number (estimable)}
\defDefault{0.10}
\defLowerBound{0.0 (exclusive)}

\defSub{length\_weight\_distribution}{The distribution of the length-weight relationship}
\defType{String}
\defDefault{normal}

\defSub{length\_bins}{The length bins}
\defType{Vector of real numbers}
\defDefault{No default}

\defSub{length\_bins\_n}{The average number in each length bin}
\defType{Vector of real numbers}
\defDefault{No default}

\defSub{units}{The units for the weight bins (grams, kilograms (kgs), or tonnes)}
\defType{String}
\defDefault{kgs}

\defSub{fishbox\_weight}{The target weight of each box}
\defType{Real number (estimable)}
\defDefault{20.0}
\defLowerBound{0.0 (exclusive)}

\defSub{weight\_bins}{The weight bins}
\defType{Vector of real numbers}
\defDefault{No default}

\subsubsection{Observation of type Proportions\_At\_Age}
\commandlabsubarg{Observation}{type}{Proportions\_At\_Age}.
\defRef{sec:Observation-ProportionsAtAge}
\label{syntax:Observation-ProportionsAtAge}

\defSub{min\_age}{The minimum age}
\defType{Non-negative integer}
\defDefault{No default}

\defSub{max\_age}{The maximum age}
\defType{Non-negative integer}
\defDefault{No default}

\defSub{plus\_group}{Is the maximum age the age plus group?}
\defType{Boolean}
\defDefault{true}

\defSub{time\_step}{The label of the time step that the observation occurs in}
\defType{String}
\defDefault{No default}

\defSub{years}{The years of the observed values}
\defType{Vector of non-negative integers}
\defDefault{No default}

\defSub{selectivities}{The labels of the selectivities}
\defType{Vector of strings}
\defDefault{true}

\defSub{process\_errors}{The process error}
\defType{Vector of real numbers (estimable)}
\defDefault{true}

\defSub{ageing\_error}{The label of ageing error to use}
\defType{String}
\defDefault{No default}

\defSub{simulated\_data\_sum\_to\_one}{Whether simulated data is discrete or scaled by totals to be proportions for each year}
\defType{Boolean}
\defDefault{true}

\defSub{sum\_to\_one}{Scale year (row) observed values by the total, so they sum = 1}
\defType{Boolean}
\defDefault{false}

\subsubsection{Observation of type Proportions\_At\_Length}
\commandlabsubarg{Observation}{type}{Proportions\_At\_Length}.
\defRef{sec:Observation-ProportionsAtLength}
\label{syntax:Observation-ProportionsAtLength}

\defSub{time\_step}{The label of the time step that the observation occurs in}
\defType{String}
\defDefault{No default}

\defSub{years}{The years for which there are observations}
\defType{Vector of non-negative integers}
\defDefault{No default}

\defSub{selectivities}{The labels of the selectivities}
\defType{Vector of strings}
\defDefault{true}

\defSub{process\_errors}{The process error}
\defType{Vector of real numbers (estimable)}
\defDefault{true}

\defSub{length\_bins}{The length bins}
\defType{Vector of real numbers}
\defDefault{true}

\defSub{sum\_to\_one}{Scale year (row) observed values by the total, so they sum = 1}
\defType{Boolean}
\defDefault{false}

\subsubsection{Observation of type Proportions\_By\_Category}
\commandlabsubarg{Observation}{type}{Proportions\_By\_Category}.
\defRef{sec:Observation-ProportionsByCategory}
\label{syntax:Observation-ProportionsByCategory}

\defSub{min\_age}{The minimum age}
\defType{Non-negative integer}
\defDefault{No default}

\defSub{max\_age}{The maximum age}
\defType{Non-negative integer}
\defDefault{No default}

\defSub{time\_step}{The label of the time step that the observation occurs in}
\defType{String}
\defDefault{No default}

\defSub{plus\_group}{Use the age plus group?}
\defType{Boolean}
\defDefault{true}

\defSub{years}{The years for which there are observations}
\defType{Vector of non-negative integers}
\defDefault{No default}

\defSub{selectivities}{The labels of the selectivities}
\defType{Vector of strings}
\defDefault{true}

\defSub{categories2}{The target categories}
\defType{Vector of strings}
\defDefault{No default}

\defSub{selectivities2}{The target selectivities}
\defType{Vector of strings}
\defDefault{No default}

\subsubsection{Observation of type Proportions\_Mature\_By\_Age}
\commandlabsubarg{Observation}{type}{Proportions\_Mature\_By\_Age}.
\defRef{sec:Observation-ProportionsMatureByAge}
\label{syntax:Observation-ProportionsMatureByAge}

\defSub{min\_age}{The minimum age}
\defType{Non-negative integer}
\defDefault{No default}

\defSub{max\_age}{The maximum age}
\defType{Non-negative integer}
\defDefault{No default}

\defSub{time\_step}{The label of time-step that the observation occurs in}
\defType{String}
\defDefault{No default}

\defSub{plus\_group}{Use the age plus group?}
\defType{Boolean}
\defDefault{true}

\defSub{years}{The years for which there are observations}
\defType{Vector of non-negative integers}
\defDefault{No default}

\defSub{ageing\_error}{The label of ageing error to use}
\defType{String}
\defDefault{No default}

\defSub{total\_categories}{All category labels that were vulnerable to sampling at the time of this observation (not including the categories already given)}
\defType{Vector of strings}
\defDefault{true}

\defSub{time\_step\_proportion}{The proportion through the mortality block of the time step when the observation is evaluated}
\defType{Real number (estimable)}
\defDefault{0.5}
\defLowerBound{0.0 (inclusive)}
\defUpperBound{1.0 (inclusive)}

\subsubsection{Observation of type Proportions\_Migrating}
\commandlabsubarg{Observation}{type}{Proportions\_Migrating}.
\defRef{sec:Observation-ProportionsMigrating}
\label{syntax:Observation-ProportionsMigrating}

\defSub{min\_age}{The minimum age}
\defType{Non-negative integer}
\defDefault{No default}

\defSub{max\_age}{The maximum age}
\defType{Non-negative integer}
\defDefault{No default}

\defSub{time\_step}{The label of the time step that the observation occurs in}
\defType{String}
\defDefault{No default}

\defSub{plus\_group}{Is the maximum age the age plus group?}
\defType{Boolean}
\defDefault{true}

\defSub{years}{The years for which there are observations}
\defType{Vector of non-negative integers}
\defDefault{No default}

\defSub{process\_errors}{The process error}
\defType{Vector of real numbers (estimable)}
\defDefault{true}

\defSub{ageing\_error}{The label of the ageing error to use}
\defType{String}
\defDefault{No default}

\defSub{process}{The process label}
\defType{String}
\defDefault{No default}

\subsubsection{Observation of type Tag\_Recapture\_By\_Age}
\commandlabsubarg{Observation}{type}{Tag\_Recapture\_By\_Age}.
\defRef{sec:Observation-TagRecaptureByAge}
\label{syntax:Observation-TagRecaptureByAge}

\defSub{max\_age}{The maximum age}
\defType{Non-negative integer}
\defDefault{No default}

\defSub{plus\_group}{Is the maximum age the age plus group?}
\defType{Boolean}
\defDefault{true}

\defSub{years}{The years for which there are observations}
\defType{Vector of non-negative integers}
\defDefault{No default}

\defSub{selectivities}{The labels of the selectivities of the scanned categories}
\defType{Vector of strings}
\defDefault{true}

\defSub{tagged\_categories}{The categories of tagged individuals}
\defType{Vector of strings}
\defDefault{No default}

\defSub{tagged\_selectivities}{The labels of the selectivities of the tagged categories}
\defType{Vector of strings}
\defDefault{No default}

\defSub{time\_step}{The label of the time step that the observation occurs in}
\defType{String}
\defDefault{No default}

\defSub{detection}{The probability of detecting a recaptured individual}
\defType{Real number (estimable)}
\defDefault{No default}
\defLowerBound{0.0 (inclusive)}
\defUpperBound{1.0 (inclusive)}

\defSub{dispersion}{The overdispersion parameter (phi)}
\defType{Vector of real numbers (estimable)}
\defDefault{true}
\defLowerBound{0.0 (inclusive)}

\defSub{time\_step\_proportion}{The proportion through the mortality block of the time step when the observation is evaluated}
\defType{Real number (estimable)}
\defDefault{0.5}
\defLowerBound{0.0 (inclusive)}
\defUpperBound{1.0 (inclusive)}

\subsubsection{Observation of type Tag\_Recapture\_By\_Fishery}
\commandlabsubarg{Observation}{type}{Tag\_Recapture\_By\_Fishery}.
\defRef{sec:Observation-TagRecaptureByFishery}
\label{syntax:Observation-TagRecaptureByFishery}

\defSub{years}{The years for which there are observations}
\defType{Vector of non-negative integers}
\defDefault{No default}

\defSub{tagged\_categories}{The categories of tagged individuals for the observation}
\defType{Vector of strings}
\defDefault{No default}

\defSub{time\_step}{The label of the time step that the observation occurs in}
\defType{Vector of strings}
\defDefault{No default}

\defSub{method\_of\_removal}{The label of the observed method of removals}
\defType{Vector of strings}
\defDefault{No default}

\defSub{mortality\_process}{The label of the mortality process for the observation}
\defType{String}
\defDefault{No default}

\defSub{reporting\_rate}{The probability of detecting a recaptured individual}
\defType{Real number (estimable)}
\defDefault{No default}
\defLowerBound{0.0 (inclusive)}
\defUpperBound{1.0 (inclusive)}

\subsubsection{Observation of type Tag\_Recapture\_By\_Length}
\commandlabsubarg{Observation}{type}{Tag\_Recapture\_By\_Length}.
\defRef{sec:Observation-TagRecaptureByLength}
\label{syntax:Observation-TagRecaptureByLength}

\defSub{time\_step}{The time step to execute in}
\defType{String}
\defDefault{No default}

\defSub{length\_bins}{The length bins}
\defType{Vector of real numbers}
\defDefault{true}

\defSub{selectivities}{The labels of the selectivities used for untagged categories}
\defType{Vector of strings}
\defDefault{true}

\defSub{tagged\_selectivities}{The labels of the tag category selectivities}
\defType{Vector of strings}
\defDefault{No default}

\defSub{detection}{The probability of detecting a recaptured individual}
\defType{Real number (estimable)}
\defDefault{No default}
\defLowerBound{0.0 (inclusive)}
\defUpperBound{1.0 (inclusive)}

\defSub{dispersion}{The overdispersion parameter (phi)}
\defType{Vector of real numbers (estimable)}
\defDefault{true}
\defLowerBound{0.0 (inclusive)}

\defSub{time\_step\_proportion}{The proportion through the mortality block of the time step when the observation is evaluated}
\defType{Real number (estimable)}
\defDefault{0.5}
\defLowerBound{0.0 (inclusive)}
\defUpperBound{1.0 (inclusive)}



\subsection{\I{Likelihoods}}
\defComLab{Likelihood}{Define an object of type \emph{Likelihood}}.
\defRef{sec:Likelihood}
\label{syntax:Likelihood}

\subsubsection{Likelihood of type Bernoulli}
\commandlabsubarg{Likelihood}{type}{Bernoulli}.
\defRef{sec:Likelihood-Bernoulli}
\label{syntax:Likelihood-Bernoulli}

The Bernoulli type has no additional subcommands.
\subsubsection{Likelihood of type Binomial}
\commandlabsubarg{Likelihood}{type}{Binomial}.
\defRef{sec:Likelihood-Binomial}
\label{syntax:Likelihood-Binomial}

The Binomial type has no additional subcommands.
\subsubsection{Likelihood of type Binomial\_Approx}
\commandlabsubarg{Likelihood}{type}{Binomial\_Approx}.
\defRef{sec:Likelihood-BinomialApprox}
\label{syntax:Likelihood-BinomialApprox}

The Binomial\_Approx type has no additional subcommands.
\subsubsection{Likelihood of type Dirichlet}
\commandlabsubarg{Likelihood}{type}{Dirichlet}.
\defRef{sec:Likelihood-Dirichlet}
\label{syntax:Likelihood-Dirichlet}

The Dirichlet type has no additional subcommands.
\subsubsection{Likelihood of type Dirichlet\_Multinomial}
\commandlabsubarg{Likelihood}{type}{Dirichlet\_Multinomial}.
\defRef{sec:Likelihood-DirichletMultinomial}
\label{syntax:Likelihood-DirichletMultinomial}

\defSub{label}{Label of the Dirichlet-multinomial distribution}
\defType{String}
\defDefault{No default}

\defSub{type}{Type of likelihood}
\defType{String}
\defDefault{No default}

\defSub{theta}{Theta parameter (account for overdispersion)}
\defType{Real number (estimable)}
\defDefault{No default}
\defLowerBound{0.0 (inclusive)}

\subsubsection{Likelihood of type Log\_Normal}
\commandlabsubarg{Likelihood}{type}{Log\_Normal}.
\defRef{sec:Likelihood-LogNormal}
\label{syntax:Likelihood-LogNormal}

The Log\_Normal type has no additional subcommands.
\subsubsection{Likelihood of type Log\_Normal\_With\_Q}
\commandlabsubarg{Likelihood}{type}{Log\_Normal\_With\_Q}.
\defRef{sec:Likelihood-LogNormalWithQ}
\label{syntax:Likelihood-LogNormalWithQ}

The Log\_Normal\_With\_Q type has no additional subcommands.
\subsubsection{Likelihood of type Multinomial}
\commandlabsubarg{Likelihood}{type}{Multinomial}.
\defRef{sec:Likelihood-Multinomial}
\label{syntax:Likelihood-Multinomial}

The Multinomial type has no additional subcommands.
\subsubsection{Likelihood of type Normal}
\commandlabsubarg{Likelihood}{type}{Normal}.
\defRef{sec:Likelihood-Normal}
\label{syntax:Likelihood-Normal}

The Normal type has no additional subcommands.
\subsubsection{Likelihood of type Poisson}
\commandlabsubarg{Likelihood}{type}{Poisson}.
\defRef{sec:Likelihood-Poisson}
\label{syntax:Likelihood-Poisson}

The Poisson type has no additional subcommands.
\subsubsection{Likelihood of type Pseudo}
\commandlabsubarg{Likelihood}{type}{Pseudo}.
\defRef{sec:Likelihood-Pseudo}
\label{syntax:Likelihood-Pseudo}

The Pseudo type has no additional subcommands.


\subsection{\I{Defining ageing error}}

Three methods for including ageing error into estimation with observations are,

\begin{itemize}
	\item None
	\item Normal
	\item Off-by-one
\end{itemize}

Each type of ageing error requires a set of subcommands and arguments specific to its type.

\defComLab{ageing\_error}{Define an object type Ageing\_Error}

\defSub{label} {Label}
\defType{string}
\defDefault{No default}

\defSub{type} {Type}
\defType{string}
\defDefault{No default}

\subsubsection[Normal]{\commandlabsubarg{ageing\_error}{type}{normal}}

\defSub{cv} {TBA}
\defType{estimable}
\defDefault{No default}

\defSub{k} {TBA}
\defType{non-negative integer}
\defDefault{0u}

\subsubsection[Off By One]{\commandlabsubarg{ageing\_error}{type}{off\_by\_one}}



\section{Report command and subcommand syntax\label{sec:report-syntax}}

\subsection{\I{Available reports}}

The report types available are,

\begin{enumerate}
  \item 
  \item
  \item
  \item
  \item
  \item
  \item
  \item
\end{enumerate}

Each type of report requires a set of subcommands and arguments specific to that report.

\subsection{\I{Report commands and subcommands}}

\section{\I{The report section: output and reports}\label{sec:Report}}\index{Reports}\index{Reports section}

The command and subcommand syntax for the estimation section is given in Section \ref{syntax:Report}.

The report section specifies the printouts and other output from the model. \CNAME\ does not, in general, produce any output unless specified by a valid \command{report} block.

\subsection{Report command block format}

Reports from \CNAME\ can be defined to print partition and states objects at a particular point in time, observation summaries, estimated and derived parameter values, and objective function values. Many of the reports below will print a specific \command{}. The most useful report is the default report (see Section~\ref{sec:Report-Default}) which will auto generate many of the reports users require.

\begin{verbatim}
		@report default
		type default
		selectivities true
		derived_quantities true
		observations true
		processes true
		catchabilities true
		time_varying true
		parameter_transformations true
		time_varying true
		
		
		@report observation_age ## label of report
		type observation		    ## Type of report
		observation age_1990	 ## label corresponding to an @observation report, shown below

		@observation age_1990
		type proportion_at_age
		year 1990
		plus_group
		etc., ...
\end{verbatim}



\subsection{Report output format}

Reports from \CNAME\ have a standard style\index{Reports ! standard style} (with the exception of \texttt{output\_parameters} and \texttt{simulated\_observation}, see below). The standard style is that reports are prefixed with an asterisk followed by a user-defined label and type of report in brackets (e.g., \texttt{*label (type)}), with the report ending with the line \texttt{*end}. For example,

\begin{verbatim}
*My_report(type)
...
... # report content
...
*end
\end{verbatim}

This report block output format should make it easier for other software packages to read and process \CNAME\ output. The \texttt{extract} functions in the \R\ \CNAME\ package use this information to identify and read \CNAME\ output.

The \texttt{output\_parameters} report does not print either a header or \texttt{*end} at the end of the report block. This is because the \texttt{output\_parameters} report is designed to provide a single line vector of the estimated parameter values, or multiple lines for more than one set, which can be read by \CNAME\ with the command \texttt{casal2 -i}. This is a specialised report for the \texttt{casal2 -o filename} command.

For estimated values in standard output use the \texttt{type=estimate\_value} report.

Reports can be defined in a \command{report} command block but may not be output, e.g., a report to print the partition for a year and/or time step that does not exist, or reporting the covariance matrix when not estimation run mode.

Certain reports are associated with certain \CNAME\ run modes. These reports are ignored by \CNAME\ and the program will not generate any output for these reports, although they must still conform to \CNAME\ syntax requirements.

Not all reports will be generated in all run modes. Some reports are only available in some run modes. For example, when simulating, only the simulation reports will be output.

\subsection{\I{Print default reports}}\index{Reports ! Default report}\label{sec:Report-Default}

This is a report type that generates a range of other default reports, The report queries model and generates default reports for all catchabilities, observations, processes, selectivities, derived parameters, time-varying parameters, parameter transformations, and projections. This report will print out in run modes \texttt{-r, -e, -f}.


\begin{verbatim}
@report default
type default
selectivities true
derived_quantities true
observations true
processes true
catchabilities true
time_varying true
parameter_transformations true
time_varying true
\end{verbatim}




\subsection{\I{Print the partition at the end of an initialisation}}\index{Reports ! Initialisation}\label{sec:Report-InitialisationPartition}

This report prints the partition following the initialisation phase, which includes the numbers of individuals in each age class and category in the partition. This report will print out in run modes \texttt{-r, -e, -f}.

\subsection{\I{Print the partition}}\index{Reports ! Partition}\label{sec:Report-Partition}

This report prints the numbers of individuals in each age class and category in the partition for each given year or given years and time step. This report is evaluated at the end of the time step in the given year(s). This report will print out in run modes \texttt{-r, -e, -f}.

\subsection{\I{Print the partition biomass}}\index{Reports ! Partition biomass}\label{sec:Report-PartitionBiomass}

This report prints the biomass in each age class and category in the partition for each given year or given years and time step. This report is evaluated at the end of the time step in the given year(s). This report will print out in run modes \texttt{-r, -e, -f}.
\ifAgeBased
% One column version

\subsection{\I{Print the age length and length weight values}}\index{Reports ! Age-length}\label{sec:Report-AgeLength}\label{sec:Report-LenthWeight}

This report prints the length and weight value for each age class and category in the partition for each given year or given years and time step. This report is evaluated at the end of the time step in the given year(s). This report will print out in run modes \texttt{-r, -e, -f}.

\begin{verbatim}
@report length_weight_at_age
type partition_mean_weight
time_step step2
years 1900:2013
\end{verbatim}

\subsection{\I{Print the ageing error misclassification matrix}}\index{Reports ! Ageing error misclassification matrix}\label{sec:Report-AgeingErrorMatrix}

This report prints the ageing error misclassification matrix used to offset observations within during model the model fitting procedure.


\else
% Two column version

\subsection{\I{Print the Growth models}}\index{Reports ! Growth-Increment}\label{sec:Report-GrowthIncrement}

This report prints the length and weight relationship as well as growth increment model used to move fish along the length partition. This will print the growth transition matrix and other important values. This report is evaluated at the end of the time step in the given year(s). This report will print out in run modes \texttt{-r, -e, -f}.

\begin{verbatim}
@report growth_increment_model
type growth_increment
time_step step2
years 1900:2013
growth_increment growth_model
\end{verbatim}

\fi % end if

\subsection{\I{Print a parameter Transformations}}\index{Reports ! Parameter Transformations}\label{sec:Report-ParameterTransformations}

This report prints a specific \command{parameter\_transformation} block with the values This report will print out in run modes \texttt{-r, -e, -m}. If you have many transformations it is best to report these using the default report see Section~\ref{sec:Report-Default}

{\small{\begin{verbatim}
		@report log_b0
		type parameter_transformation
		parameter_transformation log_b0
		\end{verbatim}}}

\subsection{\I{Print a process summary}}\index{Reports ! Processes}\label{sec:Report-Process}

Depending on the process, different summaries are produced. These reports typically detail the type of process, its parameters and other options, and any associated details. This report will print out in run modes \texttt{-r, -e, -f}.

\subsection{\I{Print derived quantities}}\index{Reports ! Derived quantities}\label{sec:Report-DerivedQuantity}

This report prints the description of the derived quantity, and the values of the derived quantity as recorded in the model state, for each year of the model, and for all years in the initialisation phase. This report will print out in run modes \texttt{-r, -e, -f}.

\subsection{\I{Print the estimated parameters}}\index{Reports ! Estimated parameters}\label{sec:Report-EstimateSummary}

This report prints a summary of the estimated parameters using the type \texttt{estimate\_summary}, including the parameter name, lower and upper bounds, the label of the prior, and its value. This report will print out in run modes \texttt{-r, -e}.

\subsection{\I{Print the MPD (the free parameters and covariance matrix)}}\index{Reports ! MPD}\label{sec:Report-MPD}

This report prints the estimated parameter values out as a vector along with the covariance matrix. The \texttt{MPD} report prints these in a format suitable for use as the starting point for an MCMC. This report will print out in run modes \texttt{-r, -e}.

\subsection{\I{Print the estimate values (the free parameters in the free parameter file format)}}\index{Reports ! Estimated parameters}\label{sec:Report-EstimateValue}

This report prints the estimated parameter values out as a vector. The \texttt{estimate\_values} report prints the name of the parameter, followed by the value for that run. This report will print out in run modes \texttt{-r, -e}.

\subsection{\I{Print the objective function}}\index{Reports ! Objective function}\label{sec:Report-ObjectiveFunction}

This report prints the total objective function value, the value of all observation likelihood components, the values of all priors, and the value of any penalties that have been incurred. If an individual model run does not incur a penalty, then the penalty will not be reported. This report will print out in run modes \texttt{-r, -e, -f}.

\subsection{\I{Print the covariance matrix}}\index{Reports ! Covariance Matrix}\label{sec:Report-CovarianceMatrix}

This report prints the covariance matrix if in estimation run mode and if the covariance has been requested by \commandlabsubarg{minimiser}{covariance}{true}.

\subsection{\I{Print the correlation matrix}}\index{Reports ! Corelation Matrix}\label{sec:Report-CorrelationMatrix}

This report prints the correlation matrix if in estimation run mode and if the covariance has been requested by \commandlabsubarg{minimiser}{covariance}{true}.

\subsection{\I{Print the Hessian matrix}}\index{Reports ! Hessian}\label{sec:Report-HessianMatrix}

This report prints the Hessian matrix if in estimation run mode and if the covariance has been requested by \commandlabsubarg{minimiser}{covariance}{true}.

\subsection{\I{Print the catchability values}}\index{Reports ! Catchability}\label{sec:Report-Catchability}

This report prints the catchability for a requested catchability.

\subsection{\I{Print observations, fits, and residuals}}\index{Reports ! Observations}\label{sec:Report-Observation}

This report prints, for each category or combination of categories, the expected values, residuals (observed $-$ expected), the error value, process error, the total error (i.e., the error value as modified by any additional process error), and the contribution to the total objective function of that individual datum in the observation.

Constants in the likelihood components are often ignored in the objective function score of individual observation values. Hence, the total score from an observation equals the contribution of the objective function scores from each individual observation value plus a constant term (if applicable). In likelihood components without a constant term, the total score from an observation will equal the contribution of the objective function scores from each individual observation value.

If \CNAME\ is in simulation run mode, then the contribution to the objective function of each observation is reported as zero.

\begin{verbatim}
		@report Tan_at_age_obs
		type observation
		observation TAN_AT_AGE
\end{verbatim}

\subsection{\I{Print simulated observations}}\index{Reports ! Simulated observations}\label{sec:Report-SimulatedObservation}

This report prints a complete set of observation values in the form specified by \commandlabsubarg{report}{type}{observation}, with observed values replaced by randomly generated simulated values. The output is in a form  suitable for use within a \CNAME\ \config, reproducing the command and subcommands from the \config. This report will print out in run mode \texttt{-s}.


Simulated reports will be produced with the following extension \texttt{.1\_1}. The first number of the extension relates to the row of the \texttt{-i}/\texttt{-I} file and the second number (separated by \texttt{\_}) represents the simulation iteration defined by the \texttt{n} argument in the configuration input \texttt{casal2 -s n}. Examples of the extension follow,
\begin{itemize}
	\item \texttt{.1\_1} indicates simulated data produced from the first row of parameters and is the first random draw
	\item \texttt{.1\_2} indicates simulated data produced from the first row of parameters and is the second random draw
	\item \texttt{.2\_10} indicates simulated data produced from the second row of parameters and is the 10\(^{th}\) random draw
\end{itemize}
\subsection{\I{Print selectivities}}\index{Reports ! Selectivities}\label{sec:Report-Selectivity}

This report prints the values of a selectivity for each age in the partition, for a given year and at then end of a given time step.

\subsection{\I{Print selectivities by year}}\index{Reports ! Selectivities By Year}\label{sec:Report-SelectivityByYear}

This report prints the values of a selectivity for each partition member for given model years. Useful when you have a model with time-varying selectivity.


\subsection{\I{Print the random number seed}}\index{Reports ! Random number seed}\label{sec:Report-RandomNumberSeed}

This report prints the random number seed used by \CNAME\ to initialise the generated random number sequence. Additional runs which use the same random number seed and the same model will produce identical outputs.

\subsection{\I{Print the results of an MCMC}}\index{Reports ! MCMC}\label{sec:Report-MCMC}

This report prints the MCMC samples, objective function values, and proposal covariance matrix following an MCMC. This report will print out in run mode \texttt{-m}.

\subsection{\I{Print the MCMC samples as they are calculated}}\index{Reports ! MCMC samples}\label{sec:Report-MCMCSample}

This report prints the MCMC samples for each new \textit{i}th sample as they are calculated while doing an MCMC. The output file will be appended with each new sample as it is calculated by \CNAME. This report will print out in run mode \texttt{-m}.

\subsection{\I{Print the MCMC objective function values as they are calculated}}\index{Reports ! MCMC objective functions}\label{sec:Report-MCMCObjective}

This report prints the MCMC objective function values, along with the proposal covariance matrix, for each new \textit{i}th sample as they are calculated while doing an MCMC. The output file will be appended with each new set of objective function values as it is calculated by \CNAME. This report will print out in run mode \texttt{-m}.

\subsection{\I{Print time varying parameters}}\index{Reports ! time varying}\label{sec:Report-TimeVarying}

This report prints all \command{time\_varying} blocks with the values and years in which they were specified. This report will print out in run modes \texttt{-r, -e, -m}.

{\small{\begin{verbatim}
		@report time_varying_parameters
		type time_varying
		\end{verbatim}}}

\subsection{\I{Tabular reporting format}}\index{Reports ! Tabular}\label{sec:Tabular}

An alternative reporting framework to the standard output is the tabular reporting format. Tabular reporting is used with multi-line \texttt{-i} input files (like the MCMC sample or \texttt{-o} outputs). Tabular reports will print out a row that will correspond with each row of the \texttt{-i} input files.

Tabular reporting is specified using the \texttt{-{}-tabular} argument (\texttt{casal2 -r -{}-tabular -i file\_name}).

Derived quantities, processes, observations, and \texttt{estimate\_values} are the only report types that can be output with this format. For each input file the output will begin with the names of each column followed by a multi-line report ending with the \texttt{*end} syntax.

These tables can be read with \R\ using the \CNAME\ \R\ package. An example usage is reading in files of MCMC posterior values of derived quantities, which can then be plotted.




\section{Other commands and subcommands\label{sec:general-syntax}}

\defComArg{include}{file}{\I{Include an external file}}

\defArg{file}{The name of the external file to include}
\defType{string}
\defDefault{No default}
\defValue{A valid external file}
\defCondition{The file name must be enclosed in double quotes}
\defExample{\command{include} \argument{\ "my\_file.txt"}}
\defNote{\command{include} does not denote the end of the previous command block as is the case for all other commands}
