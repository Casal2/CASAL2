\section{Syntax of a \CNAME\ file}\label{Sec:stru}

The general structure of \CNAME\ is blocks of subcommands to specify model, population, partition, and observation characteristics. A block always starts with \command{} symbol. Blocks describe different aspects of the model. Fundamental blocks to define in a model are

\begin{itemize}
	\item \command{model},
	\item \command{initialisation\_phase},
	\item \command{categories},
	\item \command{time\_step}, and
	\item \command{process}.
\end{itemize}

Each block has subcommands. Some subcommands will be optional and others subcommands will be required. An example of the \command{model} block:

{\small{\begin{verbatim}
		@model
		type age		## the model is age-based
		min_age 1		## the minimum age in the model
		max_age 17		## the maximum age in the model
		age_plus true	## is the last age group a plus group?
		start_year 1972	## the first year of the model
		final_year 2013 ## the last year of the model
		initialisation_phases phase1 ## label for the block @initialisation_phase
		time_steps step1 step2 ## labels for the block @time_step
		\end{verbatim}}}

The subcommands are all of the options that are part of the command \command{model}. The value for a specific subcommand is after the subcommand keyword, e.g., \texttt{min\_age} specifies the minimum age in the model and is set to $1$; this value can be any positive integer.

Different subcommands can take different types of values, which can be of type integer, floating point, string, or vector. The syntax section of the manual has information about which value type a subcommand expects. If the wrong type is used for a subcommand, for example \texttt{min\_age 1.5}, an error will be output.

Comments are indicated by starting a line with \texttt{\#}; comment lines are ignored by \CNAME. To comment out multiple lines the C++ comment syntax of \textbackslash* and *\textbackslash can be used, so that everything between these symbols will be ignored by \CNAME. This feature can be a useful tool for annotating model configuration files. 

