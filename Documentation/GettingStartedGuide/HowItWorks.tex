\section{How to run \CNAME}\label{sec:intro}

\CNAME\ is run from a console windows on Microsoft Windows, or in a terminal window on Linux. \CNAME\ is executed by typing \texttt{casal2 -[flag]}, where \texttt{flag} specifies the run type for \CNAME. When \CNAME\ is run with a specific flag, \CNAME\ reads in text files. These text files define the model structure and output.

For help on the flags available and their descriptions, type \texttt{casal2 -{}-help} or \texttt{casal2 -h}, which will print a list of flags and their usage.

\CNAME\ can be run in multiple modes. The modes and their corresponding flags include

\begin{itemize}
	\item deterministic run: \texttt{-r},
	\item parameter estimation: \texttt{-e},
	\item parameter profiling: \texttt{-p},
	\item MCMC runs: \texttt{-m}, and
	\item projections: \texttt{-f}.
\end{itemize}

There are two ways of printing output. The default configuration is to print all output to the screen, and the second option is to direct all output to a file. The second option is usually preferred for post-processing output, e.g., creating plots.

This invocation of \CNAME\ reads in a text file that defines the model configuration (\texttt{My\_model\_config.csl2}) and estimates the parameters in that model, then prints the output to the file \texttt{output.txt}:

\texttt{casal2 -e -c My\_model\_config.csl2 > output.txt}

The flag \texttt{-e} specifies the estimation run mode, the flag \texttt{-c} specifies the model configuration file to read in, and \texttt{>} directs the output to the file. If the flag \texttt{-c} and a model configuration file name are not specified, then \CNAME\ will (try to) read in the default model configuration file \texttt{config.csl2} in the current directory.

The main run modes for \CNAME:

\begin{itemize}
	
	\item \texttt{casal2 -r} will run the model from \texttt{start\_year} to \texttt{final\_year} with the parameters in the configuration file, or will use parameters specified using the \texttt{-i} functionalty.
	
	\item \texttt{casal2 -e} will run the model in \texttt{-r} many times trying to solve for the global optimum.
	
	\item \texttt{casal2 -f 1} is the projection mode and will run the model from \texttt{start\_year} to \texttt{projection\_final\_year} applying any \command{project} functionality. Since much of the \command{project} functionality is stochastic, many projections can be performed for a given set of parameters. For example, \texttt{casal2 -f 50} will generate 50 projection runs. If there are multiple candidate parameters (perhaps from an MCMC), \texttt{casal2 -f 50 -i mcmc\_params.out} will generate 50 projections for each set, which will propagate more uncertainty into the projections.
	
	\item \texttt{casal2 -s 1} is the simulation mode, which will run from \texttt{start\_year} to \texttt{final\_year}. It will calculate the expected values for all \command{observation} blocks which will then have random error applied to generate simulated data. Report commands can be specified that will generate observation files for re-estimation.
	
	\item \texttt{casal2 -m} is the MCMC mode and will run a Markov Chain Monte Carlo chain defined by subcommands in the \command{mcmc}.
\end{itemize}
