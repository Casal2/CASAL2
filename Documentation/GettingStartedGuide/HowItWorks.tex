\section{How it works}\label{sec:intro}

\CNAME\ is run from a console windows in Microsoft Windows or in a terminal window in Linux, \CNAME\ is executed by typing \texttt{casal2 -parameter}, where \texttt{parameter} defines the run type of \CNAME. Once a \CNAME\ has been executed with a certain parameter \CNAME\ reads in text files. These text files define the model structure and the output wanted. For help on the parameters available and there descriptions type \texttt{casal2 --help}, this will print a help screen. There are multiple modes that \CNAME\ can be run in, these are specified using the following command statement the program name followed by the mode parameter (e.g.\CNAME\ \texttt{ -parameter}). The modes and corresponding parameters include deterministic run \texttt{-r}, parameter estimation \texttt{-e}, parameter profiling \texttt{-p}, mcmc runs \texttt{-m}, and projections \texttt{-f}. There are two ways of printing output, the default is to print all output to screen, the second is to print output to a file. The second is usually the preferred if you intend on post processing output i.e. create plots. The following example shows how to read in text file that out model is configured in (\texttt{My\_model.txt}) and run an estimation on some parameters in that model, then print the output to a file named \texttt{output.txt}.\\
\CNAME\ \texttt{-e  -c My\_model.txt > output.txt}

\CNAME\ calls the program, \texttt{-e} tells the program it is going to do an estimation. \texttt{-c} is the parameter that gives the name of the text file with the configured model is, and \texttt{>} is the command to specify the file name where the output is printed.
