\section{Structure of a model}\label{Sec:model}

When setting up a model, it may be helpful to separate the model configuration into different files and sections, e.g.,:

\begin{enumerate}
\item population, for the model and population characteristics
\item estimation, for the parameter and penalty characteristics
\item observation, for the fishery and survey data
\item report, for the specifications for the model output
\end{enumerate}

The population file defines the categories that make up the partition, the time steps in the annual cycle and processes that occur to the partition in each time step, along with the parameters that control those processes.

The estimation file defines the estimated parameters and the prior distribution associated with parameters or processes.

The observation file defines the observations and their assumed error structure through the likelihood types which contribute to the objective function.

The report file defines the output at the end of model runs, e.g., the objective scores, residuals, derived values, and the state of the partition at a point in time.

An example \texttt{config.csl2} file may have the following commands:\\
\texttt{!include "population.csl2"}\\
\texttt{!include "estimation.csl2"}\\
\texttt{!include "observation.csl2"}\\
\texttt{!include "report.csl2"}

The \texttt{!include} function looks in the current directory for the specified filename, and reads in all of the files included in the configuration file as one model.

See the Simple example in the \texttt{Examples} subdirectory.
