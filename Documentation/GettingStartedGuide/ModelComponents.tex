\section{Components of a model}\label{Sec:comp}

Components of a model that are important to know before setting up a \CNAME\ model are

\begin{itemize}
	\item how many categories are in the partition,
	\item what processes occur to which categories in which order,
	\item where observations fit in to the model, and
	\item what the assumed state of the partition is before the model years run.
\end{itemize}

\CNAME\ runs in annual cycles, and each year is split up into time steps. Time steps are used to specify when processes such as fishing and spawning seasons occur, as well as observations such as annual surveys.

Identifying the partition is an important part of model development, and will be determined by the information available. Categories in the partition should be included if they have different biological characteristics or life history traits.

Examples why a model may have multiple categories in the partition:

\begin{itemize}
	\item differences in growth among subgroups, e.g., sex- or area-specific growth
	\item spatial or temporal differences in exploitation, e.g., a fishery may target characteristics that are associated with some subgroups
	\item disproportionate recruitment (60\% males 40\% females)
	\item tracking maturity
	\item tracking tagged fish (this process is related to observations)
\end{itemize}

