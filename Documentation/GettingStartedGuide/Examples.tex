\section{Examples}\label{Sec:examples}

\subsection{Simple Example}\label{Sec:simp}

In this example is a model for a single stock in a single area with one fishery. The partition is made up of a single category, with sex and maturity not defined in the partition. Processes and observations in a given year are defined to occur in the following order:

\begin{enumerate}
	\item Recruitment
	\item Fishing mortality with natural mortality
	\item A survey takes place in the spawning season
	\item More natural mortality
	\item At the end of the year all of the fish age
\end{enumerate}

This model has the structure:

{\small{\begin{verbatim}
		@model
		start_year 1975 # Start year
		final_year 2012 # End year
		min_age 1		# min age of all categories
		max_age 30		# max age of all categories
		age_plus true	# is the last age a plus group
		base_weight_units tonnes
		initialisation_phases Equilibrium_state
		time_steps Sep_Feb Mar_May Jun_Aug # labels for the time steps
		\end{verbatim}}}

{\small{\begin{verbatim}
		@categories
		format stock # category type
		names HAL4  # category label(s)
		age_lengths HAK4_AL # labels of age-length relationship for each category
		\end{verbatim}}}

{\small{\begin{verbatim}
		@time_step Sep_Feb 
		processes Recruitment Instantaneous_Mortality
		
		@time_step Mar_May 
		processes Instantaneous_Mortality 
		
		@time_step Jun_Aug 
		processes Ageing Instantaneous_Mortality
		\end{verbatim}}}

The \command{categories} command defines the type, label, and age-length relationship of categories that make up the partition. A category is a group of individuals that have the same attributes. Examples of such attributes include life history and growth paths. Characteristics in a partition of the population that have different attributes can be sex, maturity, area, stock, and tagging information.

An example of the \command{categories} block for a simple sex-specific model with male and female in the partition:

{\small{\begin{verbatim}
@categories
format sex        # category type
names male female # category labels
age_lengths vb_male vb_female # labels for growth categories
		\end{verbatim}}}

The \command{time\_step} command describes which processes take place in each time step and in what order. Continuing on with the \command{model} block example, where three time steps are defined in the annual cycle (\texttt{time\_steps Sep\_Feb Mar\_May Jun\_Aug}), and within each time step the processes are specified. Each process must be defined in a \command{process} block:

{\small{\begin{verbatim}
@process Recruitment
type recruitment_beverton_holt
categories HAK4
proportions 0.5 0.25 0.25
b0 44000
recruitment_multipliers 1*38
standardise_years 1976:2010
steepness 0.9
ssb SSB
age 1

@process Ageing
type ageing
categories HAK4

		
@process Instantaneous_Mortality
type mortality_instantaneous
m 0.2
time_step_proportions 0.42 0.25 0.33
selectivities One
categories HAK4
table catches
year FishingWest FishingEest
1975	80	111
1976	152	336
1977	74	1214
1978	28	6
1979	103	506
1980	481	269
1981	914	83
1982	393	203
1983	154	148
1984	224	120
1985	232	312
1986	282	80
1987	387	122
1988	385	189
1989	386	418
1990	309	689
end_table
		
table method
method  	 category 	selectivity  u_max 	time_step 	penalty
FishingWest  HAK4   	FSel 		 0.7 	Sep_Feb 	Catchmustbetaken
FishingEest  HAK4   	FSel 		 0.7 	Sep_Feb 	Catchmustbetaken
end_table
\end{verbatim}}}

The sections above define all of the processes that occur to the partitions. In the \texttt{Mortality} section natural mortality is defined with selectivity ogive \texttt{One}, and fishing mortality is applied with selectivity ogive \texttt{FSel}:

{\small{\begin{verbatim}
	@selectivity One
	type constant 
	c 1
	
	@selectivity FSel
	type double_normal 
	mu 3.82
	sigma_l 1.63
	sigma_r 17
\end{verbatim}}}

Since an age-length relationship is specified in the \command{categories} block then the \command{age\_length} block needs to be defined. This command block is used to convert age to length which is then used to convert length to weight in an age-based model:

{\small{\begin{verbatim}
	@age_length HAK4_AL		# label from the @categories block
	type von_bertalanffy
	length_weight HAK4_LW	# label for the @length_weight block
	k 0.164
	t0 -2.16
	Linf 100.8
	cv_first 0.10
	cv_last  0.10
	
	@length_weight HAK4_LW
	type basic
	units tonnes
	a 4.79e-09 
	b 2.89 
\end{verbatim}}}

An important block to complete the population definition is the \command{initialisation\_phase} block. This block specifies how the partition is initialised. This block describes the state of the partition before \texttt{start\_year} of the model, which is usually defined as an equilibrium state:

{\small{\begin{verbatim}
	@initialisation_phase phase1
	type iterative	## Type of initialisation method; see manual for more information
	years 100		## How many years to run for
\end{verbatim}}}

This block is an example of an iterative initialisation type. This block specifies that the annual cycle will iterate through 100 years, which may or may not result in the partition reaching an equilibrium state. \textbf{N.B.} when using this initialisation method check that the partition has reached an acceptable equilibrium state.

The next section to define is the observation section. The survey occurs in the second time step, and is for relative abundance:

{\small{\begin{verbatim}
@observation Survey	## label of observation
type biomass 		## type of observation
time_step Mar_May	## which time step the observation occurs
time_step_proportion 0.5	## the observation occurs half way through the time step
categories HAK4
catchability q		## The label for @catchability block
selectivities One
likelihood lognormal		## likelihood to use for the objective function
process_error 0.10
years 		1992 	1993 	1994 	1995 	
obs 		2950	3353	3303	2457
error_value 	0.41	0.52	0.91	0.61

@catchability q	## label from @observation
type free
q 0.001		## the initial value
		\end{verbatim}}}

A set of model configuration files is available in the \texttt{Examples/Simple} subdirectory.

