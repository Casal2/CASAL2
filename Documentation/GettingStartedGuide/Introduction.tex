\section{Introduction}\label{sec:introduction}

This document is an introductory help guide for \CNAME, a generalised age-structured population dynamics modelling package. \CNAME\ is run from the command prompt/terminal where it reads in text files (model configuration files) which define the model. \CNAME\ then prints output to the screen or a file.

This document is for users who are new to \CNAME. \CNAME\ is primarily used for assessing marine fish populations, and can be used to model the population dynamics of other types of animals. \CNAME's predecessor CASAL is the primary stock assessment modelling tool used in assessing New Zealand's Tier 1 stocks, and is also the standard modelling tool used by the Convention for the Conservation of Antarctic Marine Living Resources (CCAMLR) for modelling Antarctic toothfish. 

\CNAME\ is a very general, highly flexible modelling framework and has no default settings. It has several run modes, settings, and user-defined population dynamics options that can be set depending on population life history characteristics and the available data.

\CNAME\ is open source, and is covered under the GNU GPL 2.0 licence. See the terms and conditions in the \CNAME\ Technical User Manual \citep{CASAL2}, or type \texttt{casal2 -l} into the command prompt.

There is also additional information that may be useful when getting familiar with \CNAME. \CNAME\ has a comprehensive User Manual \citep{CASAL2} which has further details on model components and functionality. \CNAME\ also has a Contributors' Guide for users who would like to add functionality. The modular structure of the code base can make adding new processes, observation types, and likelihood types tractable.

If you have any questions, please contact the \CNAME\ development team at \email.

The remaining content of this chapter describes operating system requirements and details about how to run, cite, and get licensing and contact info for \CNAME.  If you are new to population modelling then section~\ref{sec:data_requirements} describes the types of data needed to run a \CNAME\ model. Section~\ref{sec:data_requirements} is a good starting point to find out if you have the data for developing a model. \CNAME\ can also be used as an operating model to generate simulated data.
 
The remaining content of this document explains how to run \CNAME, the syntax of the input configuration files that \CNAME\, and a description of an example model.

\subsection{Version\label{sec:version}}

The results for \CNAME\ models can differ between versions as issues are fixed or new features are added. The \CNAME\ version number includes a date/time stamp (\texttt{yyyy-mm-dd}), which is the revision control system UTC date for the most recent modification of the source code. The User Manual will be updated for each released version of \CNAME.

\subsection{Citing the \CNAME\ Getting Started Guide}

A reference for this document is: 

\ManualRef\index{Citation}\index{Citing \CNAME}
 
\subsection{\I{Software license}\index{GNU GPL v2 licence}}

This program and the accompanying materials are made available under the terms of the licence \href{http://www.gnu.org/licenses/old-licenses/gpl-2.0.en.html}{GNU GPL v2} which accompanies this software.

Copyright \copyright 2015-\SourceControlYearDoc, \href{http://www.niwa.co.nz}{\Organisation}. All rights reserved.

\subsection{\I{System requirements}}

\CNAME\ is available for most IBM-compatible machines running 64-bit \I{GNU/Linux} and \I{Microsoft Windows} operating systems.

Several of \CNAME's functions are highly computer intensive and a fast processor with a lot of RAM is recommended. Depending on the model specifications, some of \CNAME's functions can take a considerable amount of time (minutes to hours), and in some cases can take several days to complete a MCMC chain.

The program itself requires only a few megabytes of disk space, although output files can take up larger amounts of disk space, and some processes can use significant amounts of memory.

\subsection{\I{Necessary files}}

For both 64-bit Linux and Microsoft Windows, only the executable file \texttt{casal2} or \texttt{casal2.exe}, respectively, is required to run \CNAME with non-automatic differentiation minimisers. To use the automatic differentiation minimisers, the \texttt{.dll} (for Windows) or the \texttt{.so} (for Linux) must be in the same folder as the executable \CNAME\ file or in your system path. There is not a version of \CNAME\ for 32-bit operating systems. 

\CNAME\ has an \R\ library for post-processing of model output for use with \href{http://www.r-project.org}{\R}\ \citep{R}. See the \CNAME User Manual Chapter 17 (Post-Processing) for more detail on the \R\ package.

\subsection{Getting help\index{Getting help}\index{User assistance}\index{Notifying errors}}

\CNAME\ is distributed as unsupported software. Please notify the \CNAME\ development team of any bug reports, feature requests, or questions by opening an issue on the \CNAME\ GitHub repository (\url{https://github.com/NIWAFisheriesModelling/CASAL2/issues}). See the \CNAME\ User manual \citep{CASAL2} for the recommended template for reporting issues.

