\section{\CNAME\ build rules\label{sec:build_rules}}

This section describes the standards and requirements for code to be included within the \CNAME\ code base. 

\subsection{\CNAME\ coding practice and style}\label{subsec:code_practive}

\CNAME\ is written in C++ and follows the Google C++ style guide (see \url{https://google.github.io/styleguide/cppguide.html}). The guide is long and comprehensive, so we don't necessarily recommend that you read or understand all of its content. However, the Development Team would like you to follow the Google style of code layout and structure. Google provides a handy script to parse source code and highlight errors that do not match their coding style.

This means using good indentations for functions and loops, sensible (human readable) variable names but noting the use of the characters `\_' on the end of class variables defined in the .h files. 

\textbf{Annotate} your code. For readability we encourage you to put lots of comments in your code. This helps others read what you intended. 

On top of annotating your code we encourage developers to add descriptive logging statements (print messages) in the code. You will see this in the source code already. The purpose of this is to allow a descriptive summary of the actions being done in the model for debugging purposes and curious users. By using these, it becomes easier to identify issues, errors and creates a transparent model for users.

You can also output the text and equations in these logs that would normally be too detailed for general users to be interested, as this may allow users to verify the exact equations or processes that have been implemented. These are really there for curious users and developers.

There are different levels of logging in \CNAME\ listed below.
\begin{itemize}
	\item LOG\_MEDIUM()  usually reserved for iterative functionality (e.g. estimates during estimation phase)
	\item LOG\_FINE() the level of reporting between an actual report and a fine scale detail that end users are not interested in (Developers)
	\item LOG\_FINEST() Minor details
	\item LOG\_TRACE() put at the beginning of every function
\end{itemize}

To run \CNAME\ in log mode piping out any LOG\_FINEST and coarser logs (LOG\_MEDIUM and LOG\_FINE) you can use the following command,

\texttt{casal2 -r --loglevel finest > MyReports.csl2 2> log.out}

This will output all the logged information to \texttt{log.out}.

\subsection{Unit tests}

One of the key focusses in the \CNAME\ development is an emphasis on software integrity --- this is to help ensure that the results from implemented models are consistent and error free. As part of this, we use unit tests to check the individual components of the code base, as well as tests that run entire models in order to validate that the outputs are what we expect them to be.

\CNAME\ uses:
\begin{itemize}
	\item Google testing framework
	\item Google mocking framework
\end{itemize}

When adding unit tests, they need to be developed and tested outside of \CNAME\  first, for example in \R\ or another program like \CNAME\ e.g. CASAL or Stock Synthesis. This gives confidence that the test does not contain a calcualtion or other error. 

An example of how to add a unit test for a process is shown in Section~\ref{sec:example}

\subsection{Reporting (Optional)}

Currently \CNAME\ has reports that are \R\ compatible, i.e., all output reports produced by \CNAME\ can be read into \R\ using the standard  \textbf{CASAL2} \R\ package. If you create a new report or modify an old one, you most follow the standard so that the report is \R\ compatible.

All reports must start with,
\texttt{*label (type)}
and end with,
\texttt{*end}

Depending on what type of information you wish to report, will depend on the syntax you need to use. For example

\paragraph*{$\{$d$\}$ (Dataframe)}
Report a dataframe
{\small{\begin{verbatim}
*estimates (estimate_value)
values {d}
process[Recruitment_BOP].r0 process[Recruitment_ENLD].r0 
2e+006 8e+006
*end
\end{verbatim}}}

\paragraph*{$\{$m$\}$ (Matrix)}
Report a matrix
{\small{\begin{verbatim}
*covar (covariance_matrix)
Covariance_Matrix {m}
2.29729e+010 -742.276 -70160.5
-110126 -424507 -81300 
-36283.4 955920 -52736.2 
*end
\end{verbatim}}}

\paragraph*{$\{$L$\}$ (List)}
Report a List
{\small{\begin{verbatim}
*weight_one (partition_mean_weight)
year: 1900
ENLD.EN.notag {L}
mean_weights {L}
0.0476604 0.111575 0.199705
end {L}
age_lengths {L}
12.0314 16.2808 20.0135
end {L}
end {L}
*end
\end{verbatim}}}

\subsection{Update manual}
The syntax sections of the user manual are automatically generated from the source code. This means contributors will need to add or modify the remaining sections of the the user manual to document their changes. This is a requirement of contributed or suggested code changes, and is important for end users to be able to use the new or modified functionality.

\subsection{Builds to pass before merging changes}

Once you have made changes to the code, you must run the following builds before your changes can be considered for includion in the the main code base.

build the unittest version see Section~\ref{sec:build_environment} for how to build unittest depending on {\color{red}your} system.

Run the standard and new unit tests to check that they all pass
\texttt{DoBuild test}

And test that the debug and release of \CNAME\ compile and run.
\texttt{DoBuild debug}

Then run the second phase of unit tests (requires the debug version is built). This runs the tests that comprise of complete model runs
\texttt{DoBuild modelrunner}

Build the archive 
\texttt{DoBuild archive}


