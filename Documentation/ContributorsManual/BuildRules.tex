\section{\CNAME\ build rules\label{sec:build_rules}}

This section describes the standards and requirements for code to be included within the \CNAME\ code base. 

\subsection{\CNAME\ coding practice and style}\label{subsec:code_practive}

\CNAME\ is written in C++ and follows the Google C++ style guide (see \url{https://google.github.io/styleguide/cppguide.html}). The guide is long and comprehensive, so we don't necessarily recommend that you read or understand all of its content. However, the Development Team would like you to follow the Google style of code layout and structure. Google provides a handy script to parse source code and highlight errors that do not match their coding style.

This means using good indentations for functions and loops, sensible (human readable) variable names but noting the use of the characters `\_' on the end of class variables defined in the .h files. 

\textbf{Annotate} your code. For readability we encourage you to put lots of comments in your code. This helps others read what you intended. 

On top of annotating your code we encourage developers to add descriptive logging statements (print messages) in the code. You will see this in the source code already. The purpose of this is to allow a descriptive summary of the actions being done in the model for debugging purposes and curious users. By using these, it becomes easier to identify issues, errors and creates a transparent model for users.

You can also output the text and equations in these logs that would normally be too detailed for general users to be interested, as this may allow users to verify the exact equations or processes that have been implemented. These are really there for curious users and developers.

There are different levels of logging in \CNAME\ listed below.
\begin{itemize}
	\item LOG\_MEDIUM()  usually reserved for iterative functionality (e.g. estimates during estimation phase)
	\item LOG\_FINE() the level of reporting between an actual report and a fine scale detail that end users are not interested in (Developers)
	\item LOG\_FINEST() Minor details
	\item LOG\_TRACE() put at the beginning of every function
\end{itemize}

To run \CNAME\ in log mode piping out any LOG\_FINEST and coarser logs (LOG\_MEDIUM and LOG\_FINE) you can use the following command,

\texttt{casal2 -r --loglevel finest > MyReports.csl2 2> log.out}

This will output all the logged information to \texttt{log.out}.

\subsection{Unit tests}

One of the key focusses in the \CNAME\ development is an emphasis on software integrity --- this is to help ensure that the results from implemented models are consistent and error free. As part of this, we use unit tests to check the individual components of the code base, as well as tests that run entire models in order to validate that the outputs are what we expect them to be.

\CNAME\ uses:
\begin{itemize}
	\item Google testing framework
	\item Google mocking framework
\end{itemize}

When adding unit tests, they need to be developed and tested outside of \CNAME\  first, for example in \R\ or another program like \CNAME\ e.g. CASAL or Stock Synthesis. This gives confidence that the test does not contain a calculation or other errors. There are three different testing concepts in \CNAME\ that I will quickly explain and point to examples in the code base. The three concepts are Mocking specific classes, implementing internal models (these two concepts are implemented in source files with extension \texttt{.Test.cpp}) that have variable test cases for a specific class and overall models found in the TestModel folder.

Mocking specific classes is used to validate specific functionality and is encouraged because it is the easiest to isolate errors introduced with new code changes. Good example in the code base that utilise this are in \texttt{AgeLengths} for the \texttt{VonBertalanffy.Test.cpp} file. You can see in this file we mock the class and test the mean length calculations. Another good example of this is in the \texttt{Partition} class file \texttt{Partition.Test.cpp} which shows how \CNAME\ validate user inputs and model expectations.

Implementing internal models is the most common test resource in the code, and much of the functionality for implementing these are in the source folder \texttt{TestResources}. This method implements simple models and run test cases with differing class implementations. Nice examples of this are in the \texttt{Projects} class, for example \texttt{LogNormal.Test.cpp}. These run an Internal empty model and test output of classes from a simple run. When using this method try and only add test cases that vary the class of interest, remember you trying to tell future developers if code they have added breaks other components and they will thank you if it is easy to find why this happens. Another example is the process \texttt{TagByLength.Test.cpp}, which defines a new model and tests different functionality of the tagging process.


Modelunner tests are run using the \texttt{DoBuild} script in the BuildSystem folder. Models are located in the \texttt{TestModel} folder. These models are used to test autodiff library model runs, and other functionality such \texttt{-i} and are more holistic tests. If these fail it can be difficult to isolate the error that is why we encourage you implement the other two unit tests as much as possible. The setup for the modelrunner are defined in the python script found at \texttt{BuildSystem/BuildTools/classes/ModelRunner.py} if you wanted to add simulation TestModels and projections.


An example of how to add a unit test for a process is shown in Section~\ref{sec:example}

\subsection{Reporting (Optional)}

Currently \CNAME\ has reports that are \R\ compatible, i.e., all output reports produced by \CNAME\ can be read into \R\ using the standard  \textbf{CASAL2} \R\ package. If you create a new report or modify an old one, you most follow the standard so that the report is \R\ compatible.

All reports must start with,
\texttt{*label (type)}
and end with,
\texttt{*end}

Depending on what type of information you wish to report, will depend on the syntax you need to use. For example

\paragraph*{$\{$d$\}$ (Dataframe)}
Report a dataframe
{\small{\begin{verbatim}
*estimates (estimate_value)
values {d}
process[Recruitment_BOP].r0 process[Recruitment_ENLD].r0 
2e+006 8e+006
*end
\end{verbatim}}}

\paragraph*{$\{$m$\}$ (Matrix)}
Report a matrix
{\small{\begin{verbatim}
*covar (covariance_matrix)
Covariance_Matrix {m}
2.29729e+010 -742.276 -70160.5
-110126 -424507 -81300 
-36283.4 955920 -52736.2 
*end
\end{verbatim}}}

\paragraph*{$\{$L$\}$ (List)}
Report a List
{\small{\begin{verbatim}
*weight_one (partition_mean_weight)
year: 1900
ENLD.EN.notag {L}
mean_weights {L}
0.0476604 0.111575 0.199705
end {L}
age_lengths {L}
12.0314 16.2808 20.0135
end {L}
end {L}
*end
\end{verbatim}}}

\subsection{Update manual}
The syntax sections of the user manual are automatically generated from the source code. This means contributors will need to add or modify the remaining sections of the the user manual to document their changes. This is a requirement of contributed or suggested code changes, and is important for end users to be able to use the new or modified functionality.

\subsection{Builds to pass before merging changes}

Once you have made changes to the code, you must run the following builds before your changes can be considered for inclusion in the the main code base. 

build the unittest version see Section~\ref{sec:build_environment} for how to build unittest depending on your system.

Run the standard and new unit tests to check that they all pass, to do this first compile the test executable using the script \texttt{DoBuild test}. Then move to the directory with the location of the executable (\texttt{BuildSystem/bin/OS/test}) and run it (open a command terminal and run \texttt{casal2}) to check all the unit-tests pass.

And test that the debug and release of \CNAME\ compile and run.
\texttt{DoBuild debug}

Then run the second phase of unit tests (requires the debug version is built). This runs the tests that comprise of complete model runs
\texttt{DoBuild modelrunner}

Build the archive 
\texttt{DoBuild archive} which builds all autodiff libraries. There are small nuances between Double classes, especially when reporting the class that mean seemingly simple changes can sometimes cause a break in the full build.


