\section{Setting up \CNAME\ BuildSystem\label{sec:build_environment}}
This section will cover, how to setup the build environment on you're local machine to compile \CNAME\, on both windows and linux machines. At present the \CNAME\ build system supports Microsoft Windows 7+ and Linux (with GCC/G++ 4.9.0+). Apple OSX is currently not supported.
\subsection{Introduction}
The build system is a collection of python scripts that do various tasks. These are located in \path{CASAL2/BuildSystem/buildtools/classes/}. Each python script has it’s own set of functionality. The top level of the build system is \path{CASAL2/BuildSystem/}. In this directory you can run \texttt{doBuild.bat help} or \texttt{./doBuild.sh help} on the command line to view all of the different build modes. The build system will take one or two parameters depending on what style of build you’d like. It’s possible to build various stand-alone binaries as well as shared libraries.
\\
For example:
\\
\texttt{doBuild debug} - will build a stand alone debug binary
\\
\texttt{doBuild library debug} - will build a debug shared library
\\
\texttt{doBuild thirdparty} - will build all of the third party libraries
\\
\texttt{doBuild thirdparty betadiff} - will rebuild only the betadiff thirdparty library
\\\\
All outputs from the build system will be placed in subfolders of \path{CASAL2/BuildSystem/bin/<operating system>/<build_type>}
\\
For example:
\\
\path{CASAL2/BuildSystem/windows/debug}
\\
\path{CASAL2/BuildSystem/windows/library_release}
\\
\path{CASAL2/BuildSystem/windows/thirdparty/}
\\
\path{CASAL2/BuildSystem/linux/library_release}

\subsection{Building on Windows}
\subsubsection{Prerequisite Software}
\textbf{C++ and Fortran Compiler}
Source: tdm-gcc (MingW64) from \url{http://tdm-gcc.tdragon.net/}.
\CNAME\ is designed to compile under GCC on all platforms. At present there is no support for Microsoft Visual C++. If you had all of the third-party libraries built under Visual C++ then it’d be possible to compile \CNAME\. It’s recommend you get the latest version of TDM-GCC. At the time of writing that was 5.1.0. During installation, you must check to install the GFortran components as well as the C++ components. Some of the third-party libraries required GFortran. \textbf{Note}: A common issue is having other GCC compilers in your path for example rtools has a GCC compiler. Make sure this compiler is the first in the path otherwise you may run into errors.
\\\\
\textbf{GIT Version Control}
Source: Command line GIT from \url{https://git-scm.com/downloads}.
GIT is used to build a Version.h header file so \CNAME\ knows it’s version internally and this can be used for diagnostic purposes. During installation, you must pick the Windows/DOS command line style of installation. Selecting the Linux/*nix style will cause GIT to mangle the PATH environment variable and the build system will be unable to locate GIT.
\\\\
\textbf{MikTex Latex Processor}
Source: Portable version from http://miktex.org/portable. The documentation for \CNAME\ is a PDF manual generated from Latex. Much of the Latex has been written by hand, but the syntax sections of the manual are generated directly from the code. In order to generate the documentation you will need the MikTex latex processor.
\\\\
\textbf{Inno Setup Installer Builder (optional)}
Source: Inno Setup 5 from \url{http://www.jrsoftware.org/isdl.php}
If you wish to build a single installer for \CNAME\ you’ll need the Inno Setup 5 application installed on the machine. The installation path must be \path{C:\Program Files (x86)\Inno Setup 5\}


\subsubsection{Pre-Build Requirements}
Prior to building CASAL2 you will need to ensure you have both G++ and GIT in your path. You can check both of these by typing:
\\
\texttt{g++ --version}
\\
\texttt{git --version}
\\
It’s worth checking to ensure GFortran has been installed with the G++ compiler by typing:
\\
\texttt{gfortran --version}
\\
If you wish to build the documentation bibtex will also need to be in the path:
\\
\texttt{bibtex -version}

\subsubsection{Building \CNAME}
The build process is very simple. You can run \texttt{doBuild.bat check} to see if your build environment is ready.
\begin{enumerate}
\item Get a copy (clone) of the forked code on your local machine, mentioned in Section~\ref{sec:local_repo}: 
	
	\item Navigate to the BuildSystem folder in \path{CASAL2/BuildSystem}
	\item You need to build the third party libraries with:
	\begin{itemize}
		\item \texttt{doBuild thirdparty}
	\end{itemize}
	\item You need to build the binary you want to use:
	\begin{itemize}
		\item \texttt{doBuild release}
	\end{itemize}	
	
	\item You can build the documentation if you want:
	\begin{itemize}
		\item \texttt{doBuild documentation}
	\end{itemize}		
\end{enumerate}

\subsection{Building on Linux}
This guide has been written against a fresh install of Ubuntu 15.10. With Ubuntu we use apt-get to install new packages. You’ll need to be familiar with the package manager for your distribution to correctly install the required prerequisite software.
\subsubsection{Prerequisite Software}
\textbf{Compiler G++}
\\
Ubuntu 15.10 comes with G++ 15.10, gfortran is not installed though so we can install it with: \texttt{sudo apt-get install gfortran}.
\\
\textbf{GIT Version Control}
\\
Git isn’t installed by default but we can install it with:
\\
\texttt{sudo apt-get install git}
\\
Source: Command line GIT from \url{https://git-scm.com/downloads}
GIT is used to build a Version.h header file so \CNAME\ knows it’s version internally and this can be used for diagnostic purposes. During installation, you must pick the Windows/DOS command line style of installation. Selecting the \path{Linux/*nix} style will cause GIT to mangle the PATH environment variable and the build system will be unable to locate GIT.
\\
\textbf{CMake}
\\
CMake is required to build multiple third-party libraries and the main code base. You can do this with: 
\\
\texttt{sudo apt-get install cmake}
\\
\textbf{Python2 Modules}
\\
There are a couple of Python2  modules that are required to build \CNAME. These can be installed with:
\\
\texttt{sudo apt-get install python-dateutil}
\\
You may also need to install \textbf{datetime}, re and \textbf{distutils}. \textbf{Texlive} Latex Processor. No supported latex processors are installed with Ubuntu by default. You can install a suitable latex process with:
\\
\texttt{sudo apt-get install texlive-binaries}
\\
\texttt{sudo apt-get install texlive-latex-base}
\\
\texttt{sudo apt-get install texlive-latex-recommended}
\\
\texttt{sudo apt-get install texlive-latex-extra}
\\
Alternatively you can do:
\\
\texttt{sudo apt-get install texlive-full}
\\
But this will require ~3GBs of storage.


\subsubsection{Building \CNAME}
The build process is very simple. You can run \texttt{./doBuild.sh check} to see if your build environment is ready.
\begin{enumerate}
	\item Get a copy (clone) of the forked code on your local machine, mentioned in Section~\ref{sec:local_repo}: 
	
	\item Navigate to the BuildSystem folder in \path{CASAL2/BuildSystem}
	\item You need to build the third party libraries with:
	\begin{itemize}
		\item \texttt{./doBuild.sh thirdparty}
	\end{itemize}
	\item You need to build the binary you want to use:
	\begin{itemize}
		\item \texttt{./doBuild.sh release}
	\end{itemize}	
	
	\item You can build the documentation if you want:
	\begin{itemize}
		\item \texttt{./doBuild.sh documentation}
	\end{itemize}		
\end{enumerate}



\subsection{Troubleshooting}
\subsubsection{Third-party Libraries}
It’s possible there will be build errors or issues building the third-party libraries. If you encounter an error then it’s worth checking the log files. Each third-party build system stores a log of everything it’s doing. The files will be named
\begin{itemize}
\item casal2\_unzip.log
\item casal2\_configure.log
\item casal2\_make.log
\item casal2\_build.log
\item \dots etc
\end{itemize}


Some of the third-party libraries require very specialised environments for compiling under GCC on Windows. These libraries are packaged with MSYS (MinGW Linux style shell system). The log files for these will be found in \path{ThirdParty/<library name>/msys/1.0/<library name>/}
\\
e.g:
\\ \path{ThirdParty/adolc/msys/1.0/adolc/ADOL-C-2.5.2/casal2_make.log}

\subsubsection{Main Code Base}
If the unmodified code base does not compile the most likely cause is an issue with the third-party libraries not being built. Ensure they’ve been built, otherwise engage the \CNAME\ development team at \texttt{casal2@niwa.co.nz} to help resolve your issue.
Running \texttt{./doBuild.sh check} can also be helpful to identify any issues.


