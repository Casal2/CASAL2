\section{Introduction\label{sec:Introduction}}

The Contributors Manual provides an overview for users who wish to access the source code, compile their own version of \CNAME , or who wish to contribute to \CNAME\ source code. \CNAME\ is open source and the development team encourages users to submit and contribute changes, bug fixes, or enhancements. This manual is intended to provide a guide for users who wish to access the source code for \CNAME , build and compile  \CNAME , or contribute to the development of \CNAME . 

The \CNAME\ source code is hosted on github, and can be found at \url{https://github.com/NIWAFisheriesModelling/CASAL2}. A compiled copy of the \CNAME\ and the associated manuals and examples can be found at url{ftp://ftp.niwa.co.nz/Casal2/windows/Casal2.zip}. If you have any questions or require more information, please contact the \CNAME\ development team at \texttt{casal2@niwa.co.nz}.  

The manual first describes how to get a copy of the source code from the repository using git commands, then describes how to build and compile \CNAME , and then how to submit changes to the \CNAME\ Development Team for inclusion into the code base.

To maintain the quality and integrity of the code base and ensure that any changes are reliable and accurate, the development team has created this manual to assist contributors in understanding how to access the source code, how to compile and build, and the guidelines for submitting code changes to the \CNAME\ Development Team. We want this process to be as pain free as possible to encourage contributions. 

At the beginning of each section there will be a list of points that the section will go into at more detail. If you have experience with some aspects you can skip but if you run into trouble you can always come back to check this guide. This manual covers basics such as setting up a github profile, to using github, and all the way to compiling and modifying source code.
