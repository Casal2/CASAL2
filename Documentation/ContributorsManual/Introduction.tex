\section{Introduction\label{sec:Introduction}}
This is a comprehensive manual for users who wish to develop and contribute to \CNAME. \CNAME\ is open source and the development team encourage users to consider enhancing this program through contributing to the source code. The source code is hosted on github and can be found at \url{https://github.com/NIWAFisheriesModelling/CASAL2}. For more information you can contact the development team at \texttt{casal2@niwa.co.nz}. This manual is a step by step guide for users who wish to setup the build system and contribute to the source code. It begins by showing how to get a copy of the repository using git commands through to being able to compile \CNAME\ and finally adding changes to the master (main) repository. Once changes have been added to the master repository they will be incorporated into the latest compiled version of \CNAME\ which will be available for the public to download and use. 
\\\\
To maintain integrity of the code base and ensure confidence in users that the available version is reliable and accurate, the development team has created this manual for contributors. For the reasons above the development team will only accept changes to the repository that abide to the guidelines set out in this manual. This process and manual is intended to be pain free to encourage your contributions.
\\\\
At the beginning of each section there will be a list of points that the section will go into at great detail. If you have experience with some aspects you can skip but if you run into trouble you can always come back to check this guide. This manual covers basics such as setting up a github profile, to using github and all the way to compiling and modifying C++ code.
