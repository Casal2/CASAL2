\section{Introduction\label{sec:Introduction}}
\textbf{The contribution process has changed since this document was written} We point you \href{https://casal2.github.io/contributing/}{here}. In saying that this has a lot of useful detail that will be useful for future developers.


The Contributors Manual provides an overview for users who wish to access the source code, compile the latest version of \CNAME , or who wish to contribute to \CNAME\ source code. \CNAME\ is open source and the development team encourages users to submit and contribute changes, bug fixes, or enhancements. This document is intended to provide a guide for users who wish to undertake the aforementioned tasks. 
\\\\
The \CNAME\ source code is hosted on GitHub, and can be found at \url{https://github.com/NIWAFisheriesModelling/CASAL2}.
\\\\
A release bundle which includes a binary (both windows and Linux operating systems), manual, examples, R package and other help guides can be downloaded at \url{https://github.com/NIWAFisheriesModelling/CASAL2/releases}. The official releases can be lagged behind the latest source code on the master because the official release is a manual process. It is possible to get an executable for both windows and Linux (Ubuntu 20) using the GitHub actions artifacts \href{https://github.com/NIWAFisheriesModelling/CASAL2/actions}{see here}. You just need to be logged into your GitHub account then click on the latest successfully checked commit (indicated by a green tick), scroll to the bottom and download either \texttt{Casal2-Linux-build} or \texttt{Casal2-Windows-build}. These are only saved for 30 days so if there hasn't been a commit in a while, they may not be available.
\\\\
If you have any questions or require more information, please contact the \CNAME\ development team at \email.
\\\\
To maintain the quality and integrity of the code base and ensure that any changes are reliable and accurate, the development team has created this manual to assist contributors in understanding how to access the source code, how to compile and build, and the guidelines for submitting code changes to the \CNAME\ Development Team. We want this process to be as pain free as possible to encourage contributions. 
\\\\
This manual covers basics such as setting up a GitHub profile, to using GitHub, and all the way to compiling and modifying source code. 
\\\\
A suggestion to future contributors, before making changes to \CNAME\ it may be helpful for you to discuss a future change with members of the development team before you begin making the change. They can assist with ideas on how to add functionality with ease. The easiest forum to have a conversation on code ideas is through the issues page of GitHub found at \url{https://github.com/NIWAFisheriesModelling/CASAL2/issues}. Just open up a new issue called, for example, New Migration process and begin by talking about how you think the new feature will work, and hopefully the development team can assist or make the implementation easier for you.
