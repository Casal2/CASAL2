\section{Introduction\label{sec:Introduction}}
The Contributors Manual provides an overview for users who wish to access the source code, compile their own version of \CNAME , or who wish to contribute to \CNAME\ source code. \CNAME\ is open source and the development team encourages users to submit and contribute changes, bug fixes, or enhancements. This manual is intended to provide a guide for users who wish to access the source code for \CNAME , build and compile  \CNAME , or contribute to the development of \CNAME . 
\\\\
The \CNAME\ source code is hosted on GitHub, and can be found at \url{https://github.com/NIWAFisheriesModelling/CASAL2}.
\\\\
A Microsoft Windows bundle which includes a binary, manual, examples and other help guides can be downloaded at \url{ftp://ftp.niwa.co.nz/Casal2/windows/Casal2.zip} for the Microsoft Windows version, \url{ftp://ftp.niwa.co.nz/Casal2/windows/casal2_setup.exe} for the Microsoft Windows installer. The Linux bundle which includes a binary, manual, examples and other help guides can be downloaded at \url{ftp://ftp.niwa.co.nz/Casal2/linux/Casal2.tar.gz}.
\\\\
If you have any questions or require more information, please contact the \CNAME\ development team at \email.
\\\\
To maintain the quality and integrity of the code base and ensure that any changes are reliable and accurate, the development team has created this manual to assist contributors in understanding how to access the source code, how to compile and build, and the guidelines for submitting code changes to the \CNAME\ Development Team. We want this process to be as pain free as possible to encourage contributions. 
\\\\
This manual covers basics such as setting up a GitHub profile, to using GitHub, and all the way to compiling and modifying source code. 
\\\\
A suggestion to future contributors, before making changes to \CNAME\ it may be helpful for you to discuss a future change with the development team before you begin making the change. They can assist with ideas on how to add functionality with ease. The easiest forum to have a conversation on code ideas is through the issues page of GitHub found at \url{https://github.com/NIWAFisheriesModelling/CASAL2/issues}. Just open up a new issue called, for example, New Migration process and begin by talking about how you were thinking about adding a new idea, and hopefully the development team can assist or make the implementation easier for you.
