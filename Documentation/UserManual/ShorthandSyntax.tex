\section{\I{The Shorthand syntax section}\label{sec:ShorthandSyntax-section}}
\subsection{\I{Categories}\label{sub:categories}}
\CNAME\ allows many user defined categories so shorthand syntax has been added to aid in the readability of complex configuration scripts and partition structures. For example when defining categories you can use a comma for shortening lists of categories. The following syntax is how we would specify the categories the long way.
{\small{\begin{verbatim}
@categories 
format sex.stage
names male.immature male.mature female.immature female.mature		
\end{verbatim}}}	

for the exact same partition structure but specified in a shorter way users could define the categories as, (note the use of the commas),

{\small{\begin{verbatim}
		@categories 
		format sex.stage
		names male,female.immature,mature	
		\end{verbatim}}}

\CNAME\ asks for categories in processes and observations so that it can apply the right model dynamics to the right elements of the partition. For the same reason as defining categories shorthand syntax aids in readability and input management. An example of a process where categories need to be supplied as an input command is in ageing,

{\small{\begin{verbatim}
# 1. The standard way
@ageing my_ageing
categories male.immature male.mature female.immature female.mature

# 2. The 1st short-hand way
@ageing my_ageing
categories male,female.immature,mature

# 3. Wild Card (all categories)
@ageing my_ageing
categories *

# 4. The 2nd short-hand way
@ageing my_ageing
categories sex=male sex=female
\end{verbatim}}}

Sometimes in observations we want to amalgamate categories together for example if we had a biomass estimate of the population that was made up of both males and females in the population you can specify this using the + special character, for example

{\small{\begin{verbatim}
		@observation CPUE
		type biomass
		catchability Fishq
		time_step one
		categories male+female
		selectivities FishSel
		likelihood lognormal
		years 1992:2001
		time_step_proportion 1.0
		obs 1.50 1.10 0.93 1.33 1.53 0.90 0.68 0.75 0.57 1.23
		error_value 0.35
\end{verbatim}}}



\subsection{\I{Parameters}\label{sec:params}}
\CNAME\ also allows parameters that are of type vector or map to be referenced and estimated partially. An example of a parameter that is type vector is \texttt{ycs\_values} in a recruitment process. Let say a recruitment block was specified as follows,
{\small{\begin{verbatim}
		@process WestRecruitment
		type recruitment_beverton_holt
		r0 400000
		years
		ycs_values 1 1 1 1 1 1 1 1 
		
		An alternative specification to the sequence of values you can use an astrix to
		 shorthand repeating integers e.g.
				
		ycs_values 1*8
				
		steepness 0.9
		age 1
\end{verbatim}}}

Lets say we wanted to only estimate the last four values of the parameter \texttt{process[WestRecruitment].ycs\_values} vector. This can be done as specified in the following \command{estimate} block,

{\small{\begin{verbatim}
@estimate
parameter process[WestRecruitment].ycs_values(5:8)
type uniform
lower_bound 0.1 0.1 0.1 0.1
upper_bound  10  10  10  10 
\end{verbatim}}}

This can also be applied to parameters that are of type map, for information on what type a parameter is see the syntax section. An example of a parameter that is of type map is \command{time\_varying}\texttt{[label].type=constant}. For the following \command{time\_varying} block,

{\small{\begin{verbatim}
@time_varying q_step1
type constant
parameter catchability[Fishq].q
years 	1992	1993	1994	1995
value 	0.2		0.2		0.2		0.2	
\end{verbatim}}}

In this example a user may want to estimate only one element of the map (say 1992), but force all other years to be the same as the one estimate. This can be done in an estimate block as follows,
{\small{\begin{verbatim}
@estimate
parameter time_varying[q_step1].value(1992)
same time_varying[q_step1].value(1993:1995)
type uniform
lower_bound 0.1 0.1 0.1 0.1
upper_bound  10  10  10  10 
\end{verbatim}}}
\subsection{\I{In line declaration}\label{sec:declare}}
In line declarations can help shorten models by passing \command{} blocks, for example 
{\small{\begin{verbatim}
		@observation chatCPUE
		type biomass
		catchability [q=6.52606e-005]
		time_step one
		categories male+female
		selectivities chatFselMale chatFselFemale
		likelihood lognormal
		years 1992:2001
		time_step_proportion 1.0
		obs 1.50 1.10 0.93 1.33 1.53 0.90 0.68 0.75 0.57 1.23
		error_value 0.35
		
		@estimate 
		parameter catchability[chatTANbiomass.one].q
		type uniform_log
		lower_bound 1e-2
		upper_bound 1
		In line declaration tips
		\end{verbatim}}}

In the above code we are defining and estimating catchability without explicitly creating an \command{catchability} block


When you do an inline declaration the new object will be created with the name of the creator's \texttt{label.index}
where index will be the word if it's one-nine and the number if it's 10+, for example
{\small{\begin{verbatim}
		@mortality halfm
		selectivities [type=constant; c=1]
		
		would create
		@selectivity halfm.one
		\end{verbatim}}}

if there were 10 categories all with there own selectivity the $10^th$ selectivity would be labelled;

{\small{\begin{verbatim}
		@selectivity halfm.10
		\end{verbatim}}}