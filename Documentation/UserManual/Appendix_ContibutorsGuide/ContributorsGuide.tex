% Include the sections required for the Contributors Guide appendix

\subsection{Introduction\label{AppendixA:Introduction}}

This Appendix provides an overview for thiose who wish to access and contribute to \CNAME\ and includes instructions for compiling \CNAME. 

\CNAME\ is open source and the \CNAME\ development team encourages those who wish to contribute any suggestions, bug fixes, or enhancements to use this guide. 

The source code for \CNAME\ is hosted on GitHub and can be found at \url{https://github.com/NIWAFisheriesModelling/CASAL2}.

Release versions of \CNAME\ (including the required binaries, user manual, simple examples and other utilities) is also available on the GitHub site.

If you have any questions or require more information, please gwet in touch with the \CNAME\ development team at \email.

\subsection{Getting started\label{AppendixA:GettingStarted}}

\subsubsection{Required Software}

\CNAME\ requires a range of software to successfully compile and buid the various components. While most of the source code for \CNAME\ uses C++, other software including Git, Python 3, LaTeX, 7-Zip, Inno Setup, and \R are also required to build \CNAME.

\textbf{C++ and Fortran Compiler}

Source: tdm-gcc (MingW64) from \url{http://www.tdm-gcc.tdragon.net/}

\CNAME\ will compile using GCC on Microsoft Windows and Linux. While it may be possible to compile using alternative compilers, the \CNAME\ Development Team cannot provide any assistance or recommendations. We recommend using 64-bit TDM-GCC version 10.3.0. Ensure you select both the "fortran" and "openmp" options when installing TDM-GCC. 

\textbf{GIT Version Control}

Source: Command line GIT from \url{https://www.git-scm.com/}

\CNAME\ automatically adds a version number that is based on the latest commit identification date and tag on the GitHub repository. The command line version of GIT is used to extract this information. The resuting version identifyer is then used for the compiled binaries, R libraries, and user manual.

\textbf(Python 3}

Source: Python 3 from \url{https://www.python.org/}

The \CNAME\ code and documentation build scripts use Python 3.

\textbf{MiKTeX TeX package}

Source: MikTeX LaTeX from \url{http://www.miktex.org/}.

The documentation for \CNAME\ is generated using LaTeX. 

\textbf{7-Zip}

Source: 7-Zip from \url{http://www.7-zip.org/download.html}.

The BuildSystem calls 7zip.exe to unzip files in the build system, and to create a zipped archive of key files. we advise having 7zip available in the system path.

\textbf{Inno Setup Installer}

Source: Inno Setup 6 from \url{http://www.jrsoftware.org/isdl.php}

The building of the Microsoft Windows compatible install package requires Inno Setup 6. The installation path must be \path{C:\Program Files (x86)\Inno Setup 6\} in order for the build scripts to find and use Inno Setup.

\textbf{R}

Source: R from \url{http://www.jrsoftware.org/isdl.php}

The building of \CNAME\ \R\ libraries requires \R.












\section{Creating a forked local repository\label{sec:local_repo}}
This section will cover the following points
\begin{enumerate}
	\item Create a github profile
	\item Download git software
	\item Fork the master repository
\end{enumerate}


\subsection{Git Profile}
The first step you will have to do is create a profile on github (if you do not already have one). Github is free and easy to setup. Go to \url{https://github.com} to set up a profile if you do not already have one. Once you have set up a github account you need download git software that you use to "communicate" to repositories.


\subsection{Git Software}
As mentioned earlier the source code is hosted on github so before you can get a copy of the repository you will need to install a program that can "communicate" with github. You will need to acquire Command line GIT from \url{https://git-scm.com/downloads} this is because it is used to build a Version.h header file so \CNAME\ but this will be discussed earlier. I use tortoisegit \url{https://tortoisegit.org/download} on windows to pull, push and commit changes to repositories. However there are many programs to use and all have the same functionality.

\subsection{Forking a repository}

The main repository which contains code that is in the publicly available \CNAME\ executable is called the 'master' repository. Only the development team have permission to add, delete and change code directly with the 'master' repository. For other contributors there is a very easy way to add, delete and change code. This is through forking the master repository. This is done by going to the \CNAME\ github repository found at \url{https://github.com/NIWAFisheriesModelling/CASAL2} and selecting the fork button in the top right of the page circled in Figure~\ref{fig:fork}

\begin{figure}[!ht]
	\includegraphics[scale=0.6]{Figures/Fork_button.png}
	\caption{Creating a forked repository}\label{fig:fork}
\end{figure}
\pagebreak
This will create a copy of the \CNAME\ repository under your profile at the point of the fork. To check that you have successfully forked the repository, go to your git profile and you should see a \CNAME\ repository under your repositories, shown in Figure~\ref{fig:fork_success},

\begin{figure}[!ht]
	\includegraphics[scale=0.6]{Figures/Fork_success.png}
	\caption{Fork success}\label{fig:fork_success}
\end{figure}

An important point is that the forked repository will not automatically keep up to date with the 'master' repository. So if the development team make changes, you will want to keep your forked repository up to date. This is easily done and is explained in the next section on how to maintain and contribute to a forked repository.


 
\section{Maintaining and contributing to a forked repository\label{sec:maintain_repo}}
This section will cover the following points
\begin{enumerate}
	\item Check the status of the forked repository compared to the 'master' repository
	\item Update forked repository
	\item Make changes to your forked repository
\end{enumerate}

\subsection{Checking the status of the forked repository}

To see how your forked repository is compared to the 'master' repository. There is line that is underlined in red shown in Figure~\ref{fig:fork_status}.
	
\begin{figure}[!ht]
	\includegraphics[scale=0.6]{Figures/fork_status.png}
	\caption{Fork status}\label{fig:fork_status}
\end{figure}

This line tells us if we are up to date or as shown in Figure~\ref{fig:fork_status} one commit behind. To update this repository click the 'compare' button which is circled in red. This will bring up a page that will tell you all the changes that have occurred to the 'master' repository. There are two situations that can occur when updating a forked repository. The first and easiest is that there is no conflicts and you can merge teh 'master' changes with ease, the second is there are conflicts.
\section{Setting up \CNAME\ BuildSystem\label{sec:build_environment}}

This section descibes how to set up the environment on your local machine that will allow you to build and compile \CNAME\. The buld enviroment can be on either Microsoft Windows or Linux systems. At present the \CNAME\ build system supports Microsoft Windows 7+ and Linux (with GCC/G++ 4.9.0+). Apple OSX or other platforms are not currently supported.

\subsection{Overview}

The build system is made up of a collection of python scripts that do various tasks. These are located in \path{CASAL2/BuildSystem/buildtools/classes/}. Each python script has it’s own set of functionality and undertakes one set of actions. 

The top level of the build system can be found at \path{CASAL2/BuildSystem/}. In this directory you can run \texttt{doBuild.bat help} from the command line in Microsoft Windows systems or \texttt{./doBuild.sh help} from a terminal in Linux systems.
 - this doesn't make sense.
The build system will take one or two parameters depending on what style of build you'd like to achieve. These commands allow the building of various stand-alone binaries, shared libraries, and the documentation. Note that you will need additional software installed on your system in order to build \CNAME . These requirements are described later.

A summary of all of the doBuild arguments can be found using the command \texttt{doBuild help} in the BuildSystem directory.

The current arguments to doBuild are: 

\begin{itemize}
  \item \texttt{debug}:  Build standalone debug executable
  \item \texttt{release}: Build standalone release executable
  \item \texttt{test}: Build standalone unit tests executable
  \item \texttt{documentation}: Build the user manual
  \item \texttt{thirdparty}: Build all required third party libraries
  \item \texttt{thirdpartylean}: Build minimal third party libraries
  \item \texttt{clean}: Remove any previous debug/release build information
  \item \texttt{cleanall}: Remove all previous build information
  \item \texttt{archive}: Build a zipped archive of the application. The application is built using shared libraries so a single casal2 executable is created. 
  \item \texttt{check}: Do a check of the build system
  \item \texttt{modelrunner}: Run the test suite of models
  \item \texttt{installer}: Build an installer package
  \item \texttt{deb}: Create Linux .deb installer
  \item \texttt{library}: Build shared library for use by front end application
  \item \texttt{frontend}: Build single CASAL2 executable with all minimisers and unit tests
\end{itemize}

Valid Build Parameters: (thirdparty only)
\begin{itemize}
  \item \texttt{<libary name>}: Target third party library to build or rebuild
\end{itemize}

Valid Build parameters: (debug/release only)
\begin{itemize}
  \item \texttt{betadiff}: Use BetaDiff auto-differentiation (from CASAL)
  \item \texttt{cppad}: Use CppAD auto-differentiation
  \item \texttt{adolc}: Use ADOLC auto-differentiation in compiled executable
\end{itemize}

Valid Build parameters: (library only)
\begin{itemize}
  \item \texttt{adolc}: Build ADOLC auto-differentiation library
  \item \texttt{betadiff}: Build BetaDiff auto-differentiation library (from CASAL)
  \item \texttt{cppad}: Build CppAD auto-differentiation library
  \item \texttt{test}: Build Unit Tests library
  \item \texttt{release}: Build release library
\end{itemize}

The outputs from the build system commands will be placed in subfolders of \path{CASAL2/BuildSystem/bin/<operating system>/<build_type>}

For example:

\path{CASAL2/BuildSystem/windows/debug}

\path{CASAL2/BuildSystem/windows/library_release}

\path{CASAL2/BuildSystem/windows/thirdparty/}

\path{CASAL2/BuildSystem/linux/library_release}

\subsection{Building on Windows}

\subsubsection{Prerequisite Software}

The building of \CNAME\ requires additional build tools and software, including git version control, GCC compiler, LaTex compiler, and an Windows package builder. \CNAME\ requires specific implementations and versions in order to build. 

\textbf{C++ and Fortran Compiler}

Source: tdm-gcc (MingW64) from \url{http://www.tdm-gcc.tdragon.net/}.

\CNAME\ is designed to compile under GCC on Microsoft Windows and Linux  platforms. While it may be possible to build the package using different compilers, the \CNAME\ Development Team does provide any assistance or recommendations. We recommend using 64-bit TDM-GCC version 5.1.0. Ensure you have the "fortran" and "openmp" options installed as a part of the "gcc" dropdown tickboxes otherwise \CNAME will not compile.. \textbf{Note}: A common error that can be made is having a different GCC compiler in your path when attempting to compile. For example, rtools includes a version of the GCC compiler. We recommend removing these from your path prior to compiling.


\textbf{GIT Version Control}

Source: Command line GIT from \url{https://www.git-scm.com/downloads}.

\CNAME\ automatically adds a version number based on the GIT version of the latest commit to its repository. The command line version of GIT is used  to generate a version number for the compiled binaries, R libraries, and the manuals. 

\textbf{MiKTeX Latex Processor}

Source: Portable version from \url{http://www.miktex.org/portable}. 

The main user documentation for \CNAME\ is a PDF manual generated from LaTeX. The LaTeX syntax sections of the documentation are generated, in part, directly from the code. In order to regenerate the user documentation, you will need the MiKTeX LaTeX compiler.

\textbf{7-Zip}

Source: 7-Zip from \url{http://www.7-zip.org/download.html}.
	
The BuildSystem calls 7zip.exe to unzip files in the build system; it is advised to have this in the path.

\textbf{Inno Setup Installer Builder (optional)}

Source: Inno Setup 5 from \url{http://www.jrsoftware.org/isdl.php}

If you wish to build a Microsoft Windowes compatible Installer for \CNAME\ then you will need the Inno Setup 5 application installed on the machine. The installation path must be \path{C:\Program Files (x86)\Inno Setup 5\} in order for the build scipts to fins and use it.

\subsubsection{Pre-Build Requirements}

Prior to building \CNAME\ you will need to ensure you have both G++ and GIT in your path. You can check both of these by typing:

\texttt{g++ --version}

\texttt{git --version}

This also allows you to check that there are no alternative versions of a GCC compiler that may confuse the \CNAME\ build. 

It’s worth checking to ensure GFortran has been installed with the G++ compiler by typing:

\texttt{gfortran --version}

If you wish to build the documentation bibtex will also need to be in the path:

\texttt{bibtex -version}

\subsubsection{Building \CNAME}

The build process is relatively straightforward. You can run \texttt{doBuild check} to see if your build environment is ready.

\begin{enumerate}
  \item Get a copy (clone) of the forked code on your local machine, mentioned in Section~\ref{sec:local_repo}: 
  \item Navigate to the BuildSystem folder in \path{CASAL2/BuildSystem}
  \item You need to build the third party libraries with:
  \begin{itemize}
    \item \texttt{doBuild thirdparty}
  \end{itemize}
  \item You need to build the binary you want to use:
  \begin{itemize}
    \item \texttt{doBuild release}
  \end{itemize}	
  \item You can build the documentation if you want:
  \begin{itemize}
    \item \texttt{doBuild documentation}
  \end{itemize}		
\end{enumerate}

\subsection{Building on Linux}

This guide has been written against a fresh install of Ubuntu 15.10. With Ubuntu we use apt-get to install new packages. You’ll need to be familiar with the package manager for your distribution to correctly install the required prerequisite software.

\subsubsection{Prerequisite Software}

\textbf{Compiler G++}

Ubuntu 15.10 comes with G++ 15.10, gfortran is not installed though so we can install it with: \texttt{sudo apt-get install gfortran}.

\textbf{GIT Version Control}

Git isn't installed by default but we can install it with \texttt{sudo apt-get install git}

\CNAME\ automatically adds a version number based on the GIT version of the latest commit to its repository. The command line version of GIT is used  to generate a version number for the compiled binaries, R libraries, and the manuals. 

\textbf{CMake}

CMake is required to build multiple third-party libraries and the main code base. You can do this with \texttt{sudo apt-get install cmake}

\textbf{Python2 Modules}

There are a couple of Python2 modules that are required to build \CNAME. These can be installed with \texttt{sudo apt-get install python-dateutil}

You may also need to install \textbf{datetime}, re and \textbf{distutils}. \textbf{Texlive} Latex Processor. No supported latex processors are installed with Ubuntu by default. You can install a suitable latex process with:

\texttt{sudo apt-get install texlive-binaries}
\texttt{sudo apt-get install texlive-latex-base}
\texttt{sudo apt-get install texlive-latex-recommended}
\texttt{sudo apt-get install texlive-latex-extra}

Alternatively you can install the complete package:
\texttt{sudo apt-get install texlive-full}

\subsubsection{Building \CNAME}

The build process is relatively straighforward. You can run \texttt{./doBuild.sh check} to see if your build environment is ready.

\begin{enumerate}
	\item Get a copy (clone) of the forked code on your local machine, mentioned in Section~\ref{sec:local_repo}: 
	\item Navigate to the BuildSystem folder in \path{CASAL2/BuildSystem}
	\item You need to build the third party libraries with:
	\begin{itemize}
	    \item \texttt{./doBuild.sh thirdparty}
	\end{itemize}
	\item You need to build the binary you want to use:
	\begin{itemize}
		\item \texttt{./doBuild.sh release}
	\end{itemize}	
	\item You can build the documentation if you want:
	\begin{itemize}
		\item \texttt{./doBuild.sh documentation}
	\end{itemize}		
\end{enumerate}

\subsection{Troubleshooting}

\subsubsection{Third-party Libraries}

It's possible that there will be build errors or issues building the third-party libraries. If you encounter an error, then it’s worth checking the log files. Each third-party build system stores a log of everything it’s doing. The files will be named

\begin{itemize}
	\item casal2\_unzip.log
	\item casal2\_configure.log
	\item casal2\_make.log
	\item casal2\_build.log
	\item \dots etc,.
\end{itemize}

Some of the third-party libraries require very specialised environments for compiling under GCC on Windows. These libraries are packaged with MSYS (MinGW Linux style shell system). The log files for these will be found in \path{ThirdParty/<library name>/msys/1.0/<library name>/}

e.g: \path{ThirdParty/adolc/msys/1.0/adolc/ADOL-C-2.5.2/casal2_make.log}\\
e.g: \path{ThirdParty/boost/boost_1_58_0/casal2_build.log}

\subsubsection{Main Code Base}

If the unmodified code base does not compile, the most likely cause is an issue with the third-party libraries not being built. Ensure they have been built correctly. As they are outside the control of the Development Team, problems can arise that may require the developers of the third party libraries to resolve first. Contact the \CNAME\ development team at \texttt{casal2@niwa.co.nz} for help.




\section{\CNAME\ build rules\label{sec:build_rules}}
This section will cover, the standards contributors are expected to follow and the builds that must pass on the local machine in order to add a pull request for changes to be accepted into the 'master' repository.

\subsection{\CNAME\ coding practice}
google style, C++ 11, annotate your code

\subsection{Unit tests}
test your code externally

\subsection{Reporting}
Add reports to your code change

\subsection{Update manual}

\subsection{Builds to pass before merging changes}
build the unitest version
\texttt{DoBuild test}
run the unittest check they all pass
\texttt{casal2}
\texttt{DoBuild debug}
run the second phase of unitests which are complete model runs
\texttt{DoBuild modelrunner}
this is conditional on having a debug version previously built.
Build the archive which contains
\texttt{DoBuild archive}



\section{An example of adding a new process\label{sec:example}}
The following shows a sequence of figures and text of an example, where we are adding a new process called survivorship.
\section{Merging changes form a forked repository to a master repository\label{sec:pull_requests}}

This section describes how to merge changes from a forked repository to the master repository. 

This is under the assumptions that the contributor has followed the rules laid out in Section~\ref{sec:build_rules} and has permission from the Development Team to do so.

Text here ....
\section{Moderating Code for Maintainers\label{sec:Maintaining}}

This section is for developer responsible for maintaining the GitHub repository, specifically the best way to check a pull request. When we get a pull request the things you want to scrutinize are syntax and optimisation. Why I focus on these is because the automated build system test will tell you if the unit-tests are broken so don't waste your time looking at them. Hopefully who ever is requesting a pull-request has abided by the testing framework and this won't be an issue. \textbf{This section assumes the pull request passes all test's and builds}, please be diplomatic if this is not the truth remembering that we want to encourage collaboration on this project.



\section{Adding Reports\label{sec:reports}}
This section gives some points when adding report classes to \CNAME. There are two variables that you must define when describing a report class, these are \subcommand{model\_state\_} and \subcommand{run\_mode\_}. I will try and explain what values you should give your report, but another avenue is to look at example report classes. The \subcommand{model\_state\_} parameters tells the model when to print the report \subcommand{State::kExecute} will call the \texttt{DoExecute} after each year and \subcommand{State::kIterationComplete} will print the report once the model has run. So obviously if the report prints annual information or doesn't cache information you will want to use \subcommand{State::kExecute} e.g. \subcommand{Partition} else use \subcommand{State::kIterationComplete}. \textbf{Important note} if you add a new report that has \subcommand{model\_state\_} \subcommand{State::kExecute} you will need to tweak the R-library function \subcommand{extract.mpd()}. At line 50 of the of \texttt{extract.mpd.R} there is the following line of code.   

\begin{lstlisting}[language=R]
multi_year_reports = c("partition", "PartitionBiomass", "PartitionMeanWeight")
\end{lstlisting}

you will need to add the report label type to this vector so that R will process the report correctly.


The other command \subcommand{run\_mode\_} just tells \CNAME\ what run-mode should the report be generated in.
