\section{Introduction\label{sec:Introduction}}

\subsection{\I{About \CNAME}}

\CNAME\ is NIWA's open-source integrated assessment tool for modelling the population dynamics of marine species, including fishery stock assessments. \CNAME\ expands functionality and increases maintainability relative to its predecessor, CASAL. \CNAME\ can be used for quantitative assessments of marine populations, including fish, invertebrates, marine mammals and seabirds.

The \CNAME\ software implements a generalised age-structured population model that allows for a great deal of choice in specifying the population dynamics, parameter estimation, and model outputs. \CNAME\ is designed for flexibility. It can implement an age-structured model for a single population or multiple populations using user-defined categories such as area, sex, and maturity. These structural elements are generic and not predefined, and are easily constructed. \CNAME\ models can be used for a single population with a single anthropogenic event (in a fish population model this would be a single fishery), or for multiple species and populations, areas, and/or anthropogenic or exploitation methods, and including predator-prey interactions.

In \CNAME\ the processes in a time period and within an annual cycle are defined by the user. Observation data used for model fitting can be from many different sources, like, removals-at-size or -age from an anthropogenic or exploitation event (e.g., fishery or other human impact), research survey and other biomass indices, and mark-recapture data. Model parameters can be estimated using penalised maximum likelihood or Bayesian methods.

As well as the point estimates of the parameters, \CNAME\ can calculate the likelihood or posterior distribution profiles, and can generate Bayesian posterior distributions using Markov chain Monte Carlo methods. \CNAME\ can project population status using deterministic or stochastic population dynamics. \CNAME\ can also simulate observations from a set of given model structures.

\subsection{\I{Citing \CNAME}}

The reference for this document is \ManualRef

The peer-reviewed journal article reference for \CNAME\ is \citep{doonan_casal2}.

\subsection{\I{\CNAME~Contributors}}

The \CNAME~project was started by Alistair Dunn. The software architect is Scott Rasmussen.

Authors of this document include Ian Doonan, Craig Marsh, Kath Large, Teresa A'mar, and Alistair Dunn.

Contributors to the development include Scott Rasmussen, Alistair Dunn, Ian Doonan, Craig Marsh, Teresa A'mar, Kath Large, Sophie Mormede, and Samik Datta.

Other contributors include Matt Dunn, Jingjing Zhang, and Marco Kienzle.

\subsection{\I{Software license}}

This program and the accompanying materials are made available under the terms of the \href{http://www.opensource.org/licenses/GPL-2.0}{GNU General Public License version 2} which accompanies this software (see Section \ref{sec:gpl}).

Copyright \copyright 2016-\SourceControlYearDoc, \href{https://www.niwa.co.nz}{\Organisation}. All rights reserved.

\subsection{\I{Where to get \CNAME }}

In the first instance, see \url{https://www.niwa.co.nz/} for information about \CNAME . The \CNAME\ source code is hosted on GitHub, and can be found at \url{http://github.com/NIWAFisheriesModelling/CASAL2}\index{GitHub}.

There are installation packages available for Linux and Microsoft Windows.  The package includes the \CNAME\ executable, the \R\ library, the User Manual and other documentation, example models, and other information. The installation packages can be downloaded at \url{https://github.com/NIWAFisheriesModelling/CASAL2/releases}.

\subsection{\I{System requirements}}

\CNAME\ is available for most IBM compatible machines running 64-bit \I{Linux} and \I{Microsoft Windows} operating systems.

Several of \CNAME's tasks are computer intensive and a fast processor is recommended. Depending on the model implemented, some of the \CNAME\ tasks can take a considerable amount of processing time (minutes to hours), and in extreme cases may take several days to complete an MCMC estimate. 

Output files can be large, and the output from developing a model, sensitivity analyses, and running multiple MCMC chains can take up significant amounts of disk space\index{Disk space}. Depending on the number and type of user output requests, the output could range from a few hundred kilobytes to several hundred megabytes. When estimating model fits, several hundred megabytes of RAM may be required, depending on the spatial size of the model, number of categories, and complexity of processes and observations. For larger models, several gigabytes of RAM and disk space may occasionally be required. 

\subsection{\I{Necessary files}}

For both 64-bit Linux and Microsoft Windows, only the binary executable \texttt{casal2} or \texttt{casal2.exe} is required to run \CNAME . No other software is required. \CNAME\ is not available for 32-bit operating systems. 

\CNAME\ provides few functions for post-processing model output, and a package that allows tabulation and graphing of model outputs is recommended. Software such as \href{http://www.r-project.org}{\R}\ \citep{R} is recommended for post-processing \CNAME\ output. The \texttt{CASAL2} \R\ package is provided for parsing the \CNAME\ output into \R\ (see Section \ref{sec:post-processing}), as well as providing some diagnostic and plotting functionality.

\subsection{Getting help\index{Getting help}\index{User assistance}\index{Notifying errors}}

TODO review with the Project Manager and Project Lead

\CNAME\ is distributed as unsupported software. Please notify the \CNAME\ Development Team of any issues with or errors in \CNAME. Please contact the \emaillink. See Section \ref{sec:reporting-errors} for the template for reporting issues.

\subsection{Technical details\index{Technical specifications}}\label{sec:TechnicalDetails}

\CNAME\ was compiled on Linux using \texttt{gcc} (\url{http://gcc.gnu.org}), the C/C++ compiler developed by the GNU Project (\url{http://gcc.gnu.org}). The 64-bit Linux \index{Linux} version was compiled using \texttt{gcc} version 4.8.5 20150623 on CentOS 7.7 Linux (\url{http://www.ubuntu.com/}).

The Microsoft Windows (\url{http://www.microsoft.com})\index{Microsoft Windows} version was compiled using MingW (\url{http://www.mingw.org})\index{Mingw} \texttt{gcc} 8.1.0 (\url{http://gcc.gnu.org})\index{gcc}. The Microsoft Windows(\url{http://www.microsoft.com}) installer was built using the Inno Setup 5 (\url{http://www.jrsoftware.org/isdl.php}).

\CNAME\ includes number of different minimisers; different minimisers may perform better for some models than others. Some of the minimisers are non-automatic differentiation minimisers, and will usually work form most problems, albeit more slowly than auto-differentiation based minimisers. The non auto-differentiation minimisers are 
\begin{enumerate}
\item Numerical differences: GammaDiff, a minimiser that is closely based on the main algorithm of \cite{779}, that uses finite difference gradients\index{Finite differences minimiser}
\item DESolver: The differential evolution solver\index{Differential evolution minimiser} \citep{1442}, based on code by \href{mailto:<godwin@pushcorp.com>}{Lester E. Godwin} of \href{http://www.pushcorp.com}{PushCorp, Inc.}
\end{enumerate}

There are two auto-differentiation minimisers, based on ADOL-C. These are the
\begin{enumerate}
\item Betadiff: A minimiser from an older version of ADOL-C (v1.8.4) that was used as the automatic differentiation minimiser in the first version of CASAL \citep{1388}, with a similar minimising algorithm as used for the finite differences minimiser above, but based on the gradient from the auto-differentiation chain
\item ADOL-C: A minimiser from version of ADOL-C (v2.5.1) \citep{walther1996adolc};, with a similar minimising algorithm as used for the finite differences (GammaDiff) minimiser above, but based on the gradient from the auto-differentiation chain
\end{enumerate}

The random number generator\index{Random number generator} used in \CNAME\ uses an implementation of the Mersenne twister random number generator \citep{796}. This functionality, the command line functionality, matrix operations, and a number of other functions use the \href{http://www.boost.org/}{Boost} C++ library (Version 1.58.0)\index{Boost C++ library}.

The output from \CNAME\ may differ slightly on the different operating systems and operating system versions due to different precision arithmetic or other platform-dependent implementation details. The source code\index{\CNAME\ source code} for \CNAME\ is available in the GitHub repository at \github.

Unit tests of the underlying \CNAME\ code are carried out at build time, using the \href{https://github.com/google/googletest}{Google Test and Mock} unit testing and mocking framework. The unit test framework aims to cover a significant proportion of the key functionality within the \CNAME\ code base. The unit test code\index{Unit tests} for \CNAME\ is available as a part of the underlying source code.

