\section{Quick reference\label{sec:quick-reference}}
\defComLab{additional\_prior}{Define an object of type \emph{additional\_prior}}\par
\defSub{parameter} {Name of the parameter to generate additional prior on}
\defSub{label} {Label for teh additional prior}
\defSub{type} {Type of additional prior}
\defComLab{ageing\_error}{Define an object of type \emph{ageing\_error}}\par\par
\defSub{label} {Label of the ageing error}
\defSub{type} {Type of ageing error}
\defComLab{age\_length}{Define an object of type \emph{age\_length}}\par\par
\defSub{label} {Label of the age length relationship}
\defSub{type} {Type of age length relationship}
\defSub{time\_step\_proportions} {the fraction of the year applied in each time step that is added to the age for the purposes of evaluating the length, i.e., a value of 0.5 for a time step will evaluate the length of individuals at age+0.5 in that time step}
\defSub{distribution} {The assumed distribution for the growth curve}
\defSub{cv\_first} {CV for the first age class}
\defSub{cv\_last} {CV for last age class}
\defSub{casal\_switch} {If true, use the (less accurate) equation for the cumulative normal function as was used in the legacy version of CASAL.}
\defSub{by\_length} {Specifies if the linear interpolation of CV's is a linear function of mean length at age. Default is just by age}
\defComLab{catchability}{Define an object of type \emph{catchability}}\par\par
\defSub{label} {Label of the catchability}
\defSub{type} {Type of catchability}
\defComLab{categories}{Define an object of type \emph{categories}}\par\par
\defSub{format} {The format that the category names adhere too}
\defSub{names} {The names of the categories to be used in the model}
\defSub{years} {The years that individual categories will be active for. This overrides the model values}
\defSub{age\_lengths} {The labels of age\_length objects that are assigned to categories}
\defSub{length\_weight} {The labels of the length\_weight objects that are assigned to categories}
\defSub{age\_weight} {The labels of the age\_weight objects that are assigned to categories}
\defComLab{derived\_quantity}{Define an object of type \emph{derived\_quantity}}\par\par
\defSub{label} {Label of the derived quantity}
\defSub{type} {Type of derived quantity}
\defSub{time\_step} {The time step in which to calculate the derived quantity after}
\defSub{categories} {The list of categories to use when calculating the derived quantity}
\defSub{selectivities} {A list of one selectivity}
\defSub{time\_step\_proportion} {Proportion through the mortality block of the time step when calculated}
\defSub{time\_step\_proportion\_method} {Method for interpolating for the proportion through the mortality block}
\defSub{values} {}
\par\textbf{\commandlabsubarg{derived\_\_quantity}{type}{abundance}}\par
\par\textbf{\commandlabsubarg{derived\_\_quantity}{type}{biomass}}\par
\defComLab{estimate}{Define an object of type \emph{estimate}}\par\par
\defSub{label} {The label of the estimate}
\defSub{type} {The prior type for the estimate}
\defSub{parameter} {The name of the parameter to estimate in the model}
\defSub{lower\_bound} {The lower bound for the parameter}
\defSub{upper\_bound} {The upper bound for the parameter}
\defSub{same} {List of parameters that are constrained to have the same value as this parameter}
\defSub{estimation\_phase} {The first estimation phase to allow this to be estimated}
\defSub{mcmc} {Indicates if this parameter is estimated at the point estimate but fixed during MCMC estimation run}
\defSub{transformation} {Type of simple transformation to apply to estimate}
\defSub{transform\_with\_jacobian} {Apply jacobian during transformation}
\defSub{prior\_applies\_to\_transform} {Does the prior apply to the transformed parameter? a legacy switch, see Manual for more information}
\defComLab{estimate\_transformation}{Define an object of type \emph{estimate\_transformation}}\par\par
\defSub{label} {Label for the transformation block}
\defSub{type} {Type of transformation}
\defSub{transform\_with\_jacobian} {Apply jacobian during transformation}
\defComLab{initialisation\_phase}{Define an object of type \emph{initialisation\_phase}}\par\par
\defSub{label} {The label of the initialisation phase}
\defSub{type} {The type of initialisation}
\par\textbf{\commandlabsubarg{initialisation\_\_phase}{type}{iterative}}\par
\defSub{years} {The number of iterations (years) over which to execute this initialisation phase}
\defSub{insert\_processes} {(years) over which to execute this initialisation phase}
\defSub{exclude\_processes} {Processes in the annual cycle to be excluded from this initialisation phase}
\defSub{convergence\_years} {The iteration (year) when the test for converegence (lambda) is evaluated}
\defSub{lambda} {The maximum value of the absolute sum of differences (lambda) between the partition at year-1 and year that indicates successfull convergence}
\defComLab{length\_weight}{Define an object of type \emph{length\_weight}}\par\par
\defSub{label} {The label of the length-weight relationship}
\defSub{type} {The type of the length-weight relationship}
\defComLab{likelihood}{Define an object of type \emph{likelihood}}\par\par
\defComLab{mcmc}{Define an object of type \emph{mcmc}}\par\par
\defSub{label} {The label of the MCMC}
\defSub{type} {The type of MCMC}
\defSub{length} {The number of iterations in for the MCMC chain}
\defSub{active} {Indicates if this is the active MCMC algorithm}
\defSub{print\_default\_reports} {Indicates if the output prints the default reports}
\defSub{step\_size} {Initial stepsize (as a multiplier of the approximate covariance matrix}
\defComLab{minimiser}{Define an object of type \emph{minimiser}}\par\par
\defSub{label} {The minimiser label}
\defSub{type} {The type of minimiser to use}
\defSub{active} {Indicates if this minimiser is active}
\defSub{covariance} {Indicates if a covariance matrix should be generated}
\defComLab{model}{Define an object of type \emph{model}}\par\par
\defSub{start\_year} {Define the first year of the model, immediately following initialisation}
\defSub{final\_year} {Define the final year of the model, excluding years in the projection period}
\defSub{min\_age} {Minimum age of individuals in the population}
\defSub{max\_age} {Maximum age of individuals in the population}
\defSub{age\_plus} {Define the oldest age or extra length midpoint (plus group size) as a plus group}
\defSub{initialisation\_phases} {Define the labels of the phases of the initialisation}
\defSub{time\_steps} {Define the labels of the time steps, in the order that they are applied, to form the annual cycle}
\defSub{projection\_final\_year} {Define the final year of the model in projection mode}
\defSub{length\_bins} {The minimum length in each length bin}
\defSub{length\_plus} {Specify whether there is a length plus group or not}
\defSub{length\_plus\_group} {Mean size of length plus group}
\defSub{base\_weight\_units} {Define the units for the base weight. This will be the default unit of any weight input parameters}
\defComLab{observation}{Define an object of type \emph{observation}}\par\par
\defSub{label} {Label}
\defSub{type} {Type of observation}
\defSub{likelihood} {Type of likelihood to use}
\defSub{categories} {Category labels to use}
\defSub{delta} {Robustification value (delta) for the likelihood}
\defSub{simulation\_likelihood} {Simulation likelihood to use}
\defSub{likelihood\_multiplier} {Likelihood score multiplier}
\defSub{error\_value\_multiplier} {Error value multiplier for likelihood}
\defComLab{penalty}{Define an object of type \emph{penalty}}\par\par
\defSub{label} {The label of the penalty}
\defSub{type} {The type of penalty}
\defComLab{process}{Define an object of type \emph{process}}\par\par
\defSub{label} {The label of the process}
\defSub{type} {The type of process}
\par\textbf{\commandlabsubarg{process}{type}{growth\_basic}}\par
\defSub{categories} {The labels of the categories}
\defSub{number\_of\_growth\_episodes} {Number of growth episodes per year}
\defSub{growth\_time\_steps} {Time step in which each growth episode occurs}
\defSub{cv} {c.v. for the growth model}
\defSub{sigma\_min} {Lower bound on sigma for the growth model}
\par\textbf{\commandlabsubarg{process}{type}{mortality\_constant\_rate}}\par
\defSub{categories} {List of categories labels}
\defSub{m} {Mortality rates}
\defSub{time\_step\_ratio} {Time step ratios for the mortality rates}
\par\textbf{\commandlabsubarg{process}{type}{recruitment\_constant}}\par
\defSub{categories} {Categories}
\defSub{proportions} {Proportions}
\defSub{length\_bins} {The length bins recruits are uniformly distributed over, when recruitment occurs}
\defSub{r0} {R0}
\defComLab{profile}{Define an object of type \emph{profile}}\par\par
\defSub{label} {Label}
\defSub{steps} {The number of steps to take between the lower and upper bound}
\defSub{lower\_bound} {The lower bounds}
\defSub{upper\_bound} {The upper bounds}
\defSub{parameter} {The system parameter to profile}
\defSub{same} {A Parameter that are constrained to have the same value as the parameter being profiled}
\defComLab{project}{Define an object of type \emph{project}}\par\par
\defSub{label} {Label}
\defSub{type} {Type}
\defSub{years} {Years to recalculate the values}
\defSub{parameter} {Parameter to project}
\defSub{multiplier} {Multiplier that is applied to the projected value}
\defComLab{report}{Define an object of type \emph{report}}\par\par
\defSub{label} {The label for the report}
\defSub{type} {The type of report}
\defSub{file\_name} {The File Name if you want this report to be in a seperate file}
\defSub{write\_mode} {The write mode}
\par\textbf{\commandlabsubarg{report}{type}{initialisation\_partition\_mean\_weight}}\par
\par\textbf{\commandlabsubarg{report}{type}{partition}}\par
\defSub{time\_step} {Time Step label}
\defSub{years} {Years}
\par\textbf{\commandlabsubarg{report}{type}{partition\_biomass}}\par
\defSub{time\_step} {Time Step label}
\defSub{years} {Years}
\par\textbf{\commandlabsubarg{report}{type}{partition\_mean\_weight}}\par
\defSub{time\_step} {Time Step label}
\defSub{years} {Years}
\defComLab{selectivity}{Define an object of type \emph{selectivity}}\par\par
\defSub{label} {The label for this selectivity}
\defSub{type} {The type of selectivity}
\defSub{length\_based} {Is the selectivity length based}
\defSub{intervals} {Number of quantiles to evaluate a length based selectivity over the age length distribution}
\defSub{partition\_type} {The type of partition this selectivity will support, Defaults to same as the model}
\defSub{values} {}
\defSub{length\_values} {}
\defComLab{simulate}{Define an object of type \emph{simulate}}\par\par
\defSub{label} {Label}
\defSub{type} {Type}
\defSub{years} {Years to recalculate the values}
\defSub{parameter} {Parameter to Simulate}
\defComLab{time\_step}{Define an object of type \emph{time\_step}}\par\par
\defSub{label} {The label of the timestep}
\defSub{processes} {The labels of the processes for this time step in the order that they occur}
\defComLab{time\_varying}{Define an object of type \emph{time\_varying}}\par\par
\defSub{label} {The time-varying label}
\defSub{type} {The time-varying type}
\defSub{years} {Years in which to vary the values}
\defSub{parameter} {The name of the parameter to time vary}
