\section{Quick reference\label{sec:quick-reference}}
\defComLab{additional\_prior}{Define an object of type \emph{additional\_prior}}\par
\defSub{parameter} {The name of the parameter to generate additional prior on}
\defSub{label} {The label for the additional prior}
\defSub{type} {The type of additional prior}
\defComLab{ageing\_error}{Define an object of type \emph{ageing\_error}}\par\par
\defSub{label} {The label of the ageing error}
\defSub{type} {The type of ageing error}
\par\textbf{\commandlabsubarg{ageing\_\_error}{type}{data}}\par
\par\textbf{\commandlabsubarg{ageing\_\_error}{type}{none}}\par
\par\textbf{\commandlabsubarg{ageing\_\_error}{type}{normal}}\par
\defSub{cv} {CV of the misclassification matrix}
\defSub{k} {k defines the minimum age of individuals which can be misclassified, i.e., individuals of age less than k have no ageing error}
\par\textbf{\commandlabsubarg{ageing\_\_error}{type}{off\_by\_one}}\par
\defSub{p1} {The proportion misclassified as one year younger, e.g., the proportion of age 3 individuals that were misclassified as age 2}
\defSub{p2} {The proportion misclassified as one year older, e.g., the proportion of age 3 individuals that were misclassified as age 4}
\defSub{k} {The minimum age of animals which can be misclassified, i.e., animals of age less than k are assumed to be correctly classified}
\defComLab{age\_length}{Define an object of type \emph{age\_length}}\par\par
\defSub{label} {The label of the age length relationship}
\defSub{type} {The type of age length relationship}
\defSub{time\_step\_proportions} {The fraction of the year applied in each time step that is added to the age for the purposes of evaluating the length, i.e., a value of 0.5 for a time step will evaluate the length of individuals at age+0.5 in that time step}
\defSub{distribution} {The assumed distribution for the growth curve}
\defSub{cv\_first} {The CV for the first age class}
\defSub{cv\_last} {The CV for last age class}
\defSub{casal\_switch} {If true, use the ,less accurate, equation for the cumulative normal function as was used in the legacy version of CASAL.}
\defSub{by\_length} {Specifies if the linear interpolation of CVs is a linear function of mean length at age. Default is by age only}
\par\textbf{\commandlabsubarg{age\_\_length}{type}{data}}\par
\defSub{external\_gaps} {}
\defSub{internal\_gaps} {}
\defSub{length\_weight} {The label from an associated length-weight block}
\defSub{time\_step\_measurements\_were\_made} {The time step label for which size-at-age data are provided}
\par\textbf{\commandlabsubarg{age\_\_length}{type}{none}}\par
\par\textbf{\commandlabsubarg{age\_\_length}{type}{schnute}}\par
\defSub{y1} {The y1 parameter of the Schnute relationship}
\defSub{y2} {The y2 parameter of the Schnute relationship}
\defSub{tau1} {The $\tau_1$ parameter of the Schnute relationship}
\defSub{tau2} {The $\tau_2$ parameter of the Schnute relationship}
\defSub{a} {The $a$ parameter of the Schnute relationship}
\defSub{b} {The $b$ parameter of the Schnute relationship}
\defSub{length\_weight} {The label of the associated length-weight relationship}
\par\textbf{\commandlabsubarg{age\_\_length}{type}{von\_bertalanffy}}\par
\defSub{linf} {The $L_{infinity}$ parameter of the von Bertalanffy relationship}
\defSub{k} {The $k$ parameter of the von Bertalanffy relationship}
\defSub{t0} {The $t_0$ parameter of the von Bertalanffy relationship}
\defSub{length\_weight} {The label of the associated length-weight relationship}
\defComLab{catchability}{Define an object of type \emph{catchability}}\par\par
\defSub{label} {Label of the catchability}
\defSub{type} {The type of catchability}
\defComLab{categories}{Define an object of type \emph{categories}}\par\par
\defSub{format} {The format that the category names use}
\defSub{names} {The names of the categories to be used in the model}
\defSub{years} {The years that individual categories will be active for. This overrides the model values}
\defSub{age\_lengths} {R,The labels of age\_length objects that are assigned to categories,}
\defSub{length\_weight} {R,The labels of the length\_weight objects that are assigned to categories,}
\defSub{age\_weight} {R,The labels of the age\_weight objects that are assigned to categories,}
\defComLab{derived\_quantity}{Define an object of type \emph{derived\_quantity}}\par\par
\defSub{label} {The label of the derived quantity}
\defSub{type} {The type of derived quantity}
\defSub{time\_step} {The time step in which to calculate the derived quantity after}
\defSub{categories} {The list of categories to use when calculating the derived quantity}
\defSub{selectivities} {A list of one selectivity}
\defSub{time\_step\_proportion} {The proportion through the mortality block of the time step when calculated}
\defSub{time\_step\_proportion\_method} {The method for interpolating for the proportion through the mortality block}
\defSub{values} {}
\par\textbf{\commandlabsubarg{derived\_\_quantity}{type}{abundance}}\par
\par\textbf{\commandlabsubarg{derived\_\_quantity}{type}{biomass}}\par
\defComLab{estimate}{Define an object of type \emph{estimate}}\par\par
\defSub{label} {The label of the estimate}
\defSub{type} {The prior type for the estimate}
\defSub{parameter} {The name of the parameter to estimate in the model}
\defSub{lower\_bound} {The lower bound for the parameter}
\defSub{upper\_bound} {The upper bound for the parameter}
\defSub{same} {List of parameters that are constrained to have the same value as this parameter}
\defSub{estimation\_phase} {The first estimation phase to allow this to be estimated}
\defSub{mcmc} {Indicates if this parameter is estimated at the point estimate but fixed during MCMC estimation run}
\defSub{transformation} {Type of simple transformation to apply to estimate}
\defSub{transform\_with\_jacobian} {Apply jacobian during transformation}
\defSub{prior\_applies\_to\_transform} {Does the prior apply to the transformed parameter? a legacy switch, see Manual for more information}
\defComLab{estimate\_transformation}{Define an object of type \emph{estimate\_transformation}}\par\par
\defSub{label} {Label for the transformation block}
\defSub{type} {The type of transformation}
\defSub{transform\_with\_jacobian} {Apply Jacobian during transformation}
\defComLab{initialisation\_phase}{Define an object of type \emph{initialisation\_phase}}\par\par
\defSub{label} {The label of the initialisation phase}
\defSub{type} {The type of initialisation}
\par\textbf{\commandlabsubarg{initialisation\_\_phase}{type}{cinitial}}\par
\defSub{categories} {The list of categories for the Cinitial initialisation}
\par\textbf{\commandlabsubarg{initialisation\_\_phase}{type}{derived}}\par
\defSub{insert\_processes} {Additional processes not defined in the annual cycle that are to be inserted into this initialisation phase}
\defSub{exclude\_processes} {Processes in the annual cycle to be excluded from this initialisation phase}
\defSub{casal\_initialisation\_switch} {Run an extra annual cycle to evalaute equilibrium SSBs. Warning - if true, this may not correctly evaluate the equilibrium state. Set to true if replicating a CASAL model}
\par\textbf{\commandlabsubarg{initialisation\_\_phase}{type}{iterative}}\par
\defSub{years} {The number of iterations ,years, over which to execute this initialisation phase}
\defSub{insert\_processes} {The processes in the annual cycle to be included in this initialisation phase}
\defSub{exclude\_processes} {The processes in the annual cycle to be excluded from this initialisation phase}
\defSub{convergence\_years} {The iteration ,year, when the test for converegence ,lambda, is evaluated}
\defSub{lambda} {The maximum value of the absolute sum of differences ,lambda, between the partition at year-1 and year that indicates successful convergence}
\par\textbf{\commandlabsubarg{initialisation\_\_phase}{type}{state\_category\_by\_age}}\par
\defSub{categories} {The list of categories for the category state initialisation}
\defSub{min\_age} {The minimum age of values supplied in the definition of the category state}
\defSub{max\_age} {The maximum age of values supplied in the definition of the category state}
\defComLab{length\_weight}{Define an object of type \emph{length\_weight}}\par\par
\defSub{label} {The label of the length-weight relationship}
\defSub{type} {The type of the length-weight relationship}
\defComLab{likelihood}{Define an object of type \emph{likelihood}}\par\par
\defComLab{mcmc}{Define an object of type \emph{mcmc}}\par\par
\defSub{label} {The label of the MCMC}
\defSub{type} {The type of MCMC}
\defSub{length} {The number of iterations in for the MCMC chain}
\defSub{active} {Indicates if this is the active MCMC algorithm}
\defSub{step\_size} {Initial stepsize ,as a multiplier of the approximate covariance matrix,}
\defComLab{minimiser}{Define an object of type \emph{minimiser}}\par\par
\defSub{label} {The minimiser label}
\defSub{type} {The type of minimiser to use}
\defSub{active} {Indicates if this minimiser is active}
\defSub{covariance} {Indicates if a covariance matrix should be generated}
\defComLab{model}{Define an object of type \emph{model}}\par\par
\defSub{start\_year} {The first year of the model, immediately following initialisation}
\defSub{final\_year} {The final year of the model, excluding years in the projection period}
\defSub{min\_age} {The minimum age of individuals in the population}
\defSub{max\_age} {The maximum age of individuals in the population}
\defSub{age\_plus} {The oldest age or extra length midpoint ,plus group size, as a plus group}
\defSub{initialisation\_phases} {The labels of the phases of the initialisation}
\defSub{time\_steps} {The labels of the time steps, in the order that they are applied, to form the annual cycle}
\defSub{projection\_final\_year} {The final year of the model in projection mode}
\defSub{length\_bins} {The minimum length in each length bin}
\defSub{length\_plus} {Specify whether there is a length plus group or not}
\defSub{length\_plus\_group} {Mean length of length plus group}
\defSub{base\_weight\_units} {The units for the base weight. This will be the default unit of any weight input parameters}
\defComLab{observation}{Define an object of type \emph{observation}}\par\par
\defSub{label} {The label of the observation}
\defSub{type} {The type of observation}
\defSub{likelihood} {The type of likelihood to use}
\defSub{categories} {The category labels to use}
\defSub{delta} {The robustification value ,delta, for the likelihood}
\defSub{simulation\_likelihood} {The simulation likelihood to use}
\defSub{likelihood\_multiplier} {The likelihood score multiplier}
\defSub{error\_value\_multiplier} {The error value multiplier for likelihood}
\par\textbf{\commandlabsubarg{observation}{type}{abundance}}\par
\defSub{selectivities} {The labels of the selectivities}
\defSub{time\_step} {The label of the time step that the observation occurs in}
\par\textbf{\commandlabsubarg{observation}{type}{biomass}}\par
\defSub{catchability} {The time step of the observation}
\defSub{time\_step} {The label of the time step that the observation occurs in}
\defSub{obs} {The observed values}
\defSub{years} {The years of the observed values}
\defSub{error\_value} {The error values of the observed values ,note that the units depend on the likelihood,}
\defSub{selectivities} {The labels of the selectivities}
\defSub{process\_error} {The process error}
\defSub{age\_weight\_labels} {R,The labels for the \command{$age\_weight$} block which corresponds to each category, to use the weight calculation method for biomass calculations,}
\par\textbf{\commandlabsubarg{observation}{type}{process\_removals\_by\_age}}\par
\defSub{min\_age} {Minimum age}
\defSub{max\_age} {Maximum age}
\defSub{plus\_group} {Use age plus group}
\defSub{time\_step} {The label of time-step that the observation occurs in}
\defSub{tolerance} {Tolerance}
\defSub{years} {Years for which there are observations}
\defSub{process\_errors} {Label of process error to use}
\defSub{ageing\_error} {Label of ageing error to use}
\defSub{method\_of\_removal} {Label of observed method of removals}
\defSub{mortality\_instantaneous\_process} {The label of the mortality instantaneous process for the observation}
\par\textbf{\commandlabsubarg{observation}{type}{process\_removals\_by\_age\_retained}}\par
\defSub{min\_age} {The minimum age}
\defSub{max\_age} {The maximum age}
\defSub{plus\_group} {Use the age plus group?}
\defSub{time\_step} {The label of the time step that the observation occurs in}
\defSub{tolerance} {The tolerance}
\defSub{years} {The years for which there are observations}
\defSub{process\_errors} {The label of the process error to use}
\defSub{ageing\_error} {The label of the ageing error to use}
\defSub{method\_of\_removal} {The label of observed method of removals}
\defSub{mortality\_instantaneous\_process} {The label of the mortality instantaneous process for the observation}
\par\textbf{\commandlabsubarg{observation}{type}{process\_removals\_by\_age\_retained\_total}}\par
\defSub{min\_age} {The minimum age}
\defSub{max\_age} {The maximum age}
\defSub{plus\_group} {Use the age plus group?}
\defSub{time\_step} {The label of the time step that the observation occurs in}
\defSub{tolerance} {The tolerance}
\defSub{years} {The years for which there are observations}
\defSub{process\_errors} {The label of the process error to use}
\defSub{ageing\_error} {The label of the ageing error to use}
\defSub{method\_of\_removal} {The label of observed method of removals}
\defSub{mortality\_instantaneous\_process} {The label of the mortality instantaneous process for the observation}
\par\textbf{\commandlabsubarg{observation}{type}{process\_removals\_by\_length}}\par
\defSub{time\_step} {The time step to execute in}
\defSub{tolerance} {The tolerance for rescaling proportions}
\defSub{years} {The years for which there are observations}
\defSub{process\_errors} {The process error}
\defSub{method\_of\_removal} {The label of observed method of removals}
\defSub{length\_bins} {The length bins}
\defSub{length\_plus} {Is the last length bin a plus group?}
\defSub{mortality\_instantaneous\_process} {The label of the mortality instantaneous process for the observation}
\par\textbf{\commandlabsubarg{observation}{type}{process\_removals\_by\_length\_retained}}\par
\defSub{time\_step} {The time step to execute in}
\defSub{tolerance} {The tolerance for rescaling proportions}
\defSub{years} {The years for which there are observations}
\defSub{process\_errors} {The process error}
\defSub{method\_of\_removal} {The label of observed method of removals}
\defSub{length\_bins} {The length bins}
\defSub{length\_plus} {Is the last length bin a plus group?}
\defSub{mortality\_instantaneous\_process} {The label of the mortality instantaneous process for the observation}
\par\textbf{\commandlabsubarg{observation}{type}{process\_removals\_by\_length\_retained\_total}}\par
\defSub{time\_step} {The time step to execute in}
\defSub{tolerance} {The tolerance for rescaling proportions}
\defSub{years} {The years for which there are observations}
\defSub{process\_errors} {The process error}
\defSub{method\_of\_removal} {The label of observed method of removals}
\defSub{length\_bins} {The length bins}
\defSub{length\_plus} {Is the last length bin a plus group?}
\defSub{mortality\_instantaneous\_process} {The label of the mortality instantaneous process for the observation}
\par\textbf{\commandlabsubarg{observation}{type}{proportions\_at\_age}}\par
\defSub{min\_age} {The minimum age}
\defSub{max\_age} {The maximum age}
\defSub{plus\_group} {Use the age plus group?}
\defSub{time\_step} {The label of the time step that the observation occurs in}
\defSub{tolerance} {The tolerance on the constraint that for each year the sum of proportions in each age must equal 1, e.g., if tolerance = 0.1 then 1 - Sum,Proportions, can be as great as 0.1}
\defSub{years} {The years of the observed values}
\defSub{selectivities} {The labels of the selectivities}
\defSub{process\_errors} {The process error}
\defSub{ageing\_error} {The label of ageing error to use}
\par\textbf{\commandlabsubarg{observation}{type}{proportions\_at\_length}}\par
\defSub{time\_step} {The label of the time step that the observation occurs in}
\defSub{tolerance} {The tolerance for rescaling proportions}
\defSub{years} {The years for which there are observations}
\defSub{selectivities} {The labels of the selectivities}
\defSub{process\_errors} {The process error}
\defSub{length\_bins} {The length bins}
\par\textbf{\commandlabsubarg{observation}{type}{proportions\_by\_category}}\par
\defSub{min\_age} {The minimum age}
\defSub{max\_age} {The maximum age}
\defSub{time\_step} {The label of the time step that the observation occurs in}
\defSub{plus\_group} {Use the age plus group?}
\defSub{years} {The years for which there are observations}
\defSub{selectivities} {The labels of the selectivities}
\defSub{categories2} {The target categories}
\defSub{selectivities2} {The target selectivities}
\par\textbf{\commandlabsubarg{observation}{type}{proportions\_mature\_by\_age}}\par
\defSub{min\_age} {The minimum age}
\defSub{max\_age} {The maximum age}
\defSub{time\_step} {The label of time-step that the observation occurs in}
\defSub{plus\_group} {Use the age plus group?}
\defSub{years} {The years for which there are observations}
\defSub{ageing\_error} {The label of ageing error to use}
\defSub{total\_categories} {All category labels that were vulnerable to sampling at the time of this observation ,not including the categories already given,}
\defSub{time\_step\_proportion} {The proportion through the mortality block of the time step when the observation is evaluated}
\par\textbf{\commandlabsubarg{observation}{type}{proportions\_migrating}}\par
\defSub{min\_age} {The minimum age}
\defSub{max\_age} {The maximum age}
\defSub{time\_step} {The label of the time step that the observation occurs in}
\defSub{plus\_group} {Use the age plus group?}
\defSub{years} {The years for which there are observations}
\defSub{process\_errors} {The process error}
\defSub{ageing\_error} {The label of the ageing error to use}
\defSub{process} {The process label}
\par\textbf{\commandlabsubarg{observation}{type}{tag\_recapture\_by\_age}}\par
\defSub{min\_age} {The minimum age}
\defSub{max\_age} {The maximum age}
\defSub{plus\_group} {Use the age plus group?}
\defSub{years} {The years for which there are observations}
\defSub{categories2} {The available categories in the partition}
\defSub{selectivities} {The labels of the selectivities}
\defSub{time\_step} {The label of the time step that the observation occurs in}
\defSub{selectivities2} {The categories of tagged individuals for the observation}
\defSub{detection} {The probability of detecting a recaptured individual}
\defSub{time\_step\_proportion} {The proportion through the mortality block of the time step when the observation is evaluated}
\par\textbf{\commandlabsubarg{observation}{type}{tag\_recapture\_by\_length}}\par
\defSub{years} {The years for which there are observations}
\defSub{length\_bins} {The length bins}
\defSub{length\_plus} {Is the last length bin a plus group?}
\defSub{selectivities} {The labels of the selectivities used for untagged categories}
\defSub{tagged\_selectivities} {The labels of the tag category selectivities}
\defSub{detection} {The probability of detecting a recaptured individual}
\defSub{dispersion} {The over-dispersion parameter ,phi,}
\defSub{time\_step\_proportion} {The proportion through the mortality block of the time step when the observation is evaluated}
\defComLab{penalty}{Define an object of type \emph{penalty}}\par\par
\defSub{label} {The label of the penalty}
\defSub{type} {The type of penalty}
\defComLab{process}{Define an object of type \emph{process}}\par\par
\defSub{label} {The label of the process}
\defSub{type} {The type of process}
\par\textbf{\commandlabsubarg{process}{type}{ageing}}\par
\defSub{categories} {The labels of the categories}
\par\textbf{\commandlabsubarg{process}{type}{growth\_basic}}\par
\defSub{categories} {The labels of the categories}
\defSub{number\_of\_growth\_episodes} {The number of growth episodes per year}
\defSub{growth\_time\_steps} {The time step in which each growth episode occurs}
\defSub{cv} {C.V. for the growth model}
\defSub{sigma\_min} {The lower bound on sigma for the growth model}
\par\textbf{\commandlabsubarg{process}{type}{maturation}}\par
\defSub{from} {The list of categories to mature from}
\defSub{to} {The list of categories to mature to}
\defSub{selectivities} {The list of selectivities to use for maturation}
\defSub{years} {The years to be associated with the maturity rates}
\defSub{rates} {The rates to mature for each year}
\par\textbf{\commandlabsubarg{process}{type}{mortality\_constant\_rate}}\par
\defSub{categories} {List of categories labels}
\defSub{m} {Mortality rates}
\defSub{time\_step\_ratio} {Time step ratios for the mortality rates}
\par\textbf{\commandlabsubarg{process}{type}{mortality\_event}}\par
\defSub{categories} {Categories}
\defSub{years} {Years in which to apply the mortality process}
\defSub{catches} {The number of removals ,catches, to apply for each year}
\defSub{u\_max} {The maximum exploitation rate ,$Umax$,}
\defSub{selectivities} {The list of selectivities}
\defSub{penalty} {The label of the penalty to apply if the total number of removals cannot be taken}
\par\textbf{\commandlabsubarg{process}{type}{mortality\_event\_biomass}}\par
\defSub{categories} {Category labels}
\defSub{selectivities} {The labels of the selectivities for each of the categories}
\defSub{years} {The years in which to apply the mortality process}
\defSub{catches} {The biomass of removals ,catches, to apply for each year}
\defSub{u\_max} {The maximum exploitation rate ,$Umax$,}
\defSub{penalty} {The label of the penalty to apply if the total biomass of removals cannot be taken}
\par\textbf{\commandlabsubarg{process}{type}{mortality\_holling\_rate}}\par
\defSub{prey\_categories} {Prey Categories labels}
\defSub{predator\_categories} {Predator Categories labels}
\defSub{is\_abundance} {Is vulnerable amount of prey and predator an abundance [true] or biomass [false]}
\defSub{a} {Parameter a}
\defSub{b} {Parameter b}
\defSub{x} {This parameter controls the functional form, Holling function type 2 ,x=2, or 3 ,x=3, or generalised ,Michaelis Menten, x,=1,}
\defSub{u\_max} {The maximum exploitation rate ,$Umax$,}
\defSub{prey\_selectivities} {Selectivities for prey categories}
\defSub{predator\_selectivities} {Selectivities for predator categories}
\defSub{penalty} {Label of penalty to be applied}
\defSub{years} {Years in which to apply the mortality process}
\par\textbf{\commandlabsubarg{process}{type}{mortality\_initialisation\_event}}\par
\defSub{categories} {Categories}
\defSub{catch} {The number of removals ,catches, to apply for each year}
\defSub{u\_max} {The maximum exploitation rate ,$Umax$,}
\defSub{selectivities} {The list of selectivities}
\defSub{penalty} {The label of the penalty to apply if the total number of removals cannot be taken}
\par\textbf{\commandlabsubarg{process}{type}{mortality\_initialisation\_event\_biomass}}\par
\defSub{categories} {Categories}
\defSub{catch} {The number of removals ,catches, to apply for each year}
\defSub{u\_max} {The maximum exploitation rate ,$Umax$,}
\defSub{selectivities} {The list of selectivities}
\defSub{penalty} {The label of the penalty to apply if the total number of removals cannot be taken}
\par\textbf{\commandlabsubarg{process}{type}{mortality\_instantaneous}}\par
\defSub{categories} {The categories for instantaneous mortality}
\defSub{m} {The natural mortality rates for each category}
\defSub{time\_step\_ratio} {The time step ratios for natural mortality}
\defSub{selectivities} {The selectivities to apply on the categories for natural mortality}
\par\textbf{\commandlabsubarg{process}{type}{mortality\_instantaneous\_retained}}\par
\defSub{categories} {The categories for instantaneous mortality}
\defSub{m} {The natural mortality rates for each category}
\defSub{time\_step\_ratio} {The time step ratios for natural mortality}
\defSub{selectivities} {The selectivities to apply on the categories for natural mortality}
\par\textbf{\commandlabsubarg{process}{type}{mortality\_prey\_suitability}}\par
\defSub{prey\_categories} {Prey Categories labels}
\defSub{predator\_categories} {Predator Categories labels}
\defSub{consumption\_rate} {Predator consumption rate}
\defSub{electivities} {Prey Electivities}
\defSub{u\_max} {U max}
\defSub{prey\_selectivities} {Selectivities for prey categories}
\defSub{predator\_selectivities} {Selectivities for predator categories}
\defSub{penalty} {Label of penalty to be applied}
\defSub{years} {Year that process occurs}
\par\textbf{\commandlabsubarg{process}{type}{recruitment\_beverton\_holt}}\par
\defSub{categories} {Category labels}
\defSub{r0} {R0}
\defSub{b0} {B0}
\defSub{proportions} {Proportions}
\defSub{age} {Age at recruitment}
\defSub{ssb\_offset} {Spawning biomass year offset}
\defSub{steepness} {Steepness}
\defSub{ssb} {SSB label ,derived quantity,}
\defSub{b0\_initialisation\_phase} {Initialisation phase label that B0 is from}
\defSub{ycs\_values} {YCS Values}
\defSub{ycs\_years} {Recruitment years. A vector of years that relates to the year of the spawning event that created this cohort}
\defSub{standardise\_ycs\_years} {Years that are included for year class standardisation}
\par\textbf{\commandlabsubarg{process}{type}{recruitment\_beverton\_holt\_with\_deviations}}\par
\defSub{categories} {Category labels}
\defSub{r0} {R0}
\defSub{b0} {B0}
\defSub{proportions} {Proportions}
\defSub{age} {Age to recruit at}
\defSub{ssb\_offset} {Spawning biomass year offset}
\defSub{steepness} {Steepness}
\defSub{ssb} {SSB Label ,derived quantity,}
\defSub{sigma\_r} {Sigma R}
\defSub{b\_max} {Maximum bias adjustment}
\defSub{last\_year\_with\_no\_bias} {Last year with no bias adjustment}
\defSub{first\_year\_with\_bias} {First year with full bias adjustment}
\defSub{last\_year\_with\_bias} {Last year with full bias adjustment}
\defSub{first\_recent\_year\_with\_no\_bias} {First recent year with no bias adjustment}
\defSub{b0\_initialisation\_phase} {Initialisation phase label that b0 is from}
\defSub{deviation\_values} {Recruitment deviation values}
\defSub{deviation\_years} {Recruitment years. A vector of years that relates to the year of the spawning event that created this cohort}
\par\textbf{\commandlabsubarg{process}{type}{recruitment\_constant}}\par
\defSub{categories} {Categories}
\defSub{proportions} {Proportions}
\defSub{length\_bins} {The length bins that recruits are uniformly distributed over at the time of recruitment}
\defSub{r0} {R0}
\par\textbf{\commandlabsubarg{process}{type}{survival\_constant\_rate}}\par
\defSub{categories} {List of categories}
\defSub{s} {Survival rates}
\defSub{time\_step\_ratio} {Time step ratios for S}
\defSub{selectivities} {Selectivity label}
\par\textbf{\commandlabsubarg{process}{type}{tag\_by\_age}}\par
\defSub{from} {Categories to transition from}
\defSub{to} {Categories to transition to}
\defSub{min\_age} {Minimum age to transition}
\defSub{max\_age} {Maximum age to transition}
\defSub{penalty} {Penalty label}
\defSub{u\_max} {U Max}
\defSub{years} {Years to execute the transition in}
\defSub{initial\_mortality} {}
\defSub{initial\_mortality\_selectivity} {}
\defSub{loss\_rate} {}
\defSub{loss\_rate\_selectivities} {}
\defSub{selectivities} {}
\defSub{n} {}
\par\textbf{\commandlabsubarg{process}{type}{tag\_by\_length}}\par
\defSub{from} {Categories to transition from}
\defSub{to} {Categories to transition to}
\defSub{penalty} {Penalty label}
\defSub{u\_max} {U Max}
\defSub{initial\_mortality} {}
\defSub{initial\_mortality\_selectivity} {}
\defSub{selectivities} {}
\defSub{n} {}
\par\textbf{\commandlabsubarg{process}{type}{tag\_loss}}\par
\defSub{categories} {List of categories}
\defSub{tag\_loss\_rate} {Tag Loss rates}
\defSub{time\_step\_ratio} {Time step ratios for Tag Loss}
\defSub{tag\_loss\_type} {Type of tag loss}
\defSub{selectivities} {Selectivities}
\defSub{year} {The year the first tagging release process was executed}
\par\textbf{\commandlabsubarg{process}{type}{transition\_category}}\par
\defSub{from} {From category}
\defSub{to} {To category}
\defSub{proportions} {Proportions}
\defSub{selectivities} {Selectivity names}
\par\textbf{\commandlabsubarg{process}{type}{transition\_category\_by\_age}}\par
\defSub{from} {Categories to transition from}
\defSub{to} {Categories to transition to}
\defSub{min\_age} {Minimum age to transition}
\defSub{max\_age} {Maximum age to transition}
\defSub{penalty} {Penalty label}
\defSub{u\_max} {U max}
\defSub{years} {Years to execute the transition in}
\defComLab{profile}{Define an object of type \emph{profile}}\par\par
\defSub{label} {The label of the profile}
\defSub{steps} {The number of steps between the lower and upper bound}
\defSub{lower\_bound} {The lower bounds}
\defSub{upper\_bound} {The upper bounds}
\defSub{parameter} {The system parameter to profile}
\defSub{same} {A parameter that is constrained to have the same value as the parameter being profiled}
\defComLab{project}{Define an object of type \emph{project}}\par\par
\defSub{label} {Label}
\defSub{type} {Type}
\defSub{years} {Years to recalculate the values}
\defSub{parameter} {Parameter to project}
\defSub{multiplier} {Multiplier that is applied to the projected value}
\defComLab{report}{Define an object of type \emph{report}}\par\par
\defSub{label} {The label for the report}
\defSub{type} {The type of report}
\defSub{file\_name} {The filename for this report to be in a separate file}
\defSub{write\_mode} {The write mode}
\par\textbf{\commandlabsubarg{report}{type}{age\_length}}\par
\defSub{time\_step} {Time Step label}
\defSub{years} {Years}
\defSub{age\_length} {}
\defSub{category} {}
\par\textbf{\commandlabsubarg{report}{type}{ageing\_error\_matrix}}\par
\defSub{ageing\_error} {Ageing Error label}
\par\textbf{\commandlabsubarg{report}{type}{initialisation\_partition\_mean\_weight}}\par
\par\textbf{\commandlabsubarg{report}{type}{partition}}\par
\defSub{time\_step} {Time Step label}
\defSub{years} {Years}
\par\textbf{\commandlabsubarg{report}{type}{partition\_biomass}}\par
\defSub{time\_step} {Time Step label}
\defSub{years} {Years}
\par\textbf{\commandlabsubarg{report}{type}{partition\_mean\_length}}\par
\defSub{time\_step} {Time Step label}
\defSub{years} {Years}
\par\textbf{\commandlabsubarg{report}{type}{partition\_mean\_weight}}\par
\defSub{time\_step} {Time Step label}
\defSub{years} {Years}
\par\textbf{\commandlabsubarg{report}{type}{partition\_year\_cross\_age\_matrix}}\par
\defComLab{selectivity}{Define an object of type \emph{selectivity}}\par\par
\defSub{label} {The label for this selectivity}
\defSub{type} {The type of selectivity}
\defSub{length\_based} {Is the selectivity length based}
\defSub{intervals} {Number of quantiles to evaluate a length based selectivity over the age length distribution}
\defSub{partition\_type} {The type of partition this selectivity will support. Defaults to same as the model}
\defSub{values} {}
\defSub{length\_values} {}
\defComLab{simulate}{Define an object of type \emph{simulate}}\par\par
\defSub{label} {Label}
\defSub{type} {Type}
\defSub{years} {Years to recalculate the values}
\defSub{parameter} {Parameter to simulate}
\defComLab{time\_step}{Define an object of type \emph{time\_step}}\par\par
\defSub{label} {The label of the timestep}
\defSub{processes} {The labels of the processes for this time step in the order that they occur}
\defComLab{time\_varying}{Define an object of type \emph{time\_varying}}\par\par
\defSub{label} {The time-varying label}
\defSub{type} {The time-varying type}
\defSub{years} {Years in which to vary the values}
\defSub{parameter} {The name of the parameter to time vary}
