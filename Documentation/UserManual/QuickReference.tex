\section{Quick reference\label{sec:quick-reference}}
\defComLab{Additional\_Prior}{Define an object of type \emph{Additional\_Prior}}. \defRef{methods:AdditionalPrior}\par
\defSub{parameter} {The name of the parameter to generate additional prior on}
\defSub{label} {The label for the additional prior}
\defSub{type} {The type of additional prior}
\par\textbf\subsubsection{Additional\_Prior of type Beta}\par
\defSub{mu} {Beta distribution mean $\mu$ parameter}
\defSub{sigma} {Beta distribution variance $\sigma$ parameter}
\defSub{a} {Beta distribution lower bound, of the range $A$ parameter}
\defSub{b} {Beta distribution upper bound of the range $B$ parameter}
\par\textbf\subsubsection{Additional\_Prior of type Element\_Difference}\par
\defSub{second\_parameter} {The name of the second parameter for comparing}
\defSub{multiplier} {Multiply the penalty by this factor}
\par\textbf\subsubsection{Additional\_Prior of type Log\_Normal}\par
\defSub{mu} {The lognormal prior mean (mu) parameter}
\defSub{cv} {The lognormal variance (cv) parameter}
\par\textbf\subsubsection{Additional\_Prior of type Uniform\_Log}\par
\par\textbf\subsubsection{Additional\_Prior of type Vector\_Average}\par
\defSub{method} {Which calculation method to use: k, l, or m}
\defSub{k} {The k value to use in the calculation}
\defSub{multiplier} {Multiplier for the penalty amount}
\par\textbf\subsubsection{Additional\_Prior of type Vector\_Smoothing}\par
\defSub{log\_scale} {Should the sums of squares be calculated on the log scale?}
\defSub{multiplier} {Multiply the penalty by this factor}
\defSub{lower\_bound} {The first element to apply the penalty to in the vector}
\defSub{upper\_bound} {The last element to apply the penalty to in the vector}
\defSub{r} {Penalty applied to rth differences}
\defComLab{Ageing\_Error}{Define an object of type \emph{Ageing\_Error}}. \defRef{methods:AgeingError}\par\par
\defSub{label} {The label of the ageing error}
\defSub{type} {The type of ageing error}
\par\textbf\subsubsection{Ageing\_Error of type Data}\par
\par\textbf\subsubsection{Ageing\_Error of type None}\par
\par\textbf\subsubsection{Ageing\_Error of type Normal}\par
\defSub{cv} {CV of the misclassification matrix}
\defSub{k} {k defines the minimum age of individuals which can be misclassified, i.e., individuals of age less than k have no ageing error}
\par\textbf\subsubsection{Ageing\_Error of type Off\_By\_One}\par
\defSub{p1} {The proportion misclassified as one year younger, e.g., the proportion of age 3 individuals that were misclassified as age 2}
\defSub{p2} {The proportion misclassified as one year older, e.g., the proportion of age 3 individuals that were misclassified as age 4}
\defSub{k} {The minimum age of animals which can be misclassified, i.e., animals of age less than k are assumed to be correctly classified}
\defComLab{Age\_Length}{Define an object of type \emph{Age\_Length}}. \defRef{methods:AgeLength}\par\par
\defSub{label} {The label of the age length relationship}
\defSub{type} {The type of age length relationship}
\defSub{time\_step\_proportions} {The fraction of the year applied in each time step that is added to the age for the purposes of evaluating the length, i.e., a value of 0.5 for a time step will evaluate the length of individuals at age+0.5 in that time step}
\defSub{distribution} {The assumed distribution for the growth curve}
\defSub{cv\_first} {The CV for the first age class}
\defSub{cv\_last} {The CV for last age class}
\defSub{casal\_switch} {If true, use the (less accurate) equation for the cumulative normal function as was used in the legacy version of CASAL.}
\defSub{by\_length} {Specifies if the linear interpolation of CVs is a linear function of mean length at age. Default is by age only}
\par\textbf\subsubsection{Age\_Length of type Data}\par
\defSub{external\_gaps} {The method to use for external data gaps}
\defSub{internal\_gaps} {The method to use for internal data gaps}
\defSub{length\_weight} {The label from an associated length-weight block}
\defSub{time\_step\_measurements\_were\_made} {The time step label for which size-at-age data are provided}
\par\textbf\subsubsection{Age\_Length of type None}\par
\par\textbf\subsubsection{Age\_Length of type Schnute}\par
\defSub{y1} {The $y\_1$ parameter}
\defSub{y2} {The $y\_2$ parameter}
\defSub{tau1} {The $\tau\_1$ parameter}
\defSub{tau2} {The $\tau\_2$ parameter}
\defSub{a} {The $a$ parameter}
\defSub{b} {The $b$ parameter}
\defSub{length\_weight} {The label of the associated length-weight relationship}
\par\textbf\subsubsection{Age\_Length of type Von\_Bertalanffy}\par
\defSub{linf} {The $L\_{infinity}$ parameter}
\defSub{k} {The $k$ parameter}
\defSub{t0} {The $t\_0$ parameter}
\defSub{length\_weight} {The label of the associated length-weight relationship}
\defComLab{Catchability}{Define an object of type \emph{Catchability}}. \defRef{methods:Catchability}\par\par
\defSub{label} {Label of the catchability}
\defSub{type} {The type of catchability}
\par\textbf\subsubsection{Catchability of type Free}\par
\defSub{q} {The value of the catchability}
\par\textbf\subsubsection{Catchability of type Nuisance}\par
\defSub{lower\_bound} {The upper bound for nuisance catchability}
\defSub{upper\_bound} {The lower bound for nuisance catchability}
\defSub{q} {}
\defComLab{Categories}{Define an object of type \emph{Categories}}. \defRef{methods:Categories}\par\par
\defSub{format} {The format that the category names use}
\defSub{names} {The names of the categories to be used in the model}
\defSub{years} {The years that individual categories will be active for. This overrides the model values}
\defSub{age\_lengths} {R(The labels of age\_length objects that are assigned to categories)}
\defSub{length\_weight} {R(The labels of the length\_weight objects that are assigned to categories)}
\defSub{age\_weight} {R(The labels of the age\_weight objects that are assigned to categories)}
\defComLab{Derived\_Quantity}{Define an object of type \emph{Derived\_Quantity}}. \defRef{methods:DerivedQuantity}\par\par
\defSub{label} {The label of the derived quantity}
\defSub{type} {The type of derived quantity}
\defSub{time\_step} {The time step in which to calculate the derived quantity after}
\defSub{categories} {The list of categories to use when calculating the derived quantity}
\defSub{selectivities} {A list of one selectivity}
\defSub{time\_step\_proportion} {The proportion through the mortality block of the time step when calculated}
\defSub{time\_step\_proportion\_method} {The method for interpolating for the proportion through the mortality block}
\defSub{values} {}
\par\textbf\subsubsection{Derived\_Quantity of type Abundance}\par
\par\textbf\subsubsection{Derived\_Quantity of type Biomass}\par
\defSub{age\_weight\_labels} {The labels for the @age\_weight block which corresponds to each category, to use that weight calculation method for biomass calculations}
\defComLab{Estimate}{Define an object of type \emph{Estimate}}. \defRef{methods:Estimate}\par\par
\defSub{label} {The label of the estimate}
\defSub{type} {The prior type for the estimate}
\defSub{parameter} {The name of the parameter to estimate in the model}
\defSub{lower\_bound} {The lower bound for the parameter}
\defSub{upper\_bound} {The upper bound for the parameter}
\defSub{same} {List of parameters that are constrained to have the same value as this parameter}
\defSub{estimation\_phase} {The first estimation phase to allow this to be estimated}
\defSub{mcmc} {Indicates if this parameter is estimated at the point estimate but fixed during MCMC estimation run}
\defSub{transformation} {Type of simple transformation to apply to estimate}
\defSub{transform\_with\_jacobian} {Apply jacobian during transformation}
\defSub{prior\_applies\_to\_transform} {Does the prior apply to the transformed parameter? a legacy switch, see Manual for more information}
\par\textbf\subsubsection{Estimate of type Beta}\par
\defSub{mu} {Beta prior  mean (mu) parameter}
\defSub{sigma} {Beta prior variance (sigma) parameter}
\defSub{a} {Beta prior lower bound of the range (A) parameter}
\defSub{b} {Beta prior upper bound of the range (B) parameter}
\par\textbf\subsubsection{Estimate of type Lognormal}\par
\defSub{mu} {The lognormal prior mean (mu) parameter}
\defSub{cv} {The lognormal variance (cv) parameter}
\par\textbf\subsubsection{Estimate of type Normal}\par
\defSub{mu} {The normal prior mean (mu) parameter}
\defSub{cv} {The normal variance (standard devation) parameter}
\par\textbf\subsubsection{Estimate of type Normal\_By\_Stdev}\par
\defSub{mu} {The normal prior mean (mu) parameter}
\defSub{sigma} {The normal variance (sigma) parameter}
\defSub{lognormal\_transformation} {Add a Jacobian if the derived outcome of the estimate is assumed to be lognormal, e.g., used for rectuitment deviations in the recruitment process. See the User Manual for more information}
\par\textbf\subsubsection{Estimate of type Normal\_Log}\par
\defSub{mu} {The normal-log prior mean (mu) parameter}
\defSub{sigma} {The normal-log prior variance (standard deviation) parameter}
\par\textbf\subsubsection{Estimate of type Uniform}\par
\par\textbf\subsubsection{Estimate of type Uniform\_Log}\par
\defComLab{Estimate\_Transformation}{Define an object of type \emph{Estimate\_Transformation}}. \defRef{methods:EstimateTransformation}\par\par
\defSub{label} {Label for the transformation block}
\defSub{type} {The type of transformation}
\defSub{transform\_with\_jacobian} {Apply Jacobian during transformation}
\par\textbf\subsubsection{Estimate\_Transformation of type Average\_Difference}\par
\defSub{theta2} {The label of the @estimate block relating to the $\theta\_2$ parameter in the transformation. See the User Manual for more information}
\defSub{theta1} {The label of @estimate block relating to the $\theta\_1$ parameter in the transformation. See the User Manual for more information}
\par\textbf\subsubsection{Estimate\_Transformation of type Inverse}\par
\defSub{estimate\_label} {The label of estimate block to apply transformation. Defined as $\theta\_1$ in the documentation}
\par\textbf\subsubsection{Estimate\_Transformation of type Log}\par
\defSub{estimate\_label} {Label of estimate block to apply transformation. Defined as $\theta\_1$ in the documentation}
\par\textbf\subsubsection{Estimate\_Transformation of type Log\_Sum}\par
\defSub{theta2} {The label of the @estimate block relating to the $\theta\_2$ parameter in the transformation. See the User Manual for more information}
\defSub{theta1} {The label of @estimate block relating to the $\theta\_1$ parameter in the transformation. See the User Manual for more information}
\par\textbf\subsubsection{Estimate\_Transformation of type Orthogonal}\par
\defSub{theta2} {The label of the @estimate block relating to the $\theta\_2$ parameter in the transformation. See the User Manual for more information}
\defSub{theta1} {The label of @estimate block relating to the $\theta\_1$ parameter in the transformation. See the User Manual for more information}
\par\textbf\subsubsection{Estimate\_Transformation of type Simplex}\par
\defSub{lower\_bound} {The empirical lower bound for the simplex transformed. This should be -Inf but some of the minimisers do not allow that}
\defSub{upper\_bound} {The empirical upper bound for the simplex transformed. This should be Inf but some of the minimisers do not allow that}
\par\textbf\subsubsection{Estimate\_Transformation of type Square\_Root}\par
\defSub{estimate\_label} {The label of the estimate block to apply transformation. Defined as $\theta\_1$ in the documentation}
\par\textbf\subsubsection{Estimate\_Transformation of type Sum\_To\_One}\par
\defSub{estimate\_labels} {The label for the estimates for the sum to one transformation}
\defSub{upper\_bound} {The empirical upper bounds for the transformed parameters. There should be one less bound than parameters}
\defSub{lower\_bound} {The empirical lower bound for the transformed parameters. There should be one less bound than parameters}
\defComLab{Initialisation\_Phase}{Define an object of type \emph{Initialisation\_Phase}}. \defRef{methods:InitialisationPhase}\par\par
\defSub{label} {The label of the initialisation phase}
\defSub{type} {The type of initialisation}
\par\textbf\subsubsection{Initialisation\_Phase of type Cinitial}\par
\defSub{categories} {The list of categories for the Cinitial initialisation}
\par\textbf\subsubsection{Initialisation\_Phase of type Derived}\par
\defSub{insert\_processes} {Additional processes not defined in the annual cycle that are to be inserted into this initialisation phase}
\defSub{exclude\_processes} {Processes in the annual cycle to be excluded from this initialisation phase}
\defSub{casal\_initialisation\_switch} {Run an extra annual cycle to evaluate equilibrium SSBs. Warning - if true, this may not correctly evaluate the equilibrium state. Set to true if replicating a CASAL model}
\par\textbf\subsubsection{Initialisation\_Phase of type Iterative}\par
\defSub{years} {The number of iterations (years) over which to execute this initialisation phase}
\defSub{insert\_processes} {The processes in the annual cycle to be include in this initialisation phase}
\defSub{exclude\_processes} {The processes in the annual cycle to be excluded from this initialisation phase}
\defSub{convergence\_years} {The iteration (year) when the test for convergence ($\lambda$) is evaluated}
\defSub{lambda} {The maximum value of the absolute sum of differences (lambda) between the partition at year-1 and year that indicates successful convergence}
\par\textbf\subsubsection{Initialisation\_Phase of type State\_Category\_By\_Age}\par
\defSub{categories} {The list of categories for the category state initialisation}
\defSub{min\_age} {The minimum age of values supplied in the definition of the category state}
\defSub{max\_age} {The maximum age of values supplied in the definition of the category state}
\defComLab{Length\_Weight}{Define an object of type \emph{Length\_Weight}}. \defRef{methods:LengthWeight}\par\par
\defSub{label} {The label of the length-weight relationship}
\defSub{type} {The type of the length-weight relationship}
\par\textbf\subsubsection{Length\_Weight of type Basic}\par
\defSub{a} {The $a$ parameter ($W = a L^b$)}
\defSub{b} {The $b$ parameter ($W = a L^b$)}
\defSub{units} {The units of measure (tonnes, kgs, grams)}
\par\textbf\subsubsection{Length\_Weight of type None}\par
\defComLab{Likelihood}{Define an object of type \emph{Likelihood}}. \defRef{methods:Likelihood}\par\par
\par\textbf\subsubsection{Likelihood of type Binomial}\par
\par\textbf\subsubsection{Likelihood of type Binomial\_Approx}\par
\par\textbf\subsubsection{Likelihood of type Dirichlet}\par
\par\textbf\subsubsection{Likelihood of type Log\_Normal}\par
\par\textbf\subsubsection{Likelihood of type Log\_Normal\_With\_Q}\par
\par\textbf\subsubsection{Likelihood of type Multinomial}\par
\par\textbf\subsubsection{Likelihood of type Normal}\par
\par\textbf\subsubsection{Likelihood of type Pseudo}\par
\defComLab{MCMC}{Define an object of type \emph{MCMC}}. \defRef{methods:MCMC}\par\par
\defSub{label} {The label of the MCMC}
\defSub{type} {The MCMC method}
\defSub{length} {The number of iterations in for the MCMC chain}
\defSub{active} {Indicates if this is the active MCMC algorithm}
\defSub{step\_size} {Initial step-size (as a multiplier of the approximate covariance matrix)}
\defSub{start} {The covariance multiplier for the starting point of the MCMC}
\defSub{keep} {The spacing between recorded values in the MCMC}
\defSub{max\_correlation} {The maximum absolute correlation in the covariance matrix of the proposal distribution}
\defSub{covariance\_adjustment\_method} {The method for adjusting small variances in the covariance proposal matrix}
\defSub{correlation\_adjustment\_diff} {The minimum non-zero variance times the range of the bounds in the covariance matrix of the proposal distribution}
\defSub{proposal\_distribution} {The shape of the proposal distribution (either the t or the normal distribution)}
\defSub{df} {The degrees of freedom of the multivariate t proposal distribution}
\defSub{adapt\_stepsize\_at} {The iteration numbers in which to check and resize the MCMC stepsize}
\defSub{adapt\_covariance\_matrix\_at} {The iteration numbers in which to adapt the covariance matrix}
\defSub{adapt\_stepsize\_method} {The method to use to adapt the step size}
\par\textbf\subsubsection{MCMC of type Hamiltonian\_Monte\_Carlo}\par
\defSub{leapfrog\_steps} {Number of leapfrog steps}
\defSub{leapfrog\_delta} {Amount to leapfrog per step}
\defSub{gradient\_step\_size} {Step size to use when calculating gradient}
\par\textbf\subsubsection{MCMC of type Independence\_Metropolis}\par
\par\textbf\subsubsection{MCMC of type Random\_Walk\_Metropolis\_Hastings}\par
\defComLab{Minimiser}{Define an object of type \emph{Minimiser}}. \defRef{methods:Minimiser}\par\par
\defSub{label} {The minimiser label}
\defSub{type} {The type of minimiser to use}
\defSub{active} {Indicates if this minimiser is active}
\defSub{covariance} {Indicates if a covariance matrix should be generated}
\par\textbf\subsubsection{Minimiser of type ADOLC}\par
\defSub{iterations} {The maximum number of iterations}
\defSub{evaluations} {The maximum number of evaluations}
\defSub{tolerance} {The tolerance of the gradient for convergence}
\defSub{step\_size} {The minimum step size before minimisation fails}
\par\textbf\subsubsection{Minimiser of type Beta\_Diff}\par
\defSub{iterations} {The maximum number of iterations}
\defSub{evaluations} {The maximum number of evaluations}
\defSub{tolerance} {The tolerance of the gradient for convergence}
\par\textbf\subsubsection{Minimiser of type DESolver}\par
\defSub{population\_size} {The number of candidate solutions to have in the population}
\defSub{crossover\_probability} {The minimiser's crossover probability}
\defSub{difference\_scale} {The scale to apply to new solutions when comparing candidates}
\defSub{max\_generations} {The maximum number of iterations to run}
\defSub{tolerance} {The total variance between the population and best candidate before acceptance}
\defSub{method} {The type of candidate generation method to use}
\par\textbf\subsubsection{Minimiser of type Delta\_Diff}\par
\defSub{iterations} {Maximum number of iterations}
\defSub{evaluations} {Maximum number of evaluations}
\defSub{tolerance} {Tolerance of the gradient for convergence}
\defSub{step\_size} {Minimum Step-size before minimisation fails}
\par\textbf\subsubsection{Minimiser of type Gamma\_Diff}\par
\defSub{iterations} {The maximum number of iterations}
\defSub{evaluations} {The maximum number of evaluations}
\defSub{tolerance} {The tolerance of the gradient for convergence}
\defSub{step\_size} {The minimum step size before minimisation fails}
\defComLab{Model}{Define an object of type \emph{Model}}. \defRef{methods:Model}\par\par
\defSub{base\_weight\_units} {Define the units for the base weight. This will be the default unit of any weight input parameters}
\defSub{threads} {The number of threads to use for this model}
\par\textbf\subsubsection{Model of type Age}\par
\defSub{start\_year} {Define the first year of the model, immediately following initialisation}
\defSub{final\_year} {Define the final year of the model, excluding years in the projection period}
\defSub{min\_age} {Minimum age of individuals in the population}
\defSub{max\_age} {Maximum age of individuals in the population}
\defSub{age\_plus} {Define the oldest age or extra length midpoint (plus group size) as a plus group}
\defSub{initialisation\_phases} {Define the labels of the phases of the initialisation}
\defSub{time\_steps} {Define the labels of the time steps, in the order that they are applied, to form the annual cycle}
\defSub{projection\_final\_year} {Define the final year of the model in projection mode}
\defSub{length\_bins} {The minimum length in each length bin}
\defSub{length\_plus} {Specify whether there is a length plus group or not}
\defSub{length\_plus\_group} {Mean length of length plus group}
\par\textbf\subsubsection{Model of type Length}\par
\par\textbf\subsubsection{Model of type Multivariate}\par
\defSub{use\_random} {Use random numbers for points, not the defined MCMC algorithm}
\par\textbf\subsubsection{Model of type Pi\_Approximation}\par
\defSub{use\_random} {Use random numbers for points, not the defined MCMC algorithm}
\defComLab{Observation}{Define an object of type \emph{Observation}}. \defRef{methods:Observation}\par\par
\defSub{label} {The label of the observation}
\defSub{type} {The type of observation}
\defSub{likelihood} {The type of likelihood to use}
\defSub{categories} {The category labels to use}
\defSub{delta} {The robustification value (delta) for the likelihood}
\defSub{simulation\_likelihood} {The simulation likelihood to use}
\defSub{likelihood\_multiplier} {The likelihood score multiplier}
\defSub{error\_value\_multiplier} {The error value multiplier for likelihood}
\par\textbf\subsubsection{Observation of type Abundance}\par
\defSub{time\_step} {The label of the time step that the observation occurs in}
\defSub{catchability} {The label of the catchability coefficient (q)}
\defSub{selectivities} {The labels of the selectivities}
\defSub{process\_error} {The process error}
\defSub{years} {The years for which there are observations}
\par\textbf\subsubsection{Observation of type Biomass}\par
\defSub{time\_step} {The label of the time step that the observation occurs in}
\defSub{catchability} {The label of the catchability coefficient (q)}
\defSub{selectivities} {The labels of the selectivities}
\defSub{process\_error} {The process error}
\defSub{age\_weight\_labels} {R(The labels for the \command{$age\_weight$} block which corresponds to each category, to use the weight calculation method for biomass calculations)}
\defSub{years} {The years of the observed values}
\par\textbf\subsubsection{Observation of type Process\_Removals\_By\_Age}\par
\defSub{min\_age} {The minimum age}
\defSub{max\_age} {The maximum age}
\defSub{plus\_group} {Is the maximum age the age plus group}
\defSub{time\_step} {The label of time-step that the observation occurs in}
\defSub{tolerance} {The tolerance}
\defSub{years} {The years for which there are observations}
\defSub{process\_errors} {The label of process error to use}
\defSub{ageing\_error} {The label of the ageing error to use}
\defSub{method\_of\_removal} {The label of the observed method of removals}
\defSub{mortality\_instantaneous\_process} {The label of the mortality instantaneous process for the observation}
\par\textbf\subsubsection{Observation of type Process\_Removals\_By\_Age\_Retained}\par
\defSub{min\_age} {The minimum age}
\defSub{max\_age} {The maximum age}
\defSub{plus\_group} {Is the maximum age the age plus group?}
\defSub{time\_step} {The label of the time step that the observation occurs in}
\defSub{tolerance} {The tolerance}
\defSub{years} {The years for which there are observations}
\defSub{process\_errors} {The label of the process error to use}
\defSub{ageing\_error} {The label of the ageing error to use}
\defSub{method\_of\_removal} {The label of observed method of removals}
\defSub{mortality\_instantaneous\_process} {The label of the mortality instantaneous process for the observation}
\par\textbf\subsubsection{Observation of type Process\_Removals\_By\_Age\_Retained\_Total}\par
\defSub{min\_age} {The minimum age}
\defSub{max\_age} {The maximum age}
\defSub{plus\_group} {Is the maximum age the age plus group?}
\defSub{time\_step} {The label of the time step that the observation occurs in}
\defSub{tolerance} {The tolerance}
\defSub{years} {The years for which there are observations}
\defSub{process\_errors} {The label of the process error to use}
\defSub{ageing\_error} {The label of the ageing error to use}
\defSub{method\_of\_removal} {The label of observed method of removals}
\defSub{mortality\_instantaneous\_process} {The label of the mortality instantaneous process for the observation}
\par\textbf\subsubsection{Observation of type Process\_Removals\_By\_Length}\par
\defSub{time\_step} {The time step to execute in}
\defSub{tolerance} {The tolerance for rescaling proportions}
\defSub{years} {The years for which there are observations}
\defSub{process\_errors} {The process error}
\defSub{method\_of\_removal} {The label of observed method of removals}
\defSub{length\_bins} {The length bins}
\defSub{length\_plus} {Is the last length bin a plus group? (defaults to @model value)}
\defSub{mortality\_instantaneous\_process} {The label of the mortality instantaneous process for the observation}
\par\textbf\subsubsection{Observation of type Process\_Removals\_By\_Length\_Retained}\par
\defSub{time\_step} {The time step to execute in}
\defSub{tolerance} {The tolerance for rescaling proportions}
\defSub{years} {The years for which there are observations}
\defSub{process\_errors} {The process error}
\defSub{method\_of\_removal} {The label of observed method of removals}
\defSub{length\_bins} {The length bins}
\defSub{length\_plus} {Is the last length bin a plus group? (defaults to @model value)}
\defSub{mortality\_instantaneous\_process} {The label of the mortality instantaneous process for the observation}
\par\textbf\subsubsection{Observation of type Process\_Removals\_By\_Length\_Retained\_Total}\par
\defSub{time\_step} {The time step to execute in}
\defSub{tolerance} {The tolerance for rescaling proportions}
\defSub{years} {The years for which there are observations}
\defSub{process\_errors} {The process error}
\defSub{method\_of\_removal} {The label of observed method of removals}
\defSub{length\_bins} {The length bins}
\defSub{length\_plus} {Is the last length bin a plus group? (defaults to @model value)}
\defSub{mortality\_instantaneous\_process} {The label of the mortality instantaneous process for the observation}
\par\textbf\subsubsection{Observation of type Proportions\_At\_Age}\par
\defSub{min\_age} {The minimum age}
\defSub{max\_age} {The maximum age}
\defSub{plus\_group} {Is the maximum age the age plus group?}
\defSub{time\_step} {The label of the time step that the observation occurs in}
\defSub{tolerance} {The tolerance on the constraint that for each year the sum of proportions in each age must equal 1, e.g., if tolerance = 0.1 then 1 - Sum(Proportions) can be as great as 0.1}
\defSub{years} {The years of the observed values}
\defSub{selectivities} {The labels of the selectivities}
\defSub{process\_errors} {The process error}
\defSub{ageing\_error} {The label of ageing error to use}
\par\textbf\subsubsection{Observation of type Proportions\_At\_Length}\par
\defSub{time\_step} {The label of the time step that the observation occurs in}
\defSub{tolerance} {The tolerance for rescaling proportions}
\defSub{years} {The years for which there are observations}
\defSub{selectivities} {The labels of the selectivities}
\defSub{process\_errors} {The process error}
\defSub{length\_bins} {The length bins}
\par\textbf\subsubsection{Observation of type Proportions\_By\_Category}\par
\defSub{min\_age} {The minimum age}
\defSub{max\_age} {The maximum age}
\defSub{time\_step} {The label of the time step that the observation occurs in}
\defSub{plus\_group} {Use the age plus group?}
\defSub{years} {The years for which there are observations}
\defSub{selectivities} {The labels of the selectivities}
\defSub{categories2} {The target categories}
\defSub{selectivities2} {The target selectivities}
\par\textbf\subsubsection{Observation of type Proportions\_Mature\_By\_Age}\par
\defSub{min\_age} {The minimum age}
\defSub{max\_age} {The maximum age}
\defSub{time\_step} {The label of time-step that the observation occurs in}
\defSub{plus\_group} {Use the age plus group?}
\defSub{years} {The years for which there are observations}
\defSub{ageing\_error} {The label of ageing error to use}
\defSub{total\_categories} {All category labels that were vulnerable to sampling at the time of this observation (not including the categories already given)}
\defSub{time\_step\_proportion} {The proportion through the mortality block of the time step when the observation is evaluated}
\par\textbf\subsubsection{Observation of type Proportions\_Migrating}\par
\defSub{min\_age} {The minimum age}
\defSub{max\_age} {The maximum age}
\defSub{time\_step} {The label of the time step that the observation occurs in}
\defSub{plus\_group} {Is the maximum age the age plus group?}
\defSub{years} {The years for which there are observations}
\defSub{process\_errors} {The process error}
\defSub{ageing\_error} {The label of the ageing error to use}
\defSub{process} {The process label}
\par\textbf\subsubsection{Observation of type Tag\_Recapture\_By\_Age}\par
\defSub{min\_age} {The minimum age}
\defSub{max\_age} {The maximum age}
\defSub{plus\_group} {Is the maximum age the age plus group?}
\defSub{years} {The years for which there are observations}
\defSub{categories2} {The available categories in the partition}
\defSub{selectivities} {The labels of the selectivities}
\defSub{time\_step} {The label of the time step that the observation occurs in}
\defSub{selectivities2} {The categories of tagged individuals for the observation}
\defSub{detection} {The probability of detecting a recaptured individual}
\defSub{time\_step\_proportion} {The proportion through the mortality block of the time step when the observation is evaluated}
\par\textbf\subsubsection{Observation of type Tag\_Recapture\_By\_Length}\par
\defSub{years} {The years for which there are observations}
\defSub{time\_step} {The time step to execute in}
\defSub{length\_bins} {The length bins}
\defSub{selectivities} {The labels of the selectivities used for untagged categories}
\defSub{tagged\_selectivities} {The labels of the tag category selectivities}
\defSub{detection} {The probability of detecting a recaptured individual}
\defSub{dispersion} {The overdispersion parameter (phi)}
\defSub{time\_step\_proportion} {The proportion through the mortality block of the time step when the observation is evaluated}
\defComLab{Penalty}{Define an object of type \emph{Penalty}}. \defRef{methods:Penalty}\par\par
\defSub{label} {The label of the penalty}
\defSub{type} {The type of penalty}
\par\textbf\subsubsection{Penalty of type Process}\par
\defSub{multiplier} {The penalty multiplier}
\defSub{log\_scale} {Indicates if the sums of squares is calculated on the log scale}
\defComLab{Process}{Define an object of type \emph{Process}}. \defRef{methods:Process}\par\par
\defSub{label} {The label of the process}
\defSub{type} {The type of process}
\par\textbf\subsubsection{Process of type Ageing}\par
\defSub{categories} {The labels of the categories}
\par\textbf\subsubsection{Process of type Load\_Partition}\par
\par\textbf\subsubsection{Process of type Maturation}\par
\defSub{from} {The list of categories to mature from}
\defSub{to} {The list of categories to mature to}
\defSub{selectivities} {The list of selectivities to use for maturation}
\defSub{years} {The years to be associated with the maturity rates}
\defSub{rates} {The rates to mature for each year}
\par\textbf\subsubsection{Process of type Mortality\_Constant\_Rate}\par
\defSub{categories} {The list of category labels}
\defSub{m} {The mortality rates}
\defSub{time\_step\_ratio} {The time step ratios for the mortality rates}
\defSub{relative\_m\_by\_age} {The list of M-by-age ogives for the categories}
\par\textbf\subsubsection{Process of type Mortality\_Event}\par
\defSub{categories} {The categories}
\defSub{years} {The years in which to apply the mortality process}
\defSub{catches} {The number of removals (catches) to apply for each year}
\defSub{u\_max} {The maximum exploitation rate ($U\_{max}$)}
\defSub{selectivities} {The list of selectivities}
\defSub{penalty} {The label of the penalty to apply if the total number of removals cannot be taken}
\par\textbf\subsubsection{Process of type Mortality\_Event\_Biomass}\par
\defSub{categories} {The category labels}
\defSub{selectivities} {The labels of the selectivities for each of the categories}
\defSub{years} {The years in which to apply the mortality process}
\defSub{catches} {The biomass of removals (catches) to apply for each year}
\defSub{u\_max} {The maximum exploitation rate ($U\_{max}$)}
\defSub{penalty} {The label of the penalty to apply if the total biomass of removals cannot be taken}
\par\textbf\subsubsection{Process of type Mortality\_Holling\_Rate}\par
\defSub{prey\_categories} {The prey categories labels}
\defSub{predator\_categories} {The predator categories labels}
\defSub{is\_abundance} {Is vulnerable amount of prey and predator an abundance [true] or biomass [false]}
\defSub{a} {Parameter a}
\defSub{b} {Parameter b}
\defSub{x} {This parameter controls the functional form: Holling function type 2 (x=2) or 3 (x=3), or generalised (Michaelis Menten, x>=1)}
\defSub{u\_max} {The maximum exploitation rate ($U\_{max}$)}
\defSub{prey\_selectivities} {The selectivities for prey categories}
\defSub{predator\_selectivities} {The selectivities for predator categories}
\defSub{penalty} {The label of penalty}
\defSub{years} {The years in which to apply the mortality process}
\par\textbf\subsubsection{Process of type Mortality\_Initialisation\_Event}\par
\defSub{categories} {The categories}
\defSub{catch} {The number of removals (catches) to apply for each year}
\defSub{u\_max} {The maximum exploitation rate ($U\_{max}$)}
\defSub{selectivities} {The list of selectivities}
\defSub{penalty} {The label of the penalty to apply if the total number of removals cannot be taken}
\par\textbf\subsubsection{Process of type Mortality\_Initialisation\_Event\_Biomass}\par
\defSub{categories} {The categories}
\defSub{catch} {The number of removals (catches) to apply for each year}
\defSub{u\_max} {The maximum exploitation rate ($U\_{max}$)}
\defSub{selectivities} {The list of selectivities}
\defSub{penalty} {The label of the penalty to apply if the total number of removals cannot be taken}
\par\textbf\subsubsection{Process of type Mortality\_Instantaneous}\par
\defSub{categories} {The categories for instantaneous mortality}
\defSub{m} {The natural mortality rates for each category}
\defSub{time\_step\_ratio} {The time step ratios for natural mortality}
\defSub{relative\_m\_by\_age} {The M-by-age ogives to apply on the categories for natural mortality}
\par\textbf\subsubsection{Process of type Mortality\_Instantaneous\_Retained}\par
\defSub{categories} {The categories for instantaneous mortality}
\defSub{m} {The natural mortality rates for each category}
\defSub{time\_step\_ratio} {The time step ratios for natural mortality}
\defSub{relative\_m\_by\_age} {The M-by-age ogives to apply on the categories for natural mortality}
\par\textbf\subsubsection{Process of type Mortality\_Prey\_Suitability}\par
\defSub{prey\_categories} {The prey categories labels}
\defSub{predator\_categories} {The predator categories labels}
\defSub{consumption\_rate} {The predator consumption rate}
\defSub{electivities} {The prey electivities}
\defSub{u\_max} {The maximum exploitation rate ($U\_{max}$)}
\defSub{prey\_selectivities} {The selectivities for prey categories}
\defSub{predator\_selectivities} {The selectivities for predator categories}
\defSub{penalty} {The label of the penalty}
\defSub{years} {The year that process occurs}
\par\textbf\subsubsection{Process of type Nop}\par
\par\textbf\subsubsection{Process of type Recruitment\_Beverton\_Holt}\par
\defSub{categories} {The category labels}
\defSub{r0} {R0}
\defSub{b0} {B0}
\defSub{proportions} {The proportion for each category}
\defSub{age} {The age at recruitment}
\defSub{ssb\_offset} {The spawning biomass year offset}
\defSub{steepness} {Steepness (h)}
\defSub{ssb} {The SSB label (derived quantity)}
\defSub{b0\_initialisation\_phase} {The initialisation phase label that B0 is from}
\defSub{ycs\_values} {The YCS values}
\defSub{ycs\_years} {The recruitment years. A vector of years that relates to the year of the spawning event that created this cohort}
\defSub{standardise\_ycs\_years} {The years that are included for year class standardisation}
\par\textbf\subsubsection{Process of type Recruitment\_Beverton\_Holt\_With\_Deviations}\par
\defSub{categories} {The category labels}
\defSub{r0} {R0}
\defSub{b0} {B0}
\defSub{proportions} {The proportion for each category}
\defSub{age} {The age at recruitment}
\defSub{ssb\_offset} {The spawning biomass year offset}
\defSub{steepness} {Steepness (h)}
\defSub{ssb} {The SSB Label (derived quantity)}
\defSub{sigma\_r} {sigma R}
\defSub{b\_max} {The maximum bias adjustment}
\defSub{last\_year\_with\_no\_bias} {The last year with no bias adjustment}
\defSub{first\_year\_with\_bias} {The first year with full bias adjustment}
\defSub{last\_year\_with\_bias} {The last year with full bias adjustment}
\defSub{first\_recent\_year\_with\_no\_bias} {The first recent year with no bias adjustment}
\defSub{b0\_initialisation\_phase} {The initialisation phase label that B0 is from}
\defSub{deviation\_values} {The recruitment deviation values}
\defSub{deviation\_years} {The recruitment years. A vector of years that relates to the year of the spawning event that created this cohort}
\par\textbf\subsubsection{Process of type Recruitment\_Constant}\par
\defSub{categories} {The categories}
\defSub{proportions} {The proportion for each category}
\defSub{age} {The age}
\defSub{r0} {R0}
\par\textbf\subsubsection{Process of type Survival\_Constant\_Rate}\par
\defSub{categories} {The list of categories}
\defSub{s} {The survival rates}
\defSub{time\_step\_ratio} {The time step ratios for S}
\defSub{selectivities} {The selectivity label}
\par\textbf\subsubsection{Process of type Tag\_By\_Age}\par
\defSub{from} {The categories to transition from}
\defSub{to} {The categories to transition to}
\defSub{min\_age} {The minimum age to transition}
\defSub{max\_age} {The maximum age to transition}
\defSub{penalty} {The penalty label}
\defSub{u\_max} {The maximum exploitation rate ($U\_{max}$)}
\defSub{years} {The years to execute the transition in}
\defSub{initial\_mortality} {The initial mortality value}
\defSub{initial\_mortality\_selectivity} {The initial mortality selectivity label}
\defSub{loss\_rate} {The loss rate}
\defSub{loss\_rate\_selectivities} {The loss rate selectivity label}
\defSub{selectivities} {The selectivity labels}
\defSub{n} {N}
\par\textbf\subsubsection{Process of type Tag\_By\_Length}\par
\defSub{from} {The categories to transition from}
\defSub{to} {The categories to transition to}
\defSub{penalty} {The penalty label}
\defSub{u\_max} {The maximum exploitation rate ($U\_{max}$)}
\defSub{initial\_mortality} {}
\defSub{initial\_mortality\_selectivity} {}
\defSub{selectivities} {}
\defSub{n} {}
\par\textbf\subsubsection{Process of type Tag\_Loss}\par
\defSub{categories} {The list of categories}
\defSub{tag\_loss\_rate} {The tag loss rates}
\defSub{time\_step\_ratio} {The time step ratios for tag loss}
\defSub{tag\_loss\_type} {The type of tag loss}
\defSub{selectivities} {The selectivities}
\defSub{year} {The year the first tagging release process was executed}
\par\textbf\subsubsection{Process of type Transition\_Category}\par
\defSub{from} {The from category}
\defSub{to} {The to category}
\defSub{proportions} {The proportions}
\defSub{selectivities} {The selectivity names}
\par\textbf\subsubsection{Process of type Transition\_Category\_By\_Age}\par
\defSub{from} {The categories to transition from}
\defSub{to} {The categories to transition to}
\defSub{min\_age} {The minimum age to transition}
\defSub{max\_age} {The maximum age to transition}
\defSub{penalty} {The penalty label}
\defSub{u\_max} {The maximum exploitation rate ($U\_{max}$)}
\defSub{years} {The years to execute the transition in}
\par\textbf\subsubsection{Process of type Unit\_Tester}\par
\defComLab{Profile}{Define an object of type \emph{Profile}}. \defRef{methods:Profile}\par\par
\defSub{label} {The label of the profile}
\defSub{steps} {The number of steps between the lower and upper bound}
\defSub{lower\_bound} {The lower bounds}
\defSub{upper\_bound} {The upper bounds}
\defSub{parameter} {The system parameter to profile}
\defSub{same} {A parameter that is constrained to have the same value as the parameter being profiled}
\defComLab{Project}{Define an object of type \emph{Project}}. \defRef{methods:Project}\par\par
\defSub{label} {Label}
\defSub{type} {Type}
\defSub{years} {Years to recalculate the values}
\defSub{parameter} {Parameter to project}
\defSub{multiplier} {Multiplier that is applied to the projected value}
\par\textbf\subsubsection{Project of type Constant}\par
\defSub{values} {The values to assign to the addressable}
\par\textbf\subsubsection{Project of type Empirical\_Sampling}\par
\defSub{start\_year} {The start year of sampling}
\defSub{final\_year} {The final year of sampling}
\par\textbf\subsubsection{Project of type Log\_Normal}\par
\defSub{mean} {The mean of the lognormal process}
\defSub{sigma} {The standard deviation (sigma) of the lognormal process}
\par\textbf\subsubsection{Project of type Log\_Normal\_Empirical}\par
\defSub{mean} {The mean of the Gaussian process}
\defSub{start\_year} {The start year of sampling}
\defSub{final\_year} {The final year of sampling}
\par\textbf\subsubsection{Project of type User\_Defined}\par
\defSub{equation} {The equation to do a test run of}
\defComLab{Report}{Define an object of type \emph{Report}}. \defRef{methods:Report}\par\par
\defSub{label} {The label for the report}
\defSub{type} {The type of report}
\defSub{file\_name} {The filename for this report to be in a separate file}
\defSub{write\_mode} {The write mode}
\par\textbf\subsubsection{Report of type Addressable}\par
\defSub{parameter} {The parameter to print}
\defSub{years} {The years to print the addressable for}
\defSub{time\_step} {The time Step label}
\par\textbf\subsubsection{Report of type Age\_Length}\par
\defSub{time\_step} {The time step label}
\defSub{years} {The years for the report}
\defSub{age\_length} {The age-length label}
\defSub{category} {The category label}
\par\textbf\subsubsection{Report of type Ageing\_Error\_Matrix}\par
\defSub{ageing\_error} {The ageing error label}
\par\textbf\subsubsection{Report of type Catchability}\par
\par\textbf\subsubsection{Report of type Category\_Info}\par
\par\textbf\subsubsection{Report of type Category\_List}\par
\par\textbf\subsubsection{Report of type Correlation\_Matrix}\par
\par\textbf\subsubsection{Report of type Covariance\_Matrix}\par
\par\textbf\subsubsection{Report of type Derived\_Quantity}\par
\par\textbf\subsubsection{Report of type Equation\_Test}\par
\defSub{equation} {The equation to do a test run of}
\par\textbf\subsubsection{Report of type Estimate\_Summary}\par
\par\textbf\subsubsection{Report of type Estimate\_Value}\par
\par\textbf\subsubsection{Report of type Estimation\_Result}\par
\par\textbf\subsubsection{Report of type Hessian\_Matrix}\par
\par\textbf\subsubsection{Report of type Initialisation\_Partition}\par
\par\textbf\subsubsection{Report of type Initialisation\_Partition\_Mean\_Weight}\par
\par\textbf\subsubsection{Report of type MCMCCovariance}\par
\par\textbf\subsubsection{Report of type MCMCObjective}\par
\par\textbf\subsubsection{Report of type MCMCSample}\par
\par\textbf\subsubsection{Report of type MPD}\par
\par\textbf\subsubsection{Report of type Objective\_Function}\par
\par\textbf\subsubsection{Report of type Observation}\par
\defSub{observation} {The observation label}
\defSub{normalised\_residuals} {Print Normalised Residuals?}
\defSub{pearsons\_residuals} {Print Pearsons Residuals?}
\par\textbf\subsubsection{Report of type Output\_Parameters}\par
\par\textbf\subsubsection{Report of type Partition}\par
\defSub{time\_step} {Time Step label}
\defSub{years} {Years}
\par\textbf\subsubsection{Report of type Partition\_Biomass}\par
\defSub{time\_step} {The time step label}
\defSub{years} {The years for the report}
\par\textbf\subsubsection{Report of type Partition\_Mean\_Length}\par
\defSub{time\_step} {The time step label}
\defSub{years} {The years for the report}
\par\textbf\subsubsection{Report of type Partition\_Mean\_Weight}\par
\defSub{time\_step} {The time step label}
\defSub{years} {The years for the report}
\par\textbf\subsubsection{Report of type Partition\_Year\_Cross\_Age\_Matrix}\par
\par\textbf\subsubsection{Report of type Process}\par
\defSub{process} {The process label that is reported}
\par\textbf\subsubsection{Report of type Project}\par
\defSub{project} {The project label that is reported}
\par\textbf\subsubsection{Report of type Random\_Number\_Seed}\par
\par\textbf\subsubsection{Report of type Selectivity}\par
\defSub{selectivity} {Selectivity name}
\par\textbf\subsubsection{Report of type Simulated\_Observation}\par
\defSub{observation} {The observation label}
\par\textbf\subsubsection{Report of type Time\_Varying}\par
\defComLab{Selectivity}{Define an object of type \emph{Selectivity}}. \defRef{methods:Selectivity}\par\par
\defSub{label} {The label for the selectivity}
\defSub{type} {The type of selectivity}
\defSub{length\_based} {Is the selectivity length based?}
\defSub{intervals} {The number of quantiles to evaluate a length-based selectivity over the age-length distribution}
\defSub{partition\_type} {The type of partition that this selectivity will support. Defaults to the same as the model}
\defSub{values} {}
\defSub{length\_values} {}
\par\textbf\subsubsection{Selectivity of type All\_Values}\par
\defSub{v} {The v parameter}
\par\textbf\subsubsection{Selectivity of type All\_Values\_Bounded}\par
\defSub{l} {The low value (L)}
\defSub{h} {The high value (H)}
\defSub{v} {The v parameter}
\par\textbf\subsubsection{Selectivity of type Constant}\par
\defSub{c} {The constant value}
\par\textbf\subsubsection{Selectivity of type Constant.Mock}\par
\par\textbf\subsubsection{Selectivity of type Double\_Exponential}\par
\defSub{x0} {The X0 parameter}
\defSub{x1} {The X1 parameter}
\defSub{x2} {The X2 parameter}
\defSub{y0} {The Y0 parameter}
\defSub{y1} {The Y1 parameter}
\defSub{y2} {The Y2 parameter}
\defSub{alpha} {alpha}
\par\textbf\subsubsection{Selectivity of type Double\_Normal}\par
\defSub{mu} {The mean (mu)}
\defSub{sigma\_l} {The sigma L parameter}
\defSub{sigma\_r} {The sigma R parameter}
\defSub{alpha} {alpha}
\par\textbf\subsubsection{Selectivity of type Increasing}\par
\defSub{l} {The low value (L)}
\defSub{h} {The high value (H)}
\defSub{v} {The v parameter}
\defSub{alpha} {alpha}
\par\textbf\subsubsection{Selectivity of type Inverse\_Logistic}\par
\defSub{a50} {a50}
\defSub{ato95} {ato95}
\defSub{alpha} {alpha}
\par\textbf\subsubsection{Selectivity of type Knife\_Edge}\par
\defSub{e} {The edge value}
\defSub{alpha} {alpha}
\par\textbf\subsubsection{Selectivity of type Logistic}\par
\defSub{a50} {a50}
\defSub{ato95} {ato95}
\defSub{alpha} {alpha}
\par\textbf\subsubsection{Selectivity of type Logistic\_Producing}\par
\defSub{l} {The low value (L)}
\defSub{h} {The high value (H)}
\defSub{a50} {a50}
\defSub{ato95} {ato95}
\defSub{alpha} {alpha}
\defComLab{Simulate}{Define an object of type \emph{Simulate}}. \defRef{methods:Simulate}\par\par
\defSub{label} {The label for the simulation command block}
\defSub{type} {The type of simulation}
\defSub{years} {The years to simulate values for}
\defSub{parameter} {The parameter to generate simulated values for}
\par\textbf\subsubsection{Simulate of type Constant}\par
\defSub{value} {The value to assign to the parameter in the simulations}
\defComLab{Time\_Step}{Define an object of type \emph{Time\_Step}}. \defRef{methods:TimeStep}\par\par
\defSub{label} {The label of the time step}
\defSub{processes} {The labels of the processes for this time step, and given in the order that they occur within the time step}
\defComLab{Time\_Varying}{Define an object of type \emph{Time\_Varying}}. \defRef{methods:TimeVarying}\par\par
\defSub{label} {The label of the time-varying object}
\defSub{type} {The type of the time-varying object}
\defSub{parameter} {The name of the parameter to vary in each year}
\defSub{years} {The years in which to vary the paramter}
\par\textbf\subsubsection{Time\_Varying of type Annual\_Shift}\par
\defSub{a} {Parameter A}
\defSub{b} {Parmeter B}
\defSub{c} {Parameter C}
\defSub{scaling\_years} {The scaling years}
\defSub{values} {The values}
\par\textbf\subsubsection{Time\_Varying of type Constant}\par
\defSub{values} {The value to assign to addressable}
\par\textbf\subsubsection{Time\_Varying of type Constant.Mock}\par
\par\textbf\subsubsection{Time\_Varying of type Exogenous}\par
\defSub{a} {The shift parameter}
\defSub{exogenous\_variable} {The values of exogenous variable for each year}
\par\textbf\subsubsection{Time\_Varying of type Linear}\par
\defSub{slope} {The slope of the linear trend (i.e., the additive amount per year)}
\defSub{intercept} {The intercept of the linear trend (, i.e. the value in the first year)}
\par\textbf\subsubsection{Time\_Varying of type Random\_Draw}\par
\defSub{mean} {The mean ($\mu$) of the random draw distribution}
\defSub{sigma} {The standard deviation ($\sigma$)  of the random draw distribution}
\defSub{upper\_bound} {The upper bound for the random draw}
\defSub{upper\_bound} {The lower bound for the random draw}
\defSub{distribution} {The distribution type}
\par\textbf\subsubsection{Time\_Varying of type Random\_Walk}\par
\defSub{mean} {The mean ($\mu$) of the random walk distribution}
\defSub{sigma} {The standard deviation ($\sigma$)  of the random walk distribution}
\defSub{upper\_bound} {The upper bound for the random walk}
\defSub{upper\_bound} {The lower bound for the random walk}
\defSub{rho} {The autocorrelation parameter ($\rho$)  of the random walk distribution}
\defSub{distribution} {The distribution type}
