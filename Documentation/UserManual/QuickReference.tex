\section{Quick reference}\label{sec:QuickReference}
\small

\defComLab{additional\_prior}{Define an object of type \emph{Additional\_Prior}}

\defSub{label}{The label for the additional prior}
\defSub{parameter}{The name of the parameter for the additional prior}
\defSub{type}{The additional prior type}

\commandlabsubarg{additional\_prior}{type}{Beta}

\defSub{mu}{Beta distribution mean $\mu$ parameter}
\defSub{sigma}{Beta distribution variance $\sigma$ parameter}
\defSub{a}{Beta distribution lower bound, of the range $A$ parameter}
\defSub{b}{Beta distribution upper bound of the range $B$ parameter}

\commandlabsubarg{additional\_prior}{type}{Element\_Difference}

\defSub{second\_parameter}{The name of the second parameter for comparing}
\defSub{multiplier}{Multiply the penalty by this factor}

\commandlabsubarg{additional\_prior}{type}{Log\_Normal}

\defSub{mu}{The lognormal prior mean (mu) parameter}
\defSub{cv}{The lognormal CV parameter}

\commandlabsubarg{additional\_prior}{type}{Uniform\_Log}


\commandlabsubarg{additional\_prior}{type}{Vector\_Average}

\defSub{method}{Which calculation method to use: k, l, or m}
\defSub{k}{The k value to use in the calculation}
\defSub{multiplier}{Multiplier for the penalty amount}

\commandlabsubarg{additional\_prior}{type}{Vector\_Smoothing}

\defSub{log\_scale}{Should the sums of squares be calculated on the log scale?}
\defSub{multiplier}{Multiply the penalty by this factor}
\defSub{lower\_bound}{The first element to apply the penalty to in the vector}
\defSub{upper\_bound}{The last element to apply the penalty to in the vector}
\defSub{r}{The rth difference that the penalty is applied to}

\defComLab{ageing\_error}{Define an object of type \emph{Ageing\_Error}}

\defSub{label}{The label of the ageing error}
\defSub{type}{The type of ageing error}

\commandlabsubarg{ageing\_error}{type}{Data}

\defSub{data}{The table of data specifying the ageing misclassification matrix}
\defSub{table data}{The table of data specifying the ageing misclassification matrix}

\commandlabsubarg{ageing\_error}{type}{None}


\commandlabsubarg{ageing\_error}{type}{Normal}

\defSub{cv}{CV of the misclassification matrix}
\defSub{k}{k defines the minimum age of individuals which can be misclassified, i.e., individuals of age less than k have no ageing error}

\commandlabsubarg{ageing\_error}{type}{Off\_By\_One}

\defSub{p1}{The proportion misclassified as one year younger, e.g., the proportion of age k individuals that were misclassified as age (k-1)}
\defSub{p2}{The proportion misclassified as one year older, e.g., the proportion of age k individuals that were misclassified as age (k+1)}
\defSub{k}{The minimum age of animals which can be misclassified, i.e., animals of age less than k are assumed to be correctly classified}

\defComLab{age\_length}{Define an object of type \emph{Age\_Length}}

\defSub{label}{The label of the age length relationship}
\defSub{type}{The type of age length relationship}
\defSub{time\_step\_proportions}{The fraction of the year applied in each time step that is added to the age for the purposes of evaluating the length, i.e., a value of 0.5 for a time step will evaluate the length of individuals at age+0.5 in that time step}
\defSub{distribution}{The assumed distribution for the growth curve}
\defSub{cv\_first}{The CV for the first age class}
\defSub{cv\_last}{The CV for last age class}
\defSub{compatibility\_option}{Backwards compatibility option: either casal2 (the default) or casal to use the (less accurate) cumulative normal function from CASAL}
\defSub{by\_length}{Specifies if the linear interpolation of CVs is a linear function of mean length at age, or at age}

\commandlabsubarg{age\_length}{type}{Data}

\defSub{external\_gaps}{The method to use for external data gaps}
\defSub{internal\_gaps}{The method to use for internal data gaps}
\defSub{length\_weight}{The label from an associated length-weight block}
\defSub{time\_step\_measurements\_were\_made}{The time step label for which size-at-age data are provided}
\defSub{table}{The table of data specifying the length at age values}

\commandlabsubarg{age\_length}{type}{None}


\commandlabsubarg{age\_length}{type}{Schnute}

\defSub{y1}{The $y\_1$ parameter}
\defSub{y2}{The $y\_2$ parameter}
\defSub{tau1}{The $\tau\_1$ parameter}
\defSub{tau2}{The $\tau\_2$ parameter}
\defSub{a}{The $a$ parameter}
\defSub{b}{The $b$ parameter}
\defSub{length\_weight}{The label of the associated length-weight relationship}

\commandlabsubarg{age\_length}{type}{Von\_Bertalanffy}

\defSub{linf}{The $L\_{infinity}$ parameter}
\defSub{k}{The $k$ parameter}
\defSub{t0}{The $t\_0$ parameter}
\defSub{length\_weight}{The label of the associated length-weight relationship}

\defComLab{age\_weight}{Define an object of type \emph{Age\_Weight}}

\defSub{label}{Label of the age weight relationship}
\defSub{type}{The type of age weight}

\commandlabsubarg{age\_weight}{type}{Data}

\defSub{equilibrium\_method}{If used in an SSB calculation, what is the method to calculate equilibrium SSB}
\defSub{units}{The units of measure (grams, kilograms (kgs), or tonnes)}
\defSub{table}{The table of data specifying the age at weight values}

\commandlabsubarg{age\_weight}{type}{None}


\defComLab{assert}{Define an object of type \emph{Assert}}

\defSub{label}{The label for the assert}
\defSub{type}{The type of the assert}

\commandlabsubarg{assert}{type}{Addressable}

\defSub{parameter}{The addressable to check}
\defSub{years}{The years to check addressable}
\defSub{time\_step}{The time step to execute after}
\defSub{values}{The values to check against the addressable}
\defSub{tolerance}{The tolerance of the difference test}
\defSub{error\_type}{Report assert failures as either an error or warning}

\commandlabsubarg{assert}{type}{Objective\_Function}

\defSub{value}{Expected value of the objective function}
\defSub{tolerance}{The tolerance of the difference test}
\defSub{error\_type}{Report assert failures as either an error or warning}

\commandlabsubarg{assert}{type}{Partition}

\defSub{category}{Category to check population values for}
\defSub{values}{Values expected in the partition}
\defSub{tolerance}{The tolerance of the difference test}
\defSub{error\_type}{Report assert failures as either an error or warning}

\defComLab{catchability}{Define an object of type \emph{Catchability}}

\defSub{label}{Label of the catchability}
\defSub{type}{The type of catchability}

\commandlabsubarg{catchability}{type}{Free}

\defSub{q}{The value of the catchability}

\commandlabsubarg{catchability}{type}{Nuisance}

\defSub{lower\_bound}{The upper bound for nuisance catchability}
\defSub{upper\_bound}{The lower bound for nuisance catchability}
\defSub{q}{The value of the catchability}

\defComLab{categories}{Define an object of type \emph{Categories}}

\defSub{format}{The format that the category names use}
\defSub{names}{The names of the categories}
\defSub{age\_lengths}{The age-length relationship labels for each category}
\defSub{age\_weight}{The age-weight relationships labels for each category}

\defComLab{derived\_quantity}{Define an object of type \emph{Derived\_Quantity}}

\defSub{label}{The label of the derived quantity}
\defSub{type}{The type of derived quantity}
\defSub{time\_step}{The time step in which to calculate the derived quantity}
\defSub{categories}{The list of categories to use when calculating the derived quantity}
\defSub{selectivities}{The list of selectivities to use when calculating the derived quantity}
\defSub{time\_step\_proportion}{The proportion through the mortality block of the time step when the derived quantity is calculated}
\defSub{time\_step\_proportion\_method}{The method for interpolating for the proportion through the mortality block}

\commandlabsubarg{derived\_quantity}{type}{Abundance}


\commandlabsubarg{derived\_quantity}{type}{Biomass}

\defSub{age\_weight\_labels}{The labels for the age-weights that correspond to each category for the biomass calculation}

\defComLab{estimate}{Define an object of type \emph{Estimate}}

\defSub{label}{The label of the estimate}
\defSub{type}{The type of prior for the estimate}
\defSub{parameter}{The name of the parameter to estimate}
\defSub{lower\_bound}{The lower bound for the parameter}
\defSub{upper\_bound}{The upper bound for the parameter}
\defSub{same}{List of other parameters that are constrained to have the same value as this parameter}
\defSub{estimation\_phase}{The first estimation phase to allow this to be estimated}
\defSub{mcmc\_fixed}{Indicates if this parameter is estimated at the point estimate but fixed during MCMC estimation run}

\commandlabsubarg{estimate}{type}{Beta}

\defSub{mu}{Beta prior  mean (mu) parameter}
\defSub{sigma}{Beta prior standard deviation (sigma) parameter}
\defSub{a}{Beta prior lower bound of the range (A) parameter}
\defSub{b}{Beta prior upper bound of the range (B) parameter}

\commandlabsubarg{estimate}{type}{Lognormal}

\defSub{mu}{The lognormal prior mean (mu) parameter}
\defSub{cv}{The lognormal variance (cv) parameter}

\commandlabsubarg{estimate}{type}{Normal}

\defSub{mu}{The normal prior mean (mu) parameter}
\defSub{cv}{The normal standard deviation (sigma) parameter}

\commandlabsubarg{estimate}{type}{Normal\_By\_Stdev}

\defSub{mu}{The normal prior mean (mu) parameter}
\defSub{sigma}{The normal standard deviation (sigma) parameter}
\defSub{lognormal\_transformation}{Add a Jacobian if the derived outcome of the estimate is assumed to be lognormal, e.g., used for recruitment deviations in the recruitment process. See the User Manual for more information}

\commandlabsubarg{estimate}{type}{Normal\_Log}

\defSub{mu}{The normal-log prior mean (mu) parameter}
\defSub{sigma}{The normal-log prior standard deviation (sigma) parameter}

\commandlabsubarg{estimate}{type}{Uniform}


\commandlabsubarg{estimate}{type}{Uniform\_Log}


\defComLab{growth\_increment}{Define an object of type \emph{growth\_increment}}

\defSub{label}{The label of the growth increment model}
\defSub{type}{The type of growth increment model}
\defSub{time\_step\_proportions}{The proportion of annual increment to apply in each time-step. Must sum = 1.0}
\defSub{compatibility\_option}{Backwards compatibility option: either casal2 (the default) or casal to use the (less accurate) cumulative normal function from CASAL}
\defSub{distribution}{The assumed distribution for the growth curve}
\defSub{length\_weight}{The label from an associated length-weight block}
\defSub{cv}{The $cv$ for the growth model.}
\defSub{min\_sigma}{The minimum standard deviation for the growth model.}

\commandlabsubarg{growth\_increment}{type}{basic}

\defSub{g\_alpha}{The $g_{\alpha}$ parameter}
\defSub{g\_beta}{The $g_{\beta}$ parameter}
\defSub{l\_alpha}{The $l_{\alpha}$ parameter}
\defSub{l\_beta}{The $l_{\beta}$ parameter}

\commandlabsubarg{growth\_increment}{type}{exponential}

\defSub{g\_alpha}{The $g_{\alpha}$ parameter}
\defSub{g\_beta}{The $g_{\beta}$ parameter}
\defSub{l\_alpha}{The $l_{\alpha}$ parameter}
\defSub{l\_beta}{The $l_{\beta}$ parameter}

\commandlabsubarg{growth\_increment}{type}{none}


\defComLab{initialisation\_phase}{Define an object of type \emph{Initialisation\_Phase}}

\defSub{label}{The label of the initialisation phase}
\defSub{type}{The type of initialisation}

\commandlabsubarg{initialisation\_phase}{type}{Iterative}

\defSub{years}{The number of iterations (years) over which to execute this initialisation phase}
\defSub{insert\_processes}{The processes in the annual cycle to be include in this initialisation phase}
\defSub{exclude\_processes}{The processes in the annual cycle to be excluded from this initialisation phase}
\defSub{convergence\_years}{The iteration (year) when the test for convergence ($\lambda$) is evaluated}
\defSub{lambda}{The maximum value of the proportional summed difference between the partition at year and year+1 that indicates successful convergence}
\defSub{plus\_group}{Indicates if the convergence check applies only to the plus\_group of the partition}

\commandlabsubarg{initialisation\_phase}{type}{Cinitial}

\defSub{categories}{The list of categories for the Cinitial initialisation}
\defSub{table}{The table of data specifying the initial values by age}

\commandlabsubarg{initialisation\_phase}{type}{Derived}

\defSub{insert\_processes}{Specifies the additional processes that are not in the annual cycle, but should be be inserted into this initialisation phase}
\defSub{exclude\_processes}{Specifies the processes in the annual cycle that should be excluded from this initialisation phase}

\commandlabsubarg{initialisation\_phase}{type}{State\_Category\_By\_Age}

\defSub{categories}{The list of categories for the category state initialisation}
\defSub{min\_age}{The minimum age of values supplied in the definition of the category state}
\defSub{max\_age}{The maximum age of values supplied in the definition of the category state}
\defSub{table}{The table of data specifying the initial values by age}

\defComLab{length\_weight}{Define an object of type \emph{Length\_Weight}}

\defSub{label}{The label of the length-weight relationship}
\defSub{type}{The type of the length-weight relationship}

\commandlabsubarg{length\_weight}{type}{Basic}

\defSub{a}{The $a$ parameter ($W = a L^b$)}
\defSub{b}{The $b$ parameter ($W = a L^b$)}
\defSub{units}{The units for weights (grams, kilograms (kgs), or tonnes)}

\commandlabsubarg{length\_weight}{type}{None}


\defComLab{likelihood}{Define an object of type \emph{Likelihood}}


\commandlabsubarg{likelihood}{type}{Binomial}


\commandlabsubarg{likelihood}{type}{Binomial\_Approx}


\commandlabsubarg{likelihood}{type}{Dirichlet}


\commandlabsubarg{likelihood}{type}{Log\_Normal}


\commandlabsubarg{likelihood}{type}{Log\_Normal\_With\_Q}


\commandlabsubarg{likelihood}{type}{Multinomial}


\commandlabsubarg{likelihood}{type}{Normal}


\commandlabsubarg{likelihood}{type}{none}


\commandlabsubarg{likelihood}{type}{bernoulli}


\defComLab{mcmc}{Define an object of type \emph{MCMC}}

\defSub{label}{The label of the MCMC}
\defSub{type}{The MCMC method}
\defSub{length}{The number of iterations for the MCMC (including the burn in period)}
\defSub{burn\_in}{The number of iterations for the burn\_in period of the MCMC}
\defSub{active}{Indicates if this is the active MCMC algorithm}
\defSub{step\_size}{Initial step-size (as a multiplier of the approximate covariance matrix)}
\defSub{start}{The covariance multiplier for the starting point of the MCMC}
\defSub{keep}{The spacing between recorded values in the MCMC}
\defSub{max\_correlation}{The maximum absolute correlation in the covariance matrix of the proposal distribution}
\defSub{covariance\_adjustment\_method}{The method for adjusting small variances in the covariance proposal matrix}
\defSub{correlation\_adjustment\_diff}{The minimum non-zero variance times the range of the bounds in the covariance matrix of the proposal distribution}
\defSub{proposal\_distribution}{The shape of the proposal distribution (either the t or the normal distribution)}
\defSub{df}{The degrees of freedom of the multivariate t proposal distribution}
\defSub{adapt\_stepsize\_at}{The iteration numbers in which to check and resize the MCMC stepsize}
\defSub{adapt\_stepsize\_method}{The method to use to adapt the step size}
\defSub{adapt\_covariance\_matrix\_at}{The iteration number in which to adapt the covariance matrix}

\commandlabsubarg{mcmc}{type}{Hamiltonian}

\defSub{leapfrog\_steps}{Number of leapfrog steps}
\defSub{leapfrog\_delta}{Amount to leapfrog per step}
\defSub{gradient\_step\_size}{Step size to use when calculating gradient}

\commandlabsubarg{mcmc}{type}{Random\_Walk}


\defComLab{minimiser}{Define an object of type \emph{Minimiser}}

\defSub{label}{The minimiser label}
\defSub{type}{The type of minimiser to use}
\defSub{active}{Indicates if this minimiser is active}
\defSub{covariance}{Indicates if a covariance matrix should be generated}

\commandlabsubarg{minimiser}{type}{ADOLC}

\defSub{iterations}{The maximum number of iterations}
\defSub{evaluations}{The maximum number of evaluations}
\defSub{tolerance}{The tolerance of the gradient for convergence}
\defSub{step\_size}{The minimum step size before minimisation fails}
\defSub{parameter\_transformation}{The choice of parametrisation used to scale the parameters for ADOLC}

\commandlabsubarg{minimiser}{type}{Betadiff}

\defSub{iterations}{The maximum number of iterations}
\defSub{evaluations}{The maximum number of evaluations}
\defSub{tolerance}{The tolerance of the gradient for convergence}

\commandlabsubarg{minimiser}{type}{de\_solver}

\defSub{population\_size}{The number of candidate solutions to have in the population}
\defSub{crossover\_probability}{The minimiser's crossover probability}
\defSub{difference\_scale}{The scale to apply to new solutions when comparing candidates}
\defSub{max\_generations}{The maximum number of iterations to run}
\defSub{tolerance}{The total variance between the population and best candidate before acceptance}

\commandlabsubarg{minimiser}{type}{Deltadiff}

\defSub{iterations}{Maximum number of iterations}
\defSub{evaluations}{Maximum number of evaluations}
\defSub{tolerance}{Tolerance of the gradient for convergence}
\defSub{step\_size}{Minimum Step-size before minimisation fails}

\commandlabsubarg{minimiser}{type}{Numerical\_Differences}

\defSub{iterations}{The maximum number of iterations}
\defSub{evaluations}{The maximum number of evaluations}
\defSub{tolerance}{The tolerance of the gradient for convergence}
\defSub{step\_size}{The minimum step size before minimisation fails}

\defComLab{model}{Define an object of type \emph{Model}}

\defSub{type}{Type of model (either type=age or type=length)}
\defSub{base\_weight\_units}{Define the units for the base weight measurement unit (grams, kilograms (kgs), or tonnes). This will be the default unit of any weight input values}
\defSub{threads}{The number of threads to use for this model}

\commandlabsubarg{model}{type}{Age}

\defSub{start\_year}{Define the first year of the model, immediately following initialisation}
\defSub{final\_year}{Define the final year of the model, excluding years in the projection period}
\defSub{min\_age}{Minimum age of individuals in the population}
\defSub{max\_age}{Maximum age of individuals in the population}
\defSub{age\_plus}{Define the oldest age as a plus group}
\defSub{initialisation\_phases}{Define the labels of the phases of the initialisation}
\defSub{time\_steps}{Define the labels of the time steps, in the order that they are applied, to form the annual cycle}
\defSub{projection\_final\_year}{Define the final year of the model when running projections}
\defSub{length\_bins}{The minimum length in each length bin}
\defSub{length\_plus}{Specify whether there is a length plus group or not}
\defSub{length\_plus\_group}{Mean length of length plus group}

\defComLab{observation}{Define an object of type \emph{Observation}}

\defSub{label}{The label of the observation}
\defSub{type}{The type of observation}
\defSub{likelihood}{The type of likelihood to use}
\defSub{categories}{The category labels to use}
\defSub{delta}{The robustification value (delta) for the likelihood}
\defSub{simulation\_likelihood}{The simulation likelihood to use}
\defSub{likelihood\_multiplier}{The likelihood multiplier}
\defSub{error\_value\_multiplier}{The error value multiplier for likelihood}
\defSub{table}{The table of data specifying the observed values}
\defSub{table}{The table of data specifying the observed error values}

\commandlabsubarg{observation}{type}{Abundance}

\defSub{time\_step}{The label of the time step that the observation occurs in}
\defSub{catchability}{The label of the catchability coefficient (q)}
\defSub{selectivities}{The labels of the selectivities}
\defSub{process\_error}{The process error}
\defSub{years}{The years for which there are observations}
\defSub{table obs}{The table of data specifying the observed and error values}

\commandlabsubarg{observation}{type}{Biomass}

\defSub{time\_step}{The label of the time step that the observation occurs in}
\defSub{catchability}{The label of the catchability coefficient (q)}
\defSub{selectivities}{The labels of the selectivities}
\defSub{process\_error}{The process error}
\defSub{age\_weight\_labels}{The labels for the \command{$age\_weight$} block which corresponds to each category, to use the weight calculation method for biomass calculations)}
\defSub{years}{The years of the observed values}
\defSub{table obs}{The table of data specifying the observed and error values}

\commandlabsubarg{observation}{type}{Process\_Removals\_By\_Age}

\defSub{min\_age}{The minimum age}
\defSub{max\_age}{The maximum age}
\defSub{sum\_to\_one}{Scale year (row) observed values by the total so they sum to equal 1}
\defSub{simulated\_data\_sum\_to\_one}{Whether simulated data is discrete or scaled by totals to be proportions for each year}
\defSub{plus\_group}{Is the maximum age the age plus group}
\defSub{time\_step}{The label of time-step that the observation occurs in}
\defSub{years}{The years for which there are observations}
\defSub{process\_errors}{The process errors to use}
\defSub{ageing\_error}{The label of the ageing error to use}
\defSub{method\_of\_removal}{The label of the observed method of removals}
\defSub{mortality\_process}{The label of the mortality instantaneous process for the observation}
\defSub{table obs}{The table of data specifying the observed values}
\defSub{table error\_values}{The table of data specifying the error values}

\commandlabsubarg{observation}{type}{Process\_Removals\_By\_Age\_Retained}

\defSub{min\_age}{The minimum age}
\defSub{max\_age}{The maximum age}
\defSub{plus\_group}{Is the maximum age the age plus group?}
\defSub{time\_step}{The label of the time step that the observation occurs in}
\defSub{sum\_to\_one}{Scale the year (row) observed values by the total, so they sum to 1}
\defSub{simulated\_data\_sum\_to\_one}{Whether simulated data is discrete or scaled by totals to be proportions for each year}
\defSub{years}{The years for which there are observations}
\defSub{process\_errors}{The process errors to use}
\defSub{ageing\_error}{The label of the ageing error to use}
\defSub{method\_of\_removal}{The label of observed method of removals}
\defSub{mortality\_process}{The label of the mortality instantaneous process for the observation}
\defSub{table obs}{The table of data specifying the observed values}
\defSub{table error\_values}{The table of data specifying the error values}

\commandlabsubarg{observation}{type}{Process\_Removals\_By\_Age\_Retained\_Total}

\defSub{min\_age}{The minimum age}
\defSub{max\_age}{The maximum age}
\defSub{plus\_group}{Is the maximum age the age plus group?}
\defSub{time\_step}{The label of the time step that the observation occurs in}
\defSub{sum\_to\_one}{Scale the year (row) observed values by the total, so they sum to 1}
\defSub{simulated\_data\_sum\_to\_one}{Whether simulated data is discrete or scaled by totals to be proportions for each year}
\defSub{years}{The years for which there are observations}
\defSub{process\_errors}{The process errors to use}
\defSub{ageing\_error}{The label of the ageing error to use}
\defSub{method\_of\_removal}{The label of observed method of removals}
\defSub{mortality\_process}{The label of the mortality process for this observation}
\defSub{table obs}{The table of data specifying the observed values}
\defSub{table error\_values}{The table of data specifying the error values}

\commandlabsubarg{observation}{type}{Process\_Removals\_By\_Length}

\defSub{time\_step}{The time step to execute in}
\defSub{years}{The years for which there are observations}
\defSub{process\_errors}{The process errors to use}
\defSub{method\_of\_removal}{The label of observed method of removals}
\defSub{length\_bins}{The length bins}
\defSub{sum\_to\_one}{Scale the year (row) observed values by the total, so they sum to 1}
\defSub{simulated\_data\_sum\_to\_one}{Whether simulated data is discrete or scaled by totals to be proportions for each year}
\defSub{plus\_group}{Is the last length bin a plus group? (defaults to @model value)}
\defSub{mortality\_process}{The label of the mortality instantaneous process for the observation}
\defSub{table obs}{The table of data specifying the observed values}
\defSub{table error\_values}{The table of data specifying the error values}

\commandlabsubarg{observation}{type}{Process\_Removals\_By\_Length\_Retained}

\defSub{time\_step}{The time step to execute in}
\defSub{years}{The years for which there are observations}
\defSub{process\_errors}{The process errors to use}
\defSub{method\_of\_removal}{The label of observed method of removals}
\defSub{length\_bins}{The length bins}
\defSub{sum\_to\_one}{Scale the year (row) observed values by the total, so they sum to 1}
\defSub{simulated\_data\_sum\_to\_one}{Whether simulated data is discrete or scaled by totals to be proportions for each year}
\defSub{plus\_group}{Is the last length bin a plus group? (defaults to @model value)}
\defSub{mortality\_process}{The label of the mortality instantaneous process for the observation}
\defSub{table obs}{The table of data specifying the observed values}
\defSub{table error\_values}{The table of data specifying the error values}

\commandlabsubarg{observation}{type}{Process\_Removals\_By\_Length\_Retained\_Total}

\defSub{time\_step}{The time step to execute in}
\defSub{years}{The years for which there are observations}
\defSub{process\_errors}{The process errors to use}
\defSub{method\_of\_removal}{The label of observed method of removals}
\defSub{length\_bins}{The length bins}
\defSub{plus\_group}{Is the last length bin a plus group? (defaults to @model value)}
\defSub{sum\_to\_one}{Scale the year (row) observed values by the total, so they sum to 1}
\defSub{simulated\_data\_sum\_to\_one}{Whether simulated data is discrete or scaled by totals to be proportions for each year}
\defSub{mortality\_process}{The label of the mortality instantaneous process for the observation}
\defSub{table obs}{The table of data specifying the observed values}
\defSub{table error\_values}{The table of data specifying the error values}

\commandlabsubarg{observation}{type}{Process\_Removals\_By\_Weight}

\defSub{mortality\_process}{The label of the mortality instantaneous process for the observation}
\defSub{method\_of\_removal}{The label of observed method of removals}
\defSub{time\_step}{The time step to execute in}
\defSub{years}{The years for which there are observations}
\defSub{process\_errors}{The process errors to use}
\defSub{length\_weight\_cv}{The CV for the length-weight relationship}
\defSub{length\_weight\_distribution}{The distribution of the length-weight relationship}
\defSub{length\_bins}{The length bins}
\defSub{length\_bins\_n}{The average number in each length bin}
\defSub{units}{The units for the weight bins (grams, kilograms (kgs), or tonnes)}
\defSub{fishbox\_weight}{The target weight of each box}
\defSub{weight\_bins}{The weight bins}

\commandlabsubarg{observation}{type}{Proportions\_At\_Age}

\defSub{min\_age}{The minimum age}
\defSub{max\_age}{The maximum age}
\defSub{plus\_group}{Is the maximum age the age plus group?}
\defSub{time\_step}{The label of the time step that the observation occurs in}
\defSub{years}{The years of the observed values}
\defSub{selectivities}{The labels of the selectivities}
\defSub{process\_errors}{The process errors to use}
\defSub{ageing\_error}{The label of ageing error to use}
\defSub{sum\_to\_one}{Scale the year (row) observed values by the total, so they sum to 1}
\defSub{simulated\_data\_sum\_to\_one}{Whether simulated data is discrete or scaled by totals to be proportions for each year}
\defSub{table obs}{The table of data specifying the observed values}
\defSub{table error\_values}{The table of data specifying the error values}

\commandlabsubarg{observation}{type}{Proportions\_At\_Length}

\defSub{time\_step}{The label of the time step that the observation occurs in}
\defSub{years}{The years for which there are observations}
\defSub{selectivities}{The labels of the selectivities}
\defSub{process\_errors}{The process errors to use}
\defSub{length\_bins}{The length bins}
\defSub{plus\_group}{Is the last length bin a plus group?}
\defSub{sum\_to\_one}{Scale the year (row) observed values by the total, so they sum to 1}
\defSub{simulated\_data\_sum\_to\_one}{Whether simulated data is discrete or scaled by totals to be proportions for each year}
\defSub{table obs}{The table of data specifying the observed values}
\defSub{table error\_values}{The table of data specifying the error values}

\commandlabsubarg{observation}{type}{Proportions\_By\_Category}

\defSub{min\_age}{The minimum age}
\defSub{max\_age}{The maximum age}
\defSub{time\_step}{The label of the time step that the observation occurs in}
\defSub{plus\_group}{Use the age plus group?}
\defSub{years}{The years for which there are observations}
\defSub{selectivities}{The labels of the selectivities}
\defSub{categories2}{The target categories}
\defSub{selectivities2}{The target selectivities}

\commandlabsubarg{observation}{type}{Proportions\_Mature\_By\_Age}

\defSub{min\_age}{The minimum age}
\defSub{max\_age}{The maximum age}
\defSub{time\_step}{The label of time-step that the observation occurs in}
\defSub{plus\_group}{Use the age plus group?}
\defSub{years}{The years for which there are observations}
\defSub{ageing\_error}{The label of ageing error to use}
\defSub{total\_categories}{All category labels that were vulnerable to sampling at the time of this observation (not including the categories already given)}
\defSub{time\_step\_proportion}{The proportion through the mortality block of the time step when the observation is evaluated}

\commandlabsubarg{observation}{type}{Proportions\_Migrating}

\defSub{min\_age}{The minimum age}
\defSub{max\_age}{The maximum age}
\defSub{time\_step}{The label of the time step that the observation occurs in}
\defSub{plus\_group}{Is the maximum age the age plus group?}
\defSub{years}{The years for which there are observations}
\defSub{process\_errors}{The process errors to use}
\defSub{ageing\_error}{The label of the ageing error to use}
\defSub{process}{The process label}

\commandlabsubarg{observation}{type}{tag\_recapture\_by\_fishery}

\defSub{tagged\_categories}{The tagged categories that we want to generate recaptures for. Categories need to be space separated no use of the '+' category syntax.}
\defSub{time\_step}{The label of time-step that the observation occurs in}
\defSub{reporting\_rate}{The reporting rate for this observation}
\defSub{years}{The years for which there are observations}
\defSub{method\_of\_removal}{The label of the observed method of removals}
\defSub{mortality\_process}{The label of the mortality instantaneous process for the observation}
\defSub{table recaptured}{The table of recaptures in each year}

\commandlabsubarg{observation}{type}{Tag\_Recapture\_By\_Age}

\defSub{min\_age}{The minimum age}
\defSub{max\_age}{The maximum age}
\defSub{plus\_group}{Is the maximum age the age plus group?}
\defSub{years}{The years for which there are observations}
\defSub{time\_step}{The label of the time step that the observation occurs in}
\defSub{selectivities}{The labels of the selectivities used for untagged categories}
\defSub{tagged\_selectivities}{The labels of the tag category selectivities}
\defSub{tagged\_categories}{The  categories of tagged individuals}
\defSub{detection}{The probability of detecting a recaptured individual}
\defSub{dispersion}{The over-dispersion parameter, $\phi$}
\defSub{time\_step\_proportion}{The proportion through the mortality block of the time step when the observation is evaluated}
\defSub{table recaptured}{The table of data specifying the recaptures}
\defSub{table scanned}{The table of data specifying the scanned fish}

\commandlabsubarg{observation}{type}{Tag\_Recapture\_By\_Length}

\defSub{years}{The years for which there are observations}
\defSub{time\_step}{The time step to execute in}
\defSub{length\_bins}{The length bins}
\defSub{selectivities}{The labels of the selectivities used for untagged categories}
\defSub{tagged\_selectivities}{The labels of the tag category selectivities}
\defSub{tagged\_categories}{The  categories of tagged individuals}
\defSub{detection}{The probability of detecting a recaptured individual}
\defSub{dispersion}{The over-dispersion parameter, $\phi$}
\defSub{time\_step\_proportion}{The proportion through the mortality block of the time step when the observation is evaluated}
\defSub{table recaptured}{The table of data specifying the recaptures}
\defSub{table scanned}{The table of data specifying the scanned fish}

\commandlabsubarg{observation}{type}{age\_length}

\defSub{time\_step}{The label of the time step that the observation occurs in}
\defSub{selectivities}{The labels of the selectivities, one for each combined category}
\defSub{numerator\_categories}{A combined category label that defines categories that make up the numerator}
\defSub{year}{The year this observation occurred in}
\defSub{sample\_type}{The sample type}
\defSub{ages}{vector of observed ages}
\defSub{lengths}{vector of observed lengths}
\defSub{ageing\_error}{The label of ageing error to use}

\defComLab{observation}{Define an object of type \emph{Observation}}

\defSub{label}{The label of the observation}
\defSub{type}{The type of observation}
\defSub{likelihood}{The type of likelihood to use}
\defSub{categories}{The category labels to use}
\defSub{delta}{The robustification value (delta) for the likelihood}
\defSub{simulation\_likelihood}{The simulation likelihood to use}
\defSub{likelihood\_multiplier}{The likelihood multiplier}
\defSub{error\_value\_multiplier}{The error value multiplier for likelihood}

\commandlabsubarg{observation}{type}{Abundance}

\defSub{time\_step}{The label of the time step that the observation occurs in}
\defSub{catchability}{The label of the catchability coefficient (q)}
\defSub{selectivities}{The labels of the selectivities}
\defSub{process\_error}{The process error}
\defSub{years}{The years for which there are observations}
\defSub{table obs}{The table of data specifying the observed and error values}

\commandlabsubarg{observation}{type}{Biomass}

\defSub{time\_step}{The label of the time step that the observation occurs in}
\defSub{catchability}{The label of the catchability coefficient (q)}
\defSub{selectivities}{The labels of the selectivities}
\defSub{process\_error}{The process error}
\defSub{age\_weight\_labels}{The labels for the \command{$age\_weight$} block which corresponds to each category, to use the weight calculation method for biomass calculations)}
\defSub{years}{The years of the observed values}
\defSub{table obs}{The table of data specifying the observed and error values}

\commandlabsubarg{observation}{type}{Process\_Removals\_By\_Length}

\defSub{length\_bins}{Bespoke length bins for the observation. They need to be a subset of the \command{model} length bins.}
\defSub{sum\_to\_one}{Scale the year (row) observed values by the total, so they sum to 1}
\defSub{simulated\_data\_sum\_to\_one}{Whether simulated data are discrete or scaled by totals to be proportions for each year}
\defSub{plus\_group}{Is the maximum length bin a plus group}
\defSub{time\_step}{The label of time-step that the observation occurs in}
\defSub{years}{The years for which there are observations}
\defSub{process\_errors}{The label of process error to use}
\defSub{method\_of\_removal}{The label of the observed method of removals}
\defSub{mortality\_instantaneous\_process}{The label of the mortality instantaneous process for the observation}
\defSub{table obs}{The table of data specifying the observed values}
\defSub{table error\_values}{The table of data specifying the error values}

\commandlabsubarg{observation}{type}{Proportions\_At\_Length}

\defSub{length\_bins}{Bespoke length bins for the observation. They need to be a subset of the \command{model}{length\_bins}.}
\defSub{plus\_group}{Is the maximum length bin a plus group}
\defSub{time\_step}{The label of the time step that the observation occurs in}
\defSub{years}{The years of the observed values}
\defSub{selectivities}{The labels of the selectivities}
\defSub{process\_errors}{The process error}
\defSub{sum\_to\_one}{Scale the year (row) observed values by the total, so they sum to 1}
\defSub{simulated\_data\_sum\_to\_one}{Whether simulated data is discrete or scaled by totals to be proportions for each year}
\defSub{table obs}{The table of data specifying the observed values}
\defSub{table error\_values}{The table of data specifying the error values}

\commandlabsubarg{observation}{type}{Proportions\_By\_Category}

\defSub{length\_bins}{Bespoke length bins for the observation. They need to be a subset of the \command{model}{length\_bins}.}
\defSub{time\_step}{The label of the time step that the observation occurs in}
\defSub{plus\_group}{Use the age plus group?}
\defSub{years}{The years for which there are observations}
\defSub{selectivities}{The labels of the selectivities}
\defSub{total\_categories2}{Categories in the denominator}
\defSub{total\_selectivities}{Selectivities to apply to the total categories}

\commandlabsubarg{observation}{type}{Tag\_Recapture\_By\_Length}

\defSub{years}{The years for which there are observations}
\defSub{time\_step}{The time step to execute in}
\defSub{length\_bins}{The length bins}
\defSub{selectivities}{The labels of the selectivities used for untagged categories}
\defSub{tagged\_selectivities}{The labels of the tag category selectivities}
\defSub{detection}{The probability of detecting a recaptured individual}
\defSub{dispersion}{The overdispersion parameter (phi)}
\defSub{time\_step\_proportion}{The proportion through the mortality block of the time step when the observation is evaluated}
\defSub{table recaptured}{The table of data specifying the recaptures}
\defSub{table scanned}{The table of data specifying the scanned fish}

\commandlabsubarg{observation}{type}{tag\_recapture\_by\_length\_for\_growth}

\defSub{years}{The years for which there are observations}
\defSub{time\_step}{The time step to execute in}
\defSub{length\_bins}{The length bins}
\defSub{selectivities}{The labels of the selectivities used for untagged categories}
\defSub{tagged\_selectivities}{The labels of the tag category selectivities}
\defSub{detection}{The probability of detecting a recaptured individual}
\defSub{dispersion}{The overdispersion parameter (phi)}
\defSub{time\_step\_proportion}{The proportion through the mortality block of the time step when the observation is evaluated}
\defSub{table recaptured}{The table of data specifying the recaptures}

\defComLab{parameter\_transformation}{Define an object of type \emph{parameter\_transformation}}

\defSub{label}{Label for the transformation block}
\defSub{type}{The type of transformation}
\defSub{prior\_applies\_to\_restored\_parameters}{If the prior applies to the parameters (true) with jacobian (if it exists) or prior applies to transformed\_parameter (false) with no jacobian}
\defSub{parameters}{The label of the parameters used in the transformation}

\commandlabsubarg{parameter\_transformation}{type}{log}


\commandlabsubarg{parameter\_transformation}{type}{logistic}

\defSub{lower\_bound}{Lower bound for the transformation}
\defSub{upper\_bound}{Upper bound for the transformation}

\commandlabsubarg{parameter\_transformation}{type}{inverse}


\commandlabsubarg{parameter\_transformation}{type}{difference\_parameter}


\commandlabsubarg{parameter\_transformation}{type}{average\_difference}


\commandlabsubarg{parameter\_transformation}{type}{log\_sum}


\commandlabsubarg{parameter\_transformation}{type}{orthogonal}


\commandlabsubarg{parameter\_transformation}{type}{sum\_to\_one}


\commandlabsubarg{parameter\_transformation}{type}{simplex}

\defSub{sum\_to\_one}{Apply the sum\_to\_one constraint}

\commandlabsubarg{parameter\_transformation}{type}{sqrt}


\defComLab{penalty}{Define an object of type \emph{Penalty}}

\defSub{label}{The label of the penalty}
\defSub{type}{The type of penalty}

\commandlabsubarg{penalty}{type}{Process}

\defSub{multiplier}{The penalty multiplier}
\defSub{log\_scale}{Indicates if the sums of squares will be calculated on the log scale}

\defComLab{process}{Define an object of type \emph{Process}}

\defSub{label}{The label of the process}
\defSub{type}{The type of process}

\commandlabsubarg{process}{type}{Ageing}

\defSub{categories}{The labels of the categories to age}

\commandlabsubarg{process}{type}{Maturation}

\defSub{from}{The list of categories to mature from}
\defSub{to}{The list of categories to mature to}
\defSub{selectivities}{The list of selectivities to use for maturation}
\defSub{years}{The years to be associated with the maturity rates}
\defSub{rates}{The rates to mature for each year}

\commandlabsubarg{process}{type}{Mortality\_Constant\_Rate}

\defSub{categories}{The list of category labels}
\defSub{m}{The mortality rates}
\defSub{time\_step\_proportions}{The time step proportions for the mortality rates}
\defSub{relative\_m\_by\_age}{The list of mortality by age ogive labels for the categories}

\commandlabsubarg{process}{type}{Mortality\_Constant\_Exploitation}

\defSub{categories}{The list of category labels}
\defSub{u}{The exploitation rates}
\defSub{time\_step\_proportions}{The time step proportions for the exploitation rates}
\defSub{relative\_u\_by\_age}{The list of exploitation by age ogive labels for the categories}

\commandlabsubarg{process}{type}{Mortality\_Disease\_Rate}

\defSub{categories}{The list of category labels}
\defSub{disease\_mortality\_rate}{The disease mortality rates}
\defSub{year\_effects}{Annual deviations around the disease mortality rate}
\defSub{selectivities}{The list of selectivities}
\defSub{years}{Years in which to apply the disease mortality in}

\commandlabsubarg{process}{type}{Mortality\_Event}

\defSub{categories}{The categories}
\defSub{years}{The years in which to apply the mortality process}
\defSub{catches}{The number of removals (catches) to apply for each year}
\defSub{u\_max}{The maximum exploitation rate ($U\_{max}$)}
\defSub{selectivities}{The list of selectivities}
\defSub{penalty}{The label of the penalty to apply if the total number of removals cannot be taken}

\commandlabsubarg{process}{type}{Mortality\_Event\_Biomass}

\defSub{categories}{The category labels}
\defSub{selectivities}{The labels of the selectivities for each of the categories}
\defSub{years}{The years in which to apply the mortality process}
\defSub{catches}{The biomass of removals (catches) to apply for each year}
\defSub{u\_max}{The maximum exploitation rate ($U\_{max}$)}
\defSub{penalty}{The label of the penalty to apply if the total biomass of removals cannot be taken}

\commandlabsubarg{process}{type}{Mortality\_Holling\_Rate}

\defSub{prey\_categories}{The prey categories labels}
\defSub{predator\_categories}{The predator categories labels}
\defSub{is\_abundance}{Is vulnerable amount of prey and predator an abundance [true] or biomass [false]}
\defSub{a}{Parameter a}
\defSub{b}{Parameter b}
\defSub{x}{This parameter controls the functional form: Holling function type 2 (x=2) or 3 (x=3), or generalised (Michaelis Menten, x>=1)}
\defSub{u\_max}{The maximum exploitation rate ($U\_{max}$)}
\defSub{prey\_selectivities}{The selectivities for prey categories}
\defSub{predator\_selectivities}{The selectivities for predator categories}
\defSub{penalty}{The label of penalty}
\defSub{years}{The years in which to apply the mortality process}
\defSub{table}{The table of data specifying the predator selectivities}
\defSub{table}{The table of data specifying the prey selectivities}

\commandlabsubarg{process}{type}{mortality\_initialisation\_event}

\defSub{categories}{The categories}
\defSub{catch}{The number of removals (catches) to apply for each year}
\defSub{u\_max}{The maximum exploitation rate ($U\_{max}$)}
\defSub{selectivities}{The list of selectivities}
\defSub{penalty}{The label of the penalty to apply if the total number of removals cannot be taken}
\defSub{table}{The table of data specifying the catches for each fishery, the categories, years, and the $U_max$}

\commandlabsubarg{process}{type}{Mortality\_Initialisation\_Event\_Biomass}

\defSub{categories}{The categories}
\defSub{catch}{The number of removals (catches) to apply for each year}
\defSub{u\_max}{The maximum exploitation rate ($U\_{max}$)}
\defSub{selectivities}{The list of selectivities}
\defSub{penalty}{The label of the penalty to apply if the total number of removals cannot be taken}

\commandlabsubarg{process}{type}{mortality\_initialisation\_baranov}

\defSub{categories}{The categories}
\defSub{fishing\_mortality}{The fishing mortality to apply}
\defSub{selectivities}{The list of selectivities for each category}

\commandlabsubarg{process}{type}{mortality\_hybrid}

\defSub{categories}{The categories for to apply natural mortality to}
\defSub{m}{The natural mortality rates for each category}
\defSub{time\_step\_proportions}{The time step proportions for natural mortality}
\defSub{biomass}{Switch to indicate if the catches are biomasses or abundances}
\defSub{relative\_m\_by\_age}{The M-by-age selectivities to apply to each of the categories for natural mortality}
\defSub{max\_f}{Maximum \(F\) allowed}
\defSub{f\_iterations}{The number of tuning iterations to solve \(F\)}
\defSub{table catches}{The table of data specifying the catches for each fishery and year}
\defSub{table method}{The table of data specifying which fishery interacts with which category, selectivities, penalty, and time-step for each fishery and annual duration of the fishery.}

\commandlabsubarg{process}{type}{Mortality\_Instantaneous}

\defSub{categories}{The categories for instantaneous mortality}
\defSub{m}{The natural mortality rates for each category}
\defSub{time\_step\_proportions}{The time step proportions for natural mortality}
\defSub{biomass}{Switch to indicate if the catches are biomasses or abundances}
\defSub{relative\_m\_by\_age}{The M-by-age selectivities to apply to each of the categories for natural mortality}
\defSub{table catches}{The table of data specifying the catches for each fishery and year}
\defSub{table method}{The table of data specifying which fishery interacts with which category, selectivities, time-step, penalties and \(u_{max}\) for each fishery and year.}

\commandlabsubarg{process}{type}{Mortality\_Instantaneous\_Retained}

\defSub{categories}{The categories for instantaneous mortality}
\defSub{m}{The natural mortality rates for each category}
\defSub{time\_step\_proportions}{The time step proportions for natural mortality}
\defSub{relative\_m\_by\_age}{The M-by-age selectivities to apply on the categories for natural mortality}
\defSub{table}{The table of data specifying the catches for each fishery, the categories, years, and the $U_max$}

\commandlabsubarg{process}{type}{Mortality\_Prey\_Suitability}

\defSub{prey\_categories}{The prey categories labels}
\defSub{predator\_categories}{The predator categories labels}
\defSub{consumption\_rate}{The predator consumption rate}
\defSub{electivities}{The prey electivities}
\defSub{u\_max}{The maximum exploitation rate ($U\_{max}$)}
\defSub{prey\_selectivities}{The selectivities for prey categories}
\defSub{predator\_selectivities}{The selectivities for predator categories}
\defSub{penalty}{The label of the penalty}
\defSub{years}{The year that process occurs}

\commandlabsubarg{process}{type}{markovian\_movement}

\defSub{from}{The categories to transition from}
\defSub{to}{The categories to transition to}
\defSub{proportions}{The proportions to transition for each category}
\defSub{selectivities}{The selectivities to apply to each proportion}

\commandlabsubarg{process}{type}{null\_process}


\commandlabsubarg{process}{type}{Recruitment\_Beverton\_Holt}

\defSub{categories}{The category labels}
\defSub{r0}{R0, the mean recruitment used to scale annual recruits or initialise the model}
\defSub{b0}{B0, the SSB corresponding to R0, and used to scale annual recruits or initialise the model}
\defSub{proportions}{The proportion for each category}
\defSub{age}{The age at recruitment}
\defSub{ssb\_offset}{The spawning biomass year offset}
\defSub{steepness}{Steepness (h)}
\defSub{ssb}{The SSB label (i.e., the derived quantity label)}
\defSub{b0\_initialisation\_phase}{The initialisation phase label that B0 is from}
\defSub{ycs\_values}{Deprecated}
\defSub{ycs\_years}{Deprecated}
\defSub{standardise\_ycs\_years}{Deprecated}
\defSub{recruitment\_multipliers}{The recruitment values also termed year class strengths.}
\defSub{standardise\_years}{The years that are included for year class standardisation, they refer to the recruited year not spawning or year class year.}

\commandlabsubarg{process}{type}{Recruitment\_Beverton\_Holt\_With\_Deviations}

\defSub{categories}{The category labels}
\defSub{r0}{R0, the mean recruitment used to scale annual recruits or initialise the model}
\defSub{b0}{B0, the SSB corresponding to R0, and used to scale annual recruits or initialise the model}
\defSub{proportions}{The proportion for each category}
\defSub{age}{The age at recruitment}
\defSub{ssb\_offset}{The spawning biomass year offset}
\defSub{steepness}{Steepness (h)}
\defSub{ssb}{The SSB label (i.e., the derived quantity label)}
\defSub{sigma\_r}{The standard deviation of recruitment, $\sigma_R$}
\defSub{b\_max}{The maximum bias adjustment}
\defSub{last\_year\_with\_no\_bias}{The last year with no bias adjustment}
\defSub{first\_year\_with\_bias}{The first year with full bias adjustment}
\defSub{last\_year\_with\_bias}{The last year with full bias adjustment}
\defSub{first\_recent\_year\_with\_no\_bias}{The first recent year with no bias adjustment}
\defSub{b0\_initialisation\_phase}{The initialisation phase label that B0 is from}
\defSub{deviation\_values}{The recruitment deviation values}
\defSub{deviation\_years}{Deprecated}

\commandlabsubarg{process}{type}{Recruitment\_Ricker}

\defSub{categories}{The category labels}
\defSub{r0}{R0, the mean recruitment used to scale annual recruits or initialise the model}
\defSub{b0}{B0, the SSB corresponding to R0, and used to scale annual recruits or initialise the model}
\defSub{proportions}{The proportion for each category}
\defSub{age}{The age at recruitment}
\defSub{ssb\_offset}{The spawning biomass year offset}
\defSub{steepness}{Steepness (h)}
\defSub{ssb}{The SSB label (i.e., the derived quantity label)}
\defSub{b0\_initialisation\_phase}{The initialisation phase label that B0 is from}
\defSub{ycs\_values}{Deprecated}
\defSub{ycs\_years}{Deprecated}
\defSub{standardise\_ycs\_years}{Deprecated}
\defSub{recruitment\_multipliers}{The recruitment values also termed year class strengths.}
\defSub{standardise\_years}{The years that are included for year class standardisation, they refer to the recruited year not spawning or year class year.}

\commandlabsubarg{process}{type}{Recruitment\_Constant}

\defSub{categories}{The categories}
\defSub{proportions}{The proportion for each category}
\defSub{age}{The age at recruitment}
\defSub{r0}{R0, the recruitment used for annual recruits and initialise the model}

\commandlabsubarg{process}{type}{Survival\_Constant\_Rate}

\defSub{categories}{The list of categories}
\defSub{s}{The survival rates}
\defSub{time\_step\_proportions}{The time step proportions for the survival rate $S$}
\defSub{selectivities}{The selectivity labels for each category}

\commandlabsubarg{process}{type}{Tag\_By\_Age}

\defSub{from}{The categories that are selected for tagging (i.e, transition from)}
\defSub{to}{The categories that have tags (i.e., transition to)}
\defSub{min\_age}{The minimum age tagged}
\defSub{max\_age}{The maximum age tagged}
\defSub{penalty}{The penalty label}
\defSub{u\_max}{The maximum exploitation rate ($U\_{max}$)}
\defSub{years}{The years to execute the tagging in}
\defSub{initial\_mortality}{The initial mortality value (as a instantaneous proportion)}
\defSub{initial\_mortality\_selectivity}{The initial mortality selectivity label}
\defSub{selectivities}{The selectivity labels}
\defSub{n}{the total number of tagged fish}
\defSub{table}{The table of data specifying the either the numbers or proportions to tag from and to each category and year}
\defSub{table numbers}{The table of releases as numbers for the process}
\defSub{table proportions}{The table of releases as numbers for the process}
\defSub{tolerance}{Tolerance for checking the specified proportions sum to one}

\commandlabsubarg{process}{type}{Tag\_By\_Length}

\defSub{from}{The categories that are selected for tagging (i.e, transition from)}
\defSub{to}{The categories that have tags (i.e., transition to)}
\defSub{penalty}{The penalty label}
\defSub{u\_max}{The maximum exploitation rate ($U\_{max}$)}
\defSub{compatibility\_option}{Backwards compatibility option: either casal2 (the default) or casal. This affects the penalty and age-length calculations}
\defSub{years}{The years to execute the tagging events in}
\defSub{initial\_mortality}{The initial mortality value (as a instantaneous proportion)}
\defSub{initial\_mortality\_selectivity}{The initial mortality selectivity label}
\defSub{selectivities}{The selectivity labels}
\defSub{n}{the total number of tagged fish}
\defSub{table}{The table of data specifying the either the numbers or proportions to tag from and to each category and year}
\defSub{table numbers}{The table of releases as numbers for the process}
\defSub{table proportions}{The table of releases as numbers for the process}
\defSub{tolerance}{Tolerance for checking the specified proportions sum to one}

\commandlabsubarg{process}{type}{Tag\_Loss}

\defSub{categories}{The list of categories}
\defSub{tag\_loss\_rate}{The instantaneous tag loss rates}
\defSub{time\_step\_proportions}{The time step proportions for tag loss}
\defSub{tag\_loss\_type}{The type of tag loss}
\defSub{selectivities}{The selectivities}
\defSub{year}{The year the first tagging release process was executed}

\commandlabsubarg{process}{type}{Tag\_Loss\_Empirical}

\defSub{categories}{The list of categories}
\defSub{tag\_loss\_rate}{The instantaneous tag loss rates}
\defSub{time\_step\_proportions}{The time step proportions for tag loss}
\defSub{selectivities}{The selectivities}
\defSub{year}{The year the first tagging release process was executed}
\defSub{years\_at\_liberty}{The years at liberty that the tag\_loss\_rate applies to}

\commandlabsubarg{process}{type}{Transition\_Category}

\defSub{from}{The categories to transition from}
\defSub{to}{The categories to transition to}
\defSub{proportions}{The proportions to transition for each category}
\defSub{selectivities}{The selectivities to apply to each proportion}
\defSub{include\_in\_mortality\_block}{Include this process within the mortality block}

\commandlabsubarg{process}{type}{Transition\_Category\_By\_Age}

\defSub{from}{The categories to transition from}
\defSub{to}{The categories to transition to}
\defSub{min\_age}{The minimum age to transition}
\defSub{max\_age}{The maximum age to transition}
\defSub{penalty}{The penalty label}
\defSub{u\_max}{The maximum exploitation rate ($U\_{max}$)}
\defSub{years}{The years to execute the transition in}
\defSub{table n}{The table of numbers at age to transition from and to each category}

\defComLab{process}{Define an object of type \emph{Process}}

\defSub{label}{The label of the process}
\defSub{type}{The type of process}

\commandlabsubarg{process}{type}{Mortality\_Constant\_Rate}

\defSub{categories}{The list of category labels}
\defSub{m}{The mortality rates}
\defSub{time\_step\_proportions}{The time step proportions for the mortality rates}
\defSub{relative\_m\_by\_length}{The list of mortality by length ogive labels for the categories}

\defSub{categories}{The list of category labels}
\defSub{u}{The exploitation rates}
\defSub{time\_step\_proportions}{The time step proportions for the exploitation rates}
\defSub{relative\_u\_by\_length}{The list of exploitation by length ogive labels for the categories}

\commandlabsubarg{process}{type}{Mortality\_Disease\_Rate}

\defSub{categories}{The list of category labels}
\defSub{disease\_mortality\_rate}{The disease mortality rates}
\defSub{year\_effects}{Annual deviations around the disease mortality rate}
\defSub{selectivities}{The list of selectivities}
\defSub{years}{Years in which to apply the disease mortality in}

\commandlabsubarg{process}{type}{Mortality\_Instantaneous}

\defSub{categories}{The categories for instantaneous mortality}
\defSub{m}{The natural mortality rates for each category}
\defSub{time\_step\_proportions}{The time step proportions for natural mortality}
\defSub{biomass}{Switch to indicate if the catches are biomasses or abundances}
\defSub{relative\_m\_by\_length}{The M-by-length selectivities to apply to each of the categories for natural mortality}
\defSub{table}{The table of data specifying the catches for each fishery, the categories, years, and the $U_max$}

\commandlabsubarg{process}{type}{null\_process}


\commandlabsubarg{process}{type}{Recruitment\_Beverton\_Holt}

\defSub{categories}{The category labels}
\defSub{r0}{R0, the mean recruitment used to scale annual recruits or initialise the model}
\defSub{b0}{B0, the SSB corresponding to R0, and used to scale annual recruits or initialise the model}
\defSub{proportions}{The proportion for each category}
\defSub{initial\_mean\_length}{The initial mean length at recruitment}
\defSub{initial\_length\_cv}{The initial length cv at recruitment}
\defSub{ssb\_offset}{The spawning biomass year offset}
\defSub{steepness}{Steepness (h)}
\defSub{ssb}{The SSB label (i.e., the derived quantity label)}
\defSub{b0\_initialisation\_phase}{The initialisation phase label that B0 is from}
\defSub{ycs\_values}{Deprecated}
\defSub{ycs\_years}{Deprecated}
\defSub{standardise\_ycs\_years}{Deprecated}
\defSub{recruitment\_multipliers}{The recruitment values also termed year class strengths.}
\defSub{standardise\_years}{The years that are included for year class standardisation, they refer to the recruited year not spawning or year class year.}

\commandlabsubarg{process}{type}{Recruitment\_Constant}

\defSub{categories}{The categories}
\defSub{proportions}{The proportion for each category}
\defSub{length\_bins}{The length bin that recruits are uniformly distributed over at the time of recruitment}
\defSub{r0}{R0, the recruitment used for annual recruits and initialise the model}

\commandlabsubarg{process}{type}{Tagging}

\defSub{from}{The categories that are selected for tagging (i.e, transition from)}
\defSub{to}{The categories that have tags (i.e., transition to)}
\defSub{penalty}{The penalty label}
\defSub{u\_max}{The maximum exploitation rate ($U\_{max}$)}
\defSub{compatibility\_option}{Backwards compatibility option: either casal2 (the default) or casal. This affects the penalty and age-length calculations}
\defSub{years}{The years to execute the tagging events in}
\defSub{initial\_mortality}{The initial mortality to apply to tags as a proportion}
\defSub{initial\_mortality\_selectivity}{The initial mortality selectivity label}
\defSub{selectivities}{The selectivity labels}
\defSub{n}{The total number of tags to apply}
\defSub{table}{The table of data specifying the either the numbers or proportions to tag from and to each category and year}
\defSub{table numbers}{The table of releases as numbers for the process}
\defSub{table proportions}{The table of releases as numbers for the process}
\defSub{tolerance}{Tolerance for checking the specified proportions sum to one}

\commandlabsubarg{process}{type}{Transition\_Category}

\defSub{from}{The categories to transition from}
\defSub{to}{The categories to transition to}
\defSub{proportions}{The proportions to transition for each category}
\defSub{selectivities}{The selectivities to apply to each proportion}
\defSub{include\_in\_mortality\_block}{Include this process within the mortality block}

\defComLab{profile}{Define an object of type \emph{Profile}}

\defSub{label}{The label of the profile}
\defSub{steps}{The number of steps between the lower and upper bound}
\defSub{lower\_bound}{The lower value of the range}
\defSub{upper\_bound}{The upper value of the range}
\defSub{parameter}{The free parameter to profile}
\defSub{same}{A free parameter that is constrained to have the same value as the parameter being profiled}

\defComLab{project}{Define an object of type \emph{Project}}

\defSub{label}{Label}
\defSub{type}{Type}
\defSub{years}{Years to recalculate the values}
\defSub{parameter}{Parameter to project}
\defSub{multiplier}{Multiplier that is applied to the projected value}

\commandlabsubarg{project}{type}{Constant}

\defSub{values}{The values to assign to the addressable}

\commandlabsubarg{project}{type}{multiple\_values}

\defSub{table values}{The table of data specifying the projected values to use for each row of the \texttt{-i} or \texttt{-I} file}

\commandlabsubarg{project}{type}{Empirical\_Sampling}

\defSub{start\_year}{The start year of sampling}
\defSub{final\_year}{The final year of sampling}

\commandlabsubarg{project}{type}{Lognormal}

\defSub{mean}{The mean of the lognormal process}
\defSub{sigma}{The standard deviation (sigma) of the lognormal sampling}

\commandlabsubarg{project}{type}{Lognormal\_Empirical}

\defSub{mean}{The mean of the Gaussian process}
\defSub{start\_year}{The start year of sampling}
\defSub{final\_year}{The final year of sampling}

\defComLab{report}{Define an object of type \emph{Report}}

\defSub{label}{The report label}
\defSub{type}{The report type}
\defSub{file\_name}{The file name. If not supplied, then output is directed to standard out}
\defSub{write\_mode}{Specify if any previous file with the same name should be overwritten, appended to, or is generated using a sequential suffix}
\defSub{format}{Report output format}

\commandlabsubarg{report}{type}{Default}

\defSub{catchabilities}{Report catchabilities}
\defSub{derived\_quantities}{Report derived quantities}
\defSub{observations}{Report observations}
\defSub{processes}{Report processes}
\defSub{projects}{Report projects}
\defSub{selectivities}{Report selectivities}
\defSub{time\_varying}{Report time-varying parameters}
\defSub{parameter\_transformations}{Report all parameter transformations}

\commandlabsubarg{report}{type}{Addressable}

\defSub{parameter}{The addressable parameter name}
\defSub{years}{Define the years that the report is generated for}
\defSub{time\_step}{Defines the time-step that the report applies to}

\commandlabsubarg{report}{type}{Age\_Length}

\defSub{time\_step}{The time step label}
\defSub{years}{The years for the report}
\defSub{age\_length}{The age-length label}

\commandlabsubarg{report}{type}{growth\_increment}

\defSub{time\_step}{The time step label}
\defSub{years}{The years for the report}
\defSub{growth\_increment}{The growth-increment label}

\commandlabsubarg{report}{type}{Ageing\_Error\_Matrix}

\defSub{ageing\_error}{The ageing error label}

\commandlabsubarg{report}{type}{Catchability}

\defSub{catchability}{The catchability label}

\commandlabsubarg{report}{type}{Correlation\_Matrix}


\commandlabsubarg{report}{type}{Covariance\_Matrix}


\commandlabsubarg{report}{type}{Derived\_Quantity}

\defSub{derived\_quantity}{The derived quantity label}

\commandlabsubarg{report}{type}{Equation\_Test}

\defSub{equation}{The equation to do a test run of}

\commandlabsubarg{report}{type}{Estimate\_Summary}


\commandlabsubarg{report}{type}{Estimate\_Value}


\commandlabsubarg{report}{type}{Estimation\_Result}


\commandlabsubarg{report}{type}{Hessian\_Matrix}


\commandlabsubarg{report}{type}{Initialisation}


\commandlabsubarg{report}{type}{Initialisation\_Partition}


\commandlabsubarg{report}{type}{MCMC\_Covariance}


\commandlabsubarg{report}{type}{MCMC\_Objective}

\defSub{file\_name}{The file name. If not supplied the default filename is used}
\defSub{write\_mode}{Has a different default to the rest of the reports.}

\commandlabsubarg{report}{type}{MCMC\_Sample}

\defSub{file\_name}{The file name. If not supplied the default filename is used}
\defSub{write\_mode}{Has a different default to the rest of the reports.}

\commandlabsubarg{report}{type}{Objective\_Function}


\commandlabsubarg{report}{type}{Observation}

\defSub{observation}{The observation label}
\defSub{normalised\_residuals}{Print Normalised Residuals?}
\defSub{pearsons\_residuals}{Print Pearsons Residuals?}

\commandlabsubarg{report}{type}{Output\_Parameters}


\commandlabsubarg{report}{type}{parameter\_transformation}

\defSub{parameter\_transformation}{label of parameter transformation block}

\commandlabsubarg{report}{type}{Partition}

\defSub{time\_step}{Time Step label}
\defSub{years}{Years}

\commandlabsubarg{report}{type}{Partition\_Biomass}

\defSub{time\_step}{The time step label}
\defSub{years}{The years for the report}

\commandlabsubarg{report}{type}{Process}

\defSub{process}{The process label that is reported}

\commandlabsubarg{report}{type}{Profile}


\commandlabsubarg{report}{type}{Project}

\defSub{project}{The project label that is reported}

\commandlabsubarg{report}{type}{Random\_Number\_Seed}


\commandlabsubarg{report}{type}{Selectivity}

\defSub{selectivity}{Selectivity name}

\commandlabsubarg{report}{type}{selectivity\_by\_year}

\defSub{selectivity}{Selectivity name}
\defSub{years}{years to report the selectivity in}
\defSub{time\_step}{Time step label}

\commandlabsubarg{report}{type}{Simulated\_Observation}

\defSub{observation}{The observation label}

\commandlabsubarg{report}{type}{Time\_Varying}

\defSub{time\_varying}{The time varying label that is reported}

\defComLab{selectivity}{Define an object of type \emph{Selectivity}}

\defSub{label}{The label for the selectivity}
\defSub{type}{The type of selectivity}
\defSub{length\_based}{Is the selectivity length based?}
\defSub{intervals}{The number of quantiles to evaluate a length-based selectivity over the age-length distribution}
\defSub{values}{}
\defSub{length\_values}{}

\commandlabsubarg{selectivity}{type}{All\_Values}

\defSub{v}{The v parameter}

\commandlabsubarg{selectivity}{type}{All\_Values\_Bounded}

\defSub{l}{The low value (L)}
\defSub{h}{The high value (H)}
\defSub{v}{The v parameter}

\commandlabsubarg{selectivity}{type}{Constant}

\defSub{c}{The constant value}
\defSub{beta}{The minimum age/length for which the selectivity applies}

\commandlabsubarg{selectivity}{type}{Double\_Exponential}

\defSub{x0}{The $x0$ parameter}
\defSub{x1}{The $x1$ parameter}
\defSub{x2}{The $x2$ parameter}
\defSub{y0}{The $y0$ parameter}
\defSub{y1}{The $y1$ parameter}
\defSub{y2}{The $y2$ parameter}
\defSub{alpha}{The maximum value of the selectivity}
\defSub{beta}{The minimum age/length for which the selectivity applies}

\commandlabsubarg{selectivity}{type}{Double\_Normal}

\defSub{mu}{The mean ($\mu$}
\defSub{sigma\_l}{The left-hand variance (sigma\_l) parameter}
\defSub{sigma\_r}{The right-hand variance (sigma\_r) parameter}
\defSub{alpha}{The maximum value of the selectivity}
\defSub{beta}{The minimum age/length for which the selectivity applies}

\commandlabsubarg{selectivity}{type}{Double\_Normal\_Plateau}

\defSub{a1}{The a1 ($a1$}
\defSub{a2}{The a2 ($a2$}
\defSub{sigma\_l}{The left-hand variance (sigma\_l) parameter}
\defSub{sigma\_r}{The right-hand variance (sigma\_r) parameter}
\defSub{alpha}{The maximum value of the selectivity}
\defSub{beta}{The minimum age/length for which the selectivity applies}

\commandlabsubarg{selectivity}{type}{Double\_Normal\_Stock\_Synthesis}

\defSub{peak}{Age or length of plateau (max selectivity)}
\defSub{y0}{Transformed selectivity for the first age or length bin}
\defSub{y1}{Transformed selectivity for the last age or length bins}
\defSub{descending}{The shape of descending limb in either ages or lengths}
\defSub{ascending}{The shape of ascending limb in either ages or lengths}
\defSub{width}{width of plateau how many ages or lengths are in the plateau}
\defSub{l}{min age or first length bin}
\defSub{l}{max age or last length bin}
\defSub{alpha}{The maximum value of the selectivity}

\commandlabsubarg{selectivity}{type}{Increasing}

\defSub{l}{The low value (L)}
\defSub{h}{The high value (H)}
\defSub{v}{The v parameter}
\defSub{alpha}{The maximum value of the selectivity }

\commandlabsubarg{selectivity}{type}{Inverse\_Logistic}

\defSub{a50}{The age or length where the selectivity is \(50\%\)}
\defSub{ato95}{The age or length between \(50\%\) and \(95\%\) selective}
\defSub{alpha}{The maximum value of the selectivity }
\defSub{beta}{The minimum age/length for which the selectivity applies}

\commandlabsubarg{selectivity}{type}{Knife\_Edge}

\defSub{e}{The edge value}
\defSub{alpha}{The maximum value of the selectivity }

\commandlabsubarg{selectivity}{type}{Logistic}

\defSub{a50}{The age or length where the selectivity is \(50\%\)}
\defSub{ato95}{The age or length between \(50\%\) and \(95\%\) selective}
\defSub{alpha}{The maximum value of the selectivity}
\defSub{beta}{The minimum age/length for which the selectivity applies}

\commandlabsubarg{selectivity}{type}{Logistic\_Producing}

\defSub{l}{The low value (L)}
\defSub{h}{The high value (H)}
\defSub{a50}{The a50 parameter}
\defSub{ato95}{the ato95 parameter}
\defSub{alpha}{The maximum value of the selectivity}

\commandlabsubarg{selectivity}{type}{compound\_left}

\defSub{a50}{The a50 ($a50$}
\defSub{ato95}{The age or length between \(50\%\) and \(95\%\) selective}
\defSub{a\_min}{The (a\_min) parameter}
\defSub{left\_mean}{The left\_mean parameter}
\defSub{sigma}{The sigma parameter}

\commandlabsubarg{selectivity}{type}{compound\_right}

\defSub{a50}{The a50 ($a50$}
\defSub{ato95}{The age or length between \(50\%\) and \(95\%\) selective}
\defSub{a\_min}{The (a\_min) parameter}
\defSub{left\_mean}{The left\_mean parameter}
\defSub{to\_right\_mean}{The to\_right\_mean parameter}
\defSub{sigma}{The sigma parameter}

\commandlabsubarg{selectivity}{type}{compound\_middle}

\defSub{a50}{The a50 ($a50$}
\defSub{ato95}{The age or length between \(50\%\) and \(95\%\) selective}
\defSub{a\_min}{The (a\_min) parameter}
\defSub{left\_mean}{The left\_mean parameter}
\defSub{to\_right\_mean}{The to\_right\_mean parameter}
\defSub{sigma}{The sigma parameter}

\commandlabsubarg{selectivity}{type}{compound\_all}

\defSub{a50}{The a50 ($a50$}
\defSub{ato95}{The age or length between \(50\%\) and \(95\%\) selective}
\defSub{a\_min}{The (a\_min) parameter}
\defSub{sigma}{The sigma parameter}

\commandlabsubarg{Selectivity}{type}{multi\_selectivity}

\defSub{years}{The years for which we want to apply the corresponding selectivity in}
\defSub{selectivity\_labels}{The labels of the selectivities, one for each year}
\defSub{default\_selectivity}{The selectivity used in missing years}
\defSub{projection\_selectivity}{The selectivity used in missing years in projections}

\defComLab{simulate}{Define an object of type \emph{Simulate}}

\defSub{label}{The label for the simulation command}
\defSub{type}{The type of simulation}
\defSub{years}{The years to simulate values for}
\defSub{parameter}{The parameter to generate simulated values for}

\commandlabsubarg{simulate}{type}{Constant}

\defSub{value}{The value to assign to the parameter in the simulations}

\defComLab{time\_step}{Define an object of type \emph{Time\_Step}}

\defSub{label}{The label of the time step}
\defSub{processes}{The labels of the processes that occur in this time step, in the order that they occur}

\defComLab{time\_varying}{Define an object of type \emph{Time\_Varying}}

\defSub{label}{The label of the time-varying object}
\defSub{type}{The type of the time-varying object}
\defSub{parameter}{The name of the parameter to vary in each year}
\defSub{years}{The years in which to vary the parameter}

\commandlabsubarg{time\_varying}{type}{Annual\_Shift}

\defSub{a}{Parameter A}
\defSub{b}{Parameter B}
\defSub{c}{Parameter C}
\defSub{scaling\_years}{The scaling years}
\defSub{values}{The values}

\commandlabsubarg{time\_varying}{type}{Constant}

\defSub{values}{The value to assign to addressable}

\commandlabsubarg{time\_varying}{type}{Exogenous}

\defSub{a}{The shift parameter}
\defSub{exogenous\_variable}{The values of exogenous variable for each year}

\commandlabsubarg{time\_varying}{type}{Linear}

\defSub{slope}{The slope of the linear trend (i.e., the additive amount per year)}
\defSub{intercept}{The intercept of the linear trend (, i.e. the value in the first year)}

\commandlabsubarg{time\_varying}{type}{Random\_Draw}

\defSub{mean}{The mean ($\mu$) of the random draw distribution}
\defSub{sigma}{The standard deviation ($\sigma$)  of the random draw distribution}
\defSub{lower\_bound}{The lower bound for the random draw}
\defSub{upper\_bound}{The upper bound for the random draw}
\defSub{distribution}{The distribution type}

\commandlabsubarg{time\_varying}{type}{Random\_Walk}

\defSub{mean}{The mean ($\mu$) of the random walk distribution}
\defSub{sigma}{The standard deviation ($\sigma$)  of the random walk distribution}
\defSub{lower\_bound}{The lower bound for the random walk}
\defSub{upper\_bound}{The upper bound for the random walk}
\defSub{rho}{The autocorrelation parameter ($\rho$)  of the random walk distribution}
\defSub{distribution}{The distribution type}

\normalsize
