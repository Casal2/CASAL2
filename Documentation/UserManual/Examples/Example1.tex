This example implements a very simple single species and area model, with recruitment, maturation, natural and fishing mortality, and an annual age increment. The population structure has ages $1-30^{+}$ with a single category.

\CNAME\ default file to search for in your current working directory is \texttt{casal2.csl2}. In this example, \texttt{casal2.csl2} specifies all the files necessary to run your \CNAME\ model from your current working directory. This is done using the !\texttt{\emph{include}} command as follows.
\lstinputlisting[firstline=2,lastline=5]{../../Examples/Simple/casal2.csl2} 

Breaking up a \CNAME\ model into sections is recommended, as it aids in readability and error checking. \texttt{population.csl2} contains the population information. The model runs from 1975-2012 and is initialised over a 120 year period prior to  1975, which applies the following processes,

\begin{enumerate}
\item A Beverton-Holt recruitment process, recruiting a constant number of individuals to the first age class (i.e., $age=1$).
\item A constant mortality process representing natural mortality$(M)$. This process is repeated in all three time steps, so that each with its own time step proportion of $M$ applied.

\item An ageing process, where all individuals are aged by one year, and with a plus group accumulator age class at $age=30$.
\end{enumerate}

Following initialisation, the model runs from the years 1975 to 2012 iterating through two time-steps. The first time-step applies processes of recruitment, and  $\frac{1}{2} M_1 + F + \frac{1}{2} M_1$ processes, where $M_1$ is the proportion of $M$ applied in the first time step. The exploitation process (fishing) is applied in the years 1975--2012. Catches are defined in the catches table and attribute information on each fishery such as selectivity and time-step they are implemented are in the fisheries table in the \command{process} block.

The second time-step applies an age increment and the remaining natural mortality.

The first 28 lines of the main section of the \texttt{population.csl2} are,
% Include config file
\pagebreak
\lstinputlisting[firstline=1,lastline=28]{../../Examples/Simple/population.csl2} 

To carry out a run of the model (to verify that the model runs without any syntax errors), use the command \texttt{casal2 -r}. Note that as \CNAME\ looks for a file named \texttt{casal2.txt} by default, we can override this. Hypothetically speaking if our model was all written in \texttt{Mymodel.txt} we could call it using the \texttt{-c} command like \texttt{casal2 -r -c Mymodel.txt}.

To run an estimation, and hence estimate the parameters defined in the file \texttt{estimation.csl2} (the catchability constant $q$, recruitment $R_0$, and the selectivity parameters $a_{50}$ and $a_{to95}$), use  \texttt{casal2 -e}. Here, we have piped the output to \texttt{estimate.log} using the command \texttt{casal2 -e > estimate.log}, reports the user defined reports \texttt{reports.csl2} from the final iteration of the estimation, and successful convergence printed to screen,
{\small{\begin{verbatim}
Total elapsed time: 1 second
Completed
\end{verbatim}}}

The main part of the output from the estimation run is summarised in the file \texttt{estimate.log}, and the final MPD parameter values can be piped out as a separate report, in this case named \texttt{paramaters.out}, using the command \texttt{casal2 -e -o paramaters.out > estimate.log}.

A profile on the $R_0$ parameter can also be run, using \texttt{casal2 -p > profile.log}. See the examples folder for an example of the output.

