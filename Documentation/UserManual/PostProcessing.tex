\section{\I{Post-processing output using \R} \label{sec:post-processing}}\index{Post processing}\index{Post-processing section}

\R\ (\url{https://www.r-project.org/}) is the main application used to process and visualise output from a \CNAME\ model. \R\ is free and can be downloaded from \url{https://cran.r-project.org/}. Once you have installed \R\ you can install the \cname\ \R\ package from the file (\texttt{casal2\_1.0.tar.gz}) which is part of the \CNAME\ download.

The \CNAME\ \R\ package has functionality to parse \CNAME\ output into a list. It also has diagnostic, plotting, and summarising functions.

There are three types of output that \CNAME\ can produce, depending on the type of analysis run. These outputs are: Standard, MCMC, and Derived Quantity.

The Standard outputs are the reports that are produced in most \CNAME\ run modes, with the exception of \texttt{-s} and \texttt{-m}. The Standard output can be split into two additional categories, a single parameter run (\texttt{casal2 -r}) or a multi-parameter run (\texttt{casal2 -r -i many\_pars.out}), or running in projection mode (\texttt{-f 1}). The Standard outputs can be read into \R\ using the \texttt{extract.mpd()} function.

The second type of output is generated when doing an MCMC analysis (\texttt{casal2 -m}), which can generate two files, \texttt{mcmc\_objective.out} and \texttt{mcmc\_samples.out}. The MCMC outputs can be used to summarise convergence properties or chain behaviour, and can also be used to view marginal posteriors and quantify parameter uncertainty.

The third output type is the Derived Quantity outputs, also referred to as tabular output. The Derived Quantity output can be generated after an MCMC analysis is done, to produce the marginal posteriors for derived quantities. A commonly reported derived quantity in fisheries stock assessment modelling is the time series of spawning stock biomass. To get the posterior distributions for these derived quantities use the \texttt{-{}-tabular} flag (e.g., \texttt{casal2 -r -i mcmc\_samples.out -{}-tabular > Tabular\_report.out}). This output can then be read into \R\ using the \texttt{extract.tabular()} function.

\CNAME's reported output is written so that each \command{report} will start with a '*' and end with `*end'. This format can be used as the basis to construct functions that read \CNAME\ output to identify and read individual reports for post-processing.

The \CNAME\ \R\ \texttt{extract()} functions differ by how the expected output is structured and they each create a different \cname\ object. The \texttt{summary()} and \texttt{plot()} functions will generate different plots for the different \cname\ objects. Objects produced by the \texttt{extract()} function can be queried with \texttt{class(object)}.

The list of \cname\ \R\ functions include:

\begin{itemize}
	\item \texttt{extract.mpd()}, which parses the \CNAME\ default output into a list
	\item \texttt{extract.mcmc()}, which parses the \CNAME\ MCMC output into a list
	\item \texttt{extract.tabular()}, which parses the \CNAME\ tabular output into a list
	\item \texttt{extract.parameters()}, which parses the \CNAME\ parameter files into a list
	\item \texttt{generate.starting.pars()}, which reads in a file that contains the \command{estimate} blocks and generates 'N' starting values to test convergence (???)
	\item \texttt{burn.in.tabular()}, which omits the first 'N' rows from a \subcommand{casal2TAB} object
	\item \texttt{plot.derived\_quantities()}, which plots the derived quantities
	\item \texttt{plot.selectivities()}, which plots the selectivities
	\item \texttt{plot.ycs()}, which plots the true YCS strengths
	\item \texttt{plot.pressure()}, which plots the fishing pressures
	\item \texttt{summary()}, which summarises a model run
	\item \texttt{extract.csl2.file()}, which reads a \CNAME\ .csl2 (configuration) file into a list
	\item \texttt{write.csl2.file()}, which writes a \CNAME\ .csl2 (configuration) file from (???)
	\item \texttt{ReadSimulatedData()}, which parses \CNAME\ output from a \texttt{casal2 -s} run
	\item \texttt{Method.TA1.8()}, which returns a weighting factor for age or length composition data. See \cite{francis2011data} for more detail.
	\item \texttt{apply.dataweighting.to.csl2()}, which parses a \CNAME\ .csl2 (configuration) file that contains \command{observation} blocks, applies a weighting factor to an age or length composition data set, and generates a new \subcommand{.csl2} file with modified effective sample size values
\end{itemize}

The required and optional arguments for these functions can be queried after loading the \CNAME\ \R\ library with \texttt{library(casal2)} and using the standard \R\ help syntax \texttt{?} (e.g., \texttt{?param.profile()}). Many of the help files have example code and data to demonstrate function syntax.

\paragraph*{Standard diagnostic functions and plots for model output}

TODO (functionality description)

\texttt{plot.derived\_quantity()}

When comparing model output either: different parameters for the same model structure are being compared (Situation 1), or outputs from multiple model structures are being compared (Situation 2). These functions can be useful for both comparison types.

\begin{itemize}
	\item \texttt{plot.selectivities()}
	\item \texttt{plot.pressure()}
	\item \texttt{plot.fit()}
	\item \texttt{plot.ycs()}
\end{itemize}

\paragraph*{Data weighting}

An important component of fisheries stock assessment modelling is addressing data conflicts through the use of data weighting. There are a range of methods that can be used (\cite{francis2011data}). The \CNAME\ \R\ function is \texttt{Method.TA1.8()}. An additional function \subcommand{apply.dataweighting.to.csl2()} automatically applies a weighting factor to a specific age or length composition data in an \command{observation} block, and generates a new \subcommand{.csl2} file with modified effective sample size values.

\begin{lstlisting}
library(casal2)

## read in the reported output from a "casal2 -e" run
## ensure there is a @report block for the observation of interest.
mpd <- extract.mpd(file = "estimate.log")

## calculate weighting factor from Francis method
WeightingFactor <- Method.TA1.8(model = mpd, observation_labels = "chatTANage")

## Apply the weighting factor to the block in the Observation.csl2 file
## this call generates a new file (Observation.csl2.0) with the re-weighted effective sample sizes
apply.dataweighting.to.csl2(weighting_factor = WeightingFactor,
                                Observation_csl2_file = "Observations.csl2",
                                Observation_label = "chatTANage",
                                Observation_out_filename = "Observation.csl2.0")
\end{lstlisting}


Automating the data weighting process:

\begin{lstlisting}
library(casal2)

mpd <- extract.mpd(file = "estimate.log")

ModelFactor <- Method.TA1.8(mpd, observation_labels = c("ObserverProportionsAtAge"))

## make a back-up copy of the file Observation.csl2 before running this section

while(abs(ModelFactor - 1) > 0.01) {
	shell("betadiff & casal2 -e > estimate.log 2> log.out")
	
	new_mpd <- extract.mpd(file = "estimate.log")

	ModelFactor <- Method.TA1.8(new_mpd, observation_labels = c("ObserverProportionsAtAge"))
	
	apply.dataweighting.to.csl2(weighting_factor = ModelFactor,
                                  Observation_csl2_file = "Observation.csl2",
                                  Observation_out_filename = "Observation.csl2",
                                  Observation_label = c("ObserverProportionsAtAge"))
	print(ModelFactor)
}
\end{lstlisting}

\paragraph*{Troubleshooting the \cname\ \R\ package}

If you get this error when using one of the \texttt{extract()} functions

\begin{lstlisting}
Read 1 item
Warning messages:
1: In scan(filename, what = "", sep = "\n", fileEncoding = fileEncoding) :
embedded nul(s) found in input
2: In extract.mpd(file = "results.txt", fileEncoding = "") :
File is empty, no reports found
\end{lstlisting}

You may be able to resolve this issue by using an alternative UTF format by specifying this format with the \subcommand{fileEncoding} parameter

\begin{lstlisting}
MyOutput <- extract.mpd(file = "Estimate.log", path = getwd(), fileEncoding = "UTF-16LE")
\end{lstlisting}
