\section{Model overview\label{sec:overview}}\index{Model overview}

\CNAME~is a generalized age-structured population dynamics modelling framework of the integrated assessment type (Maunder and Punt, 2013). It allows several sources of information to be combined into a single analysis using a statistical framework so that error sources are fully  propagated into the uncertainty in the results. The model follows cohorts as numbers-at-age through time recording changes from processes acting on them such as nature mortality and fishing.  \index{Model ! About \CNAME}\index{About \CNAME} 

\CNAME~is run from the console window on Microsoft Windows or from a terminal window on Linux. \CNAME~has two sources of information: the \emph{\config} which  defines the model structure, provides observations, defines active parameters, and specifies outputs (reports); and run time actions (e.g.,  estimation of active parameters or projections) which are given by command line options and arguments (see Section~\ref{sec:running} for details).

The model is defined in terms of the \emph{state}\index{Model ! state}\index{State} at various points in the annual cycle within each year. The the annual cycle defines the time steps within a year and the order of processes. The state consists of two parts, the \emph{partition}\index{Model ! partition}\index{Partition}, and any defined \emph{derived quantities}\index{Model ! derived quantities}\index{Derived quantities}. The partition will typically change in each \emph{time-step}\index{Model ! time-steps}\index{time-steps} of every year, depending on the \emph{processes}\index{Model ! processes}\index{Processes} defined for those time-steps in the model. 

The \emph{partition} is a representation of the population at each time step, and can be considered a matrix of the numbers of individuals within each category (i.e., row) and at each age. Categories define the important population structure, such as whether a fish is male or female, in a specific area, and its maturity state, but other divisions are also possible like stock, or even species. How many categories there are and what they represent are entirely defined by the user as \CNAME~does not have any hard-wired fisheries concepts. The latter is one reason for re-writing CASAL (\CNAME's precursor), but the increased flexibility does come at a cost of more complex input files complexity.

A \emph{derived quantity} is a calculation of a selected part of the partition at some point in time, an example is spawning stock biomass (SSB). Unlike the partition, which is updated as each new time step, a derived quantity records a single value for each year of the model run. Hence, derived quantities build up a vector of values over the time period represented by the model. The derived quantity is then available to be reported or as an input into another process at some point in a subsequent year. The most common use is using SSB from a previous year in the stock-recruitment relationship to determine numbers into the first age of the partition. Another example might be having age -at-maturity inversely related to the largest size of recruitment in the previous four years.

Observations, their sampling distributions, relationship with the partition, and time sampled (time step and year) are specified in the \config. Examples are abundance indices from a research trawl survey or age compositions from the commercial catch. The relevant state can now be linked to the observation which then can generate expected values for the observations. In broad terms, the model parameters can be optimised to provide the best fit between expected values and the observations. Best fit is judged by the lowest negative log-likelihood, for which a special case would be the sums-of-squares. The manner in which observations are evaluated and how the expected values are calculated by the model is described in Section \ref{sec:observation-section}.

The model scope and population structure are  specified in the \config~ and this is similar to that in CASAL\index{CASAL} \citep{1388}, but with a diffident syntax. The model records the numbers of individuals by category and age (e.g., for a sex category, the numbers of males and females at age). In general, cohorts are added via a recruitment event, are aged annually, and are removed from the population via various forms of mortality (e.g., fishing or natural mortality). The population is assumed to be closed (i.e., no immigration or emigration from the modelled area). REPEAT?

By convention, the \config~ is a file named config.csl2 which reads in four other files so that assessment specifications are more digestible (one comprehensive file can be used instead and users do not have to use the name config.csl2). These subordinate files cover (again by convention): the model structure, population dynamics, etc. (Section \ref{sec:population-section}); the methods of estimation, priors, penalties, and the model parameters to be estimated Section \ref{sec:estimation-section}); the observation data, their associated likelihoods, and their timing (Section \ref{sec:observation-section}); and the outputs from the model (default is no outputs) (Section \ref{sec:report-section}). The \config~completely describes a model implemented in \CNAME. See Sections \ref{sec:population-syntax}, \ref{sec:estimation-syntax}, \ref{sec:observation-syntax}, and \ref{sec:report-syntax} for details and specification of \CNAME's command and subcommand syntax within the \config. 

