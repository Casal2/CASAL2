\defComLab{minimiser}{Define an object of type \emph{Minimiser}}.
\defRef{sec:Minimiser}
\label{syntax:Minimiser}

\defSub{label}{The minimiser label}
\defType{String}
\defDefault{No default}

\defSub{type}{The type of minimiser to use}
\defType{String}
\defDefault{No default}

\defSub{active}{Indicates if this minimiser is active}
\defType{Boolean}
\defDefault{false}

\defSub{covariance}{Indicates if a covariance matrix should be generated}
\defType{Boolean}
\defDefault{true}

\subsubsection{Minimiser of type ADOLC}
\commandlabsubarg{minimiser}{type}{ADOLC}.
\defRef{sec:Minimiser-ADOLC}
\label{syntax:Minimiser-ADOLC}

\defSub{iterations}{The maximum number of iterations}
\defType{Integer}
\defDefault{1000}
\defLowerBound{1 (inclusive)}

\defSub{evaluations}{The maximum number of evaluations}
\defType{Integer}
\defDefault{4000}
\defLowerBound{1 (inclusive)}

\defSub{tolerance}{The tolerance of the gradient for convergence}
\defType{Real number}
\defDefault{1e-5}
\defLowerBound{0.0 (exclusive)}

\defSub{step\_size}{The minimum step size before minimisation fails}
\defType{Real number}
\defDefault{1e-7}
\defLowerBound{0.0 (exclusive)}

\defSub{parameter\_transformation}{The choice of parametrisation used to scale the parameters for ADOLC}
\defType{string}
\defDefault{sin\_transformation}
\defValue{Either \subcommand{sin\_transformation} or \subcommand{tan\_transformation}. See \ref{sec:Minimiser-ADOLC} for more information}

\subsubsection{Minimiser of type Betadiff}
\commandlabsubarg{minimiser}{type}{Betadiff}.
\defRef{sec:Minimiser-BetaDiff}
\label{syntax:Minimiser-BetaDiff}

\defSub{iterations}{The maximum number of iterations}
\defType{Integer}
\defDefault{1000}
\defLowerBound{1 (inclusive)}

\defSub{evaluations}{The maximum number of evaluations}
\defType{Integer}
\defDefault{4000}
\defLowerBound{1 (inclusive)}

\defSub{tolerance}{The tolerance of the gradient for convergence}
\defType{Real number}
\defDefault{1e-5}
\defLowerBound{0.0 (exclusive)}

\subsubsection{Minimiser of type DESolver}
\commandlabsubarg{minimiser}{type}{de\_solver}.
\defRef{sec:Minimiser-DESolver}
\label{syntax:Minimiser-DESolver}

\defSub{population\_size}{The number of candidate solutions to have in the population}
\defType{Non-negative integer}
\defDefault{25}
\defLowerBound{1 (inclusive)}

\defSub{crossover\_probability}{The minimiser's crossover probability}
\defType{Real number}
\defDefault{0.9}
\defLowerBound{0.0 (inclusive)}
\defUpperBound{1.0 (inclusive)}

\defSub{difference\_scale}{The scale to apply to new solutions when comparing candidates}
\defType{Real number}
\defDefault{0.02}

\defSub{max\_generations}{The maximum number of iterations to run}
\defType{1000}
\defDefault{No default}

\defSub{tolerance}{The total variance between the population and best candidate before acceptance}
\defType{Real number}
\defDefault{1e-5}
\defLowerBound{0.0 (exclusive)}

\subsubsection{Minimiser of type Deltadiff}
\commandlabsubarg{minimiser}{type}{Deltadiff}.
\defRef{sec:Minimiser-DeltaDiff}
\label{syntax:Minimiser-DeltaDiff}

\defSub{iterations}{Maximum number of iterations}
\defType{Integer}
\defDefault{1000}
\defLowerBound{1 (inclusive)}

\defSub{evaluations}{Maximum number of evaluations}
\defType{Integer}
\defDefault{4000}
\defLowerBound{1 (inclusive)}

\defSub{tolerance}{Tolerance of the gradient for convergence}
\defType{Real number}
\defDefault{1e-5}
\defLowerBound{0 (exclusive)}

\defSub{step\_size}{Minimum Step-size before minimisation fails}
\defType{Real number}
\defDefault{1e-7}
\defLowerBound{0 (exclusive)}

\subsubsection{Minimiser of type Numerical Differences}
\commandlabsubarg{minimiser}{type}{Numerical\_Differences}.
\defRef{sec:Minimiser-GammaDiff}
\label{syntax:Minimiser-GammaDiff}

\defSub{iterations}{The maximum number of iterations}
\defType{Integer}
\defDefault{1000}
\defLowerBound{1 (inclusive)}

\defSub{evaluations}{The maximum number of evaluations}
\defType{Integer}
\defDefault{4000}
\defLowerBound{1 (inclusive)}

\defSub{tolerance}{The tolerance of the gradient for convergence}
\defType{Real number}
\defDefault{1e-5}
\defLowerBound{0.0 (exclusive)}

\defSub{step\_size}{The minimum step size before minimisation fails}
\defType{Real number}
\defDefault{1e-7}
\defLowerBound{0.0 (exclusive)}

