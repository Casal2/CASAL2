\defComLab{MCMC}{Define an object of type \emph{MCMC}}.
\defRef{sec:MCMC}
\label{syntax:MCMC}

\defSub{label}{The label of the MCMC}
\defType{String}
\defDefault{No default}

\defSub{type}{The MCMC method}
\defType{String}
\defDefault{No default}

\defSub{length}{The number of iterations for the MCMC (including the burn in period)}
\defType{Non-negative integer}
\defDefault{No default}
\defLowerBound{1 (inclusive)}

\defSub{burn\_in}{The number of iterations for the burn\_in period of the MCMC}
\defType{Non-negative integer}
\defDefault{0}
\defLowerBound{0 (inclusive)}

\defSub{active}{Indicates if this is the active MCMC algorithm}
\defType{Boolean}
\defDefault{true}

\defSub{step\_size}{Initial step-size (as a multiplier of the approximate covariance matrix)}
\defType{Real number}
\defDefault{The default is $2.4d^{-0.5}$}
\defLowerBound{0 (inclusive)}

\defSub{start}{The covariance multiplier for the starting point of the MCMC}
\defType{Real number}
\defDefault{0.0}
\defLowerBound{0.0 (inclusive)}
\defValue{If zero, then the MCMC starts at the point estimate (i.e., the MPD). Otherwise a random (multivariate normal) jump from the point estimate with \subcommand{start} used as the standard deviation multiplier}

\defSub{keep}{The spacing between recorded values in the MCMC}
\defType{Non-negative integer}
\defDefault{1}
\defLowerBound{1 (inclusive)}

\defSub{max\_correlation}{The maximum absolute correlation in the covariance matrix of the proposal distribution}
\defType{Real number}
\defDefault{0.8}
\defLowerBound{0.0 (exclusive)}
\defUpperBound{1.0 (inclusive)}

\defSub{covariance\_adjustment\_method}{The method for adjusting small variances in the covariance proposal matrix}
\defType{String}
\defDefault{correlation}
\defValue{Either covariance, correlation, or none}

\defSub{correlation\_adjustment\_diff}{The minimum non-zero variance times the range of the bounds in the covariance matrix of the proposal distribution}
\defType{Real number}
\defDefault{0.0001}
\defLowerBound{0.0 (exclusive)}

\defSub{proposal\_distribution}{The shape of the proposal distribution (either the t or the normal distribution)}
\defType{String}
\defDefault{t}
\defValue{Either t or normal}

\defSub{df}{The degrees of freedom of the multivariate t proposal distribution}
\defType{Non-negative integer}
\defDefault{4}
\defLowerBound{1}

\defSub{adapt\_stepsize\_at}{The iteration numbers in which to check and resize the MCMC stepsize}
\defType{Vector of non-negative integers}
\defDefault{true}
\defLowerBound{0 (inclusive)}
\defValue{If zero, then no step size adaption is applied. Otherwise must be a positive integer less than less than the burn\_in}

\defSub{adapt\_stepsize\_method}{The method to use to adapt the step size}
\defType{String}
\defDefault{ratio}

\defSub{adapt\_covariance\_matrix\_at}{The iteration number in which to adapt the covariance matrix}
\defType{Non-negative integer}
\defDefault{0}
\defLowerBound{0 (inclusive)}
\defValue{If zero, then no covariance matrix adaption is applied. Otherwise must be a positive integer that is less than the burn\_in}

\subsubsection{MCMC of type Hamiltonian\_Monte\_Carlo}
\commandlabsubarg{MCMC}{type}{Hamiltonian\_Monte\_Carlo}.
\label{syntax:MCMC-HamiltonianMonteCarlo}

\defSub{leapfrog\_steps}{Number of leapfrog steps}
\defType{Non-negative integer}
\defDefault{1}
\defLowerBound{0 (exclusive)}

\defSub{leapfrog\_delta}{Amount to leapfrog per step}
\defType{Real number}
\defDefault{1e-7}
\defLowerBound{0 (exclusive)}

\defSub{gradient\_step\_size}{Step size to use when calculating gradient}
\defType{Real number}
\defDefault{1e-7}
\defLowerBound{1e-13 (inclusive)}

\subsubsection{MCMC of type Random\_Walk\_Metropolis\_Hastings}
\commandlabsubarg{MCMC}{type}{Random\_Walk}.
\defRef{sec:MCMC-RandomWalkMetropolisHastings}
\label{syntax:MCMC-RandomWalkMetropolisHastings}

The Random\_Walk type has no additional subcommands.
