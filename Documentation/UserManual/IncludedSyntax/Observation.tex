\defComLab{observation}{Define an object of type \emph{Observation}}.
\defRef{sec:Observation}
\label{syntax:Observation}

\defSub{label}{The label of the observation}
\defType{String}
\defDefault{No default}

\defSub{type}{The type of observation}
\defType{String}
\defDefault{No default}

\defSub{likelihood}{The type of likelihood to use}
\defType{String}
\defDefault{No default}

\defSub{categories}{The category labels to use}
\defType{Vector of strings}
\defDefault{true}

\defSub{delta}{The robustification value (delta) for the likelihood}
\defType{Real number (estimable)}
\defDefault{1e-11}
\defLowerBound{0.0 (inclusive)}

\defSub{simulation\_likelihood}{The simulation likelihood to use}
\defType{String}
\defDefault{No default}

\defSub{likelihood\_multiplier}{The likelihood multiplier}
\defType{Real number (estimable)}
\defDefault{1.0}
\defLowerBound{0.0 (inclusive)}

\defSub{error\_value\_multiplier}{The error value multiplier for likelihood}
\defType{Real number (estimable)}
\defDefault{1.0}
\defLowerBound{0.0 (inclusive)}

\defSub{table}{The table of data specifying the observed values}
\defType{Data table with label = obs}
\defDefault{No default}
\defValue{A $n*m$ matrix, where $n=$ the years and $m=$ either the number of ages, lengths, or abundance/biomass observation for each year defined in the model. Each row starts with the year. The table ends with `end\_table'}
\defNote{See \ref{sec:DataTable} for more details each observation may have custom table labels.}

\defSub{table}{The table of data specifying the observed error values}
\defType{Data table with label = error\_values}
\defDefault{No default}
\defValue{A $n*m$ matrix, where $n=$ the years and $m=$ either the number of ages, lengths, or abundance/biomass observation for each year defined in the model. Each row starts with the year. The table ends with `end\_table'}
\defNote{See \ref{sec:DataTable} for more details on specifying data tables.  each observation may have custom table labels.}

\subsubsection{Observation of type Abundance}
\commandlabsubarg{observation}{type}{Abundance}.
\defRef{sec:Observation-Abundance}
\label{syntax:Observation-Abundance}

\defSub{time\_step}{The label of the time step that the observation occurs in}
\defType{String}
\defDefault{No default}

\defSub{catchability}{The label of the catchability coefficient (q)}
\defType{String}
\defDefault{No default}

\defSub{selectivities}{The labels of the selectivities}
\defType{Vector of strings}
\defDefault{true}

\defSub{process\_error}{The process error}
\defType{Real number (estimable)}
\defDefault{0.0}
\defLowerBound{0.0 (inclusive)}

\defSub{years}{The years for which there are observations}
\defType{Vector of non-negative integers}
\defDefault{No default}

\defSub{table obs}{The table of data specifying the observed and error values}
\defType{Data table with label = obs}
\defDefault{No default}
\defValue{A $n*3$ matrix, where $n=$ the years and a column for year, observation and error value. See below for example.}
\defNote{example below}
\begin{verbatim}
table obs 
# year observation error_value
1993 238.2 0.12
1994 170 0.16
1995 216.2 0.18
2004 46.9 0.20
end_table
\end{verbatim}

\subsubsection{Observation of type Biomass}
\commandlabsubarg{observation}{type}{Biomass}.
\defRef{sec:Observation-Biomass}
\label{syntax:Observation-Biomass}

\defSub{time\_step}{The label of the time step that the observation occurs in}
\defType{String}
\defDefault{No default}

\defSub{catchability}{The label of the catchability coefficient (q)}
\defType{String}
\defDefault{No default}

\defSub{selectivities}{The labels of the selectivities}
\defType{Vector of strings}
\defDefault{true}

\defSub{process\_error}{The process error}
\defType{Real number (estimable)}
\defDefault{0.0}
\defLowerBound{0.0 (inclusive)}

\defSub{age\_weight\_labels}{The labels for the \command{$age\_weight$} block which corresponds to each category, to use the weight calculation method for biomass calculations)}
\defType{Vector of strings}
\defDefault{No default}

\defSub{years}{The years of the observed values}
\defType{Vector of non-negative integers}
\defDefault{No default}

\defSub{table obs}{The table of data specifying the observed and error values}
\defType{Data table with label = obs}
\defDefault{No default}
\defValue{A $n*3$ matrix, where $n=$ the years and a column for year, observation and error value. See below for example.}
\defNote{example below}
\begin{verbatim}
table obs 
# year observation error_value
1993 238.2 0.12
1994 170 0.16
1995 216.2 0.18
2004 46.9 0.20
end_table
\end{verbatim}

\subsubsection{Observation of type Process Removals By Age}
\commandlabsubarg{observation}{type}{Process\_Removals\_By\_Age}.
\defRef{sec:Observation-ProcessRemovalsByAge}
\label{syntax:Observation-ProcessRemovalsByAge}

\defSub{min\_age}{The minimum age}
\defType{Non-negative integer}
\defDefault{No default}

\defSub{max\_age}{The maximum age}
\defType{Non-negative integer}
\defDefault{No default}

\defSub{sum\_to\_one}{Scale year (row) observed values by the total, so they sum = 1}
\defType{Boolean}
\defDefault{false}

\defSub{simulated\_data\_sum\_to\_one}{Whether simulated data is discrete or scaled by totals to be proportions for each year}
\defType{Boolean}
\defDefault{true}


\defSub{plus\_group}{Is the maximum age the age plus group}
\defType{Boolean}
\defDefault{true}

\defSub{time\_step}{The label of time-step that the observation occurs in}
\defType{Vector of strings}
\defDefault{No default}

\defSub{years}{The years for which there are observations}
\defType{Vector of non-negative integers}
\defDefault{No default}

\defSub{process\_errors}{The label of process error to use}
\defType{Vector of real numbers (estimable)}
\defDefault{true}

\defSub{ageing\_error}{The label of the ageing error to use}
\defType{String}
\defDefault{No default}

\defSub{method\_of\_removal}{The label of the observed method of removals}
\defType{Vector of strings}
\defDefault{No default}

\defSub{mortality\_instantaneous\_process}{The label of the mortality instantaneous process for the observation}
\defType{String}
\defDefault{No default}

\defSub{table obs}{The table of data specifying the observed values}
\defType{Data table with label = obs}
\defDefault{No default}
\defValue{A $n\times m$ matrix, where $n=$ the years and $m$ is categories \(\times\) length bins. See below for example.}
\defNote{example below}
\begin{verbatim}
table obs 
1993 0.1 0.2 0.3
1994 0.1 0.2 0.3
end_table
\end{verbatim}
\defSub{table error\_values}{The table of data specifying the error values}
\defType{Data table with label = error\_values}
\defDefault{No default}
\defValue{Can be specified two ways either as a $n\times 1$ matrix with an error value for each year. Or a $n\times m$ matrix, where $n=$ the years and $m$ is categories \(\times\) length bins. See below for example.}
\defNote{example below}
\begin{verbatim}
table error_values 
1993 234
1994 343
end_table
\end{verbatim}
\subsubsection{Observation of type Process Removals By Age Retained}
\commandlabsubarg{observation}{type}{Process\_Removals\_By\_Age\_Retained}.
\defRef{sec:Observation-ProcessRemovalsByAgeRetained}
\label{syntax:Observation-ProcessRemovalsByAgeRetained}

\defSub{min\_age}{The minimum age}
\defType{Non-negative integer}
\defDefault{No default}

\defSub{max\_age}{The maximum age}
\defType{Non-negative integer}
\defDefault{No default}

\defSub{plus\_group}{Is the maximum age the age plus group?}
\defType{Boolean}
\defDefault{true}

\defSub{time\_step}{The label of the time step that the observation occurs in}
\defType{Vector of strings}
\defDefault{No default}


\defSub{sum\_to\_one}{Scale year (row) observed values by the total, so they sum = 1}
\defType{Boolean}
\defDefault{false}

\defSub{simulated\_data\_sum\_to\_one}{Whether simulated data is discrete or scaled by totals to be proportions for each year}
\defType{Boolean}
\defDefault{true}

\defSub{years}{The years for which there are observations}
\defType{Vector of non-negative integers}
\defDefault{No default}

\defSub{process\_errors}{The label of the process error to use}
\defType{Vector of real numbers (estimable)}
\defDefault{true}

\defSub{ageing\_error}{The label of the ageing error to use}
\defType{String}
\defDefault{No default}

\defSub{method\_of\_removal}{The label of observed method of removals}
\defType{Vector of strings}
\defDefault{No default}

\defSub{mortality\_instantaneous\_process}{The label of the mortality instantaneous process for the observation}
\defType{String}
\defDefault{No default}

\defSub{table obs}{The table of data specifying the observed values}
\defType{Data table with label = obs}
\defDefault{No default}
\defValue{A $n\times m$ matrix, where $n=$ the years and $m$ is categories \(\times\) length bins. See below for example.}
\defNote{example below}
\begin{verbatim}
table obs 
1993 0.1 0.2 0.3
1994 0.1 0.2 0.3
end_table
\end{verbatim}
\defSub{table error\_values}{The table of data specifying the error values}
\defType{Data table with label = error\_values}
\defDefault{No default}
\defValue{Can be specified two ways either as a $n\times 1$ matrix with an error value for each year. Or a $n\times m$ matrix, where $n=$ the years and $m$ is categories \(\times\) length bins. See below for example.}
\defNote{example below}
\begin{verbatim}
table error_values 
1993 234
1994 343
end_table
\end{verbatim}
\subsubsection{Observation of type Process Removals By Age Retained Total}
\commandlabsubarg{observation}{type}{Process\_Removals\_By\_Age\_Retained\_Total}.
\defRef{sec:Observation-ProcessRemovalsByAgeRetainedTotal}
\label{syntax:Observation-ProcessRemovalsByAgeRetainedTotal}

\defSub{min\_age}{The minimum age}
\defType{Non-negative integer}
\defDefault{No default}

\defSub{max\_age}{The maximum age}
\defType{Non-negative integer}
\defDefault{No default}

\defSub{plus\_group}{Is the maximum age the age plus group?}
\defType{Boolean}
\defDefault{true}

\defSub{time\_step}{The label of the time step that the observation occurs in}
\defType{Vector of strings}
\defDefault{No default}


\defSub{sum\_to\_one}{Scale year (row) observed values by the total, so they sum = 1}
\defType{Boolean}
\defDefault{false}

\defSub{simulated\_data\_sum\_to\_one}{Whether simulated data is discrete or scaled by totals to be proportions for each year}
\defType{Boolean}
\defDefault{true}


\defSub{years}{The years for which there are observations}
\defType{Vector of non-negative integers}
\defDefault{No default}

\defSub{process\_errors}{The label of the process error to use}
\defType{Vector of real numbers (estimable)}
\defDefault{true}

\defSub{ageing\_error}{The label of the ageing error to use}
\defType{String}
\defDefault{No default}

\defSub{method\_of\_removal}{The label of observed method of removals}
\defType{Vector of strings}
\defDefault{No default}

\defSub{mortality\_instantaneous\_process}{The label of the mortality instantaneous process for the observation}
\defType{String}
\defDefault{No default}

\defSub{table obs}{The table of data specifying the observed values}
\defType{Data table with label = obs}
\defDefault{No default}
\defValue{A $n\times m$ matrix, where $n=$ the years and $m$ is categories \(\times\) length bins. See below for example.}
\defNote{example below}
\begin{verbatim}
table obs 
1993 0.1 0.2 0.3
1994 0.1 0.2 0.3
end_table
\end{verbatim}
\defSub{table error\_values}{The table of data specifying the error values}
\defType{Data table with label = error\_values}
\defDefault{No default}
\defValue{Can be specified two ways either as a $n\times 1$ matrix with an error value for each year. Or a $n\times m$ matrix, where $n=$ the years and $m$ is categories \(\times\) length bins. See below for example.}
\defNote{example below}
\begin{verbatim}
table error_values 
1993 234
1994 343
end_table
\end{verbatim}
\subsubsection{Observation of type Process Removals By Length}
\commandlabsubarg{observation}{type}{Process\_Removals\_By\_Length}.
\defRef{sec:Observation-ProcessRemovalsByLength}
\label{syntax:Observation-ProcessRemovalsByLength}

\defSub{time\_step}{The time step to execute in}
\defType{String}
\defDefault{No default}

\defSub{years}{The years for which there are observations}
\defType{Vector of non-negative integers}
\defDefault{No default}

\defSub{process\_errors}{The process error}
\defType{Vector of real numbers (estimable)}
\defDefault{true}

\defSub{method\_of\_removal}{The label of observed method of removals}
\defType{String}
\defDefault{No default}

\defSub{length\_bins}{The length bins}
\defType{Vector of real numbers (estimable)}
\defDefault{No default}


\defSub{sum\_to\_one}{Scale year (row) observed values by the total, so they sum = 1}
\defType{Boolean}
\defDefault{false}

\defSub{simulated\_data\_sum\_to\_one}{Whether simulated data is discrete or scaled by totals to be proportions for each year}
\defType{Boolean}
\defDefault{true}

\defSub{plus\_group}{Is the last length bin a plus group? (defaults to @model value)}
\defType{Boolean}
\defDefault{model}

\defSub{mortality\_instantaneous\_process}{The label of the mortality instantaneous process for the observation}
\defType{String}
\defDefault{No default}

\defSub{table obs}{The table of data specifying the observed values}
\defType{Data table with label = obs}
\defDefault{No default}
\defValue{A $n\times m$ matrix, where $n=$ the years and $m$ is categories \(\times\) length bins. See below for example.}
\defNote{example below}
\begin{verbatim}
table obs 
1993 0.1 0.2 0.3
1994 0.1 0.2 0.3
end_table
\end{verbatim}
\defSub{table error\_values}{The table of data specifying the error values}
\defType{Data table with label = error\_values}
\defDefault{No default}
\defValue{Can be specified two ways either as a $n\times 1$ matrix with an error value for each year. Or a $n\times m$ matrix, where $n=$ the years and $m$ is categories \(\times\) length bins. See below for example.}
\defNote{example below}
\begin{verbatim}
table error_values 
1993 234
1994 343
end_table
\end{verbatim}
\subsubsection{Observation of type Process Removals By Length Retained}
\commandlabsubarg{observation}{type}{Process\_Removals\_By\_Length\_Retained}.
\defRef{sec:Observation-ProcessRemovalsByLengthRetained}
\label{syntax:Observation-ProcessRemovalsByLengthRetained}

\defSub{time\_step}{The time step to execute in}
\defType{String}
\defDefault{No default}

\defSub{years}{The years for which there are observations}
\defType{Vector of non-negative integers}
\defDefault{No default}

\defSub{process\_errors}{The process error}
\defType{Vector of real numbers (estimable)}
\defDefault{true}

\defSub{method\_of\_removal}{The label of observed method of removals}
\defType{String}
\defDefault{No default}

\defSub{length\_bins}{The length bins}
\defType{Vector of real numbers (estimable)}
\defDefault{No default}


\defSub{sum\_to\_one}{Scale year (row) observed values by the total, so they sum = 1}
\defType{Boolean}
\defDefault{false}

\defSub{simulated\_data\_sum\_to\_one}{Whether simulated data is discrete or scaled by totals to be proportions for each year}
\defType{Boolean}
\defDefault{true}

\defSub{plus\_group}{Is the last length bin a plus group? (defaults to @model value)}
\defType{Boolean}
\defDefault{model}

\defSub{mortality\_instantaneous\_process}{The label of the mortality instantaneous process for the observation}
\defType{String}
\defDefault{No default}

\defSub{table obs}{The table of data specifying the observed values}
\defType{Data table with label = obs}
\defDefault{No default}
\defValue{A $n\times m$ matrix, where $n=$ the years and $m$ is categories \(\times\) length bins. See below for example.}
\defNote{example below}
\begin{verbatim}
table obs 
1993 0.1 0.2 0.3
1994 0.1 0.2 0.3
end_table
\end{verbatim}
\defSub{table error\_values}{The table of data specifying the error values}
\defType{Data table with label = error\_values}
\defDefault{No default}
\defValue{Can be specified two ways either as a $n\times 1$ matrix with an error value for each year. Or a $n\times m$ matrix, where $n=$ the years and $m$ is categories \(\times\) length bins. See below for example.}
\defNote{example below}
\begin{verbatim}
table error_values 
1993 234
1994 343
end_table
\end{verbatim}
\subsubsection{Observation of type Process Removals By Length Retained Total}
\commandlabsubarg{observation}{type}{Process\_Removals\_By\_Length\_Retained\_Total}.
\defRef{sec:Observation-ProcessRemovalsByLengthRetainedTotal}
\label{syntax:Observation-ProcessRemovalsByLengthRetainedTotal}

\defSub{time\_step}{The time step to execute in}
\defType{String}
\defDefault{No default}

\defSub{years}{The years for which there are observations}
\defType{Vector of non-negative integers}
\defDefault{No default}

\defSub{process\_errors}{The process error}
\defType{Vector of real numbers (estimable)}
\defDefault{true}

\defSub{method\_of\_removal}{The label of observed method of removals}
\defType{String}
\defDefault{No default}

\defSub{length\_bins}{The length bins}
\defType{Vector of real numbers (estimable)}
\defDefault{No default}

\defSub{plus\_group}{Is the last length bin a plus group? (defaults to @model value)}
\defType{Boolean}
\defDefault{model}


\defSub{sum\_to\_one}{Scale year (row) observed values by the total, so they sum = 1}
\defType{Boolean}
\defDefault{false}

\defSub{simulated\_data\_sum\_to\_one}{Whether simulated data is discrete or scaled by totals to be proportions for each year}
\defType{Boolean}
\defDefault{true}

\defSub{mortality\_instantaneous\_process}{The label of the mortality instantaneous process for the observation}
\defType{String}
\defDefault{No default}

\defSub{table obs}{The table of data specifying the observed values}
\defType{Data table with label = obs}
\defDefault{No default}
\defValue{A $n\times m$ matrix, where $n=$ the years and $m$ is categories \(\times\) length bins. See below for example.}
\defNote{example below}
\begin{verbatim}
table obs 
1993 0.1 0.2 0.3
1994 0.1 0.2 0.3
end_table
\end{verbatim}
\defSub{table error\_values}{The table of data specifying the error values}
\defType{Data table with label = error\_values}
\defDefault{No default}
\defValue{Can be specified two ways either as a $n\times 1$ matrix with an error value for each year. Or a $n\times m$ matrix, where $n=$ the years and $m$ is categories \(\times\) length bins. See below for example.}
\defNote{example below}
\begin{verbatim}
table error_values 
1993 234
1994 343
end_table
\end{verbatim}
\subsubsection{Observation of type Process Removals By Weight}
\commandlabsubarg{observation}{type}{Process\_Removals\_By\_Weight}.
\defRef{sec:Observation-ProcessRemovalsByWeight}
\label{syntax:Observation-ProcessRemovalsByWeight}

\defSub{mortality\_instantaneous\_process}{The label of the mortality instantaneous process for the observation}
\defType{String}
\defDefault{No default}

\defSub{method\_of\_removal}{The label of observed method of removals}
\defType{String}
\defDefault{No default}

\defSub{time\_step}{The time step to execute in}
\defType{String}
\defDefault{No default}

\defSub{years}{The years for which there are observations}
\defType{Vector of non-negative integers}
\defDefault{No default}

\defSub{process\_errors}{The process error}
\defType{Vector of real numbers (estimable)}
\defDefault{true}

\defSub{length\_weight\_cv}{The CV for the length-weight relationship}
\defType{Real number (estimable)}
\defDefault{0.10}
\defLowerBound{0.0 (exclusive)}

\defSub{length\_weight\_distribution}{The distribution of the length-weight relationship}
\defType{String}
\defDefault{normal}

\defSub{length\_bins}{The length bins}
\defType{Vector of real numbers (estimable)}
\defDefault{No default}

\defSub{length\_bins\_n}{The average number in each length bin}
\defType{Vector of real numbers (estimable)}
\defDefault{No default}

\defSub{units}{The units for the weight bins (grams, kilograms (kgs), or tonnes)}
\defType{String}
\defDefault{kgs}

\defSub{fishbox\_weight}{The target weight of each box}
\defType{Real number (estimable)}
\defDefault{20.0}
\defLowerBound{0.0 (exclusive)}

\defSub{weight\_bins}{The weight bins}
\defType{Vector of real numbers (estimable)}
\defDefault{No default}

\subsubsection{Observation of type Proportions At Age}
\commandlabsubarg{observation}{type}{Proportions\_At\_Age}.
\defRef{sec:Observation-ProportionsAtAge}
\label{syntax:Observation-ProportionsAtAge}

\defSub{min\_age}{The minimum age}
\defType{Non-negative integer}
\defDefault{No default}

\defSub{max\_age}{The maximum age}
\defType{Non-negative integer}
\defDefault{No default}

\defSub{plus\_group}{Is the maximum age the age plus group?}
\defType{Boolean}
\defDefault{true}

\defSub{time\_step}{The label of the time step that the observation occurs in}
\defType{String}
\defDefault{No default}


\defSub{years}{The years of the observed values}
\defType{Vector of non-negative integers}
\defDefault{No default}

\defSub{selectivities}{The labels of the selectivities}
\defType{Vector of strings}
\defDefault{true}

\defSub{process\_errors}{The process error}
\defType{Vector of real numbers (estimable)}
\defDefault{true}

\defSub{ageing\_error}{The label of ageing error to use}
\defType{String}
\defDefault{No default}

\defSub{sum\_to\_one}{Scale year (row) observed values by the total, so they sum = 1}
\defType{Boolean}
\defDefault{false}

\defSub{simulated\_data\_sum\_to\_one}{Whether simulated data is discrete or scaled by totals to be proportions for each year}
\defType{Boolean}
\defDefault{true}


\defSub{table obs}{The table of data specifying the observed values}
\defType{Data table with label = obs}
\defDefault{No default}
\defValue{A $n\times m$ matrix, where $n=$ the years and $m$ is categories \(\times\) length bins. See below for example.}
\defNote{example below}
\begin{verbatim}
table obs 
1993 0.1 0.2 0.3
1994 0.1 0.2 0.3
end_table
\end{verbatim}
\defSub{table error\_values}{The table of data specifying the error values}
\defType{Data table with label = error\_values}
\defDefault{No default}
\defValue{Can be specified two ways either as a $n\times 1$ matrix with an error value for each year. Or a $n\times m$ matrix, where $n=$ the years and $m$ is categories \(\times\) length bins. See below for example.}
\defNote{example below}
\begin{verbatim}
table error_values 
1993 234
1994 343
end_table
\end{verbatim}
\subsubsection{Observation of type Proportions At Length}
\commandlabsubarg{observation}{type}{Proportions\_At\_Length}.
\defRef{sec:Observation-ProportionsAtLength}
\label{syntax:Observation-ProportionsAtLength}

\defSub{time\_step}{The label of the time step that the observation occurs in}
\defType{String}
\defDefault{No default}

\defSub{years}{The years for which there are observations}
\defType{Vector of non-negative integers}
\defDefault{No default}

\defSub{selectivities}{The labels of the selectivities}
\defType{Vector of strings}
\defDefault{true}

\defSub{process\_errors}{The process error}
\defType{Vector of real numbers (estimable)}
\defDefault{true}

\defSub{length\_bins}{The length bins}
\defType{Vector of real numbers (estimable)}
\defDefault{true}

\defSub{sum\_to\_one}{Scale year (row) observed values by the total, so they sum = 1}
\defType{Boolean}
\defDefault{false}

\defSub{simulated\_data\_sum\_to\_one}{Whether simulated data is discrete or scaled by totals to be proportions for each year}
\defType{Boolean}
\defDefault{true}


\defSub{table obs}{The table of data specifying the observed values}
\defType{Data table with label = obs}
\defDefault{No default}
\defValue{A $n\times m$ matrix, where $n=$ the years and $m$ is categories \(\times\) length bins. See below for example.}
\defNote{example below}
\begin{verbatim}
table obs 
1993 0.1 0.2 0.3
1994 0.1 0.2 0.3
end_table
\end{verbatim}
\defSub{table error\_values}{The table of data specifying the error values}
\defType{Data table with label = error\_values}
\defDefault{No default}
\defValue{Can be specified two ways either as a $n\times 1$ matrix with an error value for each year. Or a $n\times m$ matrix, where $n=$ the years and $m$ is categories \(\times\) length bins. See below for example.}
\defNote{example below}
\begin{verbatim}
table error_values 
1993 234
1994 343
end_table
\end{verbatim}

\subsubsection{Observation of type Proportions By Category}
\commandlabsubarg{observation}{type}{Proportions\_By\_Category}.
\defRef{sec:Observation-ProportionsByCategory}
\label{syntax:Observation-ProportionsByCategory}

\defSub{min\_age}{The minimum age}
\defType{Non-negative integer}
\defDefault{No default}

\defSub{max\_age}{The maximum age}
\defType{Non-negative integer}
\defDefault{No default}

\defSub{time\_step}{The label of the time step that the observation occurs in}
\defType{String}
\defDefault{No default}

\defSub{plus\_group}{Use the age plus group?}
\defType{Boolean}
\defDefault{true}

\defSub{years}{The years for which there are observations}
\defType{Vector of non-negative integers}
\defDefault{No default}

\defSub{selectivities}{The labels of the selectivities}
\defType{Vector of strings}
\defDefault{true}

\defSub{categories2}{The target categories}
\defType{Vector of strings}
\defDefault{No default}

\defSub{selectivities2}{The target selectivities}
\defType{Vector of strings}
\defDefault{No default}

\subsubsection{Observation of type Proportions Mature By Age}
\commandlabsubarg{observation}{type}{Proportions\_Mature\_By\_Age}.
\defRef{sec:Observation-ProportionsMatureByAge}
\label{syntax:Observation-ProportionsMatureByAge}

\defSub{min\_age}{The minimum age}
\defType{Non-negative integer}
\defDefault{No default}

\defSub{max\_age}{The maximum age}
\defType{Non-negative integer}
\defDefault{No default}

\defSub{time\_step}{The label of time-step that the observation occurs in}
\defType{String}
\defDefault{No default}

\defSub{plus\_group}{Use the age plus group?}
\defType{Boolean}
\defDefault{true}

\defSub{years}{The years for which there are observations}
\defType{Vector of non-negative integers}
\defDefault{No default}

\defSub{ageing\_error}{The label of ageing error to use}
\defType{String}
\defDefault{No default}

\defSub{total\_categories}{All category labels that were vulnerable to sampling at the time of this observation (not including the categories already given)}
\defType{Vector of strings}
\defDefault{true}

\defSub{time\_step\_proportion}{The proportion through the mortality block of the time step when the observation is evaluated}
\defType{Real number (estimable)}
\defDefault{0.5}
\defLowerBound{0.0 (inclusive)}
\defUpperBound{1.0 (inclusive)}

\subsubsection{Observation of type Proportions Migrating}
\commandlabsubarg{observation}{type}{Proportions\_Migrating}.
\defRef{sec:Observation-ProportionsMigrating}
\label{syntax:Observation-ProportionsMigrating}

\defSub{min\_age}{The minimum age}
\defType{Non-negative integer}
\defDefault{No default}

\defSub{max\_age}{The maximum age}
\defType{Non-negative integer}
\defDefault{No default}

\defSub{time\_step}{The label of the time step that the observation occurs in}
\defType{String}
\defDefault{No default}

\defSub{plus\_group}{Is the maximum age the age plus group?}
\defType{Boolean}
\defDefault{true}

\defSub{years}{The years for which there are observations}
\defType{Vector of non-negative integers}
\defDefault{No default}

\defSub{process\_errors}{The process error}
\defType{Vector of real numbers (estimable)}
\defDefault{true}

\defSub{ageing\_error}{The label of the ageing error to use}
\defType{String}
\defDefault{No default}

\defSub{process}{The process label}
\defType{String}
\defDefault{No default}

\subsubsection{Observation of type Tag Recapture By Age}
\commandlabsubarg{observation}{type}{Tag\_Recapture\_By\_Age}.
\defRef{sec:Observation-TagRecaptureByAge}
\label{syntax:Observation-TagRecaptureByAge}

\defSub{min\_age}{The minimum age}
\defType{Non-negative integer}
\defDefault{No default}

\defSub{max\_age}{The maximum age}
\defType{Non-negative integer}
\defDefault{No default}

\defSub{plus\_group}{Is the maximum age the age plus group?}
\defType{Boolean}
\defDefault{true}

\defSub{years}{The years for which there are observations}
\defType{Vector of non-negative integers}
\defDefault{No default}

\defSub{categories2}{The available categories in the partition}
\defType{Vector of strings}
\defDefault{No default}

\defSub{selectivities}{The labels of the selectivities}
\defType{Vector of strings}
\defDefault{true}

\defSub{time\_step}{The label of the time step that the observation occurs in}
\defType{String}
\defDefault{No default}

\defSub{selectivities2}{The categories of tagged individuals for the observation}
\defType{Vector of strings}
\defDefault{No default}

\defSub{detection}{The probability of detecting a recaptured individual}
\defType{Real number (estimable)}
\defDefault{No default}
\defLowerBound{0.0 (inclusive)}
\defUpperBound{1.0 (inclusive)}

\defSub{time\_step\_proportion}{The proportion through the mortality block of the time step when the observation is evaluated}
\defType{Real number (estimable)}
\defDefault{0.5}
\defLowerBound{0.0 (inclusive)}
\defUpperBound{1.0 (inclusive)}

\defSub{table recaptured}{The table of data specifying the recaptures}
\defType{Data table with label = recaptured}
\defDefault{No default}
\defValue{A $n\times m$ matrix, where $n=$ the years and $m$ is categories \(\times\) length bins. See below for example.}
\defNote{example below}
\begin{verbatim}
table recaptured 
1993 1 32 25
1994 3 4 43
end_table
\end{verbatim}


\defSub{table scanned}{The table of data specifying the scanned fish}
\defType{Data table with label = scanned}
\defDefault{No default}
\defValue{A $n\times m$ matrix, where $n=$ the years and $m$ is categories \(\times\) length bins. See below for example.}
\defNote{example below}
\begin{verbatim}
table scanned 
1993 1 32 25
1994 3 4 43
end_table
\end{verbatim}

\subsubsection{Observation of type Tag Recapture By Length}
\commandlabsubarg{observation}{type}{Tag\_Recapture\_By\_Length}.
\defRef{sec:Observation-TagRecaptureByLength}
\label{syntax:Observation-TagRecaptureByLength}

\defSub{years}{The years for which there are observations}
\defType{Vector of non-negative integers}
\defDefault{No default}

\defSub{time\_step}{The time step to execute in}
\defType{String}
\defDefault{No default}

\defSub{length\_bins}{The length bins}
\defType{Vector of real numbers (estimable)}
\defDefault{true}

\defSub{selectivities}{The labels of the selectivities used for untagged categories}
\defType{Vector of strings}
\defDefault{true}

\defSub{tagged\_selectivities}{The labels of the tag category selectivities}
\defType{Vector of strings}
\defDefault{No default}

\defSub{detection}{The probability of detecting a recaptured individual}
\defType{Real number (estimable)}
\defDefault{No default}
\defLowerBound{0.0 (inclusive)}
\defUpperBound{1.0 (inclusive)}

\defSub{dispersion}{The overdispersion parameter (phi)}
\defType{Real number (estimable)}
\defDefault{1.0}
\defLowerBound{0.0 (inclusive)}

\defSub{time\_step\_proportion}{The proportion through the mortality block of the time step when the observation is evaluated}
\defType{Real number (estimable)}
\defDefault{0.5}
\defLowerBound{0.0 (inclusive)}
\defUpperBound{1.0 (inclusive)}


\defSub{table recaptured}{The table of data specifying the recaptures}
\defType{Data table with label = recaptured}
\defDefault{No default}
\defValue{A $n\times m$ matrix, where $n=$ the years and $m$ is categories \(\times\) length bins. See below for example.}
\defNote{example below}
\begin{verbatim}
table recaptured 
1993 1 32 25
1994 3 4 43
end_table
\end{verbatim}


\defSub{table scanned}{The table of data specifying the scanned fish}
\defType{Data table with label = scanned}
\defDefault{No default}
\defValue{A $n\times m$ matrix, where $n=$ the years and $m$ is categories \(\times\) length bins. See below for example.}
\defNote{example below}
\begin{verbatim}
table scanned 
1993 1 32 25
1994 3 4 43
end_table
\end{verbatim}
