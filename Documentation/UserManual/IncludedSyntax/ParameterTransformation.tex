\defComLab{parameter\_transformation}{Define an object of type \emph{parameter\_transformation}}.
\defRef{sec:Transformation}
\label{syntax:Transformation}

\defSub{label}{Label for the transformation block}
\defType{String}
\defDefault{No default}

\defSub{type}{The type of transformation}
\defType{String}
\defDefault{No default}

\defSub{prior\_applies\_to\_restored\_parameters}{If the prior applies to the parameters (true) with jacobian (if it exists) or prior applies to transformed\_parameter (false) with no jacobian}
\defType{bool}
\defDefault{false}

\defSub{parameters}{The label of the parameters used in the transformation}
\defType{String}
\defDefault{No default}

\subsubsection{parameter\_transformation of type Log}
\commandlabsubarg{parameter\_transformation}{type}{log}.
\label{syntax:Transformation-Log} \\ \\
The addressable parameter for this transformation is \subcommand{log\_parameter}. See Section \ref{subsec:Transformation-types}, paragraph \ref{sec:Transformation-Log}.

\subsubsection{parameter\_transformation of type Inverse}
\commandlabsubarg{parameter\_transformation}{type}{inverse}.
\label{syntax:Transformation-Inverse}  \\ \\
The addressable parameter for this transformation is \subcommand{inverse\_parameter}. See Section \ref{subsec:Transformation-types}, paragraph \ref{sec:Transformation-Inverse}.

\subsubsection{parameter\_transformation of type Average\_Difference}
\commandlabsubarg{parameter\_transformation}{type}{average\_difference}.
\label{syntax:Transformation-AverageDifference} \\ \\
The addressable parameters for this transformation are \subcommand{average\_parameter} and \subcommand{difference\_parameter}. See Section \ref{subsec:Transformation-types}, paragraph \ref{sec:Transformation-AverageDifference}.

\subsubsection{parameter\_transformation of type log\_sum}
\commandlabsubarg{parameter\_transformation}{type}{log\_sum}.
\label{syntax:Transformation-LogSum} \\ \\
The addressable parameters for this transformation are \subcommand{log\_total\_parameter} and \subcommand{difference\_parameter}. See Section \ref{subsec:Transformation-types}, paragraph \ref{sec:Transformation-LogSum}.

\subsubsection{parameter\_transformation of type orthogonal}
\commandlabsubarg{parameter\_transformation}{type}{orthogonal}.
\label{syntax:Transformation-Orthogonal} \\ \\
The addressable parameters for this transformation are \subcommand{product\_parameter} and \subcommand{quotient\_parameter}. See Section \ref{subsec:Transformation-types}, paragraph \ref{sec:Transformation-Orthogonal}.

\subsubsection{parameter\_transformation of type sum\_to\_one}
\commandlabsubarg{parameter\_transformation}{type}{sum\_to\_one}.
\label{syntax:Transformation-SumToOne} \\ \\
The addressable parameter for this transformation is \subcommand{product\_parameter}. See Section \ref{subsec:Transformation-types}, paragraph \ref{sec:Transformation-SumToOne}.

\subsubsection{parameter\_transformation of type simplex}
\commandlabsubarg{parameter\_transformation}{type}{simplex}.
\label{syntax:Transformation-Simplex}

\defSub{sum\_to\_one}{Apply the sum\_to\_one constraint}
\defType{bool}
\defDefault{true}
\defValue{If true, the parameter vector in natural space will sum to one, otherwise it will sum to the value of the length(parameter), i.e., defines the vector to have average one}

The addressable parameter for this transformation is \subcommand{simplex}. See Section \ref{subsec:Transformation-types}, paragraph \ref{sec:Transformation-Simplex}.


