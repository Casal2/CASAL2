\defComLab{process}{Define an object of type \emph{Process}}.
\defRef{sec:Process}
\label{syntax:Process}

\defSub{label}{The label of the process}
\defType{String}
\defDefault{No default}

\defSub{type}{The type of process}
\defType{String}
\defDefault{No default}

\subsubsection{Process of type Ageing}
\commandlabsubarg{process}{type}{Ageing}.
\defRef{sec:Process-Ageing}
\label{syntax:Process-Ageing}

\defSub{categories}{The labels of the categories to age}
\defType{Vector of strings}
\defDefault{No default}

%\subsubsection{Process of type Load Partition}
%\commandlabsubarg{process}{type}{Load\_Partition}.
%\defRef{sec:Process-LoadPartition}
%\label{syntax:Process-LoadPartition}
%
%\defSub{table}{The table of data specifying the n in each partition category and age}
%\defType{Data table with label = data}
%\defDefault{No default}
%\defValue{}
%\defNote{See \ref{sec:DataTable} for more details on specifying data tables}
%
\subsubsection{Process of type Maturation}
\commandlabsubarg{process}{type}{Maturation}.
\defRef{sec:Process-Maturation}
\label{syntax:Process-Maturation}

\defSub{from}{The list of categories to mature from}
\defType{Vector of strings}
\defDefault{No default}

\defSub{to}{The list of categories to mature to}
\defType{Vector of strings}
\defDefault{No default}

\defSub{selectivities}{The list of selectivities to use for maturation}
\defType{Vector of strings}
\defDefault{No default}

\defSub{years}{The years to be associated with the maturity rates}
\defType{Vector of non-negative integers}
\defDefault{No default}

\defSub{rates}{The rates to mature for each year}
\defType{Vector of real numbers (estimable)}
\defDefault{No default}

\subsubsection{Process of type Mortality Constant Rate}
\commandlabsubarg{process}{type}{Mortality\_Constant\_Rate}.
\defRef{sec:Process-MortalityConstantRate}
\label{syntax:Process-MortalityConstantRate}

\defSub{categories}{The list of category labels}
\defType{Vector of strings}
\defDefault{No default}

\defSub{m}{The mortality rates}
\defType{Real number (estimable)}
\defDefault{No default}
\defLowerBound{0.0 (inclusive)}

\defSub{time\_step\_proportions}{The time step proportions for the mortality rates}
\defType{Vector of real numbers}
\defDefault{false}
\defLowerBound{0.0 (inclusive)}
\defUpperBound{1.0 (inclusive)}
\defValue{The time step proportions must sum to one. If only one value is supplied, then the each time step is allocated an equal proportion. Otherwise the number of values must equal the number of time steps}

\defSub{relative\_m\_by\_age}{The list of mortality by age ogive labels for the categories}
\defType{Vector of strings}
\defDefault{No default}

\subsubsection{Process of type Mortality Constant Exploitation}
\commandlabsubarg{process}{type}{Mortality\_Constant\_Exploitation}.
\defRef{sec:Process-MortalityConstantExploitation}
\label{syntax:Process-MortalityConstantExploitation}

\defSub{categories}{The list of category labels}
\defType{Vector of strings}
\defDefault{No default}

\defSub{u}{The exploitation rates}
\defType{Real number (estimable)}
\defDefault{No default}
\defLowerBound{0.0 (inclusive)}
\defUpperBound{1.0 (inclusive)}

\defSub{time\_step\_proportions}{The time step proportions for the exploitation rates}
\defType{Vector of real numbers}
\defDefault{false}
\defLowerBound{0.0 (inclusive)}
\defUpperBound{1.0 (inclusive)}
\defValue{The time step proportions must sum to one. If only one value is supplied, then the each time step is allocated an equal proportion. Otherwise the number of values must equal the number of time steps}

\defSub{relative\_u\_by\_age}{The list of exploitation by age ogive labels for the categories}
\defType{Vector of strings}
\defDefault{No default}

\subsubsection{Process of type Mortality Disease Rate}
\commandlabsubarg{process}{type}{Mortality\_Disease\_Rate}.
\defRef{sec:Process-MortalityDiseaseRate}
\label{syntax:Process-MortalityDiseaseRate}

\defSub{categories}{The list of category labels}
\defType{Vector of strings}
\defDefault{No default}

\defSub{disease\_mortality\_rate}{The disease mortality rates}
\defType{Real number (estimable)}
\defDefault{No default}
\defLowerBound{0.0 (inclusive)}
\defUpperBound{10.0 (inclusive)}

\defSub{year\_effects}{Annual deviations around the disease mortality rate}
\defType{Vector of real numbers (estimable)}
\defDefault{No default}
\defLowerBound{0.0 (inclusive)}

\defSub{selectivities}{The list of selectivities}
\defType{Vector of strings}
\defDefault{No default}

\defSub{years}{Years in which to apply the disease mortality in}
\defType{Vector of non-negative integers}
\defDefault{No default}

\subsubsection{Process of type Mortality Event}
\commandlabsubarg{process}{type}{Mortality\_Event}.
\defRef{sec:Process-MortalityEvent}
\label{syntax:Process-MortalityEvent}

\defSub{categories}{The categories}
\defType{Vector of strings}
\defDefault{No default}

\defSub{years}{The years in which to apply the mortality process}
\defType{Vector of non-negative integers}
\defDefault{No default}

\defSub{catches}{The number of removals (catches) to apply for each year}
\defType{Vector of real numbers (estimable)}
\defDefault{No default}

\defSub{u\_max}{The maximum exploitation rate ($U\_{max}$)}
\defType{Real number}
\defDefault{0.99}
\defLowerBound{0.0 (inclusive)}
\defUpperBound{1.0 (inclusive)}

\defSub{selectivities}{The list of selectivities}
\defType{Vector of strings}
\defDefault{No default}

\defSub{penalty}{The label of the penalty to apply if the total number of removals cannot be taken}
\defType{String}
\defDefault{No default}

\subsubsection{Process of type Mortality Event Biomass}
\commandlabsubarg{process}{type}{Mortality\_Event\_Biomass}.
\defRef{sec:Process-MortalityEventBiomass}
\label{syntax:Process-MortalityEventBiomass}

\defSub{categories}{The category labels}
\defType{Vector of strings}
\defDefault{No default}

\defSub{selectivities}{The labels of the selectivities for each of the categories}
\defType{Vector of strings}
\defDefault{No default}

\defSub{years}{The years in which to apply the mortality process}
\defType{Vector of non-negative integers}
\defDefault{No default}

\defSub{catches}{The biomass of removals (catches) to apply for each year}
\defType{Vector of real numbers (estimable)}
\defDefault{No default}

\defSub{u\_max}{The maximum exploitation rate ($U\_{max}$)}
\defType{Real number (estimable)}
\defDefault{0.99}
\defLowerBound{0.0 (inclusive)}
\defUpperBound{1.0 (inclusive)}

\defSub{penalty}{The label of the penalty to apply if the total biomass of removals cannot be taken}
\defType{String}
\defDefault{No default}

\subsubsection{Process of type Mortality Holling Rate}
\commandlabsubarg{process}{type}{Mortality\_Holling\_Rate}.
\defRef{sec:Process-MortalityHollingRate}
\label{syntax:Process-MortalityHollingRate}

\defSub{prey\_categories}{The prey categories labels}
\defType{Vector of strings}
\defDefault{No default}

\defSub{predator\_categories}{The predator categories labels}
\defType{Vector of strings}
\defDefault{No default}

\defSub{is\_abundance}{Is vulnerable amount of prey and predator an abundance [true] or biomass [false]}
\defType{Boolean}
\defDefault{true}

\defSub{a}{Parameter a}
\defType{Real number (estimable)}
\defDefault{No default}
\defLowerBound{0.0 (inclusive)}

\defSub{b}{Parameter b}
\defType{Real number (estimable)}
\defDefault{No default}
\defLowerBound{0.0 (inclusive)}

\defSub{x}{This parameter controls the functional form: Holling function type 2 (x=2) or 3 (x=3), or generalised (Michaelis Menten, x>=1)}
\defType{Real number (estimable)}
\defDefault{No default}
\defLowerBound{1.0 (inclusive)}

\defSub{u\_max}{The maximum exploitation rate ($U\_{max}$)}
\defType{Real number}
\defDefault{0.99}
\defLowerBound{0.0 (inclusive)}
\defUpperBound{1.0 (exclusive)}

\defSub{prey\_selectivities}{The selectivities for prey categories}
\defType{Vector of strings}
\defDefault{true}

\defSub{predator\_selectivities}{The selectivities for predator categories}
\defType{Vector of strings}
\defDefault{true}

\defSub{penalty}{The label of penalty}
\defType{String}
\defDefault{No default}

\defSub{years}{The years in which to apply the mortality process}
\defType{Vector of non-negative integers}
\defDefault{No default}

\defSub{table}{The table of data specifying the predator selectivities}
\defType{Data table with label = \TODO{I don't know how this one works?}}
\defDefault{No default}
\defValue{}
\defNote{See \ref{sec:DataTable} for more details on specifying data tables}

\defSub{table}{The table of data specifying the prey selectivities}
\defType{Data table with label = \TODO{I don't know how this one works?}}
\defDefault{No default}
\defValue{}
\defNote{See \ref{sec:DataTable} for more details on specifying data tables}

\subsubsection{Process of type Mortality Initialisation Event}
\commandlabsubarg{process}{type}{mortality\_initialisation\_event}.
\defRef{sec:Process-MortalityInitialisationEvent}
\label{syntax:Process-MortalityInitialisationEvent}

\defSub{categories}{The categories}
\defType{Vector of strings}
\defDefault{No default}

\defSub{catch}{The number of removals (catches) to apply for each year}
\defType{Real number (estimable)}
\defDefault{No default}

\defSub{u\_max}{The maximum exploitation rate ($U\_{max}$)}
\defType{Real number}
\defDefault{0.99}
\defLowerBound{0.0 (inclusive)}
\defUpperBound{1.0 (exclusive)}

\defSub{selectivities}{The list of selectivities}
\defType{Vector of strings}
\defDefault{No default}

\defSub{penalty}{The label of the penalty to apply if the total number of removals cannot be taken}
\defType{String}
\defDefault{No default}

\defSub{table}{The table of data specifying the catches for each fishery, the categories, years, and the $U_max$}
\defType{Data table with label = \TODO{Craig? Can you specify this table?}}
\defDefault{No default}
\defNote{See \ref{sec:DataTable} for more details on specifying data tables}

\subsubsection{Process of type Mortality Initialisation Event Biomass}
\commandlabsubarg{process}{type}{Mortality\_Initialisation\_Event\_Biomass}.
\defRef{sec:Process-MortalityInitialisationEventBiomass}
\label{syntax:Process-MortalityInitialisationEventBiomass}

\defSub{categories}{The categories}
\defType{Vector of strings}
\defDefault{No default}

\defSub{catch}{The number of removals (catches) to apply for each year}
\defType{Real number (estimable)}
\defDefault{No default}

\defSub{u\_max}{The maximum exploitation rate ($U\_{max}$)}
\defType{Real number}
\defDefault{0.99}
\defLowerBound{0.0 (inclusive)}
\defUpperBound{1.0 (inclusive)}

\defSub{selectivities}{The list of selectivities}
\defType{Vector of strings}
\defDefault{No default}

\defSub{penalty}{The label of the penalty to apply if the total number of removals cannot be taken}
\defType{String}
\defDefault{No default}


\subsubsection{Process of type Mortality Initialisation Baranov}
\commandlabsubarg{process}{type}{mortality\_initialisation\_baranov}.
\defRef{sec:Process-MortalityInitialisationEventBiomass}
\label{syntax:Process-MortalityInitialisationBaranov}

\defSub{categories}{The categories}
\defType{Vector of strings}
\defDefault{No default}

\defSub{fishing\_mortality}{The fishing mortality to apply}
\defType{Real number (estimable)}
\defDefault{No default}

\defSub{selectivities}{The list of selectivities for each category}
\defType{Vector of strings}
\defDefault{No default}

\subsubsection{Process of type Mortality Hybrid}
\commandlabsubarg{process}{type}{mortality\_hybrid}.
\defRef{sec:Process-MortalityHybrid}
\label{syntax:Process-MortalityHybrid}

\defSub{categories}{The categories for to apply natural mortality to}
\defType{Vector of strings}
\defDefault{No default}

\defSub{m}{The natural mortality rates for each category}
\defType{Real number (estimable)}
\defDefault{No default}
\defLowerBound{0.0 (inclusive)}

\defSub{time\_step\_proportions}{The time step proportions for natural mortality}
\defType{Vector of real numbers}
\defDefault{true}
\defLowerBound{0.0 (inclusive)}
\defUpperBound{1.0 (inclusive)}
\defValue{Proportions must sum to one}

\defSub{biomass}{Switch to indicate if the catches are biomasses or abundances}
\defType{Boolean}
\defDefault{True}

\defSub{relative\_m\_by\_age}{The M-by-age selectivities to apply to each of the categories for natural mortality}
\defType{Vector of strings}
\defDefault{No default}

\defSub{max\_f}{Maximum \(F\) allowed}

\defType{Real number}
\defDefault{4.0}
\defLowerBound{0.0 (inclusive)}

\defSub{f\_iterations}{The number of tuning iterations to solve \(F\)}
\defType{integer}
\defDefault{4}
\defLowerBound{0 (inclusive)}

\defSub{table catches}{The table of data specifying the catches for each fishery and year}
\defDefault{No default}
\defNote{See below for example}
\begin{verbatim}
table catches
year Fishery1_label Fishery2_label
1993 34 34 
1994 23 34 
end_table
\end{verbatim}

\defSub{table method}{The table of data specifying which fishery interacts with which category, selectivities, penalty, and time-step for each fishery and annual duration of the fishery.}
\defDefault{No default}
\defValue{}
\defNote{See below for example}
\begin{verbatim}
table method
method  		category selectivity 	annual_duration time_step 	penalty
Fishery1_label  male   	 fish_sel 		1				step1 		none
Fishery2_label  male   	 fish_sel 		1 				step1 		none
end_table
\end{verbatim}
\subsubsection{Process of type Mortality Instantaneous}
\commandlabsubarg{process}{type}{Mortality\_Instantaneous}.
\defRef{sec:Process-MortalityInstantaneous}
\label{syntax:Process-MortalityInstantaneous}

\defSub{categories}{The categories for instantaneous mortality}
\defType{Vector of strings}
\defDefault{No default}

\defSub{m}{The natural mortality rates for each category}
\defType{Real number (estimable)}
\defDefault{No default}
\defLowerBound{0.0 (inclusive)}

\defSub{time\_step\_proportions}{The time step proportions for natural mortality}
\defType{Vector of real numbers}
\defDefault{true}
\defLowerBound{0.0 (inclusive)}
\defUpperBound{1.0 (inclusive)}
\defValue{Proportions must sum to one}

\defSub{biomass}{Switch to indicate if the catches are biomasses or abundances}
\defType{Boolean}
\defDefault{True}

\defSub{relative\_m\_by\_age}{The M-by-age selectivities to apply to each of the categories for natural mortality}
\defType{Vector of strings}
\defDefault{No default}

\defSub{table catches}{The table of data specifying the catches for each fishery and year}
\defDefault{No default}
\defValue{}
\defNote{See below for example}
\begin{verbatim}
table catches
year Fishery1_label Fishery2_label
1993 34 34 
1994 23 34 
end_table
\end{verbatim}

\defSub{table method}{The table of data specifying which fishery interacts with which category, selectivities, time-step, penalties and \(u_{max}\) for each fishery and year.}
\defDefault{No default}
\defValue{}
\defNote{See below for example}
\begin{verbatim}
table method
method  		category selectivity 	u_max 	time_step 	penalty
Fishery1_label  male   	 fish_sel 		0.9		step1 		CatchMustBeTaken
Fishery2_label  male   	 fish_sel 		0.9 	step1 		CatchMustBeTaken
end_table
\end{verbatim}

\subsubsection{Process of type Mortality Instantaneous Retained}
\commandlabsubarg{process}{type}{Mortality\_Instantaneous\_Retained}.
\defRef{sec:Process-MortalityInstantaneousRetained}
\label{syntax:Process-MortalityInstantaneousRetained}

\defSub{categories}{The categories for instantaneous mortality}
\defType{Vector of strings}
\defDefault{No default}

\defSub{m}{The natural mortality rates for each category}
\defType{Real number (estimable)}
\defDefault{No default}
\defLowerBound{0.0 (inclusive)}

\defSub{time\_step\_proportions}{The time step proportions for natural mortality}
\defType{Vector of real numbers}
\defDefault{No default}
\defLowerBound{0.0 (inclusive)}
\defUpperBound{1.0 (inclusive)}
\defValue{Proportions must sum to one}

\defSub{relative\_m\_by\_age}{The M-by-age selectivities to apply on the categories for natural mortality}
\defType{Vector of strings}
\defDefault{No default}

\defSub{table}{The table of data specifying the catches for each fishery, the categories, years, and the $U_max$}
\defType{Data table with label = \TODO{Craig? Can you specify this table?}}
\defDefault{No default}
\defValue{}
\defNote{See \ref{sec:DataTable} for more details on specifying data tables}

\subsubsection{Process of type Mortality Prey Suitability}
\commandlabsubarg{process}{type}{Mortality\_Prey\_Suitability}.
\defRef{sec:Process-MortalityPreySuitability}
\label{syntax:Process-MortalityPreySuitability}

\defSub{prey\_categories}{The prey categories labels}
\defType{Vector of strings}
\defDefault{No default}

\defSub{predator\_categories}{The predator categories labels}
\defType{Vector of strings}
\defDefault{No default}

\defSub{consumption\_rate}{The predator consumption rate}
\defType{Real number (estimable)}
\defDefault{No default}
\defLowerBound{0.0 (inclusive)}
\defUpperBound{1.0 (inclusive)}

\defSub{electivities}{The prey electivities}
\defType{Vector of real numbers (estimable)}
\defDefault{No default}
\defLowerBound{0.0 (inclusive)}
\defUpperBound{1.0 (inclusive)}

\defSub{u\_max}{The maximum exploitation rate ($U\_{max}$)}
\defType{Real number}
\defDefault{0.99}
\defLowerBound{0.0 (inclusive)}
\defUpperBound{1.0 (exclusive)}

\defSub{prey\_selectivities}{The selectivities for prey categories}
\defType{Vector of strings}
\defDefault{No default}

\defSub{predator\_selectivities}{The selectivities for predator categories}
\defType{Vector of strings}
\defDefault{No default}

\defSub{penalty}{The label of the penalty}
\defType{String}
\defDefault{No default}

\defSub{years}{The year that process occurs}
\defType{Vector of non-negative integers}
\defDefault{No default}
\subsubsection{Markovian Movement}
\commandlabsubarg{process}{type}{markovian\_movement}.
\defRef{sec:Process-MarkovianMovement}
\label{syntax:Process-MarkovianMovement}

\defSub{from}{The categories to transition from}
\defType{Vector of strings}
\defDefault{No default}
\defValue{Valid category labels}

\defSub{to}{The categories to transition to}
\defType{Vector of strings}
\defDefault{No default}
\defValue{Valid category labels}

\defSub{proportions}{The proportions to transition for each category}
\defType{Real number (estimable)}
\defDefault{No default}
\defLowerBound{0.0 (inclusive)}
\defUpperBound{1.0 (inclusive)}

\defSub{selectivities}{The selectivities to apply to each proportion}
\defType{Vector of strings}
\defDefault{No default}
\defValue{Valid selectivity labels}

\subsubsection{Process of type null process}
\commandlabsubarg{process}{type}{null\_process}.
\label{syntax:Process-Null}

The null\_process  type has no additional subcommands. Note that this process does nothing. It is included primarily as a means of replacing other processes with "no action" to allow for testing of alternative model structures.

\subsubsection{Process of type Recruitment Beverton Holt}
\commandlabsubarg{process}{type}{Recruitment\_Beverton\_Holt}.
\defRef{sec:Process-RecruitmentBevertonHolt}
\label{syntax:Process-RecruitmentBevertonHolt}

\defSub{categories}{The category labels}
\defType{Vector of strings}
\defDefault{No default}

\defSub{r0}{R0, the mean recruitment used to scale annual recruits or initialise the model}
\defType{Real number (estimable)}
\defDefault{No default}
\defLowerBound{0.0 (inclusive)}
\defValue{Use either R0 or B0, but not both}

\defSub{b0}{B0, the SSB corresponding to R0, and used to scale annual recruits or initialise the model}
\defType{Real number (estimable)}
\defDefault{No default}
\defLowerBound{0.0 (inclusive)}
\defValue{Use either R0 or B0, but not both}

\defSub{proportions}{The proportion for each category}
\defType{Real number (estimable)}
\defDefault{No default}

\defSub{age}{The age at recruitment}
\defType{Non-negative integer}
\defDefault{No default}

\defSub{ssb\_offset}{The spawning biomass year offset}
\defType{Non-negative integer}
\defDefault{No default}

\defSub{steepness}{Steepness (h)}
\defType{Real number (estimable)}
\defDefault{1.0}
\defLowerBound{0.2 (inclusive)}
\defUpperBound{1.0 (inclusive)}

\defSub{ssb}{The SSB label (i.e., the derived quantity label)}
\defType{String}
\defDefault{No default}

\defSub{b0\_initialisation\_phase}{The initialisation phase label that B0 is from}
\defType{String}
\defDefault{No default}

\defSub{ycs\_values}{Deprecated}
\defSub{ycs\_years}{Deprecated}
\defSub{standardise\_ycs\_years}{Deprecated}

\defSub{recruitment\_multipliers}{The recruitment values also termed year class strengths.}
\defType{Vector of real numbers (estimable)}
\defDefault{No default}

\defSub{standardise\_years}{The years that are included for year class standardisation, they refer to the recruited year not spawning or year class year.}
\defType{Vector of non-negative integers}
\defDefault{true}

\subsubsection{Process of type Recruitment Beverton Holt With Deviations}
\commandlabsubarg{process}{type}{Recruitment\_Beverton\_Holt\_With\_Deviations}.
\defRef{sec:Process-RecruitmentBevertonHoltWithDeviations}
\label{syntax:Process-RecruitmentBevertonHoltWithDeviations}

\defSub{categories}{The category labels}
\defType{Vector of strings}
\defDefault{No default}

\defSub{r0}{R0, the mean recruitment used to scale annual recruits or initialise the model}
\defType{Real number (estimable)}
\defDefault{No default}
\defLowerBound{0.0 (inclusive)}
\defValue{Use either R0 or B0, but not both}

\defSub{b0}{B0, the SSB corresponding to R0, and used to scale annual recruits or initialise the model}
\defType{Real number (estimable)}
\defDefault{No default}
\defLowerBound{0.0 (inclusive)}
\defValue{Use either R0 or B0, but not both}

\defSub{proportions}{The proportion for each category}
\defType{Real number (estimable)}
\defDefault{No default}

\defSub{age}{The age at recruitment}
\defType{Non-negative integer}
\defDefault{true}

\defSub{ssb\_offset}{The spawning biomass year offset}
\defType{Non-negative integer}
\defDefault{No default}

\defSub{steepness}{Steepness (h)}
\defType{Real number (estimable)}
\defDefault{1.0}
\defLowerBound{0.2 (inclusive)}
\defUpperBound{1.0 (inclusive)}

\defSub{ssb}{The SSB label (i.e., the derived quantity label)}
\defType{String}
\defDefault{No default}
\defValue(A valid derived quantity)

\defSub{sigma\_r}{The standard deviation of recruitment, $\sigma_R$}
\defType{Real number (estimable)}
\defDefault{No default}
\defLowerBound{0.0 (inclusive)}

\defSub{b\_max}{The maximum bias adjustment}
\defType{Real number (estimable)}
\defDefault{0.85}
\defLowerBound{0.0 (inclusive)}
\defUpperBound{1.0 (inclusive)}

\defSub{last\_year\_with\_no\_bias}{The last year with no bias adjustment}
\defType{Non-negative integer}
\defDefault{false}

\defSub{first\_year\_with\_bias}{The first year with full bias adjustment}
\defType{Non-negative integer}
\defDefault{false}

\defSub{last\_year\_with\_bias}{The last year with full bias adjustment}
\defType{Non-negative integer}
\defDefault{false}

\defSub{first\_recent\_year\_with\_no\_bias}{The first recent year with no bias adjustment}
\defType{Non-negative integer}
\defDefault{false}

\defSub{b0\_initialisation\_phase}{The initialisation phase label that B0 is from}
\defType{String}
\defDefault{No default}

\defSub{deviation\_values}{The recruitment deviation values}
\defType{Vector of real numbers (estimable)}
\defDefault{No default}

\defSub{deviation\_years}{Deprecated}

\subsubsection{Process of type Recruitment Ricker}
\commandlabsubarg{process}{type}{Recruitment\_Ricker}.
\defRef{sec:Process-RecruitmentRicker}
\label{syntax:Process-RecruitmentRicker}

\defSub{categories}{The category labels}
\defType{Vector of strings}
\defDefault{No default}

\defSub{r0}{R0, the mean recruitment used to scale annual recruits or initialise the model}
\defType{Real number (estimable)}
\defDefault{No default}
\defLowerBound{0.0 (inclusive)}
\defValue{Use either R0 or B0, but not both}

\defSub{b0}{B0, the SSB corresponding to R0, and used to scale annual recruits or initialise the model}
\defType{Real number (estimable)}
\defDefault{No default}
\defLowerBound{0.0 (inclusive)}
\defValue{Use either R0 or B0, but not both}

\defSub{proportions}{The proportion for each category}
\defType{Real number (estimable)}
\defDefault{No default}

\defSub{age}{The age at recruitment}
\defType{Non-negative integer}
\defDefault{No default}

\defSub{ssb\_offset}{The spawning biomass year offset}
\defType{Non-negative integer}
\defDefault{No default}

\defSub{steepness}{Steepness (h)}
\defType{Real number (estimable)}
\defDefault{1.0}
\defLowerBound{0.2 (inclusive)}
\defUpperBound{1.0 (inclusive)}

\defSub{ssb}{The SSB label (i.e., the derived quantity label)}
\defType{String}
\defDefault{No default}

\defSub{b0\_initialisation\_phase}{The initialisation phase label that B0 is from}
\defType{String}
\defDefault{No default}

\defSub{ycs\_values}{Deprecated}
\defSub{ycs\_years}{Deprecated}
\defSub{standardise\_ycs\_years}{Deprecated}

\defSub{recruitment\_multipliers}{The recruitment values also termed year class strengths.}
\defType{Vector of real numbers (estimable)}
\defDefault{No default}

\defSub{standardise\_years}{The years that are included for year class standardisation, they refer to the recruited year not spawning or year class year.}
\defType{Vector of non-negative integers}
\defDefault{true}

\subsubsection{Process of type Recruitment Constant}
\commandlabsubarg{process}{type}{Recruitment\_Constant}.
\defRef{sec:Process-RecruitmentConstant}
\label{syntax:Process-RecruitmentConstant}

\defSub{categories}{The categories}
\defType{Vector of strings}
\defDefault{No default}

\defSub{proportions}{The proportion for each category}
\defType{Real number (estimable)}
\defDefault{true}

\defSub{age}{The age at recruitment}
\defType{Non-negative integer}
\defDefault{No default}

\defSub{r0}{R0, the recruitment used for annual recruits and initialise the model}
\defType{Real number (estimable)}
\defDefault{No default}
\defLowerBound{0.0 (inclusive)}

\subsubsection{Process of type Survival Constant Rate}
\commandlabsubarg{process}{type}{Survival\_Constant\_Rate}.
\defRef{sec:Process-SurvivalConstantRate}
\label{syntax:Process-SurvivalConstantRate}

\defSub{categories}{The list of categories}
\defType{Vector of strings}
\defDefault{No default}

\defSub{s}{The survival rates}
\defType{Real number (estimable)}
\defDefault{No default}
\defLowerBound{0.0 (inclusive)}
\defUpperBound{1.0 (inclusive)}

\defSub{time\_step\_proportions}{The time step proportions for the survival rate $S$}
\defType{Vector of real numbers}
\defDefault{true}
\defLowerBound{0.0 (exclusive)}
\defUpperBound{1.0 (inclusive)}
\defValue{The proportions must sum to one}

\defSub{selectivities}{The selectivity labels for each category}
\defType{Vector of strings}
\defDefault{No default}

\subsubsection{Process of type Tag By Age}
\commandlabsubarg{process}{type}{Tag\_By\_Age}.
\defRef{sec:Process-TagByAge}
\label{syntax:Process-TagByAge}

\defSub{from}{The categories that are selected for tagging (i.e, transition from)}
\defType{Vector of strings}
\defDefault{No default}

\defSub{to}{The categories that have tags (i.e., transition to)}
\defType{Vector of strings}
\defDefault{No default}

\defSub{min\_age}{The minimum age tagged}
\defType{Non-negative integer}
\defDefault{No default}

\defSub{max\_age}{The maximum age tagged}
\defType{Non-negative integer}
\defDefault{No default}

\defSub{penalty}{The penalty label}
\defType{String}
\defDefault{No default}

\defSub{u\_max}{The maximum exploitation rate ($U\_{max}$)}
\defType{Real number}
\defDefault{0.99}
\defLowerBound{0.0 (inclusive)}
\defUpperBound{1.0 (exclusive)}

\defSub{years}{The years to execute the tagging in}
\defType{Vector of non-negative integers}
\defDefault{No default}

\defSub{initial\_mortality}{The initial mortality value}
\defType{Real number (estimable)}
\defDefault{0.0}
\defLowerBound{0.0 (inclusive)}

\defSub{initial\_mortality\_selectivity}{The initial mortality selectivity label}
\defType{String}
\defDefault{No default}

\defSub{selectivities}{The selectivity labels}
\defType{Vector of strings}
\defDefault{No default}

\defSub{n}{N}
\defType{Vector of real numbers (estimable)}
\defDefault{true}

\defSub{table}{The table of data specifying the n to tag from and to  each category, years, and the $U_max$}
\defType{Data table with label = \TODO{Craig? Can you specify this table?}}
\defDefault{No default}
\defValue{}
\defNote{See \ref{sec:DataTable} for more details on specifying data tables}

\subsubsection{Process of type Tag By Length}
\commandlabsubarg{process}{type}{Tag\_By\_Length}.
\defRef{sec:Process-TagByLength}
\label{syntax:Process-TagByLength}

\defSub{from}{The categories that are selected for tagging (i.e, transition from)}
\defType{Vector of strings}
\defDefault{No default}

\defSub{to}{The categories that have tags (i.e., transition to)}
\defType{Vector of strings}
\defDefault{No default}

\defSub{penalty}{The penalty label}
\defType{String}
\defDefault{No default}

\defSub{u\_max}{The maximum exploitation rate ($U\_{max}$)}
\defType{Real number}
\defDefault{0.99}
\defLowerBound{0.0 (inclusive)}
\defUpperBound{1.0 (exclusive)}

\defSub{compatibility\_option}{Backwards compatibility option: either casal2 (the default) or casal effects penalty and age-length calculation}
\defType{string}
\defDefault{casal2}
\defValue{Valid options are \subcommand{casal2} \& \subcommand{casal}}

\defSub{years}{The years to execute the tagging events in}
\defType{Vector of non-negative integers}
\defDefault{No default}

\defSub{initial\_mortality}{The initial mortality to apply to tags as a proportion}
\defType{Real number (estimable)}
\defDefault{0.0}
\defLowerBound{0.0 (inclusive)}
\defUpperBound{1.0 (inclusive)}

\defSub{initial\_mortality\_selectivity}{}
\defType{String}
\defDefault{No default}
\defValue{A valid selectivity label}

\defSub{selectivities}{}
\defType{Vector of strings}
\defDefault{No default}
\defValue{Valid selectivity labels}

\defSub{n}{The total number of tags to apply}
\defType{Vector of real numbers (estimable)}
\defDefault{No default}

\defSub{tolerance}{Tolerance for checking the specified proportions sum to one}
\defType{Real number}
\defDefault{1e-5}
\defLowerBound{0 (inclusive)}
\defUpperBound{1.0 (inclusive)}

\textbf{You can specify the input of this process as numbers or proportions}. See the following syntax examples on how to specify each one.

\defSub{table numbers}{The table of releases as numbers for the process}
\defType{Data table with label = numbers}
\defDefault{Can be replaced with table of type proportions see below}
\defValue{A \(n_y \times (n_l\times n_c) + 1\) matrix, where $n_y=$ is the number of years, \(n_l\) are the number of length bins defined at the \command{model} block and \(n_c\) are the number of categories. The first column is the year value for that row. See below for an example.}
\defNote{example below}
\begin{verbatim}
table numbers 
1993 34 34 23 43
1994 23 34 23 43
end_table
\end{verbatim}


\defSub{table proportions}{The table of releases as numbers for the process}
\defType{Data table with label = proportions}
\defDefault{Can be replaced with numbers table see above}
\defValue{A \(n_y \times (n_l\times n_c) + 1\) matrix, where $n_y=$ is the number of years, \(n_l\) are the number of length bins defined at the \command{model} block and \(n_c\) are the number of categories. The first column is the year value for that row. See below for an example.}
\defNote{example below}
\begin{verbatim}
n 200 300 ## need to specify n if you give proportions
table proportions 
1993 0.1 0.2 0.7
1994 0.1 0.2 0.7
end_table
\end{verbatim}

\subsubsection{Process of type Tag Loss}
\commandlabsubarg{process}{type}{Tag\_Loss}.
\defRef{sec:Process-TagLoss}
\label{syntax:Process-TagLoss}

\defSub{categories}{The list of categories}
\defType{Vector of strings}
\defDefault{No default}
\defValue{Valid category labels}

\defSub{tag\_loss\_rate}{The instantaneous tag loss rates}
\defType{Vector of real numbers (estimable)}
\defDefault{No default}
\defLowerBound{0.0 (inclusive)}
\defValue{The instantaneous rate of tag loss (supplied as either a single value that is applied to all categories, or a vector of length equal to the number of categories defined for this process)}

\defSub{time\_step\_proportions}{The time step proportions for tag loss}
\defType{Vector of real numbers (estimable)}
\defDefault{true}
\defLowerBound{0.0 (inclusive)}
\defUpperBound{1.0 (inclusive)}
\defNote{The sum of the values of time\_step\_proportions must equal 1.0}

\defSub{tag\_loss\_type}{The type of tag loss}
\defType{String}
\defDefault{single}
\defValue{Valid options are \subcommand{single} \& \subcommand{double}}

\defSub{selectivities}{The selectivities}
\defType{Vector of strings}
\defDefault{No default}

\defSub{year}{The year the first tagging release process was executed}
\defType{Non-negative integer}
\defDefault{No default}
\defNote{For the double tag loss rate, this is also assumed to be the first year in the calculation of the annual loss rates}

\subsubsection{Process of type Tag Loss Empirical}
\commandlabsubarg{process}{type}{Tag\_Loss\_Empirical}.
\defRef{sec:Process-TagLossEmpirical}
\label{syntax:Process-TagLossEmpirical}

\defSub{categories}{The list of categories}
\defType{Vector of strings}
\defDefault{No default}
\defValue{Valid category labels}

\defSub{tag\_loss\_rate}{The instantaneous tag loss rates}
\defType{Vector of real numbers (estimable)}
\defDefault{No default}
\defLowerBound{0.0 (inclusive)}
\defValue{The instantaneous rate of tag loss (supplied as either a single value that is applied to all years at liberty, or a vector of length equal to the number of years at liberty defined for this process)}

\defSub{time\_step\_proportions}{The time step proportions for tag loss}
\defType{Vector of real numbers (estimable)}
\defDefault{true}
\defLowerBound{0.0 (inclusive)}
\defUpperBound{1.0 (inclusive)}
\defNote{The sum of the values of time\_step\_proportions must equal 1.0}

\defSub{selectivities}{The selectivities}
\defType{Vector of strings}
\defDefault{No default}

\defSub{year}{The year the first tagging release process was executed}
\defType{Non-negative integer}
\defDefault{No default}
\defNote{For the tag loss rate, this is also assumed to be the first year at liberty in the calculation of the annual loss rates}

\defSub{years\_at\_liberty}{The years at liberty that the tag\_loss\_rate applies to}
\defType{Non-negative integer}
\defDefault{No default}
\defLowerBound{0 (inclusive)}
\defNote{Years at liberty must be a valid value between 0 and the maximum number of years in the model. The tag\_loss\_rate is not applied for years at liberty that are not specified}

\subsubsection{Process of type Transition Category}
\commandlabsubarg{process}{type}{Transition\_Category}.
\defRef{sec:Process-TransitionCategory}
\label{syntax:Process-TransitionCategory}

\defSub{from}{The categories to transition from}
\defType{Vector of strings}
\defDefault{No default}
\defValue{Valid category labels}

\defSub{to}{The categories to transition to}
\defType{Vector of strings}
\defDefault{No default}
\defValue{Valid category labels}

\defSub{proportions}{The proportions to transition for each category}
\defType{Real number (estimable)}
\defDefault{No default}
\defLowerBound{0.0 (inclusive)}
\defUpperBound{1.0 (inclusive)}

\defSub{selectivities}{The selectivities to apply to each proportion}
\defType{Vector of strings}
\defDefault{No default}
\defValue{Valid selectivity labels}

\defSub{include\_in\_mortality\_block}{Include this process within the mortality block}
\defType{Boolean}
\defDefault{false}
\defValue{Either true or false}
\defNote{Warning, if true, this may adversely effect derived quantities and observations that interpolate over the mortality block - see \defRef{sec:Process-TransitionCategory}}

\subsubsection{Process of type Transition Category By Age}
\commandlabsubarg{process}{type}{Transition\_Category\_By\_Age}.
\defRef{sec:Process-TransitionCategoryByAge}
\label{syntax:Process-TransitionCategoryByAge}

\defSub{from}{The categories to transition from}
\defType{Vector of strings}
\defDefault{No default}
\defValue{Valid category labels}

\defSub{to}{The categories to transition to}
\defType{Vector of strings}
\defDefault{No default}
\defValue{Valid category labels}

\defSub{min\_age}{The minimum age to transition}
\defType{Non-negative integer}
\defDefault{No default}
\defValue{Valid category labels}

\defSub{max\_age}{The maximum age to transition}
\defType{Non-negative integer}
\defDefault{No default}

\defSub{penalty}{The penalty label}
\defType{String}
\defDefault{No default}
\defValue{A valid penalty label}

\defSub{u\_max}{The maximum exploitation rate ($U\_{max}$)}
\defType{Real number (estimable)}
\defDefault{0.99}
\defLowerBound{0.0 (inclusive)}
\defUpperBound{1.0 (exclusive)}

\defSub{years}{The years to execute the transition in}
\defType{Vector of non-negative integers}
\defDefault{No default}

\defSub{table n}{The table of numbers at age to transition from and to each category}
\defType{See below for example}
\defDefault{No default}
\defValue{}
\defNote{See below for example}
\begin{verbatim}
table n
year 3 4 5 6
2008 1000 2000 3000 4000
end_table
\end{verbatim}
