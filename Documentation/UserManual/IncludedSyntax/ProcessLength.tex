\defComLab{process}{Define an object of type \emph{Process}}.
\defRef{sec:Process}
\label{syntax:Process}

\defSub{label}{The label of the process}
\defType{String}
\defDefault{No default}

\defSub{type}{The type of process}
\defType{String}
\defDefault{No default}

\subsubsection{Process of type Load Partition}
\commandlabsubarg{process}{type}{Load\_Partition}.
\defRef{sec:Process-LoadPartition}
\label{syntax:Process-LoadPartition}

\defSub{table}{The table of data specifying the n in each partition category and age}
\defType{Data table with label = \TODO{Craig? Can you specify this table?}}
\defDefault{No default}
\defValue{}
\defNote{See \ref{sec:DataTable} for more details on specifying data tables}

\subsubsection{Process of type Mortality Constant Rate}
\commandlabsubarg{process}{type}{Mortality\_Constant\_Rate}.
\defRef{sec:Process-MortalityConstantRate}
\label{syntax:Process-MortalityConstantRate}

\defSub{categories}{The list of category labels}
\defType{Vector of strings}
\defDefault{No default}

\defSub{m}{The mortality rates}
\defType{Real number (estimable)}
\defDefault{No default}
\defLowerBound{0.0 (inclusive)}

\defSub{time\_step\_proportions}{The time step proportions for the mortality rates}
\defType{Vector of real numbers}
\defDefault{false}
\defLowerBound{0.0 (inclusive)}
\defUpperBound{1.0 (inclusive)}
\defValue{The time step proportions must sum to one. If only one value is supplied, then the each time step is allocated an equal proportion. Otherwise the number of values must equal the number of time steps}

\defSub{relative\_m\_by\_length}{The list of mortality by length ogive labels for the categories}
\defType{Vector of strings}
\defDefault{No default}

\subsubsection{Process of type Mortality Constant Exploitation}\commandlabsubarg{process}{type}{Mortality\_Constant\_Exploitation}.
\defRef{sec:Process-MortalityConstantExploitation}
\label{syntax:Process-MortalityConstantExploitation}

\defSub{categories}{The list of category labels}
\defType{Vector of strings}
\defDefault{No default}

\defSub{u}{The exploitation rates}
\defType{Real number (estimable)}
\defDefault{No default}
\defLowerBound{0.0 (inclusive)}
\defUpperBound{1.0 (inclusive)}

\defSub{time\_step\_proportions}{The time step proportions for the exploitation rates}
\defType{Vector of real numbers}
\defDefault{false}
\defLowerBound{0.0 (inclusive)}
\defUpperBound{1.0 (inclusive)}
\defValue{The time step proportions must sum to one. If only one value is supplied, then the each time step is allocated an equal proportion. Otherwise the number of values must equal the number of time steps}

\defSub{relative\_u\_by\_length}{The list of exploitation by length ogive labels for the categories}
\defType{Vector of strings}
\defDefault{No default}

\subsubsection{Process of type Mortality Disease Rate}
\commandlabsubarg{process}{type}{Mortality\_Disease\_Rate}.
\defRef{sec:Process-Length-DiseaseMortalityRate}
\label{syntax:Process-MortalityDiseaseRate}

\defSub{categories}{The list of category labels}
\defType{Vector of strings}
\defDefault{No default}

\defSub{disease\_mortality\_rate}{The disease mortality rates}
\defType{Real number (estimable)}
\defDefault{No default}
\defLowerBound{0.0 (inclusive)}
\defUpperBound{10.0 (inclusive)}

\defSub{year\_effects}{Annual deviations around the disease mortality rate}
\defType{Vector of real numbers (estimable)}
\defDefault{No default}
\defLowerBound{0.0 (inclusive)}

\defSub{selectivities}{The list of selectivities}
\defType{Vector of strings}
\defDefault{No default}

\defSub{years}{Years in which to apply the disease mortality in}
\defType{Vector of non-negative integers}
\defDefault{No default}

\subsubsection{Process of type Mortality Instantaneous}
\commandlabsubarg{process}{type}{Mortality\_Instantaneous}.
\defRef{sec:Process-MortalityInstantaneous}
\label{syntax:Process-MortalityInstantaneous}

\defSub{categories}{The categories for instantaneous mortality}
\defType{Vector of strings}
\defDefault{No default}

\defSub{m}{The natural mortality rates for each category}
\defType{Real number (estimable)}
\defDefault{No default}
\defLowerBound{0.0 (inclusive)}

\defSub{time\_step\_proportions}{The time step proportions for natural mortality}
\defType{Vector of real numbers}
\defDefault{true}
\defLowerBound{0.0 (inclusive)}
\defUpperBound{1.0 (inclusive)}
\defValue{Proportions must sum to one}

\defSub{biomass}{Switch to indicate if the catches are biomasses or abundances}
\defType{Boolean}
\defDefault{True}

\defSub{relative\_m\_by\_length}{The M-by-length selectivities to apply to each of the categories for natural mortality}
\defType{Vector of strings}
\defDefault{No default}

\defSub{table}{The table of data specifying the catches for each fishery, the categories, years, and the $U_max$}
\defType{Data table with label = \TODO{Craig? Can you specify this table?}}
\defDefault{No default}
\defValue{}
\defNote{See \ref{sec:DataTable} for more details on specifying data tables}

\subsubsection{Process of type null process}
\commandlabsubarg{process}{type}{null\_process}.
\label{syntax:Process-Null}

The null\_process  type has no additional subcommands. Note that this process does nothing. It is included primarily as a means of replacing other processes with "no action" to allow for testing of alternative model structures.

\subsubsection{Process of type Recruitment Beverton Holt}
\commandlabsubarg{process}{type}{Recruitment\_Beverton\_Holt}.
\defRef{sec:Process-RecruitmentBevertonHolt}
\label{syntax:Process-RecruitmentBevertonHolt}

\defSub{categories}{The category labels}
\defType{Vector of strings}
\defDefault{No default}

\defSub{r0}{R0, the mean recruitment used to scale annual recruits or initialise the model}
\defType{Real number (estimable)}
\defDefault{No default}
\defLowerBound{0.0 (inclusive)}
\defValue{Use either R0 or B0, but not both}

\defSub{b0}{B0, the SSB corresponding to R0, and used to scale annual recruits or initialise the model}
\defType{Real number (estimable)}
\defDefault{No default}
\defLowerBound{0.0 (inclusive)}
\defValue{Use either R0 or B0, but not both}

\defSub{proportions}{The proportion for each category}
\defType{Real number (estimable)}
\defDefault{No default}

\defSub{initial\_mean\_length}{The initial mean length at recruitment}
\defType{Non-negative real number (estimable)}
\defDefault{No default}

\defSub{initial\_length\_cv}{The initial length cv at recruitment}
\defType{Non-negative real number (estimable)}
\defDefault{No default}

\defSub{ssb\_offset}{The spawning biomass year offset}
\defType{Non-negative integer}
\defDefault{No default}

\defSub{steepness}{Steepness (h)}
\defType{Real number (estimable)}
\defDefault{1.0}
\defLowerBound{0.2 (inclusive)}
\defUpperBound{1.0 (inclusive)}

\defSub{ssb}{The SSB label (i.e., the derived quantity label)}
\defType{String}
\defDefault{No default}

\defSub{b0\_initialisation\_phase}{The initialisation phase label that B0 is from}
\defType{String}
\defDefault{No default}

\defSub{ycs\_values}{Deprecated}
\defSub{ycs\_years}{Deprecated}
\defSub{standardise\_ycs\_years}{Deprecated}

\defSub{recruitment\_multipliers}{The recruitment values also termed year class strengths.}
\defType{Vector of real numbers (estimable)}
\defDefault{No default}

\defSub{standardise\_years}{The years that are included for year class standardisation, they refer to the recruited year not spawning or year class year.}
\defType{Vector of non-negative integers}
\defDefault{true}

\subsubsection{Process of type Recruitment Constant}
\commandlabsubarg{process}{type}{Recruitment\_Constant}.
\defRef{sec:Process-RecruitmentConstant}
\label{syntax:Process-RecruitmentConstant}

\defSub{categories}{The categories}
\defType{Vector of strings}
\defDefault{No default}

\defSub{proportions}{The proportion for each category}
\defType{Real number (estimable)}
\defDefault{true}

\defSub{length\_bins}{The length bin that recruits are uniformly distributed over at the time of recruitment}
\defType{Non-negative integer}
\defDefault{No default}

\defSub{r0}{R0, the recruitment used for annual recruits and initialise the model}
\defType{Real number (estimable)}
\defDefault{No default}
\defLowerBound{0.0 (inclusive)}

\subsubsection{Process of type Tagging}
\commandlabsubarg{process}{type}{Tagging}.
\defRef{sec:Process-Tagging}
\label{syntax:Process-Tagging}

\defSub{from}{The categories that are selected for tagging (i.e, transition from)}
\defType{Vector of strings}
\defDefault{No default}

\defSub{to}{The categories that have tags (i.e., transition to)}
\defType{Vector of strings}
\defDefault{No default}

\defSub{penalty}{The penalty label}
\defType{String}
\defDefault{No default}

\defSub{u\_max}{The maximum exploitation rate ($U\_{max}$)}
\defType{Real number}
\defDefault{0.99}
\defLowerBound{0.0 (inclusive)}
\defUpperBound{1.0 (exclusive)}

\defSub{compatibility\_option}{Backwards compatibility option: either casal2 (the default) or casal effects penalty and age-length calculation}
\defType{string}
\defDefault{casal2}
\defValue{Valid options are \subcommand{casal2} \& \subcommand{casal}}

\defSub{years}{The years to execute the tagging events in}
\defType{Vector of non-negative integers}
\defDefault{No default}

\defSub{initial\_mortality}{The initial mortality to apply to tags as a proportion}
\defType{Real number (estimable)}
\defDefault{0.0}
\defLowerBound{0.0 (inclusive)}
\defUpperBound{1.0 (inclusive)}

\defSub{initial\_mortality\_selectivity}{}
\defType{String}
\defDefault{No default}
\defValue{A valid selectivity label}

\defSub{selectivities}{}
\defType{Vector of strings}
\defDefault{No default}
\defValue{Valid selectivity labels}

\defSub{n}{The total number of tags to apply}
\defType{Vector of real numbers (estimable)}
\defDefault{No default}

\defSub{tolerance}{Tolerance for checking the specified proportions sum to one}
\defType{Real number}
\defDefault{1e-5}
\defLowerBound{0 (inclusive)}
\defUpperBound{1.0 (inclusive)}

\defSub{table}{The table of data specifying the n to tag from and to  each category, years, and the $U_max$}
\defType{Data table with label = \TODO{Craig? Can you specify this table?}}
\defDefault{No default}
\defValue{}
\defNote{See \ref{sec:DataTable} for more details on specifying data tables}

\subsubsection{Process of type Transition Category}
\commandlabsubarg{process}{type}{Transition\_Category}.
\defRef{sec:Process-TransitionCategory}
\label{syntax:Process-TransitionCategory}

\defSub{from}{The categories to transition from}
\defType{Vector of strings}
\defDefault{No default}
\defValue{Valid category labels}

\defSub{to}{The categories to transition to}
\defType{Vector of strings}
\defDefault{No default}
\defValue{Valid category labels}

\defSub{proportions}{The proportions to transition for each category}
\defType{Real number (estimable)}
\defDefault{No default}
\defLowerBound{0.0 (inclusive)}
\defUpperBound{1.0 (inclusive)}

\defSub{selectivities}{The selectivities to apply to each proportion}
\defType{Vector of strings}
\defDefault{No default}
\defValue{Valid selectivity labels}
