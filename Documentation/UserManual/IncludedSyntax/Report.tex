\defComLab{report}{Define an object of type \emph{Report}}.
\defRef{sec:Report}
\label{syntax:Report}

\defSub{label}{The report label}
\defType{String}
\defDefault{No default}

\defSub{type}{The report type}
\defType{String}
\defDefault{No default}

\defSub{file\_name}{The file name. If not supplied, then output is directed to standard out}
\defType{String}
\defDefault{No default}

\defSub{write\_mode}{Specify if any previous file with the same name should be overwritten, appended to, or is generated using a sequential suffix}
\defType{String}
\defDefault{overwrite}
\defValue{valid options are \subcommand{append}, \subcommand{overwrite}, \subcommand{incremental\_suffix}}

\defSub{format}{Report output format}
\defType{String}
\defDefault{r}
\defValue{Either \R\ for formatting for reading into \R\, or \texttt{none} for no formatting}

\subsubsection{Report of type Addressable}
\commandlabsubarg{report}{type}{Addressable}.
\defRef{sec:Report-Addressable}
\label{syntax:Report-Addressable}

\defSub{parameter}{The addressable parameter name}
\defType{String}
\defDefault{No default}

\defSub{years}{Define the years that the report is generated for}
\defType{Vector of non-negative integers}
\defDefault{No default}

\defSub{time\_step}{Defines the time-step that the report applies to}
\defType{String}
\defDefault{No default}
\defValue{A valid time step label}

\subsubsection{Report of type Age Length}
\commandlabsubarg{report}{type}{Age\_Length}.
\defRef{sec:Report-AgeLength}
\label{syntax:Report-AgeLength}

\defSub{time\_step}{The time step label}
\defType{String}
\defDefault{No default}
\defValue{A valid time step label}

\defSub{years}{The years for the report}
\defType{Vector of non-negative integers}
\defDefault{All years}

\defSub{age\_length}{The age-length label}
\defType{String}
\defDefault{No default}


\subsubsection{Report of type Ageing Error Matrix}
\commandlabsubarg{report}{type}{Ageing\_Error\_Matrix}.
\defRef{sec:Report-AgeingErrorMatrix}
\label{syntax:Report-AgeingErrorMatrix}

\defSub{ageing\_error}{The ageing error label}
\defType{String}
\defDefault{No default}

\subsubsection{Report of type Catchability}
\commandlabsubarg{report}{type}{Catchability}.
\defRef{sec:Report-Catchability}
\label{syntax:Report-Catchability}

\defSub{catchability}{The catchability label}
\defType{String}
\defDefault{No default}
\defValue{If not specified, then the label of the report is assumed to be the category label}

\subsubsection{Report of type Category Info}
\commandlabsubarg{report}{type}{Category\_Info}.
\defRef{sec:Report-CategoryInfo}
\label{syntax:Report-CategoryInfo}

The Category\_Info report has no additional subcommands.

\subsubsection{Report of type Category List}
\commandlabsubarg{report}{type}{Category\_List}.
\defRef{sec:Report-CategoryList}
\label{syntax:Report-CategoryList}

The Category\_List report has no additional subcommands.

\subsubsection{Report of type Correlation Matrix}
\commandlabsubarg{report}{type}{Correlation\_Matrix}.
\defRef{sec:Report-CorrelationMatrix}
\label{syntax:Report-CorrelationMatrix}

The Correlation\_Matrix report has no additional subcommands.

\subsubsection{Report of type Covariance Matrix}
\commandlabsubarg{report}{type}{Covariance\_Matrix}.
\defRef{sec:Report-CovarianceMatrix}
\label{syntax:Report-CovarianceMatrix}

The Covariance\_Matrix type has no additional subcommands.

\subsubsection{Report of type Default}
\commandlabsubarg{report}{type}{Default}.
\defRef{sec:Report-Default}
\label{syntax:Report-Default}

\defSub{catchabilities}{Report catchabilities}
\defType{Boolean}
\defDefault{false}
\defNote{Reports all valid catchabilities}

\defSub{derived\_quantities}{Report derived quantities}
\defType{Boolean}
\defDefault{false}
\defNote{Reports all valid derived quantities}

\defSub{observations}{Report observations}
\defType{Boolean}
\defDefault{false}
\defNote{Reports all valid observations}

\defSub{processes}{Report processes}
\defType{Boolean}
\defDefault{false}
\defNote{Reports all valid processes}

\defSub{projects}{Report projects}
\defType{Boolean}
\defDefault{false}
\defNote{Reports all valid projections}

\defSub{selectivities}{Report selectivities}
\defType{Boolean}
\defDefault{false}
\defNote{Reports all valid selectivities}

\subsubsection{Report of type Derived Quantity}
\commandlabsubarg{report}{type}{Derived\_Quantity}.
\defRef{sec:Report-DerivedQuantity}
\label{syntax:Report-DerivedQuantity}

\defSub{derived\_quantity}{The derived quantity label}
\defType{String}
\defDefault{No default}
\defValue{If not specified, then the label of the report is assumed to be the derived quantity label}
 
\subsubsection{Report of type Equation Test}
\commandlabsubarg{report}{type}{Equation\_Test}.
\defRef{sec:eq_parser}
\label{syntax:Report-EquationTest}

\defSub{equation}{The equation to do a test run of}
\defType{Vector of strings}
\defDefault{No default}

\subsubsection{Report of type Estimate Summary}
\commandlabsubarg{report}{type}{Estimate\_Summary}.
\defRef{sec:Report-EstimateSummary}
\label{syntax:Report-EstimateSummary}

\defValue{A summary of the estimated (free parameters)}

The Estimate\_Summary type has no additional subcommands.

\subsubsection{Report of type Estimate Value}
\commandlabsubarg{report}{type}{Estimate\_Value}.
\defRef{sec:Report-EstimateValue}
\label{syntax:Report-EstimateValue}

\defValue{The free parameters and their values, in a format suitable for use with \texttt{-i}}

The Estimate\_Value report has no additional subcommands.

\subsubsection{Report of type Estimation Result}
\commandlabsubarg{report}{type}{Estimation\_Result}.
\defRef{sec:Report-EstimationResult}
\label{syntax:Report-EstimationResult}

\defValue{A summary of the results of the minimisation}

The Estimation\_Result report has no additional subcommands.

\subsubsection{Report of type Hessian Matrix}
\commandlabsubarg{report}{type}{Hessian\_Matrix}.
\defRef{sec:Report-HessianMatrix}
\label{syntax:Report-HessianMatrix}

The Hessian\_Matrix report has no additional subcommands.

\subsubsection{Report of type Initialisation\_Partition}
\commandlabsubarg{report}{type}{Initialisation\_Partition}.
\defRef{sec:Report-InitialisationPartition}
\label{syntax:Report-InitialisationPartition}

The Initialisation\_Partition report has no additional subcommands.

\subsubsection{Report of type Initialisation Partition Mean Weight}
\commandlabsubarg{report}{type}{Initialisation\_Partition\_Mean\_Weight}.
\defRef{sec:Report-InitialisationPartitionMeanWeight}
\label{syntax:Report-InitialisationPartitionMeanWeight}

The Initialisation\_Partition\_Mean\_Weight report has no additional subcommands.

\subsubsection{Report of type MCMC Covariance}
\commandlabsubarg{report}{type}{MCMC\_Covariance}.
\defRef{sec:Report-MCMCCovariance}
\label{syntax:Report-MCMCCovariance}\\
\defValue{This will output the covariance matrices (the initial covariance matrix and the covariance matrix if adapted ) used for the MCMC chain.}

The MCMC\_Covariance report has no additional subcommands.  

\subsubsection{Report of type MCMC Objective}
\commandlabsubarg{report}{type}{MCMC\_Objective}.
\defRef{sec:Report-MCMCObjective}
\label{syntax:Report-MCMCObjective}

The MCMC\_Objective report has no additional subcommands.

\subsubsection{Report of type MCMC Sample}
\commandlabsubarg{report}{type}{MCMC\_Sample}.
\defRef{sec:Report-MCMCSample}
\label{syntax:Report-MCMCSample}

The MCMC\_Sample report has no additional subcommands.

\subsubsection{Report of type MPD}
\commandlabsubarg{report}{type}{MPD}.
\defRef{sec:Report-MPD}
\label{syntax:Report-MPD}\\
\defValue{An MPD report, consisting of the free parameters and the covariance matrix}

The MPD report has no additional subcommands.

\subsubsection{Report of type Objective Function}
\commandlabsubarg{report}{type}{Objective\_Function}.
\defRef{sec:Report-ObjectiveFunction}
\label{syntax:Report-ObjectiveFunction}

The Objective\_Function type has no additional subcommands.
\subsubsection{Report of type Observation}
\commandlabsubarg{report}{type}{Observation}.
\defRef{sec:Report-Observation}
\label{syntax:Report-Observation}

\defSub{observation}{The observation label}
\defType{String}
\defDefault{No default}

\defSub{normalised\_residuals}{Print Normalised Residuals?}
\defType{Boolean}
\defDefault{true}
\defNote{Only generated if valid for associated likelihood}

\defSub{pearsons\_residuals}{Print Pearsons Residuals?}
\defType{Boolean}
\defDefault{true}
\defNote{Only generated if valid for associated likelihood}

\subsubsection{Report of type Output Parameters}
\commandlabsubarg{report}{type}{Output\_Parameters}.
\defRef{sec:Report-OutputParameters}
\label{syntax:Report-OutputParameters}

The Output\_Parameters report has no additional subcommands.

\subsubsection{Report of type Partition}
\commandlabsubarg{report}{type}{Partition}.
\defRef{sec:Report-Partition}
\label{syntax:Report-Partition}

\defSub{time\_step}{Time Step label}
\defType{String}
\defDefault{No default}

\defSub{years}{Years}
\defType{Vector of non-negative integers}
\defDefault{All years}

\subsubsection{Report of type Partition Biomass}
\commandlabsubarg{report}{type}{Partition\_Biomass}.
\defRef{sec:Report-PartitionBiomass}
\label{syntax:Report-PartitionBiomass}

\defSub{time\_step}{The time step label}
\defType{String}
\defDefault{No default}

\defSub{years}{The years for the report}
\defType{Vector of non-negative integers}
\defDefault{All years}

\subsubsection{Report of type Partition Year Cross Age Matrix}
\commandlabsubarg{report}{type}{Partition\_Year\_Cross\_Age\_Matrix}.
\defRef{sec:Report-PartitionYearCrossAgeMatrix}
\label{syntax:Report-PartitionYearCrossAgeMatrix}

The Partition\_Year\_Cross\_Age\_Matrix report has no additional subcommands.

\subsubsection{Report of type Process}
\commandlabsubarg{report}{type}{Process}.
\defRef{sec:Report-Process}
\label{syntax:Report-Process}

\defSub{process}{The process label that is reported}
\defType{String}
\defDefault{No default}
\defValue{A valid process label}
\defValue{If not specified, then the label of the report is assumed to be the process label}


\subsubsection{Report of type Profile}
\commandlabsubarg{report}{type}{Profile}.
\defRef{sec:Profile}
\label{syntax:Report-Profile}

\subsubsection{Report of type Project}
\commandlabsubarg{report}{type}{Project}.
\defRef{sec:Project}
\label{syntax:Report-Project}

\defSub{project}{The project label that is reported}
\defType{String}
\defDefault{No default}
\defValue{If not specified, then the label of the report is assumed to be the projection label}

\subsubsection{Report of type Random Number Seed}
\commandlabsubarg{report}{type}{Random\_Number\_Seed}.
\defRef{sec:Report-RandomNumberSeed}
\label{syntax:Report-RandomNumberSeed}

The Random\_Number\_Seed type has no additional subcommands.

\subsubsection{Report of type Selectivity}
\commandlabsubarg{report}{type}{Selectivity}.
\defRef{sec:Report-Selectivity}
\label{syntax:Report-Selectivity}

\defSub{selectivity}{Selectivity name}
\defType{String}
\defDefault{No default}
\defValue{If not specified, then the label of the report is assumed to be the selectivity label}

\subsubsection{Report of type Simulated Observation}
\commandlabsubarg{report}{type}{Simulated\_Observation}.
\defRef{sec:Report-SimulatedObservation}
\label{syntax:Report-SimulatedObservation}

\defSub{observation}{The observation label}
\defType{String}
\defDefault{No default}
\defValue{If not specified, then the label of the report is assumed to be the observation label}

\subsubsection{Report of type Time Varying}
\commandlabsubarg{report}{type}{Time\_Varying}.
\defRef{sec:Report-TimeVarying}
\label{syntax:Report-TimeVarying}

\defSub{time\_varying}{The time varying label that is reported}
\defType{String}
\defDefault{No default}
\defValue{If not specified, then the label of the report is assumed to be the time varying label}

