\defComLab{initialisation\_phase}{Define an object of type \emph{Initialisation\_Phase}}.
\defRef{sec:Initialisation}
\label{syntax:InitialisationPhase}

\defSub{label}{The label of the initialisation phase}
\defType{String}
\defDefault{No default}

\defSub{type}{The type of initialisation}
\defType{String}
\defDefault{iterative}

\subsubsection{Initialisation Phase of type Cinitial}
\commandlabsubarg{initialisation\_phase}{type}{Cinitial}.
\defRef{sec:InitialisationPhase-Cinitial}
\label{syntax:InitialisationPhase-Cinitial}

\defSub{categories}{The list of categories for the Cinitial initialisation}
\defType{Vector of strings}
\defDefault{No default}

\defSub{table}{The table of data specifying the initial values by age}
\defType{Data table with label = n}
\defDefault{No default}
\defValue{A $n x m$ matrix, where $n=$ the categories and $m=$ the number of ages defined in the model. Rows of values specify the initial value of the partition. The table ends with `end\_table'}
\defNote{See \ref{sec:DataTable} for more details on specifying data tables}

\subsubsection{Initialisation Phase of type Derived}
\commandlabsubarg{initialisation\_phase}{type}{Derived}.
\defRef{sec:InitialisationPhase-Derived}
\label{syntax:InitialisationPhase-Derived}

\defSub{insert\_processes}{Specifies the additional processes that are not in the annual cycle, but should be be inserted into this initialisation phase}
\defType{Vector of strings}
\defNote{To insert a process during initialisation, it needs to subscribe to the following formate \subcommand{time\_step\_label()=process\_label}}
\defDefault{true}

\defSub{exclude\_processes}{Specifies the processes in the annual cycle that should be excluded from this initialisation phase}
\defType{Vector of strings}
\defDefault{true}


\subsubsection{Initialisation Phase of type Iterative}
\commandlabsubarg{initialisation\_phase}{type}{Iterative}.
\defRef{sec:InitialisationPhase-Iterative}
\label{syntax:InitialisationPhase-Iterative}

\defSub{years}{The number of iterations (years) over which to execute this initialisation phase}
\defType{Non-negative integer}
\defDefault{No default}

\defSub{insert\_processes}{The processes in the annual cycle to be include in this initialisation phase}
\defType{Vector of strings}
\defNote{To insert a process during initialisation, it needs to subscribe to the following formate \subcommand{time\_step\_label()=process\_label}}

\defSub{exclude\_processes}{The processes in the annual cycle to be excluded from this initialisation phase}
\defType{Vector of strings that are process labels to exclude}


\defSub{convergence\_years}{The iteration (year) when the test for convergence ($\lambda$) is evaluated}
\defType{Vector of non-negative integers}
\defDefault{true}

\defSub{lambda}{The maximum value of the absolute sum of differences (lambda) between the partition at year-1 and year that indicates successful convergence}
\defType{Real number}
\defDefault{0.0}

\subsubsection{Initialisation Phase of type State\_Category\_By\_Age}
\commandlabsubarg{initialisation\_phase}{type}{State\_Category\_By\_Age}.
\defRef{sec:InitialisationPhase-StateCategoryByAge}
\label{syntax:InitialisationPhase-StateCategoryByAge}

\defSub{categories}{The list of categories for the category state initialisation}
\defType{Vector of strings}
\defDefault{No default}

\defSub{min\_age}{The minimum age of values supplied in the definition of the category state}
\defType{Non-negative integer}
\defDefault{No default}

\defSub{max\_age}{The maximum age of values supplied in the definition of the category state}
\defType{Non-negative integer}
\defDefault{No default}

\defSub{table}{The table of data specifying the initial values by age}
\defType{Data table with label = n}
\defDefault{No default}
\defValue{A $n x m$ matrix, where $n=$ the categories and $m=$ the number of ages defined in the model. Rows of values specify the initial value of the partition. The table ends with `end\_table'}
\defNote{See \ref{sec:DataTable} for more details on specifying data tables}

