
\subsection{\I{Age-length relationship}\label{sec:AgeLength}}

The age-length relationship defines the functional form of the length-at-age (and the weight-at-length; see Section \ref{sec:MeanWeight}) of individuals at age/category within the model.

There are four length-age relationship options. The first is the naive "no relationship", where each individual has length 1 regardless of age. The others are:  von Bertalanffy relationship, the Schnute relationship, and "data" (empirical mean length-at-age for each model year).

The length-at-age relationship is used to calculate the length frequency given age, and with the length-weight relationship, the weight-at-age of individuals within an age/category. When defining length-at-age, the length-weight relationship must also be defined (see Section \ref{sec:MeanWeight}).

For most weight-based processes, derived quantities, and observations the users have the option of supplying a matrix of mean weight-at-age for each year (see section~\ref{sec:AgeWeight}). If these are supplied then this age-length-weight process is ignored.

Changes in length-at-age during the year, i.e., growth between birthdays, are represented by incrementing age as specified by the \subcommand{time\_step\_proportions} parameter.
	
\subsubsection{The `none' relationship}\index{Age-length relationshsip!None}\label{sec:AgeLength-None}

The length of each individual is 1 for all ages, and the \texttt{none} length-weight relationship must also be used.

\subsubsection{The von Bertalanffy relationship}\index{Age-length relationshsip!von Bertalanffy}\label{sec:AgeLength-VonBertalanffy}

\begin{equation}
\bar{s}(age)= L_\infty \left( 1 - \exp \left( -k \left(age-t_0 \right) \right) \right)
\end{equation}

\subsubsection{The Schnute relationship}\index{Schnute}\label{sec:AgeLength-Schnute}

\begin{equation}
\bar{s}(age)=\displaystyle\begin{cases}
  \left[ y_1^b + (y_2^b - y_1^b) \dfrac{1-\exp \left(-a(age - \tau_1) \right)}{1-\exp \left(-a(\tau_2 - \tau_1) \right)} \right]^{1/b}, & \text{if $a\ne0$ and $b\ne0$} \\
  \AddVspace
  y_1 \exp \left[ \ln \left( y_2 / y_1 \right) \dfrac{1-\exp \left(-a(age - \tau_1) \right)}{1-\exp \left(-a(\tau_2 - \tau_1) \right)} \right], & \text{if $a\ne0$ and $b=0$} \\
  \AddVspace
  \left[ y_1^b + \left( y_2^b - y_1^b \right) \dfrac{age-\tau_1}{\tau_2 - \tau_1} \right]^{1/b}, & \text{if $a=0$ and $b\ne0$} \\
  \AddVspace
  y_1 \exp \left[ \ln \left( y_2/y_1 \right) \dfrac{age-\tau_1}{\tau_2 - \tau_1} \right] , & \text{if $a=0$ and $b=0$} \\
  \end{cases}
\end{equation}

The von Bertalanffy relationship has parameters $L_\infty$, $k$, and $t_0$. The Schnute relationship \citep{836} has parameters $y_1$ and $y_2$, which are the mean lengths at reference ages $\tau_1$ and $\tau_2$, and $a$ and $b$; when $b=1$, this relationship reduces to the von Bertalanffy relationship with $k=a$.

\subsubsection{Data: matrix of size at age relationship}\index{Age-length relationship!Data}\label{sec:AgeLength-Data}

There is an option to input empirical length-at-age by year, which is an alternative to using an age-length growth model such as the von Bertalanffy and Schnute model. \CNAME\ will interpolate values for missing years across time steps. The calculations of length-at-age throughout the model years occur in the same time step.

{\small{\begin{verbatim}
@age_length   male_AL
type          data
time_step_proportions 0.0 0.0   #use age at start of time-step
length_weight wgt_male          # needed to convert numbers-at-age into catch
distribution  normal            # distribution of lengths around the mean length
cv_first      0.1               # cv of the distribution at the first age
cv_last       0.1               # cv of the distribution at the maximum age
time_step_measurements_were_made step2
internal_gaps mean
external_gaps mean
table data # first line has column labels for year and then all the model ages
year     2     3     4     5     6    7     8      9    10    11    12
1980 30.13 34.90 38.43 40.61 42.45 43.02 43.94 43.63 43.36 43.70 43.84
1981 30.33 34.78 38.03 40.15 42.22 42.89 44.21 44.07 43.99 44.32 44.64
end_table
\end{verbatim}}}

When the values for \textit{cv\_last} and \textit{cv\_first} are different, the cv used for intermediate ages is, by default, interpolate by that age's mean length. There is a legacy switch for testing the conversion of models from CASAL into \CNAME using the subcommand \textit{by\_length false} which allows the interpolation to be by age, the default setting for CASAL. CASAL also fixes to by-age interpretation when calculating the mean weight (see \ref{sec:LengthWeight-Basic}).

\subsubsection{Length distribution at age}\index{Length distribution at age}\label{sec:AgeLength-length_at_age}

When users supply an age-length class there is a subcommand \subcommand{distribution}. This describes the distribution of length for a given age. The three options are \texttt{none}, \texttt{normal} and \texttt{lognormal}. When a distribution is applied it contributes to the following dynamics. 1) applies an adjustment in mean-weight at age Equation~\ref{eq:mean_weight_with_adjustment} and 2) Used to populate an age-length transition matrix.

For any process or observation that requires the age based partition to be converted to length, an age-length transition matrix must be used for the translation. The age-length transition matrix describes a probability mass function for a specific age being in a set of length bins. Model length bins are denoted by \(l_b\) which have a minimum length value denoted by \(l_b^{min}\). The probability of a fish at age \(a\) being in length bin \(l_b\) is denoted by \(\phi_{a,l_b}\). To calculate the age-length transition matrix the following inputs are required for each age \(a\), mean length at age denoted by \(\bar{l}_a\) and standard deviation \(\sigma_a\) along with a distribution, then

\begin{equation}
	\phi_{a,l_b} = 
	\begin{cases}
		f(l_{b + 1}^{min},\bar{l}_a, \sigma_a), & \text{ for } a = a_{min}\\
		1 - f(l_{b}^{min},\bar{l}_a, \sigma_a), & \text{ for } a = a_{max} \ \& \text{ plus group}\\
		f(l_{b+1}^{min},\bar{l}_a, \sigma_a) - f(l_{b}^{min},\bar{l}_a, \sigma_a), & \text{ for } a > a_{min} \ \& \ a < a_{max} \\				
	\end{cases}
\end{equation}

where \(f(X,\mu, \sigma)\) is the cumulative density function defined by \subcommand{distribution}. The variance in these distributions are parametrised by the coefficient of variation (CV).

The CV at age can have the three following forms

\begin{enumerate}
	\item constant for all ages users just specify \subcommand{cv\_first}\\
	\item Changes linearly by age between \(a_{min}\) and \(a_{max}\) where the CV at \(a_{min}\) is defined by \subcommand{cv\_first} and the cv at \(a_{max}\) is defined by \subcommand{cv\_last}, and values inbetween are linear between these two points. Note the subcommand \subcommand{by\_length} needs to be false
	\item Changes linearly by mean length between \(a_{min}\) and \(a_{max}\) where the CV at \(a_{min}\) is defined by \subcommand{cv\_first} and the cv at \(a_{max}\) is defined by \subcommand{cv\_last}, but the values in between will be a linear interpolation based on mean length. This requires the subcommand \subcommand{by\_length} to be true;
\end{enumerate}

The CV is converted into a standard deviation as follows

\begin{align*}
	\sigma_a = CV_a \bar{l}_a & \text{ For the Normal distribution}\\
	\sigma_a = \sqrt{log(CV_a^2 + 1)} & \text{ For the Lognormal distribution}\\	
\end{align*}

When the age-length matrix is required in a model users must specify the \subcommand{length\_bin} on the \command{model}. \CNAME\ will build the age-length matrix for each year with processes; observations request it based on the model length bins. Each process and observation can then define a bespoke set of length bins for which their data represents, as long as those length bins are a subset of the model length bins. An example of this is

{\small{\begin{verbatim}
		@model
		length_bins 5 10 15 20 25 30 35 40 
		
		@observation 
		type proportions_at_length
		length_bins 10 20 30 40
\end{verbatim}}}

\CNAME\ will produce an error if input does not follow this rule.

For example the following configuration will produce an error

{\small{\begin{verbatim}
		@model
		length_bins 5 10 15 20 25 30 35 40 
		
		@observation 
		type proportions_at_length
		length_bins 7.5 17.5 27.5
		\end{verbatim}}}

because the observation length bins are not a subset of the model length bins. The reason for this business rule is, if we have coarse resolution of lengths relative to the model length bins, then the age-length matrix of the coarse length bins is a simple summation of the model age-length matrix, rather than recalculating for bespoke length bins, which is a computational task that requires many cumulative distribution function calls.

To get \CNAME\ to report the mean length-at-age, mean weight-at-age, and the age-length distribution, include the following report into your model with the years and time-steps of interest, see section~\ref{syntax:Report-AgeLength} for more information.

{\small{\begin{verbatim}
		@report age_length
		type age_length 
		age_length age_length_von_bertalanffy ## @age_length label
		years 1930:2020  ## years of interest
		time_step summer ## time-step label
		\end{verbatim}}}
	