\section{\I{The report section: output and reports}\label{sec:Report}}\index{Reports}\index{Reports section}

The command and subcommand syntax for the estimation section is given in Section \ref{syntax:Report}.

The report section specifies the printouts and other output from the model. \CNAME\ does not, in general, produce any output unless specified by a valid \command{report} block.

\subsection{Report command block format}

Reports from \CNAME\ can be defined to print partition and states objects at a particular point in time, observation summaries, estimated and derived parameter values, and objective function values. Many of the reports below will print a specific \command{}. The most useful report is the default report (see Section~\ref{sec:Report-Default}) which will auto generate many of the reports users require.

\begin{verbatim}
		@report default
		type default
		selectivities true
		derived_quantities true
		observations true
		processes true
		catchabilities true
		time_varying true
		parameter_transformations true
		time_varying true
		
		
		@report observation_age ## label of report
		type observation		    ## Type of report
		observation age_1990	 ## label corresponding to an @observation report, shown below

		@observation age_1990
		type proportion_at_age
		year 1990
		plus_group
		etc., ...
\end{verbatim}



\subsection{Report output format}

Reports from \CNAME\ have a standard style\index{Reports ! standard style} (with the exception of \texttt{output\_parameters} and \texttt{simulated\_observation}, see below). The standard style is that reports are prefixed with an asterisk followed by a user-defined label and type of report in brackets (e.g., \texttt{*label (type)}), with the report ending with the line \texttt{*end}. For example,

\begin{verbatim}
*My_report(type)
...
... # report content
...
*end
\end{verbatim}

This report block output format should make it easier for other software packages to read and process \CNAME\ output. The \texttt{extract} functions in the \R\ \CNAME\ package use this information to identify and read \CNAME\ output.

The \texttt{output\_parameters} report does not print either a header or \texttt{*end} at the end of the report block. This is because the \texttt{output\_parameters} report is designed to provide a single line vector of the estimated parameter values, or multiple lines for more than one set, which can be read by \CNAME\ with the command \texttt{casal2 -i}. This is a specialised report for the \texttt{casal2 -o filename} command.

For estimated values in standard output use the \texttt{type=estimate\_value} report.

Reports can be defined in a \command{report} command block but may not be output, e.g., a report to print the partition for a year and/or time step that does not exist, or reporting the covariance matrix when not estimation run mode.

Certain reports are associated with certain \CNAME\ run modes. These reports are ignored by \CNAME\ and the program will not generate any output for these reports, although they must still conform to \CNAME\ syntax requirements.

Not all reports will be generated in all run modes. Some reports are only available in some run modes. For example, when simulating, only the simulation reports will be output.

\subsection{\I{Print default reports}}\index{Reports ! Default report}\label{sec:Report-Default}

This is a report type that generates a range of other default reports, The report queries model and generates default reports for all catchabilities, observations, processes, selectivities, derived parameters, time-varying parameters, parameter transformations, and projections. This report will print out in run modes \texttt{-r, -e, -f}.


\begin{verbatim}
@report default
type default
selectivities true
derived_quantities true
observations true
processes true
catchabilities true
time_varying true
parameter_transformations true
time_varying true
\end{verbatim}

\subsection{\I{Print a summary of an initialisation step}}\index{Reports ! Initialisation summary}\label{sec:Report-Initialisation}

This report prints a summary of an initialisation, including convergence statistics for iterative initialisations.  This report will print out in run modes \texttt{-r, -e, -f}.

\subsection{\I{Print the partition at the end of an initialisation}}\index{Reports ! Initialisation partition}\label{sec:Report-InitialisationPartition}

This report prints the partition following the initialisation phase, which includes the numbers of individuals in each age class and category in the partition. This report will print out in run modes \texttt{-r, -e, -f}.

\subsection{\I{Print the partition}}\index{Reports ! Partition}\label{sec:Report-Partition}

This report prints the numbers of individuals in each age class and category in the partition for each given year or given years and time step. This report is evaluated at the end of the time step in the given year(s). This report will print out in run modes \texttt{-r, -e, -f}.

\subsection{\I{Print the partition biomass}}\index{Reports ! Partition biomass}\label{sec:Report-PartitionBiomass}

This report prints the biomass in each age class and category in the partition for each given year or given years and time step. This report is evaluated at the end of the time step in the given year(s). This report will print out in run modes \texttt{-r, -e, -f}.
\ifAgeBased
% One column version

\subsection{\I{Print the age length and length weight values}}\index{Reports ! Age-length}\label{sec:Report-AgeLength}\label{sec:Report-LenthWeight}

This report prints the length and weight value for each age class and category in the partition for each given year or given years and time step. This report is evaluated at the end of the time step in the given year(s). This report will print out in run modes \texttt{-r, -e, -f}.

\begin{verbatim}
@report length_weight_at_age
type partition_mean_weight
time_step step2
years 1900:2013
\end{verbatim}

\subsection{\I{Print the ageing error misclassification matrix}}\index{Reports ! Ageing error misclassification matrix}\label{sec:Report-AgeingErrorMatrix}

This report prints the ageing error misclassification matrix used to offset observations within during model the model fitting procedure.


\else
% Two column version

\subsection{\I{Print the Growth models}}\index{Reports ! Growth-Increment}\label{sec:Report-GrowthIncrement}

This report prints the length and weight relationship as well as growth increment model used to move fish along the length partition. This will print the growth transition matrix and other important values. This report is evaluated at the end of the time step in the given year(s). This report will print out in run modes \texttt{-r, -e, -f}.

\begin{verbatim}
@report growth_increment_model
type growth_increment
time_step step2
years 1900:2013
growth_increment growth_model
\end{verbatim}

\fi % end if

\subsection{\I{Print a parameter transformation}}\index{Reports ! Parameter Transformations}\label{sec:Report-ParameterTransformations}

This report prints a specific \command{parameter\_transformation} block with the values This report will print out in run modes \texttt{-r, -e, -m}. If you have many transformations it is best to report these using the default report see Section~\ref{sec:Report-Default}

{\small{\begin{verbatim}
		@report log_b0
		type parameter_transformation
		parameter_transformation log_b0
		\end{verbatim}}}

\subsection{\I{Print a process summary}}\index{Reports ! Processes}\label{sec:Report-Process}

Depending on the process, different summaries are produced. These reports typically detail the type of process, its parameters and other options, and any associated details. This report will print out in run modes \texttt{-r, -e, -f}.

\subsection{\I{Print derived quantities}}\index{Reports ! Derived quantities}\label{sec:Report-DerivedQuantity}

This report prints the description of the derived quantity, and the values of the derived quantity as recorded in the model state, for each year of the model, and for all years in the initialisation phase. This report will print out in run modes \texttt{-r, -e, -f}.

\subsection{\I{Print the estimated parameters}}\index{Reports ! Estimated parameters}\label{sec:Report-EstimateSummary}

This report prints a summary of the estimated parameters using the type \texttt{estimate\_summary}, including the parameter name, lower and upper bounds, the label of the prior, and its value. This report will print out in run modes \texttt{-r, -e}.

%\subsection{\I{Print the MPD (the free parameters and covariance matrix)}}\index{Reports ! MPD}\label{sec:Report-MPD}

%This report prints the estimated parameter values out as a vector along with the covariance matrix. The \texttt{MPD} report prints these in a format suitable for use as the starting point for an MCMC. This report will print out in run modes \texttt{-r, -e}.

\subsection{\I{Print the estimate values (the free parameters in the free parameter file format)}}\index{Reports ! Estimated parameters}\label{sec:Report-EstimateValue}

This report prints the estimated parameter values out as a vector. The \texttt{estimate\_values} report prints the name of the parameter, followed by the value for that run. This report will print out in run modes \texttt{-r, -e}.

\subsection{\I{Print the objective function}}\index{Reports ! Objective function}\label{sec:Report-ObjectiveFunction}

This report prints the total objective function value, the value of all observation likelihood components, the values of all priors, and the value of any penalties that have been incurred. If an individual model run does not incur a penalty, then the penalty will not be reported. This report will print out in run modes \texttt{-r, -e, -f}.

\subsection{\I{Print the covariance matrix}}\index{Reports ! Covariance Matrix}\label{sec:Report-CovarianceMatrix}

This report prints the covariance matrix if in estimation run mode and if the covariance has been requested by \commandlabsubarg{minimiser}{covariance}{true}.

\subsection{\I{Print the correlation matrix}}\index{Reports ! Corelation Matrix}\label{sec:Report-CorrelationMatrix}

This report prints the correlation matrix if in estimation run mode and if the covariance has been requested by \commandlabsubarg{minimiser}{covariance}{true}.

\subsection{\I{Print the Hessian matrix}}\index{Reports ! Hessian}\label{sec:Report-HessianMatrix}

This report prints the Hessian matrix if in estimation run mode and if the covariance has been requested by \commandlabsubarg{minimiser}{covariance}{true}.

\subsection{\I{Print the catchability values}}\index{Reports ! Catchability}\label{sec:Report-Catchability}

This report prints the catchability for a requested catchability.

\subsection{\I{Print observations, fits, and residuals}}\index{Reports ! Observations}\label{sec:Report-Observation}

This report prints, for each category or combination of categories, the expected values, residuals (observed $-$ expected), the error value, process error, the total error (i.e., the error value as modified by any additional process error), and the contribution to the total objective function of that individual datum in the observation.

Constants in the likelihood components are often ignored in the objective function score of individual observation values. Hence, the total score from an observation equals the contribution of the objective function scores from each individual observation value plus a constant term (if applicable). In likelihood components without a constant term, the total score from an observation will equal the contribution of the objective function scores from each individual observation value.

If \CNAME\ is in simulation run mode, then the contribution to the objective function of each observation is reported as zero.

\begin{verbatim}
		@report Tan_at_age_obs
		type observation
		observation TAN_AT_AGE
\end{verbatim}

\subsection{\I{Print simulated observations}}\index{Reports ! Simulated observations}\label{sec:Report-SimulatedObservation}

This report prints a complete set of observation values in the form specified by \commandlabsubarg{report}{type}{observation}, with observed values replaced by randomly generated simulated values. The output is in a form  suitable for use within a \CNAME\ \config, reproducing the command and subcommands from the \config. This report will print out in run mode \texttt{-s}.


Simulated reports will be produced with the following extension \texttt{.1\_1}. The first number of the extension relates to the row of the \texttt{-i}/\texttt{-I} file and the second number (separated by \texttt{\_}) represents the simulation iteration defined by the \texttt{n} argument in the configuration input \texttt{casal2 -s n}. Examples of the extension follow,
\begin{itemize}
	\item \texttt{.1\_1} indicates simulated data produced from the first row of parameters and is the first random draw
	\item \texttt{.1\_2} indicates simulated data produced from the first row of parameters and is the second random draw
	\item \texttt{.2\_10} indicates simulated data produced from the second row of parameters and is the 10\(^{th}\) random draw
\end{itemize}
\subsection{\I{Print selectivities}}\index{Reports ! Selectivities}\label{sec:Report-Selectivity}

This report prints the values of a selectivity for each age/length in the partition.

\subsection{\I{Print selectivities by year}}\index{Reports ! Selectivities By Year}\label{sec:Report-SelectivityByYear}

This report prints the values of a selectivity for each age/length for the given model years. Useful when you have a model with time-varying selectivity. See Section~\ref{syntax:Report-SelectivityByYear} for syntax information.


\subsection{\I{Print the random number seed}}\index{Reports ! Random number seed}\label{sec:Report-RandomNumberSeed}

This report prints the random number seed used by \CNAME\ to initialise the generated random number sequence. Additional runs which use the same random number seed and the same model will produce identical outputs.

\subsection{\I{Print the results of an MCMC}}\index{Reports ! MCMC}\label{sec:Report-MCMC}

This report prints the MCMC samples, objective function values, and proposal covariance matrix following an MCMC. This report will print out in run mode \texttt{-m}.

\subsection{\I{Print the MCMC samples as they are calculated}}\index{Reports ! MCMC samples}\label{sec:Report-MCMCSample}

This report prints the MCMC samples for each new \textit{i}th sample as they are calculated while doing an MCMC. The output file will be appended with each new sample as it is calculated by \CNAME. This report will print out in run mode \texttt{-m}.

\subsection{\I{Print the MCMC objective function values as they are calculated}}\index{Reports ! MCMC objective functions}\label{sec:Report-MCMCObjective}

This report prints the MCMC objective function values, along with the proposal covariance matrix, for each new \textit{i}th sample as they are calculated while doing an MCMC. The output file will be appended with each new set of objective function values as it is calculated by \CNAME. This report will print out in run mode \texttt{-m}.

\subsection{\I{Print time varying parameters}}\index{Reports ! time varying}\label{sec:Report-TimeVarying}

This report prints all \command{time\_varying} blocks with the values and years in which they were specified. This report will print out in run modes \texttt{-r, -e, -m}.

{\small{\begin{verbatim}
		@report time_varying_parameters
		type time_varying
		\end{verbatim}}}

\subsection{\I{Tabular reporting format}}\index{Reports ! Tabular}\label{sec:Tabular}

An alternative reporting framework to the standard output is the tabular reporting format. Tabular reporting is used with multi-line \texttt{-i} input files (like the MCMC sample or \texttt{-o} outputs). Tabular reports will print out a row that will correspond with each row of the \texttt{-i} input files.

Tabular reporting is specified using the \texttt{-{}-tabular} argument (\texttt{casal2 -r -{}-tabular -i file\_name}).

Derived quantities, processes, observations, and \texttt{estimate\_values} are the only report types that can be output with this format. For each input file the output will begin with the names of each column followed by a multi-line report ending with the \texttt{*end} syntax.

These tables can be read with \R\ using the \CNAME\ \R\ package. An example usage is reading in files of MCMC posterior values of derived quantities, which can then be plotted.


