\subsection{\I{Growth Increment models}\label{sec:GrowthIncrement}}

For length-based models the length structure of the partition has \(n_l\) length bins where \(l_i\) denotes length bin \(i\) with minimum length value denoted by \(lc_i\) and midpoint denoted as \(lm_i\).

\[
lm_i = 0.5 \left(lc_i + lc_{i+1}\right)
\]

In length-based models a growth transition matrix describes the probability of fish moving from length bin \(l_j\) to length bin \(l_k\). The growth increment models describe the expected mean length increment denoted by \(\mu_i\) for fish in length bin \(l_i\).

\subsubsection{The `none' model}\index{Growth Increment!None}\label{sec:GrowthIncrement-None}

All individuals have a mean length increment of 1 regardless of where they are in the length partition.

\subsubsection{The Exponential model}\index{Growth Increment!Exponential}\label{sec:GrowthIncrement-Exponential}

\begin{equation}
\mu_i = g_\alpha \left(\frac{g_\beta}{g_\alpha} \right)^{\frac{(lm_i - l_\alpha)}{(l_\beta - l_\alpha)}}
\end{equation}

\subsubsection{The Basic model}\index{Growth Increment!Basic}\label{sec:GrowthIncrement-Basic}

\begin{equation}
\mu_i = g_\alpha + (g_\beta - g_\alpha) \frac{lm_i - l_\alpha}{l_\beta - l_\alpha}
\end{equation}

%
%\subsubsection{The Schnute model}\index{Schnute Increment!Exponential}\label{sec:GrowthIncrement-Schnute}
%This is derived from \cite{haist2009multi} and doesn't look to be working as expected. It has estimable parameter \(g_{diff}\), \(g_{\alpha}\) and \(\gamma\). The generalised Schnute model applies grows individuals from time-step \(t\) to time-step \(t+1\) as
%
%\begin{equation}
%l_{t+1} = \left(l_t \exp(-\kappa\Delta t) + \zeta^{\gamma} \left(1 - \exp(-\kappa\Delta t)\right)\right)^{1/\gamma}
%\end{equation}
%%
%where, \(\Delta t\) is the proportion of annual growth between time-steps and other derived parameters are described below. The expected increment becomes
%\[
%\mu_{t} | l_t = l_{t+1} - l_t
%\]
%\begin{align*}
%	\kappa &= -\log\left(1 - \frac{a - b}{l_{\alpha}^{\gamma} - l_{\beta}^{\gamma}} \right)\\
%	\zeta &= \frac{l_{\alpha}^{\gamma} - l_{\alpha}^{\gamma}(b/a)}{1 - (b/a)}\\
%	a &= (l_{\alpha} + g_{\alpha})^{\gamma} - l_{\alpha}^{\gamma}\\	
%	b &= (l_{\beta} + g_{\beta})^{\gamma} - l_{\beta}^{\gamma}\\
%	g_{\beta} & = g_{diff} g_{\alpha}
%\end{align*}
%%
%here, \(l_{\alpha}\) and \(l_{\beta}\) are lengths that correspond to the parameters \(g_{\alpha}\) \(g_{\beta}\) which describe expected length increments for those reference lengths.  

\subsubsection{Growth transition matrix}\index{Growth Increment!Growth Transition Matrix}\label{sec:GrowthIncrement-GrowthMatrix}

The probability of a fish in length bin \(i\) moving into length bin \(j\) is defined by the growth transition matrix. This matrix is defined by length bin midpoints (\(lm_i\)) mean increment \(\mu_i\) a standard deviation and distribution assumption.

All growth increment models require a coefficient of variation (\(cv\)) and minimum standard deviation \(\sigma_{min}\). 

\[
\sigma_i = max(\sigma_{min}, \mu_i cv)
\]

The growth transition matrix in time step \(t\) denoted by \(\mathbf{G^t}\)is defined as follows

\begin{equation}
G^t_{i,j} = 
\begin{cases}
0.0 \ , & \text{ for } j < i \ \text{no negative growth}\\
f(lc_{i + 1},\mu_i, \sigma_i) \ , & \text{ for } i = j \\
1.0 - \sum\limits_{k = 1}^{n_l - 1}G^t_{i,k} \ , & \text{ for } i = n_l \ \& \ \text{plus group} \\
f(lc_{n_l + 1},\mu_i, \sigma_i) - \sum\limits_{k = 1}^{n_l - 1}G^t_{i,k} \ , & \text{ for } i = n_l \ \&  \  \text{no plus group} \\		
\end{cases}
\end{equation}

where, \(f(X,\mu, \sigma)\) is the cumulative density function defined by \subcommand{distribution}. Currently only the Normal distribution is allowed.
