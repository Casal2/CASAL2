
\section{\I{Examples}\label{sec:examples}}

\subsection{\I{An example of a simple model}\label{example1}}
\index{Examples ! Example 1}\index{Examples ! A simple non-spatial model}
This example describes a single species and area model, with recruitment, maturation, natural and fishing mortality, and an annual age increment. The population structure has ages $1-30^{+}$ with a single category.

The default \CNAME\ configuration filename is \texttt{config.csl2}. In this example, \texttt{config.csl2} specifies the files to include to run the \CNAME\ model from the current directory using the !\texttt{\emph{include}} command.

\lstinputlisting[firstline=2,lastline=5]{../../Examples/Simple/config.csl2}

It is recommended to separate the sections of a \CNAME\ model for enhancing readability and error checking, and including the files in a version control system.

The file  \texttt{population.csl2} contains the population information. The model years are from 1975 through 2012, with 3 time steps. The model is initialised over a 120 year period prior to 1975 and applies the following processes

\begin{itemize}
\item A Beverton-Holt recruitment process, recruiting a constant number of individuals to the first age class (i.e., $age=1$).

\item A constant mortality process representing natural mortality$(M)$. This process is repeated in all 3 time steps, so that a proportion of $M$ is applied in each time step.

\item An ageing process, where all individuals are aged by one year, and with a plus group accumulator age class at $age=30$.
\end{itemize}

Following initialisation, the model runs from the years 1975 to 2012 iterating through 3 time steps.

The first time step applies processes of recruitment, and  $\frac{1}{2} M_1 + F + \frac{1}{2} M_1$ processes, where $M_1$ is the proportion of $M$ applied in the first time step. The exploitation process (fishing) is applied in the years 1975 - 2012. Catches are defined in the catches table and attributes for each fishery, such as selectivity and time step they are implemented, are in the fisheries table in the \command{process} block.

The second time step applies an age increment and the remaining natural mortality.

The third time step applies \TODO{complete this}.

The first 28 lines of the main section of the \texttt{population.csl2}:

% Include config file
\lstinputlisting[firstline=1,lastline=28]{../../Examples/Simple/config/population.csl2}

To run the model to verify that the model runs without any syntax errors, use the command \texttt{casal2 -r}. Since \CNAME\ reads in the default filename \texttt{config.csl2}, this filename can be overridden. For example, if the model is in file \texttt{Mymodel.txt}, then this filename would be specified using the \texttt{-c} option, \texttt{casal2 -r -c Mymodel.txt}.

To estimate the parameters defined in the file \texttt{estimation.csl2} (the catchability constant $q$, recruitment $R_0$, and the selectivity parameters $a_{50}$ and $a_{to95}$), use \texttt{casal2 -e}. The output has been redirected to file  \texttt{estimate.log} using the command \texttt{casal2 -e > estimate.log}. Reports for the user-defined reports \texttt{reports.csl2} from the final iteration of the estimation are output to the file \texttt{estimate.log}, and successful convergence is printed to the screen

{\small{\begin{verbatim}
Total elapsed time: 1 second
Completed
\end{verbatim}}}

The main output from the estimation run is summarised in the file \texttt{estimate.log}, and the final MPD parameter values can also be redirected as a separate report, in this case named \texttt{paramaters.out}, using the command \texttt{casal2 -e -o paramaters.out > estimate.log}.

A profile on the $R_0$ parameter can be run, using \texttt{casal2 -p > profile.log}. See the examples folder for the example of the output.




KATH note below, will be useful to copy that document across.
Examples on Input file specification go to the file Input File Specification.odt found in CASAL2/Documentation/Software Development




\subsection{\I{In line declaration}\label{sec:declare}}

In line declarations can help shorten models by by passing \command{} blocks, for example 
{\small{\begin{verbatim}
		@observation chatCPUE
		type biomass
		catchability [q=6.52606e-005]
		time_step one
		categories male+female
		selectivities chatFselMale chatFselFemale
		likelihood lognormal
		years 1992:2001
		time_step_proportion 1.0
		obs 1.50 1.10 0.93 1.33 1.53 0.90 0.68 0.75 0.57 1.23
		error_value 0.35
		
		@estimate 
		parameter catchability[chatTANbiomass.one].q
		type uniform_log
		lower_bound 1e-2
		upper_bound 1
		In line declaration tips
		\end{verbatim}}}

In the above code we are defining and estimating catchabbility without explicitely creating an \command{catchability} block


When you do an inline declaration the new object will be created with the name of the creator's label.$<$index$>$
where index will be the word if it's one-nine and the number if it's 10+, for example
{\small{\begin{verbatim}
@mortality halfm
selectivities [type=constant; c=1]

would create
@selectivity halfm.one
		\end{verbatim}}}

if there were 10 categories all with there own selectivity the $10^th$ selectivity would be labelled;

{\small{\begin{verbatim}
@selectivity halfm.10
\end{verbatim}}}