
\section{\I{Examples}\label{sec:examples}}

\subsection{Input File Specification}
The file format used for \CNAME\ is based on the formats used for CASAL and SPM. It's a standard text file that contains definitions organised into blocks.
\\
Without exception, every object specified in a configuration file is part of a block. At the top level blocks have a one-to-one relationships with components in the system.
\\
Example:
\\
{\small{\begin{verbatim}
@block1 label
parameter value
parameter value_1 value 2

@block2 label
parameter value
table table_name
column_1 column_2
data_1 data_2
data_3 data_4
end_table
\end{verbatim}}}

Some general notes about writing configuration files:
\begin{enumerate}
	\item Whitespace can be used freely. Tabs and spaces are both accepted
	\item A block ends only at the beginning of a new block or end of final configuration file
	\item You can include another configuration file from anywhere
	\item Included files are placed inline, so you can continue a block in a new file
	\item The configuration files support inline declarations of objects
\end{enumerate}

\subsubsection{Keywords And Reserved Characters}
In order to allow efficient creation of input files CASAL2's file format contains special keywords and characters that cannot be used for labels etc.

\paragraph*{\command Block Definitions}
Every new block in the configuration file must start with a block definition character. The reserved character for this is the \command character\\
Example:
{\small{\begin{verbatim}
@block1 <label>
type <type>

@block2 <label>
type <type>
\end{verbatim}}}

\paragraph*{'type' Keyword}
The 'type' keyword is used for declaring the sub-type of a defined block. Any block object that has multiple sub-types will use the type keyword.\\
Example:
{\small{\begin{verbatim}
@block1 <label>
type <sub_type>

@block2 <label>
type <sub_type>
\end{verbatim}}}

\paragraph*{\# (Single-Line Comment)}
Comments are supported in the configuration file in either single-line (to end-of-line) or multi-line\\
Example:
{\small{\begin{verbatim}
@block <label>
type <sub_type> #Descriptive comment
#parameter <value_1> – This whole line is commented out
parameter <value_1> #<value_2>(value_2 is commented out)
\end{verbatim}}}

\paragraph*{$\{ \}$ (Multi-Line Comment)}
Multiple line comments are supported by surrounding the comments in \{ and \}\\
Example:
{\small{\begin{verbatim}
@block <label>
type <sub_type>
parameter <value_1>
parameter <value_1> <value_2>

{ Do not load this process
	@block <label>
	type <sub_type>
	parameter <value_1>
	parameter <value_1> <value_2>
}
\end{verbatim}}}

\paragraph*{':' (Range Specifier)}
The range specifier allows you to specify a range of values at once instead of having to input them manually. Ranges can be either incremental or decremental.\\
Example:
{\small{\begin{verbatim}
@process my_recruitment_process
type constant_recruitment
years_to_run 1999:2009 #With range specifier

@process my_mortality_process
type natural_mortality
years_to_run 2000 2001 2002 2003 2004 2005 2006 2007 #Without range specifier
\end{verbatim}}}

\paragraph*{',' (List Specifier)}
When a parameter supports multiple values in a single entry you can use the list specifier to supply multiple values as a single parameter.\\
Example:
{\small{\begin{verbatim}
@categories
format sex.stage
names male,female.immature,mature #With list specifier

@categories
format sex.stage
names male.immature male.mature female.immature female.mature #Without list specifier
\end{verbatim}}}

\paragraph*{'table' and 'end\_table' Keyword}
The table keyword is used to define a table of information used as a parameter. The line following the table declaration must contain a list of columns to be used. Following lines are rows of the table. Each row must have the same number of values as the number of columns specified.
The table definition must end with the 'end\_table' keyword on it's own line.
The first row of a table will be the name of the columns if required.\\
Example:
{\small{\begin{verbatim}
@block <label>
type <sub_type>
parameter <value_1>
table <table_label>
<column_1> <column_2> <column_n>
<row1_value1> <row1_value2> <row1_valueN>
<row2_value1> <row2_value2> <row2_valueN>
end_table
\end{verbatim}}}

\paragraph*{[ ] (Inline Declarations)}
When an object takes the label of a target object as a parameter this can be replaced with an inline declaration. An inline declaration is a complete declaration of an object one 1 line. This is designed to allow the configuration writer to simplify the configuration writing process.\\
Example:
{\small{\begin{verbatim}
#With inline declaration with label specified for time step
@model
time_steps step_one=[type=iterative; processes=recruitment ageing]

#With inline declaration with default label (model.1)
@model
time_steps [type=iterative; processes=recruitment ageing]

#Without inline declaration
@model
time_steps step_one

@time_step step_one
processes recruitment ageing
\end{verbatim}}}

\paragraph*{Categories}
The \CNAME\ model is essentially a 2-dimensional model. The model partition is:\\
Categories x Ages/Lengths.
\\\\
Each category supports the ability to have a different range of ages/lengths and accessibility during different time periods.
\\\\
Because each category is quite complicated the syntax for defining categories has been structured to allow complex definitions using a simple short-hand structure.
\\\\
The "format" parameter allows you to tell the model the structure of the category labels. By using a "." (period) character between each segment we can utilise this later in the model to do short-hand lookups of categories.

The "names" parameter is a list of the category names. The syntax of these names will need to match the "format" parameter so \CNAME\ can organise and search on them. Using the "list specifier" and range characters we can shorten this parameter significantly.\\
Example:
{\small{\begin{verbatim}
@categories
format sex.stage.tag
names male.immature.notag male.immature.2001 male.mature.notag male.mature.2001

names male.immature #Invalid: No tag information
names female #Invalid: no stage of tag information
names female.immature.notag.1 #Invalid: Extra format segment not defined

names male,female.immature,mature.notag,2001:2005 #OK!
#Without short-hand. You'd have to write:
names male.immature.notag male.immature.2001 male.immature.2002 male.immature.2003 male.immature.2004 male.immature.2005 male.mature.notag male.mature.2001 male.mature.2002 male.mature.2003 male.mature.2004 male.mature.2005 female.immature.notag female.immature.2001 female.immature.2002 female.immature.2003 female.immature.2004 female.immature.2005 female.mature.notag female.mature.2001 female.mature.2002 female.mature.2003 female.mature.2004 female.mature.2005
\end{verbatim}}}

When we have specific data for a year in a category we don't want the model to process this category during other years (or the initialisation stages). We can define a list of years where each category will be available, this will override the default of all years in the model. Any category where you overwrite the default will no longer be accessible in the initialisation phases.
\\
Examples:
{\small{\begin{verbatim}
@model
start_year 1998
final_year 2010

@categories
format sex.stage.tag
names male,female.immature,mature.notag,2001:2005 #OK!
years tag=2001=1999:2003  tag=2005=2003:2007
# Categories with the tag value “2001” will be available during years 1999, 2000, 2001, 2002 and 2003
# Categories with the tag value “2005” will be available during the years 2003, 2004, 2005, 2006, 2007
\end{verbatim}}}


\subsection{\I{An example of a simple model}\label{example1}}
\index{Examples ! Example 1}\index{Examples ! A simple non-spatial model}
This example describes a single species and area model, with recruitment, maturation, natural and fishing mortality, and an annual age increment. The population structure has ages $1-30^{+}$ with a single category.

The default \CNAME\ configuration filename is \texttt{config.csl2}. In this example, \texttt{config.csl2} specifies the files to include to run the \CNAME\ model from the current directory using the !\texttt{\emph{include}} command.

\lstinputlisting[firstline=2,lastline=5]{../../Examples/Simple/config.csl2}

It is recommended to separate the sections of a \CNAME\ model for enhancing readability and error checking, and including the files in a version control system.

The file  \texttt{population.csl2} contains the population information. The model years are from 1975 through 2012, with 3 time steps. The model is initialised over a 120 year period prior to 1975 and applies the following processes

\begin{itemize}
\item A Beverton-Holt recruitment process, recruiting a constant number of individuals to the first age class (i.e., $age=1$).

\item A constant mortality process representing natural mortality$(M)$. This process is repeated in all 3 time steps, so that a proportion of $M$ is applied in each time step.

\item An ageing process, where all individuals are aged by one year, and with a plus group accumulator age class at $age=30$.
\end{itemize}

Following initialisation, the model runs from the years 1975 to 2012 iterating through 3 time steps.

The first time step applies processes of recruitment, and  $\frac{1}{2} M_1 + F + \frac{1}{2} M_1$ processes, where $M_1$ is the proportion of $M$ applied in the first time step. The exploitation process (fishing) is applied in the years 1975 - 2012. Catches are defined in the catches table and attributes for each fishery, such as selectivity and time step they are implemented, are in the fisheries table in the \command{process} block.

The second time step applies an age increment and the remaining natural mortality.

The third time step applies \TODO{complete this}.

The first 28 lines of the main section of the \texttt{population.csl2}:

% Include config file
\lstinputlisting[firstline=1,lastline=28]{../../Examples/Simple/config/population.csl2}

To run the model to verify that the model runs without any syntax errors, use the command \texttt{casal2 -r}. Since \CNAME\ reads in the default filename \texttt{config.csl2}, this filename can be overridden. For example, if the model is in file \texttt{Mymodel.txt}, then this filename would be specified using the \texttt{-c} option, \texttt{casal2 -r -c Mymodel.txt}.

To estimate the parameters defined in the file \texttt{estimation.csl2} (the catchability constant $q$, recruitment $R_0$, and the selectivity parameters $a_{50}$ and $a_{to95}$), use \texttt{casal2 -e}. The output has been redirected to file  \texttt{estimate.log} using the command \texttt{casal2 -e > estimate.log}. Reports for the user-defined reports \texttt{reports.csl2} from the final iteration of the estimation are output to the file \texttt{estimate.log}, and successful convergence is printed to the screen

{\small{\begin{verbatim}
Total elapsed time: 1 second
Completed
\end{verbatim}}}

The main output from the estimation run is summarised in the file \texttt{estimate.log}, and the final MPD parameter values can also be redirected as a separate report, in this case named \texttt{paramaters.out}, using the command \texttt{casal2 -e -o paramaters.out > estimate.log}.

A profile on the $R_0$ parameter can be run, using \texttt{casal2 -p > profile.log}. See the examples folder for the example of the output.





\subsection{More examples of shorthand syntax and use of \CNAME's reserved and key characters}\label{sec:ShorthandSyntax-section}
\paragraph*{\I{Categories}\label{sub:categories}}
\CNAME\ allows many user defined categories so shorthand syntax has been added to aid in the readability of complex configuration scripts and partition structures. For example when defining categories you can use a comma for shortening lists of categories. The following syntax is how we would specify the categories the long way.
{\small{\begin{verbatim}
		@categories 
		format sex.stage
		names male.immature male.mature female.immature female.mature		
		\end{verbatim}}}	

for the exact same partition structure but specified in a shorter way users could define the categories as, (note the use of the list character ','),

{\small{\begin{verbatim}
		@categories 
		format sex.stage
		names male,female.immature,mature	
		\end{verbatim}}}

\CNAME\ asks for categories in processes and observations so that it can apply the right model dynamics to the right elements of the partition. For the same reason as defining categories shorthand syntax aids in readability and input management. An example of a process where categories need to be supplied as an input command is in ageing,

{\small{\begin{verbatim}
		# 1. The standard way
		@ageing my_ageing
		categories male.immature male.mature female.immature female.mature
		
		# 2. The 1st short-hand way
		@ageing my_ageing
		categories male,female.immature,mature
		
		# 3. Wild Card (all categories)
		@ageing my_ageing
		categories *
		
		# 4. The 2nd short-hand way
		@ageing my_ageing
		categories sex=male sex=female
		\end{verbatim}}}

Sometimes in observations we want to amalgamate categories together for example if we had a biomass estimate of the population that was made up of both males and females in the population you can specify this using the + special character, for example

{\small{\begin{verbatim}
		@observation CPUE
		type biomass
		catchability Fishq
		time_step one
		categories male+female
		selectivities FishSel
		likelihood lognormal
		years 1992:2001
		time_step_proportion 1.0
		obs 1.50 1.10 0.93 1.33 1.53 0.90 0.68 0.75 0.57 1.23
		error_value 0.35
		\end{verbatim}}}

Shorthand syntax can be useful when applying processes to a select group of categories from the partition, for example. If we wanted to apply a spawning migration to the mature categories in the partition and the partition was defined by the categories below,
{\small{\begin{verbatim}
		@categories 
		format area.maturity.tag
		names north.immature.notag,2011 north.mature.notag,2011 south.immature.notag,2011
		south.mature.notag,2011		
		\end{verbatim}}}	

If we wanted to migrate a portion of the mature population from the southern area to the northern are you could use the following syntax,

{\small{\begin{verbatim}
		@process spawn_migration
		type transition_category 		
		from format=south.mature.*	
		to format=north.mature.*
		proportions 1.0
		selectivities One
		\end{verbatim}}}


\paragraph*{\I{Parameters}\label{sec:params}}
\CNAME\ also allows parameters that are of type vector or map to be referenced and estimated partially. An example of a parameter that is type vector is \texttt{ycs\_values} in a recruitment process. Let say a recruitment block was specified as follows,
{\small{\begin{verbatim}
		@process WestRecruitment
		type recruitment_beverton_holt
		r0 400000
		years
		ycs_values 1 1 1 1 1 1 1 1 
		
		An alternative specification to the sequence of values you can use an astrix to
		shorthand repeating integers e.g.
		
		ycs_values 1*8
		
		steepness 0.9
		age 1
		\end{verbatim}}}

Lets say we wanted to only estimate the last four values of the parameter \texttt{process[WestRecruitment].ycs\_values} vector. This can be done as specified in the following \command{estimate} block,

{\small{\begin{verbatim}
		@estimate
		parameter process[WestRecruitment].ycs_values(5:8)
		type uniform
		lower_bound 0.1 0.1 0.1 0.1
		upper_bound  10  10  10  10 
		\end{verbatim}}}

Note the first element of a vector is indexed by 1. This syntax can be applied to parameters that are of type map as well, for information on what type a parameter is see the syntax section. An example of a parameter that is of type map is \command{time\_varying}\texttt{[label].type=constant}. For the following \command{time\_varying} block,

{\small{\begin{verbatim}
		@time_varying q_step1
		type constant
		parameter catchability[Fishq].q
		years 	1992	1993	1994	1995
		value 	0.2		0.2		0.2		0.2	
		\end{verbatim}}}

In this example a user may want to estimate only one element of the map (say 1992), but force all other years to be the same as the one estimate. This can be done in an estimate block as follows,
{\small{\begin{verbatim}
		@estimate
		parameter time_varying[q_step1].value(1992)
		same time_varying[q_step1].value(1993:1995)
		type uniform
		lower_bound 0.1 0.1 0.1 0.1
		upper_bound  10  10  10  10 
		\end{verbatim}}}
\paragraph*{\I{In line declaration}\label{sec:declare}}
In line declarations can help shorten models by passing \command{} blocks, for example 
{\small{\begin{verbatim}
		@observation chatCPUE
		type biomass
		catchability [q=6.52606e-005]
		time_step one
		categories male+female
		selectivities chatFselMale chatFselFemale
		likelihood lognormal
		years 1992:2001
		time_step_proportion 1.0
		obs 1.50 1.10 0.93 1.33 1.53 0.90 0.68 0.75 0.57 1.23
		error_value 0.35
		
		@estimate 
		parameter catchability[chatTANbiomass.one].q
		type uniform_log
		lower_bound 1e-2
		upper_bound 1
		In line declaration tips
		\end{verbatim}}}

In the above code we are defining and estimating catchability without explicitly creating an \command{catchability} block


When you do an inline declaration the new object will be created with the name of the creator's \texttt{label.index}
where index will be the word if it's one-nine and the number if it's 10+, for example
{\small{\begin{verbatim}
		@mortality halfm
		selectivities [type=constant; c=1]
		
		would create
		@selectivity halfm.one
		\end{verbatim}}}

if there were 10 categories all with there own selectivity the $10^th$ selectivity would be labelled;

{\small{\begin{verbatim}
		@selectivity halfm.10
		\end{verbatim}}}
