
\section{\I{Syntax conventions, examples and niceties}\label{sec:examples}}

\subsection{Input File Specification}

The file format used for \CNAME\ is based on the command block formats used in CASAL and SPM. It is a text file that contains definitions organised into blocks.

Every object specified in a configuration file is part of a block. At the top level blocks have a one-to-one relationships with components in the system.

Example:

{\small{\begin{verbatim}
@block1 label
parameter value
parameter value_1 value 2

@block2 label
parameter value
table table_name
column_1 column_2
data_1 data_2
data_3 data_4
end_table
\end{verbatim}}}

Some general notes about configuration files:

\begin{itemize}
	\item Whitespace can be used freely. Tabs and spaces are both valid.
	\item A block ends only at the beginning of a new block or at the end of the final configuration file.
	\item Configuration files can include other configuration files.
	\item Included files are placed in-line, so a block can be continued in a new file.
	\item The configuration files support in-line declarations of objects.
\end{itemize}

\subsubsection{Keywords And Reserved Characters}

In order to allow efficient creation of input files, the \CNAME\ file format has special keywords and characters that cannot be used for labels.

\paragraph*{\command Block Definitions}

Each block in the configuration file must start with the block definition character, which is the "\command" character.

Example:

{\small{\begin{verbatim}
@block1 <label>
type <type>

@block2 <label>
type <type>
\end{verbatim}}}

\paragraph*{The 'type' Keyword}

The 'type' keyword is used for declaring the sub-type of a defined block. Any block object that has multiple sub-types will use the \texttt{type} keyword.

Example:

{\small{\begin{verbatim}
@block1 <label>
type <sub_type>

@block2 <label>
type <sub_type>
\end{verbatim}}}

\paragraph*{\# (Single Line Comments)}

Comments are supported in the configuration file on one line (to the end of that line) or over multiple lines.  Comments on single lines start with the "\#" character.

Example:

{\small{\begin{verbatim}
@block <label>
type <sub_type> # Descriptive comment
# parameter <value_1> *** This whole line is commented out
parameter <value_1> # <value_2> *** value_2 is commented out
\end{verbatim}}}

\paragraph*{\commentstart\ \commentend\ (Multiple Line Comments)}

Multiple line comments are supported by surrounding the comments in \commentstart\ and \commentend

Example:

{\small{\begin{verbatim}
@block <label>
type <sub_type>
parameter <value_1>
parameter <value_1> <value_2>

/*
	Do not load this process
	@block <label>
	type <sub_type>
	parameter <value_1>
	parameter <value_1> <value_2>
*/
\end{verbatim}}}

\paragraph*{$\{ \}$ (Indexing Parameters)}

Individual elements of a vector can be referenced using the \texttt{\{ \}} syntax. For example, when estimating \subcommand{ycs\_values} a range or block of YCS values can be referenced.

Example:

{\small{\begin{verbatim}
		@estimate YCS
		parameter process[Recruitment].ycs_values{1975:2012}
		type uniform
		lower_bound
		upper_bound
		\end{verbatim}}}

\paragraph*{':' (Range Specifier)}

The range specifier "\texttt{:}" allows specifying a range of values instead of specifying each value explicitly. Ranges can be either incremental or decremental.

Example:

{\small{\begin{verbatim}
@process my_recruitment_process
type constant_recruitment
# With the range specifier
years_to_run 1999:2009

@process my_mortality_process
type natural_mortality
# Without the range specifier
years_to_run 2000 2001 2002 2003 2004 2005 2006 2007
\end{verbatim}}}

\paragraph*{',' (List Specifier)}

When a parameter supports multiple values in a single entry, the list specifier "\texttt{,}" can be used to define multiple values as a single parameter.

Example:

{\small{\begin{verbatim}
@categories
format sex.stage
# With the list specifier
names male,female.immature,mature

@categories
format sex.stage
# Without the list specifier
names male.immature male.mature female.immature female.mature
\end{verbatim}}}

\paragraph*{'table' and 'end\_table' Keyword}

The table keyword \texttt{table} is used to define a tabular block of values used as a parameter. The first line following the \texttt{table} declaration must contain a list of columns to be used. The subsequent lines are rows of the table. Each row must have the same number of values as the number of columns specified. The table definition must end with the "\texttt{end\_table}" keyword on its own line.

Example:

{\small{\begin{verbatim}
@block <label>
type <sub_type>
parameter <value_1>
table <table_label>
<column_label_1> <column_label_2> ...  <column_label_N>
<row1_value_1>   <row1_value_2>   ...  <row1_value_N>
<row2_value_1>   <row2_value_2>   ...  <row2_value_N>
end_table
\end{verbatim}}}

\paragraph*{$[\ ]$ (in-line Declarations)}

When an object takes the label of a target object as a parameter, the label can be replaced with an in-line declaration. An in-line declaration "\texttt{$[\ ]$}" is a complete declaration of an object on one line. This feature is designed to allow simplifying the configuration definition.

Example:

{\small{\begin{verbatim}
@model
# With in-line declaration with label specified for time step
time_steps step_one=[type=iterative; processes=recruitment ageing]

@model
# With in-line declaration with default label (model.1)
time_steps [type=iterative; processes=recruitment ageing]


# Without in-line declaration
@model
time_steps step_one

@time_step step_one
processes recruitment ageing
\end{verbatim}}}

\paragraph*{Categories}

The \CNAME\ population representation is essentially a 2-dimensional structure. The partition is:

\textbf{Categories x Ages or Lengths}

Each category allows for a different range of ages or lengths and accessibility during different time periods.

Because each category can be quite complicated, the syntax for defining categories has been structured to allow for complex definitions using a simple shorthand structure.

The "format" parameter allows for defining the structure of the category labels. Using a "." (period) character between each segment allows for shorthand lookups of categories.

The "names" parameter is a list of the category names. The syntax of these names is required to match the "format" parameter so \CNAME\ can organise and search on them. Using the "list specifier" and range characters this parameter can be shortened.

Example:

{\small{\begin{verbatim}
@categories
format sex.stage.tag
names male.immature.notag male.immature.2001 male.mature.notag male.mature.2001

names male.immature # Invalid: No tag information
names female # Invalid: no stage of tag information
names female.immature.notag.1 # Invalid: Additional format segment not defined

names male,female.immature,mature.notag,2001:2005 # Valid
# Without the shorthand syntax these categories would be written:
names male.immature.notag male.immature.2001 male.immature.2002 male.immature.2003 male.immature.2004 male.immature.2005 male.mature.notag male.mature.2001 male.mature.2002 male.mature.2003 male.mature.2004 male.mature.2005 female.immature.notag female.immature.2001 female.immature.2002 female.immature.2003 female.immature.2004 female.immature.2005 female.mature.notag female.mature.2001 female.mature.2002 female.mature.2003 female.mature.2004 female.mature.2005
\end{verbatim}}}

Specific data for a year in a category can be set up so that this category is not to be processed during specific years or in the initialisation phases. A list of years can be provided for each category to indicate in which year(s) it is to be available. These years which will override the default setting of all years in the model. Any category with the default years overridden will no longer be accessible in the initialisation phases.

Examples:

{\small{\begin{verbatim}
@model
start_year 1998
final_year 2010

@categories
format sex.stage.tag
names male,female.immature,mature.notag,2001:2005 # Valid
# Categories with the tag value "2001" are available in years 1999, 2000, 2001, 2002, 2003
# Categories with the tag value "2005" are available in years 2003, 2004, 2005, 2006, 2007
years tag=2001=1999:2003  tag=2005=2003:2007
\end{verbatim}}}

\subsection{More examples of shorthand syntax and use of reserved and key characters}\label{sec:ShorthandSyntax-section}

\paragraph*{\I{Categories}\label{sub:categories}}

\CNAME\ allows for many user-defined categories so shorthand syntax has been added to aid in the definition of complex configuration labelling and partition structures. For example, when defining categories a comma "\texttt{,}" can be used to shorten lists of categories.

This syntax is the long way:

{\small{\begin{verbatim}
		@categories
		format sex.stage
		names male.immature male.mature female.immature female.mature
		\end{verbatim}}}

For the exact same partition structure specified in a shorter way:

{\small{\begin{verbatim}
		@categories
		format sex.stage
		names male,female.immature,mature
		\end{verbatim}}}

\CNAME\ requires categories in processes and observations so that the correct model dynamics can be applied to the correct elements of the partition.

An example of a process where categories are required as an input command is for ageing

{\small{\begin{verbatim}
		# 1. The standard way
		@ageing my_ageing
		categories male.immature male.mature female.immature female.mature

		# 2. The first shorthand way
		@ageing my_ageing
		categories male,female.immature,mature

		# 3. Wild Card (all categories)
		@ageing my_ageing
		categories *

		# 4. The second shorthand way
		@ageing my_ageing
		categories sex=male sex=female
		\end{verbatim}}}

To combine/aggregate categories together, use the "\texttt{+}" special character. For example, this feature can be used to specify that the total biomass of the population is made up of both males and females.

For example,

{\small{\begin{verbatim}
		@observation CPUE
		type biomass
		catchability Fishq
		time_step one
		categories male+female
		selectivities FishSel
		likelihood lognormal
		time_step_proportion 1.0
		years 1992:2001
		table obs
		1992    1.50    0.35
		1993    1.10    0.35
		1994    0.93    0.35
		1995    1.33    0.35
		1996    1.53    0.35
		1997    0.90    0.35
		1998    0.68    0.35
		1999    0.75    0.35
		2000    0.57    0.35
		2001    1.23    0.35
		end_table
		\end{verbatim}}}

This combination/aggregation functionality can be used to compare an observation to the total combined population:

{\small{\begin{verbatim}
		@observation CPUE
		type biomass
		catchability Fishq
		time_step one
		categories *+
		selectivities FishSel
		likelihood lognormal
		time_step_proportion 1.0
		years 1992:2001
		table obs
		1992    1.50    0.35
		1993    1.10    0.35
		1994    0.93    0.35
		1995    1.33    0.35
		1996    1.53    0.35
		1997    0.90    0.35
		1998    0.68    0.35
		1999    0.75    0.35
		2000    0.57    0.35
		2001    1.23    0.35
		end_table
		\end{verbatim}}}

If \subcommand{male} and \subcommand{female} are the only categories in a population, then this is the same syntax as the command block above it.

Shorthand syntax can be useful when applying processes to a select group of categories from the partition.

For example, to apply a spawning migration to the mature categories in the partition and the partition was defined:

{\small{\begin{verbatim}
		@categories
		format area.maturity.tag
		names north.immature.notag,2011 north.mature.notag,2011 south.immature.notag,2011
		south.mature.notag,2011
		\end{verbatim}}}

Then, to migrate a portion of the mature population from the southern area to the northern area:

{\small{\begin{verbatim}
		@process spawn_migration
		type transition_category
		from format=south.mature.*
		to format=north.mature.*
		proportions 1.0
		selectivities One
		\end{verbatim}}}


\paragraph*{\I{Parameters}\label{sec:params}}

\CNAME\ also allows parameters that are of type vector or map to be referenced and estimated fully or partially. An example of a parameter that is type vector is \texttt{ycs\_values} in a recruitment process.

For example, a recruitment block:

{\small{\begin{verbatim}
		@process WestRecruitment
		type recruitment_beverton_holt
		r0 400000
		years
		ycs_values 1 1 1 1 1 1 1 1
		ycs_years 1975:1983
		# An alternative method to specify a sequence of values
		# use an asterix to represent a vector of repeating integers
		ycs_values 1*8
		steepness 0.9
		age 1
		\end{verbatim}}}

To estimate the last four years of the parameter \texttt{process[WestRecruitment].ycs\_values} only can be specified as

{\small{\begin{verbatim}
		@estimate
		parameter process[WestRecruitment].ycs_values{1980:1983}
		type uniform
		lower_bound 0.1 0.1 0.1 0.1
		upper_bound  10  10  10  10
		\end{verbatim}}}

Note that the first element of a vector is indexed by 1. This syntax can be applied to parameters that are of type map as well. For information on what type a parameter is see the syntax section.

An example of a parameter that is of type map is \command{time\_varying}\texttt{[label].type=constant}.

For a \command{time\_varying} block

{\small{\begin{verbatim}
		@time_varying q_step1
		type constant
		parameter catchability[Fishq].q
		years 	1992	1993	1994	1995
		value 	0.2		0.2		0.2		0.2
		\end{verbatim}}}

For example, to estimate only one element of the map (say 1992), and force all other years to be the same as the one estimate, can be done in the \command{estimate} block using \texttt{same}:

{\small{\begin{verbatim}
		@estimate
		parameter time_varying[q_step1].value{1992}
		same time_varying[q_step1].value{1993:1995}
		type uniform
		lower_bound 0.1 0.1 0.1 0.1
		upper_bound  10  10  10  10
		\end{verbatim}}}
	
\paragraph*{\I{In-line declaration}\label{sec:declare}}

In-line declarations can help shorten models by passing \command{} blocks.

For example,

{\small{\begin{verbatim}
		@observation chatCPUE
		type biomass
		catchability [q=6.52606e-005]
		time_step one
		categories male+female
		selectivities chatFselMale chatFselFemale
		likelihood lognormal
		time_step_proportion 1.0
		years 1992:2001
		table obs
		1992    1.50    0.35
		1993    1.10    0.35
		1994    0.93    0.35
		1995    1.33    0.35
		1996    1.53    0.35
		1997    0.90    0.35
		1998    0.68    0.35
		1999    0.75    0.35
		2000    0.57    0.35
		2001    1.23    0.35
		end_table

		@estimate
		parameter catchability[chatTANbiomass.one].q
		type uniform_log
		lower_bound 1e-2
		upper_bound 1
		\end{verbatim}}}

In the above code catchability is defined and estimated without explicitly creating a \command{catchability} block.

When an in-line declaration is made, the new object will be created with the name of the creator's \texttt{label.index}, where \texttt{index} is the word "\texttt{one}" through "\texttt{nine}" if it is 1 through 9, and the number if it is 10+.

For example,

{\small{\begin{verbatim}
		@mortality halfm
		selectivities [type=constant; c=1]

		would create
		@selectivity halfm.one
		\end{verbatim}}}

If there are 10 categories, each with its own selectivity, the $10^{th}$ selectivity is labelled

{\small{\begin{verbatim}
		@selectivity halfm.10
		\end{verbatim}}}

\subsection{Processes}

Processes are special in how they can be defined. Throughout this document specifying a process has been

{\small{\begin{verbatim}
		@process Recruitment
		type recruitment_beverton_holt
		\end{verbatim}}}
	
However, for convenience and clarity, this block can also be specified as

{\small{\begin{verbatim}
		@recruitment Recruitment
		type beverton_holt
		\end{verbatim}}}

The difference is that the keyword \subcommand{process} can be replaced with the first word of the process type. In the example above this is the \subcommand{recruitment} process. This option can be used to create more succinct model configurations.

More examples:

{\small{\begin{verbatim}
		@mortality Fishing_and_M
		type instantaneous
		\end{verbatim}}}

{\small{\begin{verbatim}
		@transition Migration
		type category
		\end{verbatim}}}

\subsection{An example of a simple model\label{example1}}\index{A simple non-spatial model example}

This example describes a single species and area model, with recruitment, maturation, natural and fishing mortality, and an annual age increment. The population structure has ages $1-30^{+}$ with a single category.

The default \CNAME\ configuration filename is \texttt{config.csl2}. In this example, \texttt{config.csl2} specifies the files to include to run the \CNAME\ model from the current directory using the !\texttt{\emph{include}} command.

\lstinputlisting[firstline=2,lastline=5]{../../Examples/Simple/config.csl2}

It is recommended to separate the sections of a \CNAME\ model for enhancing readability and error checking, and including the files in a version control system.

The file  \texttt{population.csl2} contains the population information. The model years are from 1975 through 2012, with 3 time steps. The model is initialised over a 120 year period prior to 1975 and applies the following processes

\begin{itemize}
\item A Beverton-Holt recruitment process, recruiting a constant number of individuals to the first age class (i.e., $age=1$).

\item A constant mortality process representing natural mortality$(M)$. This process is repeated in all 3 time steps, so that a proportion of $M$ is applied in each time step.

\item An ageing process, where all individuals are aged by one year, and with a plus group accumulator age class at $age=30$.
\end{itemize}

Following initialisation, the model runs from the years 1975 to 2012 iterating through 3 time steps.

The first time step applies processes of recruitment, and  $\frac{1}{2} M_1 + F + \frac{1}{2} M_1$ processes, where $M_1$ is the proportion of $M$ applied in the first time step. The exploitation process (fishing) is applied in the years 1975 - 2012. Catches are defined in the catches table and attributes for each fishery, such as selectivity and time step they are implemented, are in the fisheries table in the \command{process} block.

The second time step applies an age increment and the remaining natural mortality.

The third time step applies \TODO{complete this}.

The first 28 lines of the main section of the \texttt{population.csl2}:

% Include config file
\lstinputlisting[firstline=1,lastline=28]{../../Examples/Simple/config/population.csl2}

To run the model to verify that the model runs without any syntax errors, use the command \texttt{casal2 -r}. Since \CNAME\ reads in the default filename \texttt{config.csl2}, this filename can be overridden. For example, if the model is in file \texttt{Mymodel.txt}, then this filename would be specified using the \texttt{-c} option, \texttt{casal2 -r -c Mymodel.txt}.

To estimate the parameters defined in the file \texttt{estimation.csl2} (the catchability constant $q$, recruitment $R_0$, and the selectivity parameters $a_{50}$ and $a_{to95}$), use \texttt{casal2 -e}. The output has been redirected to file  \texttt{estimate.log} using the command \texttt{casal2 -e > estimate.log}. Reports for the user-defined reports \texttt{reports.csl2} from the final iteration of the estimation are output to the file \texttt{estimate.log}, and successful convergence is printed to the screen

{\small{\begin{verbatim}
Total elapsed time: 1 second
Completed
\end{verbatim}}}

The main output from the estimation run is summarised in the file \texttt{estimate.log}, and the final MPD parameter values can also be redirected as a separate report, in this case named \texttt{paramaters.out}, using the command \texttt{casal2 -e -o paramaters.out > estimate.log}.

A profile on the $R_0$ parameter can be run, using \texttt{casal2 -p > profile.log}. See the examples folder for the example of the output.



%\subsection{An example of a simple spatial model\label{example1}}\index{A simple spatial model example}

%\input{Examples/Example2}

