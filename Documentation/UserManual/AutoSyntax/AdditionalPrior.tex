\defComLab{Additional\_Prior}{Define an object of type \emph{Additional\_Prior}}.
\defRef{sec:AdditionalPrior}
\label{syntax:AdditionalPrior}

\defSub{label}{The label for the additional prior}
\defType{String}
\defDefault{No default}

\defSub{parameter}{The name of the parameter for the additional prior}
\defType{String}
\defDefault{No default}

\defSub{type}{The additional prior type}
\defType{String}
\defDefault{No default}

\subsubsection{Additional\_Prior of type Beta}
\commandlabsubarg{Additional\_Prior}{type}{Beta}.
\defRef{sec:AdditionalPrior-Beta}
\label{syntax:AdditionalPrior-Beta}

\defSub{mu}{Beta distribution mean (mu) parameter}
\defType{Real number (estimable)}
\defDefault{No default}

\defSub{sigma}{Beta distribution variance (sigma) parameter}
\defType{Real number (estimable)}
\defDefault{No default}
\defLowerBound{0.0 (exclusive)}

\defSub{a}{Beta distribution lower bound, of the range (A) parameter}
\defType{Real number (estimable)}
\defDefault{No default}

\defSub{b}{Beta distribution upper bound of the range (B) parameter}
\defType{Real number (estimable)}
\defDefault{No default}

\subsubsection{Additional\_Prior of type Element\_Difference}
\commandlabsubarg{Additional\_Prior}{type}{Element\_Difference}.
\defRef{sec:AdditionalPrior-ElementDifference}
\label{syntax:AdditionalPrior-ElementDifference}

\defSub{second\_parameter}{The name of the second parameter for comparing}
\defType{String}
\defDefault{No default}

\defSub{multiplier}{Multiply the penalty by this factor}
\defType{Real number (estimable)}
\defDefault{1}

\subsubsection{Additional\_Prior of type Log\_Normal}
\commandlabsubarg{Additional\_Prior}{type}{Log\_Normal}.
\defRef{sec:AdditionalPrior-LogNormal}
\label{syntax:AdditionalPrior-LogNormal}

\defSub{mu}{The lognormal prior mean (mu) parameter}
\defType{Real number (estimable)}
\defDefault{No default}
\defLowerBound{0.0 (exclusive)}

\defSub{cv}{The lognormal CV parameter}
\defType{Real number (estimable)}
\defDefault{No default}
\defLowerBound{0.0 (exclusive)}

\subsubsection{Additional\_Prior of type Ratio}
\commandlabsubarg{Additional\_Prior}{type}{Ratio}.
\defRef{sec:AdditionalPrior-Ratio}
\label{syntax:AdditionalPrior-Ratio}

\defSub{second\_parameter}{The name of the parameter on the denominator}
\defType{String}
\defDefault{No default}

\defSub{mu}{The lognormal prior mean (mu) of the ratio}
\defType{Real number (estimable)}
\defDefault{No default}
\defLowerBound{0.0 (exclusive)}

\defSub{cv}{The lognormal CV parameter for the ratio}
\defType{Real number (estimable)}
\defDefault{No default}
\defLowerBound{0.0 (exclusive)}

\subsubsection{Additional\_Prior of type Uniform\_Log}
\commandlabsubarg{Additional\_Prior}{type}{Uniform\_Log}.
\defRef{sec:AdditionalPrior-UniformLog}
\label{syntax:AdditionalPrior-UniformLog}

The Uniform\_Log type has no additional subcommands.
\subsubsection{Additional\_Prior of type Vector\_Average}
\commandlabsubarg{Additional\_Prior}{type}{Vector\_Average}.
\defRef{sec:AdditionalPrior-VectorAverage}
\label{syntax:AdditionalPrior-VectorAverage}

\defSub{method}{Which calculation method to use: k, l, or m}
\defType{String}
\defDefault{k}

\defSub{k}{The k value to use in the calculation}
\defType{Real number (estimable)}
\defDefault{No default}

\defSub{multiplier}{Multiplier for the penalty amount}
\defType{Real number (estimable)}
\defDefault{1}

\subsubsection{Additional\_Prior of type Vector\_Smoothing}
\commandlabsubarg{Additional\_Prior}{type}{Vector\_Smoothing}.
\defRef{sec:AdditionalPrior-VectorSmoothing}
\label{syntax:AdditionalPrior-VectorSmoothing}

\defSub{log\_scale}{Should the sums of squares be calculated on the log scale?}
\defType{Boolean}
\defDefault{false}

\defSub{multiplier}{Multiply the penalty by this factor}
\defType{Real number (estimable)}
\defDefault{1}

\defSub{lower\_bound}{The first element to apply the penalty to in the vector}
\defType{Non-negative integer}
\defDefault{0}

\defSub{upper\_bound}{The last element to apply the penalty to in the vector}
\defType{Non-negative integer}
\defDefault{0}

\defSub{r}{The rth difference that the penalty is applied to}
\defType{Non-negative integer}
\defDefault{2}

