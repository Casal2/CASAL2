\defComLab{Addressable\_Transformation}{Define an object of type \emph{Addressable\_Transformation}}.
\defRef{sec:AddressableTransformation}
\label{syntax:AddressableTransformation}

\defSub{label}{Label of the transformation}
\defType{String}
\defDefault{No default}

\defSub{type}{The type of transformation}
\defType{String}
\defDefault{No default}

\defSub{parameters}{The parameters used in the transformation}
\defType{Vector of strings}
\defDefault{No default}

\defSub{prior\_applies\_to\_restored\_parameters}{If the prior applies to the parameters (true) with jacobian (if it exists) or prior applies to transformed\_parameter (false) with no jacobian}
\defType{Boolean}
\defDefault{false}

\subsubsection{Addressable\_Transformation of type Average\_Difference}
\commandlabsubarg{Addressable\_Transformation}{type}{Average\_Difference}.
\defRef{sec:AddressableTransformation-AverageDifference}
\label{syntax:AddressableTransformation-AverageDifference}

The Average\_Difference type has no additional subcommands.
\subsubsection{Addressable\_Transformation of type Difference}
\commandlabsubarg{Addressable\_Transformation}{type}{Difference}.
\defRef{sec:AddressableTransformation-Difference}
\label{syntax:AddressableTransformation-Difference}

\defSub{difference\_parameter}{The difference between the parameters}
\defType{Real number (estimable)}
\defDefault{false}

\subsubsection{Addressable\_Transformation of type Inverse}
\commandlabsubarg{Addressable\_Transformation}{type}{Inverse}.
\defRef{sec:AddressableTransformation-Inverse}
\label{syntax:AddressableTransformation-Inverse}

The Inverse type has no additional subcommands.
\subsubsection{Addressable\_Transformation of type Log}
\commandlabsubarg{Addressable\_Transformation}{type}{Log}.
\defRef{sec:AddressableTransformation-Log}
\label{syntax:AddressableTransformation-Log}

The Log type has no additional subcommands.
\subsubsection{Addressable\_Transformation of type Log\_Sum}
\commandlabsubarg{Addressable\_Transformation}{type}{Log\_Sum}.
\defRef{sec:AddressableTransformation-LogSum}
\label{syntax:AddressableTransformation-LogSum}

The Log\_Sum type has no additional subcommands.
\subsubsection{Addressable\_Transformation of type Logistic}
\commandlabsubarg{Addressable\_Transformation}{type}{Logistic}.
\defRef{sec:AddressableTransformation-Logistic}
\label{syntax:AddressableTransformation-Logistic}

\defSub{lower\_bound}{Lower bound of parameter space}
\defType{Real number (estimable)}
\defDefault{true}

\defSub{upper\_bound}{Upper bound of parameter space}
\defType{Real number (estimable)}
\defDefault{true}

\subsubsection{Addressable\_Transformation of type Orthogonal}
\commandlabsubarg{Addressable\_Transformation}{type}{Orthogonal}.
\defRef{sec:AddressableTransformation-Orthogonal}
\label{syntax:AddressableTransformation-Orthogonal}

The Orthogonal type has no additional subcommands.
\subsubsection{Addressable\_Transformation of type Simplex}
\commandlabsubarg{Addressable\_Transformation}{type}{Simplex}.
\defRef{sec:AddressableTransformation-Simplex}
\label{syntax:AddressableTransformation-Simplex}

\defSub{sum\_to\_one}{if true, then the parameter vector in natural space will sum to 1, otherwise it will sum to a value equal to length(parameter)}
\defType{Boolean}
\defDefault{true}

\subsubsection{Addressable\_Transformation of type Square\_Root}
\commandlabsubarg{Addressable\_Transformation}{type}{Square\_Root}.
\defRef{sec:AddressableTransformation-SquareRoot}
\label{syntax:AddressableTransformation-SquareRoot}

The Square\_Root type has no additional subcommands.
\subsubsection{Addressable\_Transformation of type Sum\_To\_One}
\commandlabsubarg{Addressable\_Transformation}{type}{Sum\_To\_One}.
\defRef{sec:AddressableTransformation-SumToOne}
\label{syntax:AddressableTransformation-SumToOne}

The Sum\_To\_One type has no additional subcommands.
