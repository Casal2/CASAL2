\defComLab{Process}{Define an object of type \emph{Process}}.
\defRef{sec:Process}
\label{syntax:Process}

\defSub{label}{The label of the process}
\defType{String}
\defDefault{No default}

\defSub{type}{The type of process}
\defType{String}
\defDefault{No default}

\subsubsection{Process of type Ageing}
\commandlabsubarg{Process}{type}{Ageing}.
\defRef{sec:Process-Ageing}
\label{syntax:Process-Ageing}

\defSub{categories}{The labels of the categories to age}
\defType{Vector of strings}
\defDefault{No default}

\subsubsection{Process of type Load\_Partition}
\commandlabsubarg{Process}{type}{Load\_Partition}.
\defRef{sec:Process-LoadPartition}
\label{syntax:Process-LoadPartition}

The Load\_Partition type has no additional subcommands.
\subsubsection{Process of type Markovian\_Movement}
\commandlabsubarg{Process}{type}{Markovian\_Movement}.
\defRef{sec:Process-MarkovianMovement}
\label{syntax:Process-MarkovianMovement}

\defSub{from}{The categories to transition from}
\defType{Vector of strings}
\defDefault{No default}

\defSub{to}{The categories to transition to}
\defType{Vector of strings}
\defDefault{No default}

\defSub{proportions}{The proportions to transition for each category}
\defType{Addressable}
\defDefault{No default}
\defLowerBound{0.0 (inclusive)}
\defUpperBound{1.0 (inclusive)}

\defSub{selectivities}{The selectivities to apply to each proportion}
\defType{Vector of strings}
\defDefault{No default}

\defSub{proportions}{}
\defType{Addressable}
\defDefault{No default}

\subsubsection{Process of type Maturation}
\commandlabsubarg{Process}{type}{Maturation}.
\defRef{sec:Process-Maturation}
\label{syntax:Process-Maturation}

\defSub{from}{The list of categories to mature from}
\defType{Vector of strings}
\defDefault{No default}

\defSub{to}{The list of categories to mature to}
\defType{Vector of strings}
\defDefault{No default}

\defSub{selectivities}{The list of selectivities to use for maturation}
\defType{Vector of strings}
\defDefault{No default}

\defSub{years}{The years to be associated with the maturity rates}
\defType{Vector of non-negative integers}
\defDefault{No default}

\defSub{rates}{The rates to mature for each year}
\defType{Vector of real numbers (estimable)}
\defDefault{No default}

\subsubsection{Process of type Mortality}
\commandlabsubarg{Process}{type}{Mortality}.
\defRef{sec:Process-Mortality}
\label{syntax:Process-Mortality}

The Mortality type has no additional subcommands.
\subsubsection{Process of type Nop}
\commandlabsubarg{Process}{type}{Nop}.
\defRef{sec:Process-Nop}
\label{syntax:Process-Nop}

The Nop type has no additional subcommands.
\subsubsection{Process of type Recruitment\_Beverton\_Holt}
\commandlabsubarg{Process}{type}{Recruitment\_Beverton\_Holt}.
\defRef{sec:Process-RecruitmentBevertonHolt}
\label{syntax:Process-RecruitmentBevertonHolt}

\defSub{categories}{The category labels}
\defType{Vector of strings}
\defDefault{No default}

\defSub{r0}{R0, the mean recruitment used to scale annual recruits or initialise the model}
\defType{Real number (estimable)}
\defDefault{false}
\defLowerBound{0.0 (inclusive)}

\defSub{b0}{B0, the SSB corresponding to R0, and used to scale annual recruits or initialise the model}
\defType{Real number (estimable)}
\defDefault{false}
\defLowerBound{0.0 (inclusive)}

\defSub{proportions}{The proportion for each category}
\defType{Real number (estimable)}
\defDefault{No default}

\defSub{age}{The age at recruitment}
\defType{Non-negative integer}
\defDefault{true}

\defSub{ssb\_offset}{The spawning biomass year offset}
\defType{Non-negative integer}
\defDefault{true}

\defSub{steepness}{Steepness (h)}
\defType{Real number (estimable)}
\defDefault{1.0}
\defLowerBound{0.2 (inclusive)}
\defUpperBound{1.0 (inclusive)}

\defSub{ssb}{The SSB label (i.e., the derived quantity label)}
\defType{String}
\defDefault{No default}

\defSub{b0\_initialisation\_phase}{The initialisation phase label that B0 is from}
\defType{String}
\defDefault{No default}

\defSub{recruitment\_multipliers}{The YCS values}
\defType{Vector of real numbers (estimable)}
\defDefault{No default}

\defSub{standardise\_years}{The years that are included for year class standardisation}
\defType{Vector of non-negative integers}
\defDefault{true}

\defSub{ycs\_values}{The YCS values}
\defType{Vector of real numbers (estimable)}
\defDefault{true}

\defSub{ycs\_years}{The recruitment years. A vector of years that relates to the year of the spawning event that created this cohort}
\defType{Vector of non-negative integers}
\defDefault{true}

\subsubsection{Process of type Recruitment\_Beverton\_Holt\_With\_Deviations}
\commandlabsubarg{Process}{type}{Recruitment\_Beverton\_Holt\_With\_Deviations}.
\defRef{sec:Process-RecruitmentBevertonHoltWithDeviations}
\label{syntax:Process-RecruitmentBevertonHoltWithDeviations}

\defSub{categories}{The category labels}
\defType{Vector of strings}
\defDefault{No default}

\defSub{r0}{R0, the mean recruitment used to scale annual recruits or initialise the model}
\defType{Real number (estimable)}
\defDefault{false}
\defLowerBound{0.0 (inclusive)}

\defSub{b0}{B0, the SSB corresponding to R0, and used to scale annual recruits or initialise the model}
\defType{Real number (estimable)}
\defDefault{false}
\defLowerBound{0.0 (inclusive)}

\defSub{proportions}{The proportion for each category}
\defType{Real number (estimable)}
\defDefault{No default}

\defSub{age}{The age at recruitment}
\defType{Non-negative integer}
\defDefault{true}

\defSub{ssb\_offset}{The spawning biomass year offset}
\defType{Non-negative integer}
\defDefault{true}

\defSub{steepness}{Steepness (h)}
\defType{Real number (estimable)}
\defDefault{1.0}
\defLowerBound{0.2 (inclusive)}
\defUpperBound{1.0 (inclusive)}

\defSub{ssb}{The SSB label (i.e., the derived quantity label))}
\defType{String}
\defDefault{No default}

\defSub{sigma\_r}{The standard deviation of recruitment, sigma\_R}
\defType{Real number (estimable)}
\defDefault{No default}
\defLowerBound{0.0 (inclusive)}

\defSub{b\_max}{The maximum bias adjustment}
\defType{Real number (estimable)}
\defDefault{0.85}
\defLowerBound{0.0 (inclusive)}
\defUpperBound{1.0 (inclusive)}

\defSub{last\_year\_with\_no\_bias}{The last year (recruited year) with no bias adjustment}
\defType{Non-negative integer}
\defDefault{false}

\defSub{first\_year\_with\_bias}{The first year (recruited year) with full bias adjustment}
\defType{Non-negative integer}
\defDefault{false}

\defSub{last\_year\_with\_bias}{The last year (recruited year) with full bias adjustment}
\defType{Non-negative integer}
\defDefault{false}

\defSub{first\_recent\_year\_with\_no\_bias}{The first recent year (recruited year) with no bias adjustment}
\defType{Non-negative integer}
\defDefault{false}

\defSub{b0\_initialisation\_phase}{The initialisation phase label that B0 is from}
\defType{String}
\defDefault{No default}

\defSub{deviation\_values}{The recruitment deviation values}
\defType{Vector of real numbers (estimable)}
\defDefault{No default}

\defSub{deviation\_years}{The recruitment years. A vector of years that relates to the year of the spawning event that created this cohort}
\defType{Vector of non-negative integers}
\defDefault{true}

\subsubsection{Process of type Recruitment\_Constant}
\commandlabsubarg{Process}{type}{Recruitment\_Constant}.
\defRef{sec:Process-RecruitmentConstant}
\label{syntax:Process-RecruitmentConstant}

\defSub{categories}{The categories}
\defType{Vector of strings}
\defDefault{No default}

\defSub{proportions}{The proportion for each category}
\defType{Real number (estimable)}
\defDefault{true}

\defSub{age}{The age at recruitment}
\defType{Non-negative integer}
\defDefault{No default}

\defSub{r0}{R0, the recruitment used for annual recruits and initialise the model}
\defType{Real number (estimable)}
\defDefault{No default}
\defLowerBound{0.0 (inclusive)}

\subsubsection{Process of type Recruitment\_Ricker}
\commandlabsubarg{Process}{type}{Recruitment\_Ricker}.
\defRef{sec:Process-RecruitmentRicker}
\label{syntax:Process-RecruitmentRicker}

\defSub{categories}{The category labels}
\defType{Vector of strings}
\defDefault{No default}

\defSub{r0}{R0, the mean recruitment used to scale annual recruits or initialise the model}
\defType{Real number (estimable)}
\defDefault{false}
\defLowerBound{0.0 (inclusive)}

\defSub{b0}{B0, the SSB corresponding to R0, and used to scale annual recruits or initialise the model}
\defType{Real number (estimable)}
\defDefault{false}
\defLowerBound{0.0 (inclusive)}

\defSub{proportions}{The proportion for each category}
\defType{Real number (estimable)}
\defDefault{No default}

\defSub{age}{The age at recruitment}
\defType{Non-negative integer}
\defDefault{true}

\defSub{ssb\_offset}{The spawning biomass year offset}
\defType{Non-negative integer}
\defDefault{true}

\defSub{steepness}{Steepness (h)}
\defType{Real number (estimable)}
\defDefault{1.0}
\defLowerBound{0.2 (inclusive)}
\defUpperBound{1.0 (inclusive)}

\defSub{ssb}{The SSB label (i.e., the derived quantity label)}
\defType{String}
\defDefault{No default}

\defSub{b0\_initialisation\_phase}{The initialisation phase label that B0 is from}
\defType{String}
\defDefault{No default}

\defSub{recruitment\_multipliers}{The YCS values}
\defType{Vector of real numbers (estimable)}
\defDefault{No default}

\defSub{standardise\_years}{The years that are included for year class standardisation}
\defType{Vector of non-negative integers}
\defDefault{true}

\defSub{ycs\_values}{The YCS values}
\defType{Vector of real numbers (estimable)}
\defDefault{true}

\defSub{ycs\_years}{The recruitment years. A vector of years that relates to the year of the spawning event that created this cohort}
\defType{Vector of non-negative integers}
\defDefault{true}

\subsubsection{Process of type Tag\_By\_Age}
\commandlabsubarg{Process}{type}{Tag\_By\_Age}.
\defRef{sec:Process-TagByAge}
\label{syntax:Process-TagByAge}

\defSub{to}{The categories that are being tagged (i.e., transition to)}
\defType{Vector of strings}
\defDefault{No default}

\defSub{min\_age}{The minimum age to transition}
\defType{Non-negative integer}
\defDefault{model}

\defSub{max\_age}{The maximum age to transition}
\defType{Non-negative integer}
\defDefault{model}

\defSub{penalty}{The penalty label}
\defType{String}
\defDefault{No default}

\defSub{u\_max}{The maximum exploitation rate, U\_max}
\defType{Real number}
\defDefault{0.99}
\defLowerBound{0.0 (inclusive)}
\defUpperBound{1.0 (exclusive)}

\defSub{years}{The years to execute the tagging events in}
\defType{Vector of non-negative integers}
\defDefault{No default}

\defSub{initial\_mortality}{The initial mortality to apply to tags as a proportion}
\defType{Real number (estimable)}
\defDefault{0.0}
\defLowerBound{0.0 (inclusive)}
\defUpperBound{1.0 (inclusive)}

\defSub{initial\_mortality\_selectivity}{The initial mortality selectivity label}
\defType{String}
\defDefault{No default}

\defSub{selectivities}{Selectivity of the tagging process}
\defType{Vector of strings}
\defDefault{No default}

\defSub{n}{The number of individuals tagged}
\defType{Vector of real numbers (estimable)}
\defDefault{true}

\defSub{tolerance}{Tolerance for checking the specificed proportions sum to one}
\defType{Real number}
\defDefault{1e-5}
\defLowerBound{0 (inclusive)}
\defUpperBound{1.0 (inclusive)}

\subsubsection{Process of type Tag\_By\_Length}
\commandlabsubarg{Process}{type}{Tag\_By\_Length}.
\defRef{sec:Process-TagByLength}
\label{syntax:Process-TagByLength}

\defSub{to}{The categories that are being tagged (i.e., transition to)}
\defType{Vector of strings}
\defDefault{No default}

\defSub{penalty}{The penalty label}
\defType{String}
\defDefault{No default}

\defSub{u\_max}{The maximum exploitation rate, U\_max}
\defType{Real number}
\defDefault{0.99}
\defLowerBound{0.0 (inclusive)}
\defUpperBound{1.0 (exclusive)}

\defSub{years}{The years to execute the tagging events in}
\defType{Vector of non-negative integers}
\defDefault{No default}

\defSub{initial\_mortality}{The initial mortality to apply to tags as a proportion}
\defType{Real number (estimable)}
\defDefault{0.0}
\defLowerBound{0.0 (inclusive)}
\defUpperBound{1.0 (inclusive)}

\defSub{initial\_mortality\_selectivity}{The initial mortality selectivity label}
\defType{String}
\defDefault{No default}

\defSub{selectivities}{Selectivity of the tagging process}
\defType{Vector of strings}
\defDefault{No default}

\defSub{n}{The number of individuals tagged}
\defType{Vector of real numbers (estimable)}
\defDefault{true}

\defSub{tolerance}{Tolerance for checking the specificed proportions sum to one}
\defType{Real number}
\defDefault{1e-5}
\defLowerBound{0 (inclusive)}
\defUpperBound{1.0 (inclusive)}

\defSub{compatibility\_option}{Backwards compatibility option: either casal2 (the default) or casal: effects penalty and age-length calculation}
\defType{String}
\defDefault{casal2}

\subsubsection{Process of type Tag\_Loss}
\commandlabsubarg{Process}{type}{Tag\_Loss}.
\defRef{sec:Process-TagLoss}
\label{syntax:Process-TagLoss}

\defSub{categories}{The list of categories to apply}
\defType{Vector of strings}
\defDefault{No default}

\defSub{tag\_loss\_rate}{The tag loss rates}
\defType{Real number (estimable)}
\defDefault{No default}
\defLowerBound{0.0 (inclusive)}

\defSub{time\_step\_proportions}{The time step proportions for tag loss}
\defType{Vector of real numbers (estimable)}
\defDefault{false}
\defLowerBound{0.0 (inclusive)}

\defSub{tag\_loss\_type}{The type of tag loss}
\defType{String}
\defDefault{single}

\defSub{selectivities}{The selectivities}
\defType{Vector of strings}
\defDefault{No default}

\defSub{year}{The year the first tagging release process was executed}
\defType{Non-negative integer}
\defDefault{No default}

\subsubsection{Process of type Tag\_Loss\_Empirical}
\commandlabsubarg{Process}{type}{Tag\_Loss\_Empirical}.
\defRef{sec:Process-TagLossEmpirical}
\label{syntax:Process-TagLossEmpirical}

\defSub{categories}{The list of categories to apply}
\defType{Vector of strings}
\defDefault{No default}

\defSub{time\_step\_proportions}{The time step proportions for tag loss}
\defType{Vector of real numbers (estimable)}
\defDefault{true}
\defLowerBound{0.0 (inclusive)}
\defUpperBound{1.0 (inclusive)}

\defSub{selectivities}{The selectivities}
\defType{Vector of strings}
\defDefault{No default}

\defSub{year}{The year the first tagging release process was executed}
\defType{Non-negative integer}
\defDefault{No default}

\defSub{years\_at\_liberty}{Years at liberty to apply the annual tag loss rates of tags}
\defType{Vector of non-negative integers}
\defDefault{No default}
\defLowerBound{0 (inclusive)}

\subsubsection{Process of type Transition\_Category}
\commandlabsubarg{Process}{type}{Transition\_Category}.
\defRef{sec:Process-TransitionCategory}
\label{syntax:Process-TransitionCategory}

\defSub{from}{The categories to transition from}
\defType{Vector of strings}
\defDefault{No default}

\defSub{to}{The categories to transition to}
\defType{Vector of strings}
\defDefault{No default}

\defSub{proportions}{The proportions to transition for each category}
\defType{Vector of real numbers (estimable)}
\defDefault{No default}
\defLowerBound{0.0 (inclusive)}
\defUpperBound{1.0 (inclusive)}

\defSub{selectivities}{The selectivities to apply to each proportion}
\defType{Vector of strings}
\defDefault{No default}

\defSub{include\_in\_mortality\_block}{Is the process is in the mortality block}
\defType{Boolean}
\defDefault{false}

\subsubsection{Process of type Transition\_Category\_By\_Age}
\commandlabsubarg{Process}{type}{Transition\_Category\_By\_Age}.
\defRef{sec:Process-TransitionCategoryByAge}
\label{syntax:Process-TransitionCategoryByAge}

\defSub{from}{The categories to transition from}
\defType{Vector of strings}
\defDefault{No default}

\defSub{to}{The categories to transition to}
\defType{Vector of strings}
\defDefault{No default}

\defSub{min\_age}{The minimum age to transition}
\defType{Non-negative integer}
\defDefault{No default}

\defSub{max\_age}{The maximum age to transition}
\defType{Non-negative integer}
\defDefault{No default}

\defSub{penalty}{The penalty label}
\defType{String}
\defDefault{No default}

\defSub{u\_max}{The maximum exploitation rate ($U\_{max}$)}
\defType{Real number (estimable)}
\defDefault{0.99}
\defLowerBound{0.0 (inclusive)}
\defUpperBound{1.0 (inclusive)}

\defSub{years}{The years to execute the transition in}
\defType{Vector of non-negative integers}
\defDefault{No default}

\subsubsection{Process of type Unit\_Tester}
\commandlabsubarg{Process}{type}{Unit\_Tester}.
\defRef{sec:Process-UnitTester}
\label{syntax:Process-UnitTester}

The Unit\_Tester type has no additional subcommands.
