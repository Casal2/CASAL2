\section{Quick reference\label{sec:quick-reference}}

\defComLab{additional\_prior}{An object of type \emph{additional\_prior}}\par
\defSub{parameter} {Name of the parameter to generate the additional prior on}
\defSub{label} {Label for the additional prior}
\defSub{type} {Type of prior}

\defComLab{ageing\_error}{An object of type \emph{ageing\_error}}\par\par
\defSub{label} {Label of the ageing error}
\defSub{type} {Type of ageing error}
\par\textbf{\commandlabsubarg{ageing\_error}{type}{data}}\par
\par\textbf{\commandlabsubarg{ageing\_error}{type}{none}}\par
\par\textbf{\commandlabsubarg{ageing\_error}{type}{normal}}\par
\defSub{cv} {The CV of the misclassification matrix}
\defSub{k} {k defines the minimum age of individuals which can be misclassified, e.g., individuals of age less than k have no ageing error}
\par\textbf{\commandlabsubarg{ageing\_error}{type}{off\_by\_one}}\par
\defSub{p1} {proportion misclassified as one year younger, e.g., the proportion of age 3 individuals that were misclassified as age 2}
\defSub{p2} {proportion misclassified as one year older, e.g., the proportion of age 3 individuals that were misclassified as age 4}
\defSub{k} {The minimum age of fish which can be misclassified, i.e., fish of age less than k are assumed to be correctly classified}

\defComLab{age\_length}{An object of type \emph{age\_length}}\par\par
\defSub{label} {Label of the age length relationship}
\defSub{type} {Type of age length relationship}
\defSub{time\_step\_proportions} {the fraction of the year applied in each time step that is added to the age for the purposes of evaluating the length, i.e., a value of 0.5 for a time step will evaluate the length of individuals at age+0.5 in that time step}
\defSub{distribution} {The distribution for the growth curve}
\defSub{cv\_first} {The CV for the first age class}
\defSub{cv\_last} {The CV for last age class}
\defSub{casal\_switch} {If true, use the less accurate equation for the cumulative normal function as was used in the legacy version of CASAL.}
\defSub{by\_length} {Specifies if the linear interpolation of CVs is a linear function of mean length at age. Default is just by age}

\par\textbf{\commandlabsubarg{age\_length}{type}{data}}\par
\defSub{external\_gaps} {TODO}
\defSub{internal\_gaps} {TODO}
\defSub{length\_weight} {The label from an associated length-weight block}
\defSub{time\_step\_measurements\_were\_made} {Time step label for which size-at-age data are provided}

\par\textbf{\commandlabsubarg{age\_length}{type}{none}}\par
\par\textbf{\commandlabsubarg{age\_length}{type}{schnute}}\par
\defSub{y1} {The y1 parameter of the Schnute relationship}
\defSub{y2} {The y2 parameter of the Schnute relationship}
\defSub{tau1} {The $\tau_1$ parameter of the Schnute relationship}
\defSub{tau2} {The $\tau_2$ parameter of the Schnute relationship}
\defSub{a} {The $a$ parameter of the Schnute relationship}
\defSub{b} {The $b$ parameter of the Schnute relationship}
\defSub{length\_weight} {The label of the associated length-weight relationship}

\par\textbf{\commandlabsubarg{age\_length}{type}{von\_bertalanffy}}\par
\defSub{linf} {The $L_{\infty}$ parameter of the von Bertalanffy relationship}
\defSub{k} {The $k$ parameter of the von Bertalanffy relationship}
\defSub{t0} {The $t_0$ parameter of the von Bertalanffy relationship}
\defSub{length\_weight} {The label of the associated length-weight relationship}

\defComLab{catchability}{An object of type \emph{catchability}}\par\par
\defSub{label} {Label of the catchability}
\defSub{type} {Type of catchability}

\defComLab{categories}{An object of type \emph{categories}}\par\par
\defSub{format} {The format that the category names adhere to}
\defSub{names} {The names of the categories to be used in the model}
\defSub{years} {The years that individual categories will be active for; this setting overrides the model values}
\defSub{age\_lengths} {The labels of age\_length objects that are assigned to categories}
\defSub{length\_weight} {The labels of the length\_weight objects that are assigned to categories}
\defSub{age\_weight} {The labels of the age\_weight objects that are assigned to categories}

\defComLab{derived\_quantity}{An object of type \emph{derived\_quantity}}\par\par
\defSub{label} {Label of the derived quantity}
\defSub{type} {Type of derived quantity}
\defSub{time\_step} {The time step in which to calculate the derived quantity}
\defSub{categories} {The list of categories to use when calculating the derived quantity}
\defSub{selectivities} {A list of one selectivity}
\defSub{time\_step\_proportion} {Proportion through the mortality block of the time step when calculated}
\defSub{time\_step\_proportion\_method} {The method for interpolating for the proportion through the mortality block}
\defSub{values} {TODO}

\par\textbf{\commandlabsubarg{derived\_quantity}{type}{abundance}}\par
\par\textbf{\commandlabsubarg{derived\_quantity}{type}{biomass}}\par

\defComLab{estimate}{An object of type \emph{estimate}}\par\par
\defSub{label} {The label of the estimate}
\defSub{type} {The prior type for the estimate}
\defSub{parameter} {The name of the parameter to estimate in the model}
\defSub{lower\_bound} {The lower bound for the parameter}
\defSub{upper\_bound} {The upper bound for the parameter}
\defSub{same} {List of parameters that are constrained to have the same value as this parameter}
\defSub{estimation\_phase} {The first estimation phase in which this parameter is estimated}
\defSub{mcmc} {Indicates if this parameter is estimated at the point estimate and fixed during MCMC estimation run}
\defSub{transformation} {Type of simple transformation to apply to estimate}
\defSub{transform\_with\_jacobian} {Apply jacobian during transformation}
\defSub{prior\_applies\_to\_transform} {Does the prior apply to the transformed parameter? This switch is a legacy switch; see the TODO Manual for more information}

\defComLab{estimate\_transformation}{An object of type \emph{estimate\_transformation}}\par\par
\defSub{label} {Label for the transformation block}
\defSub{type} {Type of transformation}
\defSub{transform\_with\_jacobian} {Apply jacobian during transformation}

\defComLab{initialisation\_phase}{An object of type \emph{initialisation\_phase}}\par\par
\defSub{label} {The label of the initialisation phase}
\defSub{type} {The type of initialisation}
\par\textbf{\commandlabsubarg{initialisation\_phase}{type}{cinitial}}\par
\defSub{categories} {The list of categories for the Cinitial initialisation}
\par\textbf{\commandlabsubarg{initialisation\_phase}{type}{derived}}\par
\defSub{insert\_processes} {Additional processes not defined in the annual cycle, that are to be inserted into this initialisation phase}
\defSub{exclude\_processes} {Processes in the annual cycle to be excluded from this initialisation phase}
\defSub{casal\_initialisation\_switch} {Run an extra annual cycle to evalaute equilibrium SSBs. Warning: if this switch is true, then this may not correctly evaluate the equilibrium state. Use true if attempting to replicate a legacy CASAL model}
\par\textbf{\commandlabsubarg{initialisation\_phase}{type}{iterative}}\par
\defSub{years} {The number of iterations (years) over which to execute this initialisation phase}
\defSub{insert\_processes} {years over which to execute this initialisation phase}
\defSub{exclude\_processes} {Processes in the annual cycle to be excluded from this initialisation phase}
\defSub{convergence\_years} {The iteration (year) when the test for converegence (lambda) is evaluated}
\defSub{lambda} {The maximum value of the absolute sum of differences between the partition at year-1 and year that indicates successful convergence}
\par\textbf{\commandlabsubarg{initialisation\_phase}{type}{state\_category\_by\_age}}\par
\defSub{categories} {The list of categories for the category state initialisation}
\defSub{min\_age} {The minimum age of values supplied in the definition of the category state}
\defSub{max\_age} {The maximum age of values supplied in the definition of the category state}

\defComLab{length\_weight}{An object of type \emph{length\_weight}}\par\par
\defSub{label} {The label of the length-weight relationship}
\defSub{type} {The type of length-weight relationship}

\defComLab{likelihood}{An object of type \emph{likelihood}}\par\par

\defComLab{mcmc}{An object of type \emph{mcmc}}\par\par
\defSub{label} {The label of the MCMC}
\defSub{type} {The type of MCMC}
\defSub{length} {The number of iterations in the MCMC chain}
\defSub{active} {Indicates if this is the active MCMC algorithm}
\defSub{print\_default\_reports} {Indicates if the output prints the default reports}
\defSub{step\_size} {Initial step size (as a multiplier of the approximate covariance matrix)}

\defComLab{minimiser}{An object of type \emph{minimiser}}\par\par
\defSub{label} {The label of the minimiser}
\defSub{type} {The type of minimiser}
\defSub{active} {Indicates if this minimiser is active}
\defSub{covariance} {Indicates if the covariance matrix should be calculated}

\defComLab{model}{An object of type \emph{model}}\par\par
\defSub{start\_year} {The first year of the model, immediately following initialisation}
\defSub{final\_year} {The final year of the model, excluding years in the projection period}
\defSub{min\_age} {The minimum age of individuals in the population}
\defSub{max\_age} {The maximum age of individuals in the population}
\defSub{age\_plus} {The oldest age or extra length midpoint (plus group age or size) as a plus group}
\defSub{initialisation\_phases} {The labels of the phases of the initialisation}
\defSub{time\_steps} {The labels of the time steps, in time sequential order, to form the annual cycle}
\defSub{projection\_final\_year} {The final year of the model in projection mode}
\defSub{length\_bins} {The minimum length in each length bin}
\defSub{length\_plus} {The switch to specify if there is a length plus group or not}
\defSub{length\_plus\_group} {Mean length of length plus group}
\defSub{base\_weight\_units} {The units for the base weight. This will be the default unit of any weight input parameters}

\defComLab{observation}{An object of type \emph{observation}}\par\par
\defSub{label} {The label of the observation}
\defSub{type} {The type of observation}
\defSub{likelihood} {The type of likelihood to use}
\defSub{categories} {The category labels to use}
\defSub{delta} {The robustification value for the likelihood}
\defSub{simulation\_likelihood} {The simulation likelihood}
\defSub{likelihood\_multiplier} {The likelihood score multiplier}
\defSub{error\_value\_multiplier} {The error value multiplier for the likelihood}

\par\textbf{\commandlabsubarg{observation}{type}{abundance}}\par
\defSub{selectivities} {The label of the selectivity for the observation}
\defSub{time\_step} {The label of time-step that the observation occurs in}

\par\textbf{\commandlabsubarg{observation}{type}{biomass}}\par
\defSub{catchability} {The label of the catchability for the observation}
\defSub{time\_step} {The label of time-step that the observation occurs in}
\defSub{obs} {The observed values}
\defSub{years} {The years of the observed values}
\defSub{error\_value} {The error values of the observed values; the units depend on the likelihood}
\defSub{selectivities} {The label of the selectivity}
\defSub{process\_error} {The value for process error}
\defSub{age\_weight\_labels} {The labels for the \command{$age\_weight$} block which corresponds to each category}

\par\textbf{\commandlabsubarg{observation}{type}{process\_removals\_by\_age}}\par
\defSub{min\_age} {The minimum age for the observation}
\defSub{max\_age} {The maximum age for the observation}
\defSub{plus\_group} {The switch to indicate the use of the age plus group}
\defSub{time\_step} {The label of the time-step that the observation occurs in}
\defSub{tolerance} {The tolerance}
\defSub{years} {The years for which there are observations}
\defSub{process\_errors} {The label of process error for the observation}
\defSub{ageing\_error} {The label of ageing error for the observation}
\defSub{method\_of\_removal} {The label of observed method of removals}
\defSub{mortality\_instantaneous\_process} {The label of the mortality instantaneous process for the observation}

\par\textbf{\commandlabsubarg{observation}{type}{process\_removals\_by\_age\_retained}}\par
\defSub{min\_age} {The minimum age for the observation}
\defSub{max\_age} {The maximum age for the observation}
\defSub{plus\_group} {The switch to indicate the use of the age plus group}
\defSub{time\_step} {The label of the time-step that the observation occurs in}
\defSub{tolerance} {The tolerance}
\defSub{years} {The years for which there are observations}
\defSub{process\_errors} {The label of process error for the observation}
\defSub{ageing\_error} {The label of ageing error for the observation}
\defSub{method\_of\_removal} {The label of observed method of removals}
\defSub{mortality\_instantaneous\_process} {The label of the mortality instantaneous process for the observation}

\par\textbf{\commandlabsubarg{observation}{type}{process\_removals\_by\_age\_retained\_total}}\par
\defSub{min\_age} {The minimum age for the observation}
\defSub{max\_age} {The maximum age for the observation}
\defSub{plus\_group} {The switch to indicate the use of the age plus group}
\defSub{time\_step} {The label of the time-step that the observation occurs in}
\defSub{tolerance} {The tolerance}
\defSub{years} {The years for which there are observations}
\defSub{process\_errors} {The label of process error for the observation}
\defSub{ageing\_error} {The label of ageing error for the observation}
\defSub{method\_of\_removal} {The label of observed method of removals}
\defSub{mortality\_instantaneous\_process} {The label of the mortality instantaneous process for the observation}

\par\textbf{\commandlabsubarg{observation}{type}{process\_removals\_by\_length}}\par
\defSub{length\_bins} {The length bins for the observation}
\defSub{time\_step} {The label of the time-step that the observation occurs in}
\defSub{length\_plus} {The switch to indicate that the last bin is a plus group}
\defSub{tolerance} {The tolerance for rescaling proportions}
\defSub{years} {The years for which there are observations}
\defSub{process\_errors} {The value of process error}
\defSub{method\_of\_removal} {The label of observed method of removals}
\defSub{mortality\_instantaneous\_process} {The label of the mortality instantaneous process for the observation}

\par\textbf{\commandlabsubarg{observation}{type}{process\_removals\_by\_length\_retained}}\par
\defSub{length\_bins} {The length bins for the observation}
\defSub{time\_step} {The label of the time-step that the observation occurs in}
\defSub{length\_plus} {The switch to indicate that the last bin is a plus group}
\defSub{tolerance} {The tolerance for rescaling proportions}
\defSub{years} {The years for which there are observations}
\defSub{process\_errors} {The value of process error}
\defSub{method\_of\_removal} {The label of observed method of removals}
\defSub{mortality\_instantaneous\_process} {The label of the mortality instantaneous process for the observation}

\par\textbf{\commandlabsubarg{observation}{type}{process\_removals\_by\_length\_retained\_total}}\par
\defSub{length\_bins} {The length bins for the observation}
\defSub{time\_step} {The label of the time-step that the observation occurs in}
\defSub{length\_plus} {The switch to indicate that the last bin is a plus group}
\defSub{tolerance} {The tolerance for rescaling proportions}
\defSub{years} {The years for which there are observations}
\defSub{process\_errors} {The value of process error}
\defSub{method\_of\_removal} {The label of observed method of removals}
\defSub{mortality\_instantaneous\_process} {The label of the mortality instantaneous process for the observation}

\par\textbf{\commandlabsubarg{observation}{type}{proportions\_at\_age}}\par
\defSub{min\_age} {The minimum age}
\defSub{max\_age} {The maximum age}
\defSub{plus\_group} {The switch to indicate the use of the age plus group}
\defSub{time\_step} {The label of time-step that the observation occurs in}
\defSub{tolerance} {Tolerance on the constraint that for each year the sum of proportions in each age must equal one, e.g., tolerance = 0.1 then 1 - Sum(Proportions) can be as large as 0.1}
\defSub{years} {The years of the observed values}
\defSub{selectivities} {The label of the selectivity}
\defSub{process\_errors} {The value of the process error}
\defSub{ageing\_error} {The label of ageing error to use}

\par\textbf{\commandlabsubarg{observation}{type}{proportions\_at\_length}}\par
\defSub{time\_step} {The label of time-step that the observation occurs in}
\defSub{tolerance} {The tolerance for rescaling proportions}
\defSub{years} {The years for which there are observations}
\defSub{selectivities} {The label of the selectivity}
\defSub{process\_errors} {The value of the process error}

\par\textbf{\commandlabsubarg{observation}{type}{proportions\_by\_category}}\par
\defSub{min\_age} {The minimum age}
\defSub{max\_age} {The maximum age}
\defSub{time\_step} {The label of time-step that the observation occurs in}
\defSub{plus\_group} {The switch to indicate the use of the age plus group}
\defSub{years} {The years for which there are observations}
\defSub{selectivities} {The label of the selectivity}
\defSub{categories2} {Target Categories}
\defSub{selectivities2} {Target Selectivities}

\par\textbf{\commandlabsubarg{observation}{type}{proportions\_mature\_by\_age}}\par
\defSub{min\_age} {The minimum age}
\defSub{max\_age} {The maximum age}
\defSub{time\_step} {The label of time-step that the observation occurs in}
\defSub{plus\_group} {The switch to indicate the use of the age plus group}
\defSub{years} {The years for which there are observations}
\defSub{ageing\_error} {The label of the ageing error}
\defSub{total\_categories} {All category labels that were vulnerable to sampling at the time of this observation, not including the categories already given}
\defSub{time\_step\_proportion} {Proportion through the mortality block of the time step when the observation is evaluated}

\par\textbf{\commandlabsubarg{observation}{type}{proportions\_migrating}}\par
\defSub{min\_age} {The minimum age}
\defSub{max\_age} {The maximum age}
\defSub{time\_step} {The label of time-step that the observation occurs in}
\defSub{plus\_group} {The switch to indicate the use of the age plus group}
\defSub{years} {The years for which there are observations}
\defSub{process\_error} {The value of the process error}
\defSub{ageing\_error} {The label of the ageing error}
\defSub{process} {The process label}

\par\textbf{\commandlabsubarg{observation}{type}{tag\_recapture\_by\_age}}\par
\defSub{min\_age} {The minimum age}
\defSub{max\_age} {The maximum age}
\defSub{plus\_group} {The switch to indicate the use of the age plus group}
\defSub{years} {The years for which there are observations}
\defSub{categories2} {The available categories in the partition}
\defSub{selectivities} {The label of the selectivity}
\defSub{time\_step} {The label of time-step that the observation occurs in}
\defSub{selectivities2} {The categories of tagged individuals for the observation}
\defSub{detection} {The probability of detecting a recaptured individual}
\defSub{time\_step\_proportion} {The proportion through the mortality block of the time step when the observation is evaluated}

\par\textbf{\commandlabsubarg{observation}{type}{tag\_recapture\_by\_length}}\par
\defSub{years} {The years for which there are observations}
\defSub{length\_bins} {The length bins}
\defSub{length\_plus} {The switch to indicate that the last bin is a plus group}
\defSub{selectivities} {The label of the selectivity for untagged categories}
\defSub{tagged\_selectivities} {The labels of the tag category selectivity}
\defSub{detection} {The probability of detecting a recaptured individual}
\defSub{dispersion} {The over-dispersion parameter $\phi$}
\defSub{time\_step\_proportion} {The proportion through the mortality block of the time step when the observation is evaluated}

\defComLab{penalty}{An object of type \emph{penalty}}\par\par
\defSub{label} {The label of the penalty}
\defSub{type} {The type of penalty}

\defComLab{process}{An object of type \emph{process}}\par\par
\defSub{label} {The label of the process}
\defSub{type} {The type of process}

\par\textbf{\commandlabsubarg{process}{type}{ageing}}\par
\defSub{categories} {The labels of the categories}

\par\textbf{\commandlabsubarg{process}{type}{growth\_basic}}\par
\defSub{categories} {The labels of the categories}
\defSub{number\_of\_growth\_episodes} {The number of growth episodes per year}
\defSub{growth\_time\_steps} {The label of the time-step in which each growth episode occurs}
\defSub{cv} {The CV for the growth model}
\defSub{sigma\_min} {The lower bound on sigma for the growth model}

\par\textbf{\commandlabsubarg{process}{type}{maturation}}\par
\defSub{from} {The list of categories to mature from}
\defSub{to} {The list of categories to mature to}
\defSub{selectivities} {The list of selectivities to use for maturation}
\defSub{years} {The years to be associated with rates}
\defSub{rates} {The rates to mature for each year}

\par\textbf{\commandlabsubarg{process}{type}{mortality\_constant\_rate}}\par
\defSub{categories} {The list of categories labels}
\defSub{m} {The mortality rates}
\defSub{time\_step\_ratio} {The time step ratios for the mortality rates}

\par\textbf{\commandlabsubarg{process}{type}{mortality\_event}}\par
\defSub{categories} {The category labels}
\defSub{years} {The years in which to apply the mortality process}
\defSub{catches} {The number of removals (catches) to apply for each year}
\defSub{u\_max} {The maximum exploitation rate, $U_{max}$}
\defSub{selectivities} {The list of selectivities}
\defSub{penalty} {The label of the penalty to apply if the total removals cannot be taken}

\par\textbf{\commandlabsubarg{process}{type}{mortality\_event\_biomass}}\par
\defSub{categories} {The category labels}
\defSub{selectivities} {The labels of the selectivities for each of the categories}
\defSub{years} {The years in which to apply the mortality process}
\defSub{catches} {The biomass of removals (catches) to apply for each year}
\defSub{u\_max} {The maximum exploitation rate, $U_{max}$}
\defSub{penalty} {The label of the penalty to apply if the total biomass of removals cannot be taken}

\par\textbf{\commandlabsubarg{process}{type}{mortality\_holling\_rate}}\par
\defSub{prey\_categories} {The prey category labels}
\defSub{predator\_categories} {The predator category labels}
\defSub{is\_abundance} {The switch to indicate vulnerable amount of prey and predator as abundance [true] or biomass [false]}
\defSub{a} {Holling parameter a}
\defSub{b} {Holling parameter b}
\defSub{x} {This parameter specifies the functional form: Holling function type 2 (x=2) or 3 (x=3), or generalised Michaelis Menten 1 (x=1)}
\defSub{u\_max} {The maximum exploitation rate, $U_{max}$}
\defSub{prey\_selectivities} {The selectivities for the prey categories}
\defSub{predator\_selectivities} {The selectivities for the predator categories}
\defSub{penalty} {The label of the penalty to be applied}
\defSub{years} {The years in which to apply the mortality process}

\par\textbf{\commandlabsubarg{process}{type}{mortality\_initialisation\_event}}\par
\defSub{categories} {The labels of the categories}
\defSub{catch} {The number of removals (catches) to apply for each year}
\defSub{u\_max} {The maximum exploitation rate, $U_{max}$}
\defSub{selectivities} {The list of selectivities}
\defSub{penalty} {The label of the penalty to apply if the total removals cannot be taken}

\par\textbf{\commandlabsubarg{process}{type}{mortality\_initialisation\_event\_biomass}}\par
\defSub{categories} {The labels of the categories}
\defSub{catch} {The number of removals (catches) to apply for each year}
\defSub{u\_max} {The maximum exploitation rate, $U_{max}$}
\defSub{selectivities} {The list of selectivities}
\defSub{penalty} {The label of the penalty to apply if the total removals cannot be taken}

\par\textbf{\commandlabsubarg{process}{type}{mortality\_instantaneous}}\par
\defSub{categories} {The categories for instantaneous mortality}
\defSub{m} {The natural mortality rates for each category}
\defSub{time\_step\_ratio} {The time step ratios for natural mortality}
\defSub{selectivities} {The selectivities to apply on the categories for natural mortality}

\par\textbf{\commandlabsubarg{process}{type}{mortality\_instantaneous\_retained}}\par
\defSub{categories} {The categories for instantaneous mortality}
\defSub{m} {The natural mortality rates for each category}
\defSub{time\_step\_ratio} {The time step ratios for natural mortality}
\defSub{selectivities} {The selectivities to apply on the categories for natural mortality}

\par\textbf{\commandlabsubarg{process}{type}{mortality\_prey\_suitability}}\par
\defSub{prey\_categories} {The prey category labels}
\defSub{predator\_categories} {The predator category labels}
\defSub{consumption\_rate} {The predator consumption rate}
\defSub{electivities} {The prey electivities}
\defSub{u\_max} {The maximum exploitation rate, $U_{max}$}
\defSub{prey\_selectivities} {The selectivities for the prey categories}
\defSub{predator\_selectivities} {The selectivities for the predator categories}
\defSub{penalty} {The label of penalty to be applied}
\defSub{years} {The years in which the process occurs}

\par\textbf{\commandlabsubarg{process}{type}{recruitment\_beverton\_holt}}\par
\defSub{categories} {The category labels}
\defSub{r0} {The equilibrium number of recruits, $R_0$}
\defSub{b0} {The equilibrium (female) spawning biomass, $B_0$}
\defSub{proportions} {The proportions (of what?)}
\defSub{age} {The age at recruitment}
\defSub{ssb\_offset} {The spawning biomass year offset}
\defSub{steepness} {Steepness, $h$}
\defSub{ssb} {SSB label (a derived quantity)}
\defSub{b0\_initialisation\_phase} {The initialisation phase label that "b0" is from}
\defSub{ycs\_values} {YCS Values}
\defSub{ycs\_years} {The recruitment years: a vector of years that relates to the year of the spawning event that created this cohort}
\defSub{standardise\_ycs\_years} {The years that are included for year class standardisation}

\par\textbf{\commandlabsubarg{process}{type}{recruitment\_beverton\_holt\_with\_deviations}}\par
\defSub{categories} {The category labels}
\defSub{r0} {The equilibrium number of recruits, $R_0$}
\defSub{b0} {The equilibrium (female) spawning biomass, $B_0$}
\defSub{proportions} {The proportions (of what?)}
\defSub{age} {The age at recruitment}
\defSub{ssb\_offset} {The spawning biomass year offset}
\defSub{steepness} {Steepness, $h$}
\defSub{ssb} {SSB label (a derived quantity)}
\defSub{sigma\_r} {The recruitment standard deviation, $\sigma_R$}
\defSub{b\_max} {The maximum bias adjustment}
\defSub{last\_year\_with\_no\_bias} {The last year with no bias adjustment}
\defSub{first\_year\_with\_bias} {The first year with full bias adjustment}
\defSub{last\_year\_with\_bias} {The last year with full bias adjustment}
\defSub{first\_recent\_year\_with\_no\_bias} {The first recent year with no bias adjustment}
\defSub{b0\_initialisation\_phase} {The initialisation phase label that "b0" is from}
\defSub{deviation\_values} {The recruitment deviation values}
\defSub{deviation\_years} {The recruitment years: a vector of years that relates to the year of the spawning event that created the cohort}

\par\textbf{\commandlabsubarg{process}{type}{recruitment\_constant}}\par
\defSub{categories} {The category labels}
\defSub{proportions} {The proportions (of what?)}
\defSub{length\_bins} {The length bins that the recruits are uniformly distributed over when recruitment occurs}
\defSub{r0} {The equilibrium number of recruits, $R_0$}

\par\textbf{\commandlabsubarg{process}{type}{survival\_constant\_rate}}\par
\defSub{categories} {The list of categories}
\defSub{s} {The survival rates}
\defSub{time\_step\_ratio} {The time step ratios for S}
\defSub{selectivities} {The selectivity label}

\par\textbf{\commandlabsubarg{process}{type}{tag\_by\_age}}\par
\defSub{from} {The categories to transition from}
\defSub{to} {The categories to transition to}
\defSub{min\_age} {The minimum age to transition}
\defSub{max\_age} {The maximum age to transition}
\defSub{penalty} {The penalty label}
\defSub{u\_max} {The maximum exploitation rate, $U_{max}$}
\defSub{years} {The years to execute the transition in}
\defSub{initial\_mortality} {TODO}
\defSub{initial\_mortality\_selectivity} {TODO}
\defSub{loss\_rate} {TODO}
\defSub{loss\_rate\_selectivities} {TODO}
\defSub{selectivities} {TODO}
\defSub{n} {TODO}

\par\textbf{\commandlabsubarg{process}{type}{tag\_by\_length}}\par
\defSub{from} {The categories to transition from}
\defSub{to} {The categories to transition to}
\defSub{penalty} {The penalty label}
\defSub{u\_max} {The maximum exploitation rate, $U_{max}$}
\defSub{years} {The years to execute the transition in}
\defSub{initial\_mortality} {TODO}
\defSub{initial\_mortality\_selectivity} {TODO}
\defSub{selectivities} {TODI}
\defSub{n} {TODO}

\par\textbf{\commandlabsubarg{process}{type}{tag\_loss}}\par
\defSub{categories} {The list of categories}
\defSub{tag\_loss\_rate} {The tag loss rates}
\defSub{time\_step\_ratio} {The time step ratios for Tag Loss}
\defSub{tag\_loss\_type} {The type of tag loss}
\defSub{selectivities} {The selectivities}
\defSub{year} {The year the first tagging release process was executed}

\par\textbf{\commandlabsubarg{process}{type}{transition\_category}}\par
\defSub{from} {The categories to transition from}
\defSub{to} {The categories to transition to}
\defSub{proportions} {The proportions (of what?)}
\defSub{selectivities} {The selectivity labels}

\par\textbf{\commandlabsubarg{process}{type}{transition\_category\_by\_age}}\par
\defSub{from} {The categories to transition from}
\defSub{to} {The categories to transition to}
\defSub{min\_age} {The minimum age to transition}
\defSub{max\_age} {The maximum age to transition}
\defSub{penalty} {The penalty label}
\defSub{u\_max} {The maximum exploitation rate, $U_{max}$}
\defSub{years} {The years to execute the transition in}

\defComLab{profile}{An object of type \emph{profile}}\par\par
\defSub{label} {The label of the profile}
\defSub{steps} {The number of steps to take between the lower and upper bound}
\defSub{lower\_bound} {The lower bounds}
\defSub{upper\_bound} {The upper bounds}
\defSub{parameter} {The system parameter to profile}
\defSub{same} {A parameter that is constrained to have the same value as the parameter being profiled}

\defComLab{project}{An object of type \emph{project}}\par\par
\defSub{label} {The label of the projections}
\defSub{type} {The type of projections}
\defSub{years} {The years to recalculate the values}
\defSub{parameter} {The parameter to project}
\defSub{multiplier} {The value of the multiplier that is applied to the projected value}

\defComLab{report}{An object of type \emph{report}}\par\par
\defSub{label} {The label for the report}
\defSub{type} {The type of report}
\defSub{file\_name} {The file name to output this report to a separate file}
\defSub{write\_mode} {The write mode}

\par\textbf{\commandlabsubarg{report}{type}{age\_length}}\par
\defSub{time\_step} {The time step label}
\defSub{years} {The years (for what?)}
\defSub{age\_length} {TODO}
\defSub{category} {TODO}

\par\textbf{\commandlabsubarg{report}{type}{ageing\_error\_matrix}}\par
\defSub{ageing\_error} {The ageing error label}

\par\textbf{\commandlabsubarg{report}{type}{initialisation\_partition\_mean\_weight}}\par

\par\textbf{\commandlabsubarg{report}{type}{partition}}\par
\defSub{time\_step} {The time step label}
\defSub{years} {The years (for what?)}

\par\textbf{\commandlabsubarg{report}{type}{partition\_biomass}}\par
\defSub{time\_step} {The time step label}
\defSub{years} {The years (for what?)}

\par\textbf{\commandlabsubarg{report}{type}{partition\_mean\_weight}}\par
\defSub{time\_step} {The time step label}
\defSub{years} {The years (for what?)}

\par\textbf{\commandlabsubarg{report}{type}{partition\_year\_cross\_age\_matrix}}\par

\defComLab{selectivity}{An object of type \emph{selectivity}}\par\par
\defSub{label} {The label for this selectivity}
\defSub{type} {The type of selectivity}
\defSub{length\_based} {The switch to indicate that the selectivity is length based}
\defSub{intervals} {The number of quantiles to evaluate a length based selectivity over the age length distribution}
\defSub{partition\_type} {The type of partition this selectivity will support; defaults to the same type as the model}
\defSub{values} {TODO}
\defSub{length\_values} {TODO}

\defComLab{simulate}{An object of type \emph{simulate}}\par\par
\defSub{label} {The label of the simulation}
\defSub{type} {The type of simulation}
\defSub{years} {The years to recalculate the values}
\defSub{parameter} {The parameter to simulate}

\defComLab{time\_step}{An object of type \emph{time\_step}}\par\par
\defSub{label} {The label of the time step}
\defSub{processes} {The labels of the processes for this time step in the order that they occur}

\defComLab{time\_varying}{An object of type \emph{time\_varying}}\par\par
\defSub{label} {The time-varying label}
\defSub{type} {The time-varying type}
\defSub{years} {The years in which to vary the values}
\defSub{parameter} {The name of the parameter to vary over time}
