\documentclass{article}
\usepackage{graphicx}
\usepackage{amsmath}
\newcommand{\command}[1] {\texttt{@#1}}
\begin{document}
	\title{Catch at age Documentation}
	\author{Craig Marsh and Ian Doonan}
	\maketitle
	
	This is a technical document to show why there is a change in code base for how catch at age observations were implemented from CASAL to CASAL2. In CASAL \texttt{\command catch\_at} observations occur in the middle of the fishing process. In CASAL2 we don't derive an expectation for an observation in the process, reasons why (a model may not have \texttt{\command catch\_at} observation so if you embed an observation in a process it removes flexibility). The information we do have is the state of the partition before and after the process. This document shows that we can have equivalent expectations, using two slightly different implementation of the same observation. When natural mortality and a fishing event occur in the same time step the user can choose to implement instantaneous mortality or Baranov. For now CASAL2 does not have the Baranov equation implemented, so that is ignored from here on, we are also going to ignore age structure and only have the one category for simplicity\\
	
	\underline{Context}: Firstly the instantaneous fishing process is described. \\
				$$P_1 = \text{Partition before instantaneous mortality}$$
				$$P_2 = \text{Partition after instantaneous mortality}$$
				Instantaneous Mortality  $= e^{-m/2}(1-u)e^{-m/2}$, where $m$ is natural mortality and $u$ is the fishing exploitation for that time step. This states that fishing is assumed to occur, once half natural mortlity has occured. Thus the partition before fishing occurs ($P_{begin}$), is defined as,
				
				$$P_{begin} = P_1e^{-m/2}$$
				and the partition after fishing occurs ($P_{end}$) is defined as,
				\begin{equation}\label{eq:cas}
				P_{end} = P_{begin}(1-u)
				\end{equation}

				
	CASAL compares catch at age observations to the partition at the following point in the method.
	 $$Catch = P_1e^{-m/2}u$$
	The above information is requested during the process is implemented in CASAL. In CASAL 2 we only know the values $P_1$ and $P_2$. The following shows that if we have knowledge on $P_1, P_2, m$ and $u$ we can replicate CASAL's partition where Catch at age observations are compared. 
		\begin{equation}
		\begin{split}
			P_2 = P_{end}e^{-m/2} & = P_1e^{-m/2}(1-u)e^{-m/2}\\
			& = (P_1e^{-m/2} - P_1e^{-m/2}u)e^{-m/2}\\
			& = P_1e^{-m} - P_1e^{-m}u\\	
		\end{split}
		\end{equation}

		\begin{equation}\label{eq:dif}
		P_2 - P_1 = P_1e^{-m} - P_1e^{-m}u - P_1\\
		\end{equation}
		
	We first start by rearanging Equation~\ref{eq:dif} for $P_1e^{-m}u$,
	  
		\begin{equation}\label{eq:rearrange}
		 P_1e^{-m}u = P_1e^{-m} - P_2\\
		\end{equation}	
		
	For comparison of CASAL we need $P_1e^{-m/2}u$ so multiply both sides by $e^{m/2}$
			    
		\begin{equation}\label{eq:final}
		\begin{split}
		P_1e^{-m/2}u &= (P_1e^{-m} - P_2)e^{m/2}\\
		&= P_1e^{-m/2} - P_2e^{m/2}
		\end{split}
		\end{equation}    
		
	An adjustment must be made if $u$ contains multiple fisheries i.e. $u = u_1 + u_2$,

		$$  P_1e^{-m/2}(u_1 + u_2) = (P_1e^{-m} - P_2)e^{m/2}$$
		$$  P_1e^{-m/2}u_1(u_1 + u_2) = (P_1e^{-m} - P_2)e^{m/2}u_1 $$
		\begin{equation}\label{eq:cas2}
		P_1e^{-m/2}u_1 = (P_1e^{-m} - P_2)e^{m/2}\frac{u_1}{(u_1 + u_2)}
		\end{equation}


CASAL implements Equation~\ref{eq:cas}

CASAL2 implements Equation~\ref{eq:cas2}
	
	
	
\end{document}